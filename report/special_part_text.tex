\subsection{Расчетные формулы} 
\begin{equation}
\label{eq:L_T_integral}
q_{ч}=P Ce, \: q_{км}=\frac{q_{ч}}{3.6 V}, \: L_{кс}= \int_{m_к}^{m_н} \frac{dm}{q_{км}},\: T_{кс}= \int_{m_к}^{m_н} \frac{dm}{q_{ч}},
\end{equation}

\begin{equation}
\label{eq:P_potr}
P_п = \frac{mg}{K} 
\end{equation}



\begin{equation}
    P_р(M, H)= P_{р\, 11} \frac{p_H}{p_{H=11}}
\end{equation}

\begin{equation}
    P_р(M, H) = \bar{P}_0 m g \tilde{P}(H,\, M) 
\end{equation}

\[ 
q_ч = Ce \frac{mg}{K}, \: q_{км} = \frac{mg Ce}{3.6KV}, \: L_{кс}= \frac{3.6}{g}\int_{m_к}^{m_н} \frac{KV}{Ce m}
\, dm, \: T_{кс}= \frac{1}{g} \int_{m_к}^{m_н} \frac{K}{Ce m}\, dm
\]
$C_{ya}, C_{xa}$ из курсовой работы №1 по динамике полета.

%\subsection{Задачи}
%По мере уменьшения массы из-за выгорания топлива в крейсерском полете будет уменьшаться $P_п$ из формулы (\ref{eq:P_potr}), что ведет к уменьшению расхода топлива.  
%Провдем такие количественные анализы:  
%\begin{enumerate}
%    \item Влияние массы на изменение экономической скорости.  
%    \item Оптимальную траекторию с учетом выгорания топлива.
%    \item Найти моменты смены эшелона для перехода на экономиечески выгодный эшелон. 
%    \item Разница в расходах топлива при полете на постоянной высоте и со сменой высоты.
%\end{enumerate}
