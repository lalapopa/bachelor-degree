\subsection{Расчетные формулы} 

\begin{equation}
\label{eq:L_T_integral}
q_{ч}=P_{р} Ce, \: q_{км}=\frac{q_{ч}}{3.6 V}, \: L_{кс}= \int_{m_к}^{m_н} \frac{dm}{q_{км}},\: T_{кс}= \int_{m_к}^{m_н} \frac{dm}{q_{ч}},
\end{equation}
\begin{equation}
    P_п(M, H)= \frac{mg}{K} 
    \label{eq:P_potr}
\end{equation}
\begin{equation}
    P_р(M, H)= P_{р_{11}} \frac{p_H}{p_{H=11}},
\end{equation}
\begin{equation}
    P_р(M, H) = \bar{P}_0 m g \tilde{P}(H,\, M),
\end{equation}
\begin{equation}
Ce = Ce_0 \tilde{Ce}(H,M) \hat{Ce}_{др}(R),
\end{equation}
\begin{equation}
    L_{кс}= \frac{3.6}{\bar{P}_0 Ce_0g} \int_{m_к}^{m_н} \frac{V}{m
    \tilde{P}(H,\,M) \tilde{Ce}(H,\, M) \hat{Ce}_{др}(\bar{R})}   \, dm,
\end{equation}
\begin{equation}
    T_{кс}= \frac{1}{g} \int_{m_к}^{m_н} \frac{1}{m \tilde{P}(H,\,M)
\tilde{Ce}(H,\, M) \hat{Ce}_{др}(\bar{R})}\, dm
\end{equation}

$C_{ya}, C_{xa}$ из курсовой работы №1 по динамике полета.

\subsection{Задачи}
По мере уменьшения массы из-за выгорания топлива в крейсерском полете будет уменьшаться $P_п$ из формулы (\ref{eq:P_potr}), что ведет к изменению расхода топлива.  

Проведем такие количественные анализы:  
\begin{enumerate}
    \item Влияние массы на изменение экономической скорости.
    \item Оптимальную траекторию с учетом выгорания топлива.
    \item Найти моменты смены эшелона для перехода на экономически выгодный эшелон. 
    \item Разница в расходах топлива при полете на постоянной высоте и со сменой высоты.
\end{enumerate}
