\section{Вывод}
В данном разделе была получена траектория эшелонированного полета для
обеспечения минимального расхода топлива. Такая траектория с исходными данными
самолета прототипа дает разницу в 0.11 \% по сравнению с оптимальной
траекторией в количестве израсходованного топлива. Что дает разницу в избытке
топлива на 10 полетов равной в 346.5 кг. К сравнению при полете на одной высоте
разница составляет 1.19 \%, что дает избыток топлива на 10 полетов равный 3667 
кг.

Отсюда следует, что экономически выгодно выполнять крейсерский полет на
эшелоне, который обеспечивает минимум километрового расхода топлива. 

