\renewcommand{\thechapterfont}{\normalsize\bfseries} % Номер главы полужирным
\renewcommand{\prethechapter}{} % Убираем слово "глава"
\renewcommand{\chapteralign}{\raggedright}
\renewcommand{\postthechapter}{.~} % ставим точку и пробел после номер

\renewcommand{\appendixfont}{\normalsize\bfseries}
\renewcommand{\chapterfont}{\normalsize\bfseries}
\renewcommand{\sectionfont}{\normalsize\bfseries}
\renewcommand{\subsectionfont}{\normalsize\bfseries}
\renewcommand{\subsubsectionfont}{\normalsize\bfseries}
\renewcommand{\tocprethechapter}{} % в оглавлении убираем слово "Глава"

\setcounter{tocdepth}{4} % Глубина tableofcontent
\setcounter{secnumdepth}{4} % Глубина tableofcontent
\pagestyle{footcenter}
\chapterpagestyle{footcenter} % Расположение номеров страниц снизу

% Рисуники и таблица
\usepackage[tableposition=top]{caption}
\usepackage{subcaption}
\DeclareCaptionLabelFormat{gostfigure}{Рисунок #2}
\DeclareCaptionLabelFormat{gosttable}{Таблица #2}
\DeclareCaptionLabelSeparator{gost}{~---~}
\captionsetup{labelsep=gost}
\captionsetup[figure]{labelformat=gostfigure}
\captionsetup[table]{labelformat=gosttable}
\renewcommand{\thesubfigure}{\asbuk{subfigure}}

