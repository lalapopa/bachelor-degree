\section{Описание объекта управления}

Один из разрабатываемых режимов для САУ -- это система автоматической
стабилизации высоты, применяемая в системе автоматического пилотирования.
Дополнительные данные необходимые для расчетов приведены в таблицах \ref{tab:data},
\ref{tab:aero_coeff}. 

Управление самолетом в полете осуществляется отклонением руля высоты,
стабилизатора, руля направления, элеронов и спойлеров. Система ручного
управления необратимая бустерная, с возможностью перехода на ручное управление. 

Самолет оснащен системой автоматического управления САУ-1Т-2Б, которая является
частью пилотажно-навигационного коплекса ПК-76. САУ-1Т-2Б обеспечивает:
автоматическое и директорное пилотирование по заданному маршруту в диапазонах
высот от 400 м до максимальной высоты полета в режимах набора высота,
горизонтального полета и снижения, заход на посадку до высоты 60 м в
автоматическом и директором режимах.


\begin{table}[H]
    \centering
    \caption{Исходные данные самолета Ил-76}
    \label{tab:data}
    \begin{tabular}{|p{0.2\textwidth}|c|}
        \hline
        Параметр & Значение \\ 
        \hline
        $b_a$ & 6.436 м \\
        \hline
        $ \delta_в$ & $15^\circ ... -21^\circ$\\
        \hline
        $\varphi$ & $ +2^\circ ... -8^\circ$ \\
        \hline
        $\bar{x}_\text{т}$ & 0.45 \\
        \hline
        $I_z$ & $ 19\cdot10^6 \ \text{кг}\,\text{м}^2$\\
        \hline
    \end{tabular}
\end{table}

\begin{table}[H]
\centering
\caption{Значения производных коэффициентов аэродинамических сил и моментов для разных чисел Маха}
\label{tab:aero_coeff}
\begin{tabular}{|c|c|c|c|c|c|c|c|c|}
\hline
$M$ & $C_y^\alpha$ & $\bar{x}_{F}$ & $m_z^{\bar{\omega}_z}$ & $m_{z}^{\bar{\dot{\alpha}}}$ & $m_z^{\delta_в}$ \\
\hline
0.3 &        5.160 &         0.695 &                 -11.09 &                        -7.75 &          -2.7215 \\
0.4 &        5.160 &         0.690 &                 -11.09 &                        -7.75 &          -2.7215 \\
0.5 &        5.160 &         0.695 &                 -11.09 &                        -7.75 &          -2.7215 \\
0.6 &        5.160 &         0.710 &                 -11.09 &                        -7.75 &          -2.7215 \\
0.7 &        5.350 &         0.728 &                 -11.09 &                        -7.75 &          -2.7215 \\
0.8 &        6.150 &         0.764 &                 -11.09 &                        -7.75 &          -2.7215 \\
\hline
\end{tabular}
\end{table}

\subsection{Построение области высот и скоростей}
Аналогичный расчет был проведен в разделе \ref{sec:flight_param}. 
Исходя из области высот и скоростей (рисунок \ref{fig:H_M}), найдем узловые
точки для расчета коэффициентов обратных связей которые представлены в таблице
\ref{tab:fa}.

\begin{table}[H]
    \centering
    \caption{Узловые точки для расчета}
    \label{tab:fa}
\begin{tabular}{|r|rrrrrr|}
    \hline
    $H,\ м$ & \multicolumn{6}{c|}{$M$}\\
\hline
0 & 0.240 & 0.302 & 0.364 & 0.426 & 0.488 & 0.612 \\
\hline
2000 & 0.270 & 0.337 & 0.404 & 0.471 & 0.537 & 0.671 \\
\hline
4000 & 0.307 & 0.372 & 0.438 & 0.503 & 0.568 & 0.699 \\
\hline
6000 & 0.352 & 0.414 & 0.477 & 0.539 & 0.601 & 0.726 \\
\hline
8000 & 0.406 & 0.463 & 0.519 & 0.575 & 0.631 & 0.744 \\
\hline
10000 & 0.475 & 0.519 & 0.563 & 0.607 & 0.651 & 0.739 \\
\hline
11558 & 0.544 & 0.564 & 0.584 & 0.604 & 0.624 & 0.664 \\
\hline
\end{tabular}
\end{table}

\subsection{Выбор параметров привода}
Приближенно привод можно представить как:
\[
    W_\text{пр} = \frac{1}{ T_\text{пр}^2 p^2 + 2 \xi_{\text{пр}}T_\text{пр}+ 1}, 
\]
где $\xi_{\text{пр}}=0.7$. Для нахождения $T_{\text{пр}}$ найдем собственные частоты для самолета
\[
\omega_\text{с} = \sqrt{-\bar{M}_z^\alpha - \bar{M}_z^{ \omega_z} \bar{Y}^\alpha},
\]
во всех узловых точках. Выберем $\omega_{max}$ --- максимальное значение
$\omega_{\text{с}}$ из всей рассчитанной области. Найдем $T_{{\text{пр}}_{\text{теор}}} =
\frac{1}{10\omega_{max}}$.
Из ряда: 
\[
  T_{\text{пр}}^* = [0.02 \ 0.025 \ 0.003 \ 0.035 \ 0.04 \ 0.045 \ 0.05]  
\]
выберем ближайшее значение к $T_{{\text{пр}}_{\text{теор}}}$ которое будет $T_{\text{пр}}$.

Расчеты по нахождению $\omega_с$ сведены в таблицу \ref{tab:driver_tab}, откуда:
\[
    \omega_{max} = 2.2517,\ T_{\text{пр}} = 0.045.
\]

\begin{table}[H]
    \centering
    \caption{Результаты расчета $\omega_{\text{с}}$}
    \label{tab:driver_tab}
    \begin{tabular}{|r|rrrrrr|}
    \hline
    $H,\ м$ & \multicolumn{6}{c|}{Значения $\omega_с$ для узловых точек}\\
    \hline
0 & 0.85414 & 1.07530 & 1.29070 & 1.51050 & 1.74080 & 2.2517\cellcolor{green} \\
2000 & 0.83279 & 1.03530 & 1.23460 & 1.44800 & 1.67410 &                  2.1879 \\
4000 & 0.81355 & 0.98132 & 1.15520 & 1.33700 & 1.53730 &                  1.9895 \\
6000 & 0.79583 & 0.93461 & 1.08250 & 1.24080 & 1.40870 &                  1.8298 \\
8000 & 0.78184 & 0.89540 & 1.01380 & 1.14180 & 1.28030 &                  1.6339 \\
10000 & 0.78185 & 0.86125 & 0.94614 & 1.03410 & 1.13170 &                  1.3725 \\
11558 & 0.79699 & 0.83101 & 0.86535 & 0.90071 & 0.93916 &                  1.0177 \\
    \hline
    \end{tabular}
\end{table}

\subsection{Вывод}
В данном разделе были получены узловые точки для расчетов из области высот и скоростей. Также были определены параметры привода, которые равны:
\[
    \xi_{пр} = 0.7,\ T_{пр} = 0.045.
\]

