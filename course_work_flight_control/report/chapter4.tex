\section{Нелинейное моделирование САУ}
Нелинейное моделирование будет проводится для скоростного режима $M_\text{кр}$
на крейсерской высоте $H = 10000\ \text{м}$ для двух максимальных скоростей
отклонения руля высоты $\dot{\delta}_\text{в max} = 15\, \frac{\text{град.}}{\text{сек.}},\ 60\, \frac{\text{град.}}{\text{сек.}}$. Также будут введены ограничения на:

\begin{itemize}
    \item Диапазон отклонения руля высоты $\delta_\text{в} = -21^\circ ... 15^\circ$ 
    \item Диапазон угла тангажа в наборе $\vartheta = -6.5^\circ...6.5^\circ$
\end{itemize}

Схема нелинейной модели из <<Simulink>> представлена на рисунке \ref{fig:nonlinear_model_simulink}. 
Блок с названием <<i\_H>> соответствует коэффициенту $i_H$, <<K\_theta\_int>>
---  $K_\vartheta$, <<W\_p>> --- $W_{п}$, <<d\_w\_d\_v>> --- $\left\{ \frac{\Delta
\omega_z}{\Delta \delta_{в}} \right\} $, <<K\_omega\_z\_int>> --- $K_{\omega_z}$,
<<W\_H\_theta>> --- $\left\{ \frac{\Delta H}{\Delta \vartheta} \right\} $.

\begin{figure}[H]
    \centering
    \includegraphics[clip, trim=0.4cm 6cm 0.4cm 7cm , width=1.00\textwidth]{./figures/model.pdf}
    \caption{Схема нелинейной модели}
    \label{fig:nonlinear_model_simulink}
\end{figure}

\subsection{Сравнение для разных максимальных скоростей отклонения руля высоты}

Графики изменения $\Delta H$, $\delta_{в}$, $\omega_z$, $\vartheta$ для
$\dot{\delta}_\text{в max} = 15\, \frac{\text{град.}}{\text{сек.}},\ 60\, \frac{\text{град.}}{\text{сек.}}$
представлены на рисунках \ref{fig:model_DD_Delta_H},
\ref{fig:model_DD_delta_elevator}, \ref{fig:model_DD_omega_z}, \ref{fig:model_DD_theta}.

\begin{figure}[H]
    \begin{minipage}{0.48\textwidth}
    \centering
    \resizebox{1.1\linewidth}{!}{%% Creator: Matplotlib, PGF backend
%%
%% To include the figure in your LaTeX document, write
%%   \input{<filename>.pgf}
%%
%% Make sure the required packages are loaded in your preamble
%%   \usepackage{pgf}
%%
%% Figures using additional raster images can only be included by \input if
%% they are in the same directory as the main LaTeX file. For loading figures
%% from other directories you can use the `import` package
%%   \usepackage{import}
%%
%% and then include the figures with
%%   \import{<path to file>}{<filename>.pgf}
%%
%% Matplotlib used the following preamble
%%   \usepackage[warn]{mathtext}
%%   \usepackage[T2A]{fontenc}
%%   \usepackage[utf8]{inputenc}
%%   \usepackage[english,russian]{babel}
%%
\begingroup%
\makeatletter%
\begin{pgfpicture}%
\pgfpathrectangle{\pgfpointorigin}{\pgfqpoint{5.882184in}{4.404267in}}%
\pgfusepath{use as bounding box, clip}%
\begin{pgfscope}%
\pgfsetbuttcap%
\pgfsetmiterjoin%
\definecolor{currentfill}{rgb}{1.000000,1.000000,1.000000}%
\pgfsetfillcolor{currentfill}%
\pgfsetlinewidth{0.000000pt}%
\definecolor{currentstroke}{rgb}{1.000000,1.000000,1.000000}%
\pgfsetstrokecolor{currentstroke}%
\pgfsetdash{}{0pt}%
\pgfpathmoveto{\pgfqpoint{0.000000in}{0.000000in}}%
\pgfpathlineto{\pgfqpoint{5.882184in}{0.000000in}}%
\pgfpathlineto{\pgfqpoint{5.882184in}{4.404267in}}%
\pgfpathlineto{\pgfqpoint{0.000000in}{4.404267in}}%
\pgfpathclose%
\pgfusepath{fill}%
\end{pgfscope}%
\begin{pgfscope}%
\pgfsetbuttcap%
\pgfsetmiterjoin%
\definecolor{currentfill}{rgb}{1.000000,1.000000,1.000000}%
\pgfsetfillcolor{currentfill}%
\pgfsetlinewidth{0.000000pt}%
\definecolor{currentstroke}{rgb}{0.000000,0.000000,0.000000}%
\pgfsetstrokecolor{currentstroke}%
\pgfsetstrokeopacity{0.000000}%
\pgfsetdash{}{0pt}%
\pgfpathmoveto{\pgfqpoint{0.724269in}{0.608267in}}%
\pgfpathlineto{\pgfqpoint{5.684269in}{0.608267in}}%
\pgfpathlineto{\pgfqpoint{5.684269in}{4.304267in}}%
\pgfpathlineto{\pgfqpoint{0.724269in}{4.304267in}}%
\pgfpathclose%
\pgfusepath{fill}%
\end{pgfscope}%
\begin{pgfscope}%
\pgfpathrectangle{\pgfqpoint{0.724269in}{0.608267in}}{\pgfqpoint{4.960000in}{3.696000in}}%
\pgfusepath{clip}%
\pgfsetrectcap%
\pgfsetroundjoin%
\pgfsetlinewidth{0.803000pt}%
\definecolor{currentstroke}{rgb}{0.690196,0.690196,0.690196}%
\pgfsetstrokecolor{currentstroke}%
\pgfsetdash{}{0pt}%
\pgfpathmoveto{\pgfqpoint{0.724269in}{0.608267in}}%
\pgfpathlineto{\pgfqpoint{0.724269in}{4.304267in}}%
\pgfusepath{stroke}%
\end{pgfscope}%
\begin{pgfscope}%
\pgfsetbuttcap%
\pgfsetroundjoin%
\definecolor{currentfill}{rgb}{0.000000,0.000000,0.000000}%
\pgfsetfillcolor{currentfill}%
\pgfsetlinewidth{0.803000pt}%
\definecolor{currentstroke}{rgb}{0.000000,0.000000,0.000000}%
\pgfsetstrokecolor{currentstroke}%
\pgfsetdash{}{0pt}%
\pgfsys@defobject{currentmarker}{\pgfqpoint{0.000000in}{-0.048611in}}{\pgfqpoint{0.000000in}{0.000000in}}{%
\pgfpathmoveto{\pgfqpoint{0.000000in}{0.000000in}}%
\pgfpathlineto{\pgfqpoint{0.000000in}{-0.048611in}}%
\pgfusepath{stroke,fill}%
}%
\begin{pgfscope}%
\pgfsys@transformshift{0.724269in}{0.608267in}%
\pgfsys@useobject{currentmarker}{}%
\end{pgfscope}%
\end{pgfscope}%
\begin{pgfscope}%
\definecolor{textcolor}{rgb}{0.000000,0.000000,0.000000}%
\pgfsetstrokecolor{textcolor}%
\pgfsetfillcolor{textcolor}%
\pgftext[x=0.724269in,y=0.511045in,,top]{\color{textcolor}\rmfamily\fontsize{14.000000}{16.800000}\selectfont \(\displaystyle {0}\)}%
\end{pgfscope}%
\begin{pgfscope}%
\pgfpathrectangle{\pgfqpoint{0.724269in}{0.608267in}}{\pgfqpoint{4.960000in}{3.696000in}}%
\pgfusepath{clip}%
\pgfsetrectcap%
\pgfsetroundjoin%
\pgfsetlinewidth{0.803000pt}%
\definecolor{currentstroke}{rgb}{0.690196,0.690196,0.690196}%
\pgfsetstrokecolor{currentstroke}%
\pgfsetdash{}{0pt}%
\pgfpathmoveto{\pgfqpoint{1.550935in}{0.608267in}}%
\pgfpathlineto{\pgfqpoint{1.550935in}{4.304267in}}%
\pgfusepath{stroke}%
\end{pgfscope}%
\begin{pgfscope}%
\pgfsetbuttcap%
\pgfsetroundjoin%
\definecolor{currentfill}{rgb}{0.000000,0.000000,0.000000}%
\pgfsetfillcolor{currentfill}%
\pgfsetlinewidth{0.803000pt}%
\definecolor{currentstroke}{rgb}{0.000000,0.000000,0.000000}%
\pgfsetstrokecolor{currentstroke}%
\pgfsetdash{}{0pt}%
\pgfsys@defobject{currentmarker}{\pgfqpoint{0.000000in}{-0.048611in}}{\pgfqpoint{0.000000in}{0.000000in}}{%
\pgfpathmoveto{\pgfqpoint{0.000000in}{0.000000in}}%
\pgfpathlineto{\pgfqpoint{0.000000in}{-0.048611in}}%
\pgfusepath{stroke,fill}%
}%
\begin{pgfscope}%
\pgfsys@transformshift{1.550935in}{0.608267in}%
\pgfsys@useobject{currentmarker}{}%
\end{pgfscope}%
\end{pgfscope}%
\begin{pgfscope}%
\definecolor{textcolor}{rgb}{0.000000,0.000000,0.000000}%
\pgfsetstrokecolor{textcolor}%
\pgfsetfillcolor{textcolor}%
\pgftext[x=1.550935in,y=0.511045in,,top]{\color{textcolor}\rmfamily\fontsize{14.000000}{16.800000}\selectfont \(\displaystyle {5}\)}%
\end{pgfscope}%
\begin{pgfscope}%
\pgfpathrectangle{\pgfqpoint{0.724269in}{0.608267in}}{\pgfqpoint{4.960000in}{3.696000in}}%
\pgfusepath{clip}%
\pgfsetrectcap%
\pgfsetroundjoin%
\pgfsetlinewidth{0.803000pt}%
\definecolor{currentstroke}{rgb}{0.690196,0.690196,0.690196}%
\pgfsetstrokecolor{currentstroke}%
\pgfsetdash{}{0pt}%
\pgfpathmoveto{\pgfqpoint{2.377602in}{0.608267in}}%
\pgfpathlineto{\pgfqpoint{2.377602in}{4.304267in}}%
\pgfusepath{stroke}%
\end{pgfscope}%
\begin{pgfscope}%
\pgfsetbuttcap%
\pgfsetroundjoin%
\definecolor{currentfill}{rgb}{0.000000,0.000000,0.000000}%
\pgfsetfillcolor{currentfill}%
\pgfsetlinewidth{0.803000pt}%
\definecolor{currentstroke}{rgb}{0.000000,0.000000,0.000000}%
\pgfsetstrokecolor{currentstroke}%
\pgfsetdash{}{0pt}%
\pgfsys@defobject{currentmarker}{\pgfqpoint{0.000000in}{-0.048611in}}{\pgfqpoint{0.000000in}{0.000000in}}{%
\pgfpathmoveto{\pgfqpoint{0.000000in}{0.000000in}}%
\pgfpathlineto{\pgfqpoint{0.000000in}{-0.048611in}}%
\pgfusepath{stroke,fill}%
}%
\begin{pgfscope}%
\pgfsys@transformshift{2.377602in}{0.608267in}%
\pgfsys@useobject{currentmarker}{}%
\end{pgfscope}%
\end{pgfscope}%
\begin{pgfscope}%
\definecolor{textcolor}{rgb}{0.000000,0.000000,0.000000}%
\pgfsetstrokecolor{textcolor}%
\pgfsetfillcolor{textcolor}%
\pgftext[x=2.377602in,y=0.511045in,,top]{\color{textcolor}\rmfamily\fontsize{14.000000}{16.800000}\selectfont \(\displaystyle {10}\)}%
\end{pgfscope}%
\begin{pgfscope}%
\pgfpathrectangle{\pgfqpoint{0.724269in}{0.608267in}}{\pgfqpoint{4.960000in}{3.696000in}}%
\pgfusepath{clip}%
\pgfsetrectcap%
\pgfsetroundjoin%
\pgfsetlinewidth{0.803000pt}%
\definecolor{currentstroke}{rgb}{0.690196,0.690196,0.690196}%
\pgfsetstrokecolor{currentstroke}%
\pgfsetdash{}{0pt}%
\pgfpathmoveto{\pgfqpoint{3.204269in}{0.608267in}}%
\pgfpathlineto{\pgfqpoint{3.204269in}{4.304267in}}%
\pgfusepath{stroke}%
\end{pgfscope}%
\begin{pgfscope}%
\pgfsetbuttcap%
\pgfsetroundjoin%
\definecolor{currentfill}{rgb}{0.000000,0.000000,0.000000}%
\pgfsetfillcolor{currentfill}%
\pgfsetlinewidth{0.803000pt}%
\definecolor{currentstroke}{rgb}{0.000000,0.000000,0.000000}%
\pgfsetstrokecolor{currentstroke}%
\pgfsetdash{}{0pt}%
\pgfsys@defobject{currentmarker}{\pgfqpoint{0.000000in}{-0.048611in}}{\pgfqpoint{0.000000in}{0.000000in}}{%
\pgfpathmoveto{\pgfqpoint{0.000000in}{0.000000in}}%
\pgfpathlineto{\pgfqpoint{0.000000in}{-0.048611in}}%
\pgfusepath{stroke,fill}%
}%
\begin{pgfscope}%
\pgfsys@transformshift{3.204269in}{0.608267in}%
\pgfsys@useobject{currentmarker}{}%
\end{pgfscope}%
\end{pgfscope}%
\begin{pgfscope}%
\definecolor{textcolor}{rgb}{0.000000,0.000000,0.000000}%
\pgfsetstrokecolor{textcolor}%
\pgfsetfillcolor{textcolor}%
\pgftext[x=3.204269in,y=0.511045in,,top]{\color{textcolor}\rmfamily\fontsize{14.000000}{16.800000}\selectfont \(\displaystyle {15}\)}%
\end{pgfscope}%
\begin{pgfscope}%
\pgfpathrectangle{\pgfqpoint{0.724269in}{0.608267in}}{\pgfqpoint{4.960000in}{3.696000in}}%
\pgfusepath{clip}%
\pgfsetrectcap%
\pgfsetroundjoin%
\pgfsetlinewidth{0.803000pt}%
\definecolor{currentstroke}{rgb}{0.690196,0.690196,0.690196}%
\pgfsetstrokecolor{currentstroke}%
\pgfsetdash{}{0pt}%
\pgfpathmoveto{\pgfqpoint{4.030935in}{0.608267in}}%
\pgfpathlineto{\pgfqpoint{4.030935in}{4.304267in}}%
\pgfusepath{stroke}%
\end{pgfscope}%
\begin{pgfscope}%
\pgfsetbuttcap%
\pgfsetroundjoin%
\definecolor{currentfill}{rgb}{0.000000,0.000000,0.000000}%
\pgfsetfillcolor{currentfill}%
\pgfsetlinewidth{0.803000pt}%
\definecolor{currentstroke}{rgb}{0.000000,0.000000,0.000000}%
\pgfsetstrokecolor{currentstroke}%
\pgfsetdash{}{0pt}%
\pgfsys@defobject{currentmarker}{\pgfqpoint{0.000000in}{-0.048611in}}{\pgfqpoint{0.000000in}{0.000000in}}{%
\pgfpathmoveto{\pgfqpoint{0.000000in}{0.000000in}}%
\pgfpathlineto{\pgfqpoint{0.000000in}{-0.048611in}}%
\pgfusepath{stroke,fill}%
}%
\begin{pgfscope}%
\pgfsys@transformshift{4.030935in}{0.608267in}%
\pgfsys@useobject{currentmarker}{}%
\end{pgfscope}%
\end{pgfscope}%
\begin{pgfscope}%
\definecolor{textcolor}{rgb}{0.000000,0.000000,0.000000}%
\pgfsetstrokecolor{textcolor}%
\pgfsetfillcolor{textcolor}%
\pgftext[x=4.030935in,y=0.511045in,,top]{\color{textcolor}\rmfamily\fontsize{14.000000}{16.800000}\selectfont \(\displaystyle {20}\)}%
\end{pgfscope}%
\begin{pgfscope}%
\pgfpathrectangle{\pgfqpoint{0.724269in}{0.608267in}}{\pgfqpoint{4.960000in}{3.696000in}}%
\pgfusepath{clip}%
\pgfsetrectcap%
\pgfsetroundjoin%
\pgfsetlinewidth{0.803000pt}%
\definecolor{currentstroke}{rgb}{0.690196,0.690196,0.690196}%
\pgfsetstrokecolor{currentstroke}%
\pgfsetdash{}{0pt}%
\pgfpathmoveto{\pgfqpoint{4.857602in}{0.608267in}}%
\pgfpathlineto{\pgfqpoint{4.857602in}{4.304267in}}%
\pgfusepath{stroke}%
\end{pgfscope}%
\begin{pgfscope}%
\pgfsetbuttcap%
\pgfsetroundjoin%
\definecolor{currentfill}{rgb}{0.000000,0.000000,0.000000}%
\pgfsetfillcolor{currentfill}%
\pgfsetlinewidth{0.803000pt}%
\definecolor{currentstroke}{rgb}{0.000000,0.000000,0.000000}%
\pgfsetstrokecolor{currentstroke}%
\pgfsetdash{}{0pt}%
\pgfsys@defobject{currentmarker}{\pgfqpoint{0.000000in}{-0.048611in}}{\pgfqpoint{0.000000in}{0.000000in}}{%
\pgfpathmoveto{\pgfqpoint{0.000000in}{0.000000in}}%
\pgfpathlineto{\pgfqpoint{0.000000in}{-0.048611in}}%
\pgfusepath{stroke,fill}%
}%
\begin{pgfscope}%
\pgfsys@transformshift{4.857602in}{0.608267in}%
\pgfsys@useobject{currentmarker}{}%
\end{pgfscope}%
\end{pgfscope}%
\begin{pgfscope}%
\definecolor{textcolor}{rgb}{0.000000,0.000000,0.000000}%
\pgfsetstrokecolor{textcolor}%
\pgfsetfillcolor{textcolor}%
\pgftext[x=4.857602in,y=0.511045in,,top]{\color{textcolor}\rmfamily\fontsize{14.000000}{16.800000}\selectfont \(\displaystyle {25}\)}%
\end{pgfscope}%
\begin{pgfscope}%
\pgfpathrectangle{\pgfqpoint{0.724269in}{0.608267in}}{\pgfqpoint{4.960000in}{3.696000in}}%
\pgfusepath{clip}%
\pgfsetrectcap%
\pgfsetroundjoin%
\pgfsetlinewidth{0.803000pt}%
\definecolor{currentstroke}{rgb}{0.690196,0.690196,0.690196}%
\pgfsetstrokecolor{currentstroke}%
\pgfsetdash{}{0pt}%
\pgfpathmoveto{\pgfqpoint{5.684269in}{0.608267in}}%
\pgfpathlineto{\pgfqpoint{5.684269in}{4.304267in}}%
\pgfusepath{stroke}%
\end{pgfscope}%
\begin{pgfscope}%
\pgfsetbuttcap%
\pgfsetroundjoin%
\definecolor{currentfill}{rgb}{0.000000,0.000000,0.000000}%
\pgfsetfillcolor{currentfill}%
\pgfsetlinewidth{0.803000pt}%
\definecolor{currentstroke}{rgb}{0.000000,0.000000,0.000000}%
\pgfsetstrokecolor{currentstroke}%
\pgfsetdash{}{0pt}%
\pgfsys@defobject{currentmarker}{\pgfqpoint{0.000000in}{-0.048611in}}{\pgfqpoint{0.000000in}{0.000000in}}{%
\pgfpathmoveto{\pgfqpoint{0.000000in}{0.000000in}}%
\pgfpathlineto{\pgfqpoint{0.000000in}{-0.048611in}}%
\pgfusepath{stroke,fill}%
}%
\begin{pgfscope}%
\pgfsys@transformshift{5.684269in}{0.608267in}%
\pgfsys@useobject{currentmarker}{}%
\end{pgfscope}%
\end{pgfscope}%
\begin{pgfscope}%
\definecolor{textcolor}{rgb}{0.000000,0.000000,0.000000}%
\pgfsetstrokecolor{textcolor}%
\pgfsetfillcolor{textcolor}%
\pgftext[x=5.684269in,y=0.511045in,,top]{\color{textcolor}\rmfamily\fontsize{14.000000}{16.800000}\selectfont \(\displaystyle {30}\)}%
\end{pgfscope}%
\begin{pgfscope}%
\definecolor{textcolor}{rgb}{0.000000,0.000000,0.000000}%
\pgfsetstrokecolor{textcolor}%
\pgfsetfillcolor{textcolor}%
\pgftext[x=3.204269in,y=0.277745in,,top]{\color{textcolor}\rmfamily\fontsize{14.000000}{16.800000}\selectfont \(\displaystyle t,\ с\)}%
\end{pgfscope}%
\begin{pgfscope}%
\pgfpathrectangle{\pgfqpoint{0.724269in}{0.608267in}}{\pgfqpoint{4.960000in}{3.696000in}}%
\pgfusepath{clip}%
\pgfsetrectcap%
\pgfsetroundjoin%
\pgfsetlinewidth{0.803000pt}%
\definecolor{currentstroke}{rgb}{0.690196,0.690196,0.690196}%
\pgfsetstrokecolor{currentstroke}%
\pgfsetdash{}{0pt}%
\pgfpathmoveto{\pgfqpoint{0.724269in}{0.608267in}}%
\pgfpathlineto{\pgfqpoint{5.684269in}{0.608267in}}%
\pgfusepath{stroke}%
\end{pgfscope}%
\begin{pgfscope}%
\pgfsetbuttcap%
\pgfsetroundjoin%
\definecolor{currentfill}{rgb}{0.000000,0.000000,0.000000}%
\pgfsetfillcolor{currentfill}%
\pgfsetlinewidth{0.803000pt}%
\definecolor{currentstroke}{rgb}{0.000000,0.000000,0.000000}%
\pgfsetstrokecolor{currentstroke}%
\pgfsetdash{}{0pt}%
\pgfsys@defobject{currentmarker}{\pgfqpoint{-0.048611in}{0.000000in}}{\pgfqpoint{-0.000000in}{0.000000in}}{%
\pgfpathmoveto{\pgfqpoint{-0.000000in}{0.000000in}}%
\pgfpathlineto{\pgfqpoint{-0.048611in}{0.000000in}}%
\pgfusepath{stroke,fill}%
}%
\begin{pgfscope}%
\pgfsys@transformshift{0.724269in}{0.608267in}%
\pgfsys@useobject{currentmarker}{}%
\end{pgfscope}%
\end{pgfscope}%
\begin{pgfscope}%
\definecolor{textcolor}{rgb}{0.000000,0.000000,0.000000}%
\pgfsetstrokecolor{textcolor}%
\pgfsetfillcolor{textcolor}%
\pgftext[x=0.529131in, y=0.538840in, left, base]{\color{textcolor}\rmfamily\fontsize{14.000000}{16.800000}\selectfont \(\displaystyle {0}\)}%
\end{pgfscope}%
\begin{pgfscope}%
\pgfpathrectangle{\pgfqpoint{0.724269in}{0.608267in}}{\pgfqpoint{4.960000in}{3.696000in}}%
\pgfusepath{clip}%
\pgfsetrectcap%
\pgfsetroundjoin%
\pgfsetlinewidth{0.803000pt}%
\definecolor{currentstroke}{rgb}{0.690196,0.690196,0.690196}%
\pgfsetstrokecolor{currentstroke}%
\pgfsetdash{}{0pt}%
\pgfpathmoveto{\pgfqpoint{0.724269in}{1.176482in}}%
\pgfpathlineto{\pgfqpoint{5.684269in}{1.176482in}}%
\pgfusepath{stroke}%
\end{pgfscope}%
\begin{pgfscope}%
\pgfsetbuttcap%
\pgfsetroundjoin%
\definecolor{currentfill}{rgb}{0.000000,0.000000,0.000000}%
\pgfsetfillcolor{currentfill}%
\pgfsetlinewidth{0.803000pt}%
\definecolor{currentstroke}{rgb}{0.000000,0.000000,0.000000}%
\pgfsetstrokecolor{currentstroke}%
\pgfsetdash{}{0pt}%
\pgfsys@defobject{currentmarker}{\pgfqpoint{-0.048611in}{0.000000in}}{\pgfqpoint{-0.000000in}{0.000000in}}{%
\pgfpathmoveto{\pgfqpoint{-0.000000in}{0.000000in}}%
\pgfpathlineto{\pgfqpoint{-0.048611in}{0.000000in}}%
\pgfusepath{stroke,fill}%
}%
\begin{pgfscope}%
\pgfsys@transformshift{0.724269in}{1.176482in}%
\pgfsys@useobject{currentmarker}{}%
\end{pgfscope}%
\end{pgfscope}%
\begin{pgfscope}%
\definecolor{textcolor}{rgb}{0.000000,0.000000,0.000000}%
\pgfsetstrokecolor{textcolor}%
\pgfsetfillcolor{textcolor}%
\pgftext[x=0.431216in, y=1.107055in, left, base]{\color{textcolor}\rmfamily\fontsize{14.000000}{16.800000}\selectfont \(\displaystyle {20}\)}%
\end{pgfscope}%
\begin{pgfscope}%
\pgfpathrectangle{\pgfqpoint{0.724269in}{0.608267in}}{\pgfqpoint{4.960000in}{3.696000in}}%
\pgfusepath{clip}%
\pgfsetrectcap%
\pgfsetroundjoin%
\pgfsetlinewidth{0.803000pt}%
\definecolor{currentstroke}{rgb}{0.690196,0.690196,0.690196}%
\pgfsetstrokecolor{currentstroke}%
\pgfsetdash{}{0pt}%
\pgfpathmoveto{\pgfqpoint{0.724269in}{1.744698in}}%
\pgfpathlineto{\pgfqpoint{5.684269in}{1.744698in}}%
\pgfusepath{stroke}%
\end{pgfscope}%
\begin{pgfscope}%
\pgfsetbuttcap%
\pgfsetroundjoin%
\definecolor{currentfill}{rgb}{0.000000,0.000000,0.000000}%
\pgfsetfillcolor{currentfill}%
\pgfsetlinewidth{0.803000pt}%
\definecolor{currentstroke}{rgb}{0.000000,0.000000,0.000000}%
\pgfsetstrokecolor{currentstroke}%
\pgfsetdash{}{0pt}%
\pgfsys@defobject{currentmarker}{\pgfqpoint{-0.048611in}{0.000000in}}{\pgfqpoint{-0.000000in}{0.000000in}}{%
\pgfpathmoveto{\pgfqpoint{-0.000000in}{0.000000in}}%
\pgfpathlineto{\pgfqpoint{-0.048611in}{0.000000in}}%
\pgfusepath{stroke,fill}%
}%
\begin{pgfscope}%
\pgfsys@transformshift{0.724269in}{1.744698in}%
\pgfsys@useobject{currentmarker}{}%
\end{pgfscope}%
\end{pgfscope}%
\begin{pgfscope}%
\definecolor{textcolor}{rgb}{0.000000,0.000000,0.000000}%
\pgfsetstrokecolor{textcolor}%
\pgfsetfillcolor{textcolor}%
\pgftext[x=0.431216in, y=1.675270in, left, base]{\color{textcolor}\rmfamily\fontsize{14.000000}{16.800000}\selectfont \(\displaystyle {40}\)}%
\end{pgfscope}%
\begin{pgfscope}%
\pgfpathrectangle{\pgfqpoint{0.724269in}{0.608267in}}{\pgfqpoint{4.960000in}{3.696000in}}%
\pgfusepath{clip}%
\pgfsetrectcap%
\pgfsetroundjoin%
\pgfsetlinewidth{0.803000pt}%
\definecolor{currentstroke}{rgb}{0.690196,0.690196,0.690196}%
\pgfsetstrokecolor{currentstroke}%
\pgfsetdash{}{0pt}%
\pgfpathmoveto{\pgfqpoint{0.724269in}{2.312913in}}%
\pgfpathlineto{\pgfqpoint{5.684269in}{2.312913in}}%
\pgfusepath{stroke}%
\end{pgfscope}%
\begin{pgfscope}%
\pgfsetbuttcap%
\pgfsetroundjoin%
\definecolor{currentfill}{rgb}{0.000000,0.000000,0.000000}%
\pgfsetfillcolor{currentfill}%
\pgfsetlinewidth{0.803000pt}%
\definecolor{currentstroke}{rgb}{0.000000,0.000000,0.000000}%
\pgfsetstrokecolor{currentstroke}%
\pgfsetdash{}{0pt}%
\pgfsys@defobject{currentmarker}{\pgfqpoint{-0.048611in}{0.000000in}}{\pgfqpoint{-0.000000in}{0.000000in}}{%
\pgfpathmoveto{\pgfqpoint{-0.000000in}{0.000000in}}%
\pgfpathlineto{\pgfqpoint{-0.048611in}{0.000000in}}%
\pgfusepath{stroke,fill}%
}%
\begin{pgfscope}%
\pgfsys@transformshift{0.724269in}{2.312913in}%
\pgfsys@useobject{currentmarker}{}%
\end{pgfscope}%
\end{pgfscope}%
\begin{pgfscope}%
\definecolor{textcolor}{rgb}{0.000000,0.000000,0.000000}%
\pgfsetstrokecolor{textcolor}%
\pgfsetfillcolor{textcolor}%
\pgftext[x=0.431216in, y=2.243486in, left, base]{\color{textcolor}\rmfamily\fontsize{14.000000}{16.800000}\selectfont \(\displaystyle {60}\)}%
\end{pgfscope}%
\begin{pgfscope}%
\pgfpathrectangle{\pgfqpoint{0.724269in}{0.608267in}}{\pgfqpoint{4.960000in}{3.696000in}}%
\pgfusepath{clip}%
\pgfsetrectcap%
\pgfsetroundjoin%
\pgfsetlinewidth{0.803000pt}%
\definecolor{currentstroke}{rgb}{0.690196,0.690196,0.690196}%
\pgfsetstrokecolor{currentstroke}%
\pgfsetdash{}{0pt}%
\pgfpathmoveto{\pgfqpoint{0.724269in}{2.881128in}}%
\pgfpathlineto{\pgfqpoint{5.684269in}{2.881128in}}%
\pgfusepath{stroke}%
\end{pgfscope}%
\begin{pgfscope}%
\pgfsetbuttcap%
\pgfsetroundjoin%
\definecolor{currentfill}{rgb}{0.000000,0.000000,0.000000}%
\pgfsetfillcolor{currentfill}%
\pgfsetlinewidth{0.803000pt}%
\definecolor{currentstroke}{rgb}{0.000000,0.000000,0.000000}%
\pgfsetstrokecolor{currentstroke}%
\pgfsetdash{}{0pt}%
\pgfsys@defobject{currentmarker}{\pgfqpoint{-0.048611in}{0.000000in}}{\pgfqpoint{-0.000000in}{0.000000in}}{%
\pgfpathmoveto{\pgfqpoint{-0.000000in}{0.000000in}}%
\pgfpathlineto{\pgfqpoint{-0.048611in}{0.000000in}}%
\pgfusepath{stroke,fill}%
}%
\begin{pgfscope}%
\pgfsys@transformshift{0.724269in}{2.881128in}%
\pgfsys@useobject{currentmarker}{}%
\end{pgfscope}%
\end{pgfscope}%
\begin{pgfscope}%
\definecolor{textcolor}{rgb}{0.000000,0.000000,0.000000}%
\pgfsetstrokecolor{textcolor}%
\pgfsetfillcolor{textcolor}%
\pgftext[x=0.431216in, y=2.811701in, left, base]{\color{textcolor}\rmfamily\fontsize{14.000000}{16.800000}\selectfont \(\displaystyle {80}\)}%
\end{pgfscope}%
\begin{pgfscope}%
\pgfpathrectangle{\pgfqpoint{0.724269in}{0.608267in}}{\pgfqpoint{4.960000in}{3.696000in}}%
\pgfusepath{clip}%
\pgfsetrectcap%
\pgfsetroundjoin%
\pgfsetlinewidth{0.803000pt}%
\definecolor{currentstroke}{rgb}{0.690196,0.690196,0.690196}%
\pgfsetstrokecolor{currentstroke}%
\pgfsetdash{}{0pt}%
\pgfpathmoveto{\pgfqpoint{0.724269in}{3.449343in}}%
\pgfpathlineto{\pgfqpoint{5.684269in}{3.449343in}}%
\pgfusepath{stroke}%
\end{pgfscope}%
\begin{pgfscope}%
\pgfsetbuttcap%
\pgfsetroundjoin%
\definecolor{currentfill}{rgb}{0.000000,0.000000,0.000000}%
\pgfsetfillcolor{currentfill}%
\pgfsetlinewidth{0.803000pt}%
\definecolor{currentstroke}{rgb}{0.000000,0.000000,0.000000}%
\pgfsetstrokecolor{currentstroke}%
\pgfsetdash{}{0pt}%
\pgfsys@defobject{currentmarker}{\pgfqpoint{-0.048611in}{0.000000in}}{\pgfqpoint{-0.000000in}{0.000000in}}{%
\pgfpathmoveto{\pgfqpoint{-0.000000in}{0.000000in}}%
\pgfpathlineto{\pgfqpoint{-0.048611in}{0.000000in}}%
\pgfusepath{stroke,fill}%
}%
\begin{pgfscope}%
\pgfsys@transformshift{0.724269in}{3.449343in}%
\pgfsys@useobject{currentmarker}{}%
\end{pgfscope}%
\end{pgfscope}%
\begin{pgfscope}%
\definecolor{textcolor}{rgb}{0.000000,0.000000,0.000000}%
\pgfsetstrokecolor{textcolor}%
\pgfsetfillcolor{textcolor}%
\pgftext[x=0.333300in, y=3.379916in, left, base]{\color{textcolor}\rmfamily\fontsize{14.000000}{16.800000}\selectfont \(\displaystyle {100}\)}%
\end{pgfscope}%
\begin{pgfscope}%
\pgfpathrectangle{\pgfqpoint{0.724269in}{0.608267in}}{\pgfqpoint{4.960000in}{3.696000in}}%
\pgfusepath{clip}%
\pgfsetrectcap%
\pgfsetroundjoin%
\pgfsetlinewidth{0.803000pt}%
\definecolor{currentstroke}{rgb}{0.690196,0.690196,0.690196}%
\pgfsetstrokecolor{currentstroke}%
\pgfsetdash{}{0pt}%
\pgfpathmoveto{\pgfqpoint{0.724269in}{4.017559in}}%
\pgfpathlineto{\pgfqpoint{5.684269in}{4.017559in}}%
\pgfusepath{stroke}%
\end{pgfscope}%
\begin{pgfscope}%
\pgfsetbuttcap%
\pgfsetroundjoin%
\definecolor{currentfill}{rgb}{0.000000,0.000000,0.000000}%
\pgfsetfillcolor{currentfill}%
\pgfsetlinewidth{0.803000pt}%
\definecolor{currentstroke}{rgb}{0.000000,0.000000,0.000000}%
\pgfsetstrokecolor{currentstroke}%
\pgfsetdash{}{0pt}%
\pgfsys@defobject{currentmarker}{\pgfqpoint{-0.048611in}{0.000000in}}{\pgfqpoint{-0.000000in}{0.000000in}}{%
\pgfpathmoveto{\pgfqpoint{-0.000000in}{0.000000in}}%
\pgfpathlineto{\pgfqpoint{-0.048611in}{0.000000in}}%
\pgfusepath{stroke,fill}%
}%
\begin{pgfscope}%
\pgfsys@transformshift{0.724269in}{4.017559in}%
\pgfsys@useobject{currentmarker}{}%
\end{pgfscope}%
\end{pgfscope}%
\begin{pgfscope}%
\definecolor{textcolor}{rgb}{0.000000,0.000000,0.000000}%
\pgfsetstrokecolor{textcolor}%
\pgfsetfillcolor{textcolor}%
\pgftext[x=0.333300in, y=3.948131in, left, base]{\color{textcolor}\rmfamily\fontsize{14.000000}{16.800000}\selectfont \(\displaystyle {120}\)}%
\end{pgfscope}%
\begin{pgfscope}%
\definecolor{textcolor}{rgb}{0.000000,0.000000,0.000000}%
\pgfsetstrokecolor{textcolor}%
\pgfsetfillcolor{textcolor}%
\pgftext[x=0.277745in,y=2.456267in,,bottom,rotate=90.000000]{\color{textcolor}\rmfamily\fontsize{14.000000}{16.800000}\selectfont \(\displaystyle \Delta H,\ м\)}%
\end{pgfscope}%
\begin{pgfscope}%
\pgfpathrectangle{\pgfqpoint{0.724269in}{0.608267in}}{\pgfqpoint{4.960000in}{3.696000in}}%
\pgfusepath{clip}%
\pgfsetrectcap%
\pgfsetroundjoin%
\pgfsetlinewidth{2.007500pt}%
\definecolor{currentstroke}{rgb}{0.121569,0.466667,0.705882}%
\pgfsetstrokecolor{currentstroke}%
\pgfsetdash{}{0pt}%
\pgfpathmoveto{\pgfqpoint{0.724269in}{0.608267in}}%
\pgfpathlineto{\pgfqpoint{1.000375in}{0.609378in}}%
\pgfpathlineto{\pgfqpoint{1.032781in}{0.612079in}}%
\pgfpathlineto{\pgfqpoint{1.058407in}{0.616414in}}%
\pgfpathlineto{\pgfqpoint{1.081058in}{0.622474in}}%
\pgfpathlineto{\pgfqpoint{1.102386in}{0.630474in}}%
\pgfpathlineto{\pgfqpoint{1.123218in}{0.640667in}}%
\pgfpathlineto{\pgfqpoint{1.144381in}{0.653535in}}%
\pgfpathlineto{\pgfqpoint{1.166370in}{0.669555in}}%
\pgfpathlineto{\pgfqpoint{1.189847in}{0.689469in}}%
\pgfpathlineto{\pgfqpoint{1.215309in}{0.714044in}}%
\pgfpathlineto{\pgfqpoint{1.243250in}{0.744186in}}%
\pgfpathlineto{\pgfqpoint{1.273175in}{0.779798in}}%
\pgfpathlineto{\pgfqpoint{1.304423in}{0.820488in}}%
\pgfpathlineto{\pgfqpoint{1.336829in}{0.866387in}}%
\pgfpathlineto{\pgfqpoint{1.370887in}{0.918560in}}%
\pgfpathlineto{\pgfqpoint{1.406599in}{0.977397in}}%
\pgfpathlineto{\pgfqpoint{1.444295in}{1.043846in}}%
\pgfpathlineto{\pgfqpoint{1.484306in}{1.118958in}}%
\pgfpathlineto{\pgfqpoint{1.526797in}{1.203540in}}%
\pgfpathlineto{\pgfqpoint{1.572098in}{1.298767in}}%
\pgfpathlineto{\pgfqpoint{1.620375in}{1.405510in}}%
\pgfpathlineto{\pgfqpoint{1.672125in}{1.525411in}}%
\pgfpathlineto{\pgfqpoint{1.727842in}{1.660221in}}%
\pgfpathlineto{\pgfqpoint{1.787858in}{1.811353in}}%
\pgfpathlineto{\pgfqpoint{1.852999in}{1.981531in}}%
\pgfpathlineto{\pgfqpoint{1.926573in}{2.180182in}}%
\pgfpathlineto{\pgfqpoint{2.039495in}{2.492505in}}%
\pgfpathlineto{\pgfqpoint{2.151426in}{2.799890in}}%
\pgfpathlineto{\pgfqpoint{2.219378in}{2.980024in}}%
\pgfpathlineto{\pgfqpoint{2.276583in}{3.125530in}}%
\pgfpathlineto{\pgfqpoint{2.327837in}{3.249938in}}%
\pgfpathlineto{\pgfqpoint{2.374957in}{3.358579in}}%
\pgfpathlineto{\pgfqpoint{2.418935in}{3.454496in}}%
\pgfpathlineto{\pgfqpoint{2.460599in}{3.540109in}}%
\pgfpathlineto{\pgfqpoint{2.500114in}{3.616316in}}%
\pgfpathlineto{\pgfqpoint{2.537975in}{3.684584in}}%
\pgfpathlineto{\pgfqpoint{2.574349in}{3.745658in}}%
\pgfpathlineto{\pgfqpoint{2.609399in}{3.800236in}}%
\pgfpathlineto{\pgfqpoint{2.643293in}{3.848961in}}%
\pgfpathlineto{\pgfqpoint{2.676194in}{3.892417in}}%
\pgfpathlineto{\pgfqpoint{2.708103in}{3.930932in}}%
\pgfpathlineto{\pgfqpoint{2.739186in}{3.965012in}}%
\pgfpathlineto{\pgfqpoint{2.769442in}{3.994942in}}%
\pgfpathlineto{\pgfqpoint{2.799037in}{4.021145in}}%
\pgfpathlineto{\pgfqpoint{2.828135in}{4.043977in}}%
\pgfpathlineto{\pgfqpoint{2.856738in}{4.063625in}}%
\pgfpathlineto{\pgfqpoint{2.885010in}{4.080366in}}%
\pgfpathlineto{\pgfqpoint{2.912951in}{4.094344in}}%
\pgfpathlineto{\pgfqpoint{2.940727in}{4.105762in}}%
\pgfpathlineto{\pgfqpoint{2.968338in}{4.114722in}}%
\pgfpathlineto{\pgfqpoint{2.995949in}{4.121363in}}%
\pgfpathlineto{\pgfqpoint{3.023725in}{4.125770in}}%
\pgfpathlineto{\pgfqpoint{3.051831in}{4.127983in}}%
\pgfpathlineto{\pgfqpoint{3.080434in}{4.127997in}}%
\pgfpathlineto{\pgfqpoint{3.109533in}{4.125784in}}%
\pgfpathlineto{\pgfqpoint{3.139458in}{4.121266in}}%
\pgfpathlineto{\pgfqpoint{3.170375in}{4.114323in}}%
\pgfpathlineto{\pgfqpoint{3.202285in}{4.104861in}}%
\pgfpathlineto{\pgfqpoint{3.235682in}{4.092610in}}%
\pgfpathlineto{\pgfqpoint{3.270567in}{4.077421in}}%
\pgfpathlineto{\pgfqpoint{3.307437in}{4.058916in}}%
\pgfpathlineto{\pgfqpoint{3.346786in}{4.036632in}}%
\pgfpathlineto{\pgfqpoint{3.388946in}{4.010159in}}%
\pgfpathlineto{\pgfqpoint{3.434909in}{3.978619in}}%
\pgfpathlineto{\pgfqpoint{3.485831in}{3.940918in}}%
\pgfpathlineto{\pgfqpoint{3.544194in}{3.894861in}}%
\pgfpathlineto{\pgfqpoint{3.615783in}{3.835393in}}%
\pgfpathlineto{\pgfqpoint{3.729367in}{3.737700in}}%
\pgfpathlineto{\pgfqpoint{3.856343in}{3.629325in}}%
\pgfpathlineto{\pgfqpoint{3.932397in}{3.567367in}}%
\pgfpathlineto{\pgfqpoint{3.997373in}{3.517249in}}%
\pgfpathlineto{\pgfqpoint{4.056231in}{3.474608in}}%
\pgfpathlineto{\pgfqpoint{4.111287in}{3.437429in}}%
\pgfpathlineto{\pgfqpoint{4.163698in}{3.404692in}}%
\pgfpathlineto{\pgfqpoint{4.214125in}{3.375797in}}%
\pgfpathlineto{\pgfqpoint{4.263063in}{3.350302in}}%
\pgfpathlineto{\pgfqpoint{4.310845in}{3.327900in}}%
\pgfpathlineto{\pgfqpoint{4.357799in}{3.308322in}}%
\pgfpathlineto{\pgfqpoint{4.404258in}{3.291342in}}%
\pgfpathlineto{\pgfqpoint{4.450551in}{3.276782in}}%
\pgfpathlineto{\pgfqpoint{4.496845in}{3.264552in}}%
\pgfpathlineto{\pgfqpoint{4.543303in}{3.254577in}}%
\pgfpathlineto{\pgfqpoint{4.590258in}{3.246774in}}%
\pgfpathlineto{\pgfqpoint{4.637874in}{3.241122in}}%
\pgfpathlineto{\pgfqpoint{4.686482in}{3.237600in}}%
\pgfpathlineto{\pgfqpoint{4.736413in}{3.236228in}}%
\pgfpathlineto{\pgfqpoint{4.787997in}{3.237052in}}%
\pgfpathlineto{\pgfqpoint{4.841565in}{3.240146in}}%
\pgfpathlineto{\pgfqpoint{4.897943in}{3.245649in}}%
\pgfpathlineto{\pgfqpoint{4.957794in}{3.253750in}}%
\pgfpathlineto{\pgfqpoint{5.022439in}{3.264777in}}%
\pgfpathlineto{\pgfqpoint{5.093863in}{3.279267in}}%
\pgfpathlineto{\pgfqpoint{5.175703in}{3.298208in}}%
\pgfpathlineto{\pgfqpoint{5.277383in}{3.324145in}}%
\pgfpathlineto{\pgfqpoint{5.486034in}{3.380399in}}%
\pgfpathlineto{\pgfqpoint{5.617639in}{3.414409in}}%
\pgfpathlineto{\pgfqpoint{5.684269in}{3.430404in}}%
\pgfpathlineto{\pgfqpoint{5.684269in}{3.430404in}}%
\pgfusepath{stroke}%
\end{pgfscope}%
\begin{pgfscope}%
\pgfpathrectangle{\pgfqpoint{0.724269in}{0.608267in}}{\pgfqpoint{4.960000in}{3.696000in}}%
\pgfusepath{clip}%
\pgfsetbuttcap%
\pgfsetroundjoin%
\pgfsetlinewidth{2.007500pt}%
\definecolor{currentstroke}{rgb}{1.000000,0.498039,0.054902}%
\pgfsetstrokecolor{currentstroke}%
\pgfsetdash{{12.800000pt}{3.200000pt}{2.000000pt}{3.200000pt}}{0.000000pt}%
\pgfpathmoveto{\pgfqpoint{0.724269in}{0.608267in}}%
\pgfpathlineto{\pgfqpoint{0.975245in}{0.609370in}}%
\pgfpathlineto{\pgfqpoint{1.005335in}{0.612218in}}%
\pgfpathlineto{\pgfqpoint{1.031623in}{0.616923in}}%
\pgfpathlineto{\pgfqpoint{1.056423in}{0.623619in}}%
\pgfpathlineto{\pgfqpoint{1.080562in}{0.632453in}}%
\pgfpathlineto{\pgfqpoint{1.104535in}{0.643621in}}%
\pgfpathlineto{\pgfqpoint{1.128674in}{0.657354in}}%
\pgfpathlineto{\pgfqpoint{1.153143in}{0.673865in}}%
\pgfpathlineto{\pgfqpoint{1.178274in}{0.693543in}}%
\pgfpathlineto{\pgfqpoint{1.204231in}{0.716738in}}%
\pgfpathlineto{\pgfqpoint{1.231181in}{0.743855in}}%
\pgfpathlineto{\pgfqpoint{1.259287in}{0.775350in}}%
\pgfpathlineto{\pgfqpoint{1.288717in}{0.811732in}}%
\pgfpathlineto{\pgfqpoint{1.319634in}{0.853562in}}%
\pgfpathlineto{\pgfqpoint{1.352205in}{0.901451in}}%
\pgfpathlineto{\pgfqpoint{1.386594in}{0.956052in}}%
\pgfpathlineto{\pgfqpoint{1.422967in}{1.018062in}}%
\pgfpathlineto{\pgfqpoint{1.461490in}{1.088208in}}%
\pgfpathlineto{\pgfqpoint{1.502493in}{1.167575in}}%
\pgfpathlineto{\pgfqpoint{1.546141in}{1.256991in}}%
\pgfpathlineto{\pgfqpoint{1.592765in}{1.357657in}}%
\pgfpathlineto{\pgfqpoint{1.642695in}{1.470841in}}%
\pgfpathlineto{\pgfqpoint{1.696429in}{1.598259in}}%
\pgfpathlineto{\pgfqpoint{1.754295in}{1.741310in}}%
\pgfpathlineto{\pgfqpoint{1.816957in}{1.902265in}}%
\pgfpathlineto{\pgfqpoint{1.885570in}{2.084791in}}%
\pgfpathlineto{\pgfqpoint{1.971047in}{2.318960in}}%
\pgfpathlineto{\pgfqpoint{2.160850in}{2.840939in}}%
\pgfpathlineto{\pgfqpoint{2.225826in}{3.011773in}}%
\pgfpathlineto{\pgfqpoint{2.281543in}{3.152172in}}%
\pgfpathlineto{\pgfqpoint{2.331805in}{3.272921in}}%
\pgfpathlineto{\pgfqpoint{2.378263in}{3.378848in}}%
\pgfpathlineto{\pgfqpoint{2.421746in}{3.472551in}}%
\pgfpathlineto{\pgfqpoint{2.462914in}{3.556072in}}%
\pgfpathlineto{\pgfqpoint{2.502098in}{3.630624in}}%
\pgfpathlineto{\pgfqpoint{2.539629in}{3.697333in}}%
\pgfpathlineto{\pgfqpoint{2.575671in}{3.756946in}}%
\pgfpathlineto{\pgfqpoint{2.610557in}{3.810405in}}%
\pgfpathlineto{\pgfqpoint{2.644285in}{3.858072in}}%
\pgfpathlineto{\pgfqpoint{2.677021in}{3.900529in}}%
\pgfpathlineto{\pgfqpoint{2.708765in}{3.938106in}}%
\pgfpathlineto{\pgfqpoint{2.739682in}{3.971306in}}%
\pgfpathlineto{\pgfqpoint{2.769938in}{4.000567in}}%
\pgfpathlineto{\pgfqpoint{2.799533in}{4.026122in}}%
\pgfpathlineto{\pgfqpoint{2.828466in}{4.048207in}}%
\pgfpathlineto{\pgfqpoint{2.856903in}{4.067161in}}%
\pgfpathlineto{\pgfqpoint{2.885010in}{4.083257in}}%
\pgfpathlineto{\pgfqpoint{2.912951in}{4.096710in}}%
\pgfpathlineto{\pgfqpoint{2.940727in}{4.107618in}}%
\pgfpathlineto{\pgfqpoint{2.968338in}{4.116086in}}%
\pgfpathlineto{\pgfqpoint{2.996114in}{4.122278in}}%
\pgfpathlineto{\pgfqpoint{3.024055in}{4.126223in}}%
\pgfpathlineto{\pgfqpoint{3.052327in}{4.127959in}}%
\pgfpathlineto{\pgfqpoint{3.081095in}{4.127481in}}%
\pgfpathlineto{\pgfqpoint{3.110359in}{4.124763in}}%
\pgfpathlineto{\pgfqpoint{3.140450in}{4.119725in}}%
\pgfpathlineto{\pgfqpoint{3.171533in}{4.112249in}}%
\pgfpathlineto{\pgfqpoint{3.203773in}{4.102188in}}%
\pgfpathlineto{\pgfqpoint{3.237501in}{4.089302in}}%
\pgfpathlineto{\pgfqpoint{3.272882in}{4.073371in}}%
\pgfpathlineto{\pgfqpoint{3.310247in}{4.054076in}}%
\pgfpathlineto{\pgfqpoint{3.350093in}{4.030962in}}%
\pgfpathlineto{\pgfqpoint{3.393079in}{4.003407in}}%
\pgfpathlineto{\pgfqpoint{3.440034in}{3.970610in}}%
\pgfpathlineto{\pgfqpoint{3.492445in}{3.931219in}}%
\pgfpathlineto{\pgfqpoint{3.553122in}{3.882737in}}%
\pgfpathlineto{\pgfqpoint{3.629506in}{3.818693in}}%
\pgfpathlineto{\pgfqpoint{3.913053in}{3.578148in}}%
\pgfpathlineto{\pgfqpoint{3.979517in}{3.526318in}}%
\pgfpathlineto{\pgfqpoint{4.039367in}{3.482415in}}%
\pgfpathlineto{\pgfqpoint{4.095085in}{3.444262in}}%
\pgfpathlineto{\pgfqpoint{4.147826in}{3.410804in}}%
\pgfpathlineto{\pgfqpoint{4.198583in}{3.381213in}}%
\pgfpathlineto{\pgfqpoint{4.247687in}{3.355135in}}%
\pgfpathlineto{\pgfqpoint{4.295634in}{3.332166in}}%
\pgfpathlineto{\pgfqpoint{4.342754in}{3.312041in}}%
\pgfpathlineto{\pgfqpoint{4.389378in}{3.294533in}}%
\pgfpathlineto{\pgfqpoint{4.435671in}{3.279516in}}%
\pgfpathlineto{\pgfqpoint{4.481965in}{3.266834in}}%
\pgfpathlineto{\pgfqpoint{4.528423in}{3.256415in}}%
\pgfpathlineto{\pgfqpoint{4.575213in}{3.248202in}}%
\pgfpathlineto{\pgfqpoint{4.622663in}{3.242132in}}%
\pgfpathlineto{\pgfqpoint{4.671106in}{3.238189in}}%
\pgfpathlineto{\pgfqpoint{4.720706in}{3.236394in}}%
\pgfpathlineto{\pgfqpoint{4.771794in}{3.236777in}}%
\pgfpathlineto{\pgfqpoint{4.825031in}{3.239418in}}%
\pgfpathlineto{\pgfqpoint{4.880749in}{3.244425in}}%
\pgfpathlineto{\pgfqpoint{4.939938in}{3.252004in}}%
\pgfpathlineto{\pgfqpoint{5.003591in}{3.262433in}}%
\pgfpathlineto{\pgfqpoint{5.073362in}{3.276159in}}%
\pgfpathlineto{\pgfqpoint{5.152722in}{3.294102in}}%
\pgfpathlineto{\pgfqpoint{5.249111in}{3.318281in}}%
\pgfpathlineto{\pgfqpoint{5.399730in}{3.358691in}}%
\pgfpathlineto{\pgfqpoint{5.577794in}{3.405883in}}%
\pgfpathlineto{\pgfqpoint{5.684103in}{3.431708in}}%
\pgfpathlineto{\pgfqpoint{5.684269in}{3.431746in}}%
\pgfpathlineto{\pgfqpoint{5.684269in}{3.431746in}}%
\pgfusepath{stroke}%
\end{pgfscope}%
\begin{pgfscope}%
\pgfpathrectangle{\pgfqpoint{0.724269in}{0.608267in}}{\pgfqpoint{4.960000in}{3.696000in}}%
\pgfusepath{clip}%
\pgfsetbuttcap%
\pgfsetroundjoin%
\pgfsetlinewidth{4.015000pt}%
\definecolor{currentstroke}{rgb}{0.172549,0.627451,0.172549}%
\pgfsetstrokecolor{currentstroke}%
\pgfsetdash{{4.000000pt}{6.600000pt}}{0.000000pt}%
\pgfpathmoveto{\pgfqpoint{0.724269in}{0.608267in}}%
\pgfpathlineto{\pgfqpoint{0.889437in}{0.608267in}}%
\pgfpathlineto{\pgfqpoint{0.890925in}{3.449343in}}%
\pgfpathlineto{\pgfqpoint{5.684269in}{3.449343in}}%
\pgfpathlineto{\pgfqpoint{5.684269in}{3.449343in}}%
\pgfusepath{stroke}%
\end{pgfscope}%
\begin{pgfscope}%
\pgfsetrectcap%
\pgfsetmiterjoin%
\pgfsetlinewidth{0.803000pt}%
\definecolor{currentstroke}{rgb}{0.000000,0.000000,0.000000}%
\pgfsetstrokecolor{currentstroke}%
\pgfsetdash{}{0pt}%
\pgfpathmoveto{\pgfqpoint{0.724269in}{0.608267in}}%
\pgfpathlineto{\pgfqpoint{0.724269in}{4.304267in}}%
\pgfusepath{stroke}%
\end{pgfscope}%
\begin{pgfscope}%
\pgfsetrectcap%
\pgfsetmiterjoin%
\pgfsetlinewidth{0.803000pt}%
\definecolor{currentstroke}{rgb}{0.000000,0.000000,0.000000}%
\pgfsetstrokecolor{currentstroke}%
\pgfsetdash{}{0pt}%
\pgfpathmoveto{\pgfqpoint{5.684269in}{0.608267in}}%
\pgfpathlineto{\pgfqpoint{5.684269in}{4.304267in}}%
\pgfusepath{stroke}%
\end{pgfscope}%
\begin{pgfscope}%
\pgfsetrectcap%
\pgfsetmiterjoin%
\pgfsetlinewidth{0.803000pt}%
\definecolor{currentstroke}{rgb}{0.000000,0.000000,0.000000}%
\pgfsetstrokecolor{currentstroke}%
\pgfsetdash{}{0pt}%
\pgfpathmoveto{\pgfqpoint{0.724269in}{0.608267in}}%
\pgfpathlineto{\pgfqpoint{5.684269in}{0.608267in}}%
\pgfusepath{stroke}%
\end{pgfscope}%
\begin{pgfscope}%
\pgfsetrectcap%
\pgfsetmiterjoin%
\pgfsetlinewidth{0.803000pt}%
\definecolor{currentstroke}{rgb}{0.000000,0.000000,0.000000}%
\pgfsetstrokecolor{currentstroke}%
\pgfsetdash{}{0pt}%
\pgfpathmoveto{\pgfqpoint{0.724269in}{4.304267in}}%
\pgfpathlineto{\pgfqpoint{5.684269in}{4.304267in}}%
\pgfusepath{stroke}%
\end{pgfscope}%
\begin{pgfscope}%
\pgfsetbuttcap%
\pgfsetmiterjoin%
\definecolor{currentfill}{rgb}{1.000000,1.000000,1.000000}%
\pgfsetfillcolor{currentfill}%
\pgfsetfillopacity{0.800000}%
\pgfsetlinewidth{1.003750pt}%
\definecolor{currentstroke}{rgb}{0.800000,0.800000,0.800000}%
\pgfsetstrokecolor{currentstroke}%
\pgfsetstrokeopacity{0.800000}%
\pgfsetdash{}{0pt}%
\pgfpathmoveto{\pgfqpoint{3.595778in}{0.705489in}}%
\pgfpathlineto{\pgfqpoint{5.548158in}{0.705489in}}%
\pgfpathquadraticcurveto{\pgfqpoint{5.587047in}{0.705489in}}{\pgfqpoint{5.587047in}{0.744378in}}%
\pgfpathlineto{\pgfqpoint{5.587047in}{1.925096in}}%
\pgfpathquadraticcurveto{\pgfqpoint{5.587047in}{1.963985in}}{\pgfqpoint{5.548158in}{1.963985in}}%
\pgfpathlineto{\pgfqpoint{3.595778in}{1.963985in}}%
\pgfpathquadraticcurveto{\pgfqpoint{3.556889in}{1.963985in}}{\pgfqpoint{3.556889in}{1.925096in}}%
\pgfpathlineto{\pgfqpoint{3.556889in}{0.744378in}}%
\pgfpathquadraticcurveto{\pgfqpoint{3.556889in}{0.705489in}}{\pgfqpoint{3.595778in}{0.705489in}}%
\pgfpathclose%
\pgfusepath{stroke,fill}%
\end{pgfscope}%
\begin{pgfscope}%
\pgfsetrectcap%
\pgfsetroundjoin%
\pgfsetlinewidth{2.007500pt}%
\definecolor{currentstroke}{rgb}{0.121569,0.466667,0.705882}%
\pgfsetstrokecolor{currentstroke}%
\pgfsetdash{}{0pt}%
\pgfpathmoveto{\pgfqpoint{3.634667in}{1.732882in}}%
\pgfpathlineto{\pgfqpoint{4.023556in}{1.732882in}}%
\pgfusepath{stroke}%
\end{pgfscope}%
\begin{pgfscope}%
\definecolor{textcolor}{rgb}{0.000000,0.000000,0.000000}%
\pgfsetstrokecolor{textcolor}%
\pgfsetfillcolor{textcolor}%
\pgftext[x=4.179111in,y=1.664827in,left,base]{\color{textcolor}\rmfamily\fontsize{14.000000}{16.800000}\selectfont \(\displaystyle \dot{\delta}_{{в}_{max}}=15 \frac{град.}{сек.}\)}%
\end{pgfscope}%
\begin{pgfscope}%
\pgfsetbuttcap%
\pgfsetroundjoin%
\pgfsetlinewidth{2.007500pt}%
\definecolor{currentstroke}{rgb}{1.000000,0.498039,0.054902}%
\pgfsetstrokecolor{currentstroke}%
\pgfsetdash{{12.800000pt}{3.200000pt}{2.000000pt}{3.200000pt}}{0.000000pt}%
\pgfpathmoveto{\pgfqpoint{3.634667in}{1.277089in}}%
\pgfpathlineto{\pgfqpoint{4.023556in}{1.277089in}}%
\pgfusepath{stroke}%
\end{pgfscope}%
\begin{pgfscope}%
\definecolor{textcolor}{rgb}{0.000000,0.000000,0.000000}%
\pgfsetstrokecolor{textcolor}%
\pgfsetfillcolor{textcolor}%
\pgftext[x=4.179111in,y=1.209034in,left,base]{\color{textcolor}\rmfamily\fontsize{14.000000}{16.800000}\selectfont \(\displaystyle \dot{\delta}_{{в}_{max}}=60 \frac{град.}{сек.}\)}%
\end{pgfscope}%
\begin{pgfscope}%
\pgfsetbuttcap%
\pgfsetroundjoin%
\pgfsetlinewidth{4.015000pt}%
\definecolor{currentstroke}{rgb}{0.172549,0.627451,0.172549}%
\pgfsetstrokecolor{currentstroke}%
\pgfsetdash{{4.000000pt}{6.600000pt}}{0.000000pt}%
\pgfpathmoveto{\pgfqpoint{3.634667in}{0.903822in}}%
\pgfpathlineto{\pgfqpoint{4.023556in}{0.903822in}}%
\pgfusepath{stroke}%
\end{pgfscope}%
\begin{pgfscope}%
\definecolor{textcolor}{rgb}{0.000000,0.000000,0.000000}%
\pgfsetstrokecolor{textcolor}%
\pgfsetfillcolor{textcolor}%
\pgftext[x=4.179111in,y=0.835767in,left,base]{\color{textcolor}\rmfamily\fontsize{14.000000}{16.800000}\selectfont \(\displaystyle \Delta H_{зад}\)}%
\end{pgfscope}%
\end{pgfpicture}%
\makeatother%
\endgroup%
}
    \caption{Изменение высоты для различных $\dot{\delta}_\text{в max}$}
    \label{fig:model_DD_Delta_H}
    \end{minipage}
    \hfill
    \begin{minipage}{0.48\textwidth}
    \centering
    \resizebox{1.1\linewidth}{!}{%% Creator: Matplotlib, PGF backend
%%
%% To include the figure in your LaTeX document, write
%%   \input{<filename>.pgf}
%%
%% Make sure the required packages are loaded in your preamble
%%   \usepackage{pgf}
%%
%% Figures using additional raster images can only be included by \input if
%% they are in the same directory as the main LaTeX file. For loading figures
%% from other directories you can use the `import` package
%%   \usepackage{import}
%%
%% and then include the figures with
%%   \import{<path to file>}{<filename>.pgf}
%%
%% Matplotlib used the following preamble
%%   \usepackage[warn]{mathtext}
%%   \usepackage[T2A]{fontenc}
%%   \usepackage[utf8]{inputenc}
%%   \usepackage[english,russian]{babel}
%%
\begingroup%
\makeatletter%
\begin{pgfpicture}%
\pgfpathrectangle{\pgfpointorigin}{\pgfqpoint{6.092171in}{4.404267in}}%
\pgfusepath{use as bounding box, clip}%
\begin{pgfscope}%
\pgfsetbuttcap%
\pgfsetmiterjoin%
\definecolor{currentfill}{rgb}{1.000000,1.000000,1.000000}%
\pgfsetfillcolor{currentfill}%
\pgfsetlinewidth{0.000000pt}%
\definecolor{currentstroke}{rgb}{1.000000,1.000000,1.000000}%
\pgfsetstrokecolor{currentstroke}%
\pgfsetdash{}{0pt}%
\pgfpathmoveto{\pgfqpoint{0.000000in}{0.000000in}}%
\pgfpathlineto{\pgfqpoint{6.092171in}{0.000000in}}%
\pgfpathlineto{\pgfqpoint{6.092171in}{4.404267in}}%
\pgfpathlineto{\pgfqpoint{0.000000in}{4.404267in}}%
\pgfpathclose%
\pgfusepath{fill}%
\end{pgfscope}%
\begin{pgfscope}%
\pgfsetbuttcap%
\pgfsetmiterjoin%
\definecolor{currentfill}{rgb}{1.000000,1.000000,1.000000}%
\pgfsetfillcolor{currentfill}%
\pgfsetlinewidth{0.000000pt}%
\definecolor{currentstroke}{rgb}{0.000000,0.000000,0.000000}%
\pgfsetstrokecolor{currentstroke}%
\pgfsetstrokeopacity{0.000000}%
\pgfsetdash{}{0pt}%
\pgfpathmoveto{\pgfqpoint{0.934256in}{0.608267in}}%
\pgfpathlineto{\pgfqpoint{5.894256in}{0.608267in}}%
\pgfpathlineto{\pgfqpoint{5.894256in}{4.304267in}}%
\pgfpathlineto{\pgfqpoint{0.934256in}{4.304267in}}%
\pgfpathclose%
\pgfusepath{fill}%
\end{pgfscope}%
\begin{pgfscope}%
\pgfpathrectangle{\pgfqpoint{0.934256in}{0.608267in}}{\pgfqpoint{4.960000in}{3.696000in}}%
\pgfusepath{clip}%
\pgfsetrectcap%
\pgfsetroundjoin%
\pgfsetlinewidth{0.803000pt}%
\definecolor{currentstroke}{rgb}{0.690196,0.690196,0.690196}%
\pgfsetstrokecolor{currentstroke}%
\pgfsetdash{}{0pt}%
\pgfpathmoveto{\pgfqpoint{0.934256in}{0.608267in}}%
\pgfpathlineto{\pgfqpoint{0.934256in}{4.304267in}}%
\pgfusepath{stroke}%
\end{pgfscope}%
\begin{pgfscope}%
\pgfsetbuttcap%
\pgfsetroundjoin%
\definecolor{currentfill}{rgb}{0.000000,0.000000,0.000000}%
\pgfsetfillcolor{currentfill}%
\pgfsetlinewidth{0.803000pt}%
\definecolor{currentstroke}{rgb}{0.000000,0.000000,0.000000}%
\pgfsetstrokecolor{currentstroke}%
\pgfsetdash{}{0pt}%
\pgfsys@defobject{currentmarker}{\pgfqpoint{0.000000in}{-0.048611in}}{\pgfqpoint{0.000000in}{0.000000in}}{%
\pgfpathmoveto{\pgfqpoint{0.000000in}{0.000000in}}%
\pgfpathlineto{\pgfqpoint{0.000000in}{-0.048611in}}%
\pgfusepath{stroke,fill}%
}%
\begin{pgfscope}%
\pgfsys@transformshift{0.934256in}{0.608267in}%
\pgfsys@useobject{currentmarker}{}%
\end{pgfscope}%
\end{pgfscope}%
\begin{pgfscope}%
\definecolor{textcolor}{rgb}{0.000000,0.000000,0.000000}%
\pgfsetstrokecolor{textcolor}%
\pgfsetfillcolor{textcolor}%
\pgftext[x=0.934256in,y=0.511045in,,top]{\color{textcolor}\rmfamily\fontsize{14.000000}{16.800000}\selectfont \(\displaystyle {0}\)}%
\end{pgfscope}%
\begin{pgfscope}%
\pgfpathrectangle{\pgfqpoint{0.934256in}{0.608267in}}{\pgfqpoint{4.960000in}{3.696000in}}%
\pgfusepath{clip}%
\pgfsetrectcap%
\pgfsetroundjoin%
\pgfsetlinewidth{0.803000pt}%
\definecolor{currentstroke}{rgb}{0.690196,0.690196,0.690196}%
\pgfsetstrokecolor{currentstroke}%
\pgfsetdash{}{0pt}%
\pgfpathmoveto{\pgfqpoint{1.760923in}{0.608267in}}%
\pgfpathlineto{\pgfqpoint{1.760923in}{4.304267in}}%
\pgfusepath{stroke}%
\end{pgfscope}%
\begin{pgfscope}%
\pgfsetbuttcap%
\pgfsetroundjoin%
\definecolor{currentfill}{rgb}{0.000000,0.000000,0.000000}%
\pgfsetfillcolor{currentfill}%
\pgfsetlinewidth{0.803000pt}%
\definecolor{currentstroke}{rgb}{0.000000,0.000000,0.000000}%
\pgfsetstrokecolor{currentstroke}%
\pgfsetdash{}{0pt}%
\pgfsys@defobject{currentmarker}{\pgfqpoint{0.000000in}{-0.048611in}}{\pgfqpoint{0.000000in}{0.000000in}}{%
\pgfpathmoveto{\pgfqpoint{0.000000in}{0.000000in}}%
\pgfpathlineto{\pgfqpoint{0.000000in}{-0.048611in}}%
\pgfusepath{stroke,fill}%
}%
\begin{pgfscope}%
\pgfsys@transformshift{1.760923in}{0.608267in}%
\pgfsys@useobject{currentmarker}{}%
\end{pgfscope}%
\end{pgfscope}%
\begin{pgfscope}%
\definecolor{textcolor}{rgb}{0.000000,0.000000,0.000000}%
\pgfsetstrokecolor{textcolor}%
\pgfsetfillcolor{textcolor}%
\pgftext[x=1.760923in,y=0.511045in,,top]{\color{textcolor}\rmfamily\fontsize{14.000000}{16.800000}\selectfont \(\displaystyle {5}\)}%
\end{pgfscope}%
\begin{pgfscope}%
\pgfpathrectangle{\pgfqpoint{0.934256in}{0.608267in}}{\pgfqpoint{4.960000in}{3.696000in}}%
\pgfusepath{clip}%
\pgfsetrectcap%
\pgfsetroundjoin%
\pgfsetlinewidth{0.803000pt}%
\definecolor{currentstroke}{rgb}{0.690196,0.690196,0.690196}%
\pgfsetstrokecolor{currentstroke}%
\pgfsetdash{}{0pt}%
\pgfpathmoveto{\pgfqpoint{2.587589in}{0.608267in}}%
\pgfpathlineto{\pgfqpoint{2.587589in}{4.304267in}}%
\pgfusepath{stroke}%
\end{pgfscope}%
\begin{pgfscope}%
\pgfsetbuttcap%
\pgfsetroundjoin%
\definecolor{currentfill}{rgb}{0.000000,0.000000,0.000000}%
\pgfsetfillcolor{currentfill}%
\pgfsetlinewidth{0.803000pt}%
\definecolor{currentstroke}{rgb}{0.000000,0.000000,0.000000}%
\pgfsetstrokecolor{currentstroke}%
\pgfsetdash{}{0pt}%
\pgfsys@defobject{currentmarker}{\pgfqpoint{0.000000in}{-0.048611in}}{\pgfqpoint{0.000000in}{0.000000in}}{%
\pgfpathmoveto{\pgfqpoint{0.000000in}{0.000000in}}%
\pgfpathlineto{\pgfqpoint{0.000000in}{-0.048611in}}%
\pgfusepath{stroke,fill}%
}%
\begin{pgfscope}%
\pgfsys@transformshift{2.587589in}{0.608267in}%
\pgfsys@useobject{currentmarker}{}%
\end{pgfscope}%
\end{pgfscope}%
\begin{pgfscope}%
\definecolor{textcolor}{rgb}{0.000000,0.000000,0.000000}%
\pgfsetstrokecolor{textcolor}%
\pgfsetfillcolor{textcolor}%
\pgftext[x=2.587589in,y=0.511045in,,top]{\color{textcolor}\rmfamily\fontsize{14.000000}{16.800000}\selectfont \(\displaystyle {10}\)}%
\end{pgfscope}%
\begin{pgfscope}%
\pgfpathrectangle{\pgfqpoint{0.934256in}{0.608267in}}{\pgfqpoint{4.960000in}{3.696000in}}%
\pgfusepath{clip}%
\pgfsetrectcap%
\pgfsetroundjoin%
\pgfsetlinewidth{0.803000pt}%
\definecolor{currentstroke}{rgb}{0.690196,0.690196,0.690196}%
\pgfsetstrokecolor{currentstroke}%
\pgfsetdash{}{0pt}%
\pgfpathmoveto{\pgfqpoint{3.414256in}{0.608267in}}%
\pgfpathlineto{\pgfqpoint{3.414256in}{4.304267in}}%
\pgfusepath{stroke}%
\end{pgfscope}%
\begin{pgfscope}%
\pgfsetbuttcap%
\pgfsetroundjoin%
\definecolor{currentfill}{rgb}{0.000000,0.000000,0.000000}%
\pgfsetfillcolor{currentfill}%
\pgfsetlinewidth{0.803000pt}%
\definecolor{currentstroke}{rgb}{0.000000,0.000000,0.000000}%
\pgfsetstrokecolor{currentstroke}%
\pgfsetdash{}{0pt}%
\pgfsys@defobject{currentmarker}{\pgfqpoint{0.000000in}{-0.048611in}}{\pgfqpoint{0.000000in}{0.000000in}}{%
\pgfpathmoveto{\pgfqpoint{0.000000in}{0.000000in}}%
\pgfpathlineto{\pgfqpoint{0.000000in}{-0.048611in}}%
\pgfusepath{stroke,fill}%
}%
\begin{pgfscope}%
\pgfsys@transformshift{3.414256in}{0.608267in}%
\pgfsys@useobject{currentmarker}{}%
\end{pgfscope}%
\end{pgfscope}%
\begin{pgfscope}%
\definecolor{textcolor}{rgb}{0.000000,0.000000,0.000000}%
\pgfsetstrokecolor{textcolor}%
\pgfsetfillcolor{textcolor}%
\pgftext[x=3.414256in,y=0.511045in,,top]{\color{textcolor}\rmfamily\fontsize{14.000000}{16.800000}\selectfont \(\displaystyle {15}\)}%
\end{pgfscope}%
\begin{pgfscope}%
\pgfpathrectangle{\pgfqpoint{0.934256in}{0.608267in}}{\pgfqpoint{4.960000in}{3.696000in}}%
\pgfusepath{clip}%
\pgfsetrectcap%
\pgfsetroundjoin%
\pgfsetlinewidth{0.803000pt}%
\definecolor{currentstroke}{rgb}{0.690196,0.690196,0.690196}%
\pgfsetstrokecolor{currentstroke}%
\pgfsetdash{}{0pt}%
\pgfpathmoveto{\pgfqpoint{4.240923in}{0.608267in}}%
\pgfpathlineto{\pgfqpoint{4.240923in}{4.304267in}}%
\pgfusepath{stroke}%
\end{pgfscope}%
\begin{pgfscope}%
\pgfsetbuttcap%
\pgfsetroundjoin%
\definecolor{currentfill}{rgb}{0.000000,0.000000,0.000000}%
\pgfsetfillcolor{currentfill}%
\pgfsetlinewidth{0.803000pt}%
\definecolor{currentstroke}{rgb}{0.000000,0.000000,0.000000}%
\pgfsetstrokecolor{currentstroke}%
\pgfsetdash{}{0pt}%
\pgfsys@defobject{currentmarker}{\pgfqpoint{0.000000in}{-0.048611in}}{\pgfqpoint{0.000000in}{0.000000in}}{%
\pgfpathmoveto{\pgfqpoint{0.000000in}{0.000000in}}%
\pgfpathlineto{\pgfqpoint{0.000000in}{-0.048611in}}%
\pgfusepath{stroke,fill}%
}%
\begin{pgfscope}%
\pgfsys@transformshift{4.240923in}{0.608267in}%
\pgfsys@useobject{currentmarker}{}%
\end{pgfscope}%
\end{pgfscope}%
\begin{pgfscope}%
\definecolor{textcolor}{rgb}{0.000000,0.000000,0.000000}%
\pgfsetstrokecolor{textcolor}%
\pgfsetfillcolor{textcolor}%
\pgftext[x=4.240923in,y=0.511045in,,top]{\color{textcolor}\rmfamily\fontsize{14.000000}{16.800000}\selectfont \(\displaystyle {20}\)}%
\end{pgfscope}%
\begin{pgfscope}%
\pgfpathrectangle{\pgfqpoint{0.934256in}{0.608267in}}{\pgfqpoint{4.960000in}{3.696000in}}%
\pgfusepath{clip}%
\pgfsetrectcap%
\pgfsetroundjoin%
\pgfsetlinewidth{0.803000pt}%
\definecolor{currentstroke}{rgb}{0.690196,0.690196,0.690196}%
\pgfsetstrokecolor{currentstroke}%
\pgfsetdash{}{0pt}%
\pgfpathmoveto{\pgfqpoint{5.067589in}{0.608267in}}%
\pgfpathlineto{\pgfqpoint{5.067589in}{4.304267in}}%
\pgfusepath{stroke}%
\end{pgfscope}%
\begin{pgfscope}%
\pgfsetbuttcap%
\pgfsetroundjoin%
\definecolor{currentfill}{rgb}{0.000000,0.000000,0.000000}%
\pgfsetfillcolor{currentfill}%
\pgfsetlinewidth{0.803000pt}%
\definecolor{currentstroke}{rgb}{0.000000,0.000000,0.000000}%
\pgfsetstrokecolor{currentstroke}%
\pgfsetdash{}{0pt}%
\pgfsys@defobject{currentmarker}{\pgfqpoint{0.000000in}{-0.048611in}}{\pgfqpoint{0.000000in}{0.000000in}}{%
\pgfpathmoveto{\pgfqpoint{0.000000in}{0.000000in}}%
\pgfpathlineto{\pgfqpoint{0.000000in}{-0.048611in}}%
\pgfusepath{stroke,fill}%
}%
\begin{pgfscope}%
\pgfsys@transformshift{5.067589in}{0.608267in}%
\pgfsys@useobject{currentmarker}{}%
\end{pgfscope}%
\end{pgfscope}%
\begin{pgfscope}%
\definecolor{textcolor}{rgb}{0.000000,0.000000,0.000000}%
\pgfsetstrokecolor{textcolor}%
\pgfsetfillcolor{textcolor}%
\pgftext[x=5.067589in,y=0.511045in,,top]{\color{textcolor}\rmfamily\fontsize{14.000000}{16.800000}\selectfont \(\displaystyle {25}\)}%
\end{pgfscope}%
\begin{pgfscope}%
\pgfpathrectangle{\pgfqpoint{0.934256in}{0.608267in}}{\pgfqpoint{4.960000in}{3.696000in}}%
\pgfusepath{clip}%
\pgfsetrectcap%
\pgfsetroundjoin%
\pgfsetlinewidth{0.803000pt}%
\definecolor{currentstroke}{rgb}{0.690196,0.690196,0.690196}%
\pgfsetstrokecolor{currentstroke}%
\pgfsetdash{}{0pt}%
\pgfpathmoveto{\pgfqpoint{5.894256in}{0.608267in}}%
\pgfpathlineto{\pgfqpoint{5.894256in}{4.304267in}}%
\pgfusepath{stroke}%
\end{pgfscope}%
\begin{pgfscope}%
\pgfsetbuttcap%
\pgfsetroundjoin%
\definecolor{currentfill}{rgb}{0.000000,0.000000,0.000000}%
\pgfsetfillcolor{currentfill}%
\pgfsetlinewidth{0.803000pt}%
\definecolor{currentstroke}{rgb}{0.000000,0.000000,0.000000}%
\pgfsetstrokecolor{currentstroke}%
\pgfsetdash{}{0pt}%
\pgfsys@defobject{currentmarker}{\pgfqpoint{0.000000in}{-0.048611in}}{\pgfqpoint{0.000000in}{0.000000in}}{%
\pgfpathmoveto{\pgfqpoint{0.000000in}{0.000000in}}%
\pgfpathlineto{\pgfqpoint{0.000000in}{-0.048611in}}%
\pgfusepath{stroke,fill}%
}%
\begin{pgfscope}%
\pgfsys@transformshift{5.894256in}{0.608267in}%
\pgfsys@useobject{currentmarker}{}%
\end{pgfscope}%
\end{pgfscope}%
\begin{pgfscope}%
\definecolor{textcolor}{rgb}{0.000000,0.000000,0.000000}%
\pgfsetstrokecolor{textcolor}%
\pgfsetfillcolor{textcolor}%
\pgftext[x=5.894256in,y=0.511045in,,top]{\color{textcolor}\rmfamily\fontsize{14.000000}{16.800000}\selectfont \(\displaystyle {30}\)}%
\end{pgfscope}%
\begin{pgfscope}%
\definecolor{textcolor}{rgb}{0.000000,0.000000,0.000000}%
\pgfsetstrokecolor{textcolor}%
\pgfsetfillcolor{textcolor}%
\pgftext[x=3.414256in,y=0.277745in,,top]{\color{textcolor}\rmfamily\fontsize{14.000000}{16.800000}\selectfont \(\displaystyle t,\ с\)}%
\end{pgfscope}%
\begin{pgfscope}%
\pgfpathrectangle{\pgfqpoint{0.934256in}{0.608267in}}{\pgfqpoint{4.960000in}{3.696000in}}%
\pgfusepath{clip}%
\pgfsetrectcap%
\pgfsetroundjoin%
\pgfsetlinewidth{0.803000pt}%
\definecolor{currentstroke}{rgb}{0.690196,0.690196,0.690196}%
\pgfsetstrokecolor{currentstroke}%
\pgfsetdash{}{0pt}%
\pgfpathmoveto{\pgfqpoint{0.934256in}{0.716044in}}%
\pgfpathlineto{\pgfqpoint{5.894256in}{0.716044in}}%
\pgfusepath{stroke}%
\end{pgfscope}%
\begin{pgfscope}%
\pgfsetbuttcap%
\pgfsetroundjoin%
\definecolor{currentfill}{rgb}{0.000000,0.000000,0.000000}%
\pgfsetfillcolor{currentfill}%
\pgfsetlinewidth{0.803000pt}%
\definecolor{currentstroke}{rgb}{0.000000,0.000000,0.000000}%
\pgfsetstrokecolor{currentstroke}%
\pgfsetdash{}{0pt}%
\pgfsys@defobject{currentmarker}{\pgfqpoint{-0.048611in}{0.000000in}}{\pgfqpoint{-0.000000in}{0.000000in}}{%
\pgfpathmoveto{\pgfqpoint{-0.000000in}{0.000000in}}%
\pgfpathlineto{\pgfqpoint{-0.048611in}{0.000000in}}%
\pgfusepath{stroke,fill}%
}%
\begin{pgfscope}%
\pgfsys@transformshift{0.934256in}{0.716044in}%
\pgfsys@useobject{currentmarker}{}%
\end{pgfscope}%
\end{pgfscope}%
\begin{pgfscope}%
\definecolor{textcolor}{rgb}{0.000000,0.000000,0.000000}%
\pgfsetstrokecolor{textcolor}%
\pgfsetfillcolor{textcolor}%
\pgftext[x=0.333334in, y=0.646617in, left, base]{\color{textcolor}\rmfamily\fontsize{14.000000}{16.800000}\selectfont \(\displaystyle {\ensuremath{-}0.25}\)}%
\end{pgfscope}%
\begin{pgfscope}%
\pgfpathrectangle{\pgfqpoint{0.934256in}{0.608267in}}{\pgfqpoint{4.960000in}{3.696000in}}%
\pgfusepath{clip}%
\pgfsetrectcap%
\pgfsetroundjoin%
\pgfsetlinewidth{0.803000pt}%
\definecolor{currentstroke}{rgb}{0.690196,0.690196,0.690196}%
\pgfsetstrokecolor{currentstroke}%
\pgfsetdash{}{0pt}%
\pgfpathmoveto{\pgfqpoint{0.934256in}{1.214985in}}%
\pgfpathlineto{\pgfqpoint{5.894256in}{1.214985in}}%
\pgfusepath{stroke}%
\end{pgfscope}%
\begin{pgfscope}%
\pgfsetbuttcap%
\pgfsetroundjoin%
\definecolor{currentfill}{rgb}{0.000000,0.000000,0.000000}%
\pgfsetfillcolor{currentfill}%
\pgfsetlinewidth{0.803000pt}%
\definecolor{currentstroke}{rgb}{0.000000,0.000000,0.000000}%
\pgfsetstrokecolor{currentstroke}%
\pgfsetdash{}{0pt}%
\pgfsys@defobject{currentmarker}{\pgfqpoint{-0.048611in}{0.000000in}}{\pgfqpoint{-0.000000in}{0.000000in}}{%
\pgfpathmoveto{\pgfqpoint{-0.000000in}{0.000000in}}%
\pgfpathlineto{\pgfqpoint{-0.048611in}{0.000000in}}%
\pgfusepath{stroke,fill}%
}%
\begin{pgfscope}%
\pgfsys@transformshift{0.934256in}{1.214985in}%
\pgfsys@useobject{currentmarker}{}%
\end{pgfscope}%
\end{pgfscope}%
\begin{pgfscope}%
\definecolor{textcolor}{rgb}{0.000000,0.000000,0.000000}%
\pgfsetstrokecolor{textcolor}%
\pgfsetfillcolor{textcolor}%
\pgftext[x=0.333334in, y=1.145558in, left, base]{\color{textcolor}\rmfamily\fontsize{14.000000}{16.800000}\selectfont \(\displaystyle {\ensuremath{-}0.20}\)}%
\end{pgfscope}%
\begin{pgfscope}%
\pgfpathrectangle{\pgfqpoint{0.934256in}{0.608267in}}{\pgfqpoint{4.960000in}{3.696000in}}%
\pgfusepath{clip}%
\pgfsetrectcap%
\pgfsetroundjoin%
\pgfsetlinewidth{0.803000pt}%
\definecolor{currentstroke}{rgb}{0.690196,0.690196,0.690196}%
\pgfsetstrokecolor{currentstroke}%
\pgfsetdash{}{0pt}%
\pgfpathmoveto{\pgfqpoint{0.934256in}{1.713926in}}%
\pgfpathlineto{\pgfqpoint{5.894256in}{1.713926in}}%
\pgfusepath{stroke}%
\end{pgfscope}%
\begin{pgfscope}%
\pgfsetbuttcap%
\pgfsetroundjoin%
\definecolor{currentfill}{rgb}{0.000000,0.000000,0.000000}%
\pgfsetfillcolor{currentfill}%
\pgfsetlinewidth{0.803000pt}%
\definecolor{currentstroke}{rgb}{0.000000,0.000000,0.000000}%
\pgfsetstrokecolor{currentstroke}%
\pgfsetdash{}{0pt}%
\pgfsys@defobject{currentmarker}{\pgfqpoint{-0.048611in}{0.000000in}}{\pgfqpoint{-0.000000in}{0.000000in}}{%
\pgfpathmoveto{\pgfqpoint{-0.000000in}{0.000000in}}%
\pgfpathlineto{\pgfqpoint{-0.048611in}{0.000000in}}%
\pgfusepath{stroke,fill}%
}%
\begin{pgfscope}%
\pgfsys@transformshift{0.934256in}{1.713926in}%
\pgfsys@useobject{currentmarker}{}%
\end{pgfscope}%
\end{pgfscope}%
\begin{pgfscope}%
\definecolor{textcolor}{rgb}{0.000000,0.000000,0.000000}%
\pgfsetstrokecolor{textcolor}%
\pgfsetfillcolor{textcolor}%
\pgftext[x=0.333334in, y=1.644499in, left, base]{\color{textcolor}\rmfamily\fontsize{14.000000}{16.800000}\selectfont \(\displaystyle {\ensuremath{-}0.15}\)}%
\end{pgfscope}%
\begin{pgfscope}%
\pgfpathrectangle{\pgfqpoint{0.934256in}{0.608267in}}{\pgfqpoint{4.960000in}{3.696000in}}%
\pgfusepath{clip}%
\pgfsetrectcap%
\pgfsetroundjoin%
\pgfsetlinewidth{0.803000pt}%
\definecolor{currentstroke}{rgb}{0.690196,0.690196,0.690196}%
\pgfsetstrokecolor{currentstroke}%
\pgfsetdash{}{0pt}%
\pgfpathmoveto{\pgfqpoint{0.934256in}{2.212867in}}%
\pgfpathlineto{\pgfqpoint{5.894256in}{2.212867in}}%
\pgfusepath{stroke}%
\end{pgfscope}%
\begin{pgfscope}%
\pgfsetbuttcap%
\pgfsetroundjoin%
\definecolor{currentfill}{rgb}{0.000000,0.000000,0.000000}%
\pgfsetfillcolor{currentfill}%
\pgfsetlinewidth{0.803000pt}%
\definecolor{currentstroke}{rgb}{0.000000,0.000000,0.000000}%
\pgfsetstrokecolor{currentstroke}%
\pgfsetdash{}{0pt}%
\pgfsys@defobject{currentmarker}{\pgfqpoint{-0.048611in}{0.000000in}}{\pgfqpoint{-0.000000in}{0.000000in}}{%
\pgfpathmoveto{\pgfqpoint{-0.000000in}{0.000000in}}%
\pgfpathlineto{\pgfqpoint{-0.048611in}{0.000000in}}%
\pgfusepath{stroke,fill}%
}%
\begin{pgfscope}%
\pgfsys@transformshift{0.934256in}{2.212867in}%
\pgfsys@useobject{currentmarker}{}%
\end{pgfscope}%
\end{pgfscope}%
\begin{pgfscope}%
\definecolor{textcolor}{rgb}{0.000000,0.000000,0.000000}%
\pgfsetstrokecolor{textcolor}%
\pgfsetfillcolor{textcolor}%
\pgftext[x=0.333334in, y=2.143440in, left, base]{\color{textcolor}\rmfamily\fontsize{14.000000}{16.800000}\selectfont \(\displaystyle {\ensuremath{-}0.10}\)}%
\end{pgfscope}%
\begin{pgfscope}%
\pgfpathrectangle{\pgfqpoint{0.934256in}{0.608267in}}{\pgfqpoint{4.960000in}{3.696000in}}%
\pgfusepath{clip}%
\pgfsetrectcap%
\pgfsetroundjoin%
\pgfsetlinewidth{0.803000pt}%
\definecolor{currentstroke}{rgb}{0.690196,0.690196,0.690196}%
\pgfsetstrokecolor{currentstroke}%
\pgfsetdash{}{0pt}%
\pgfpathmoveto{\pgfqpoint{0.934256in}{2.711808in}}%
\pgfpathlineto{\pgfqpoint{5.894256in}{2.711808in}}%
\pgfusepath{stroke}%
\end{pgfscope}%
\begin{pgfscope}%
\pgfsetbuttcap%
\pgfsetroundjoin%
\definecolor{currentfill}{rgb}{0.000000,0.000000,0.000000}%
\pgfsetfillcolor{currentfill}%
\pgfsetlinewidth{0.803000pt}%
\definecolor{currentstroke}{rgb}{0.000000,0.000000,0.000000}%
\pgfsetstrokecolor{currentstroke}%
\pgfsetdash{}{0pt}%
\pgfsys@defobject{currentmarker}{\pgfqpoint{-0.048611in}{0.000000in}}{\pgfqpoint{-0.000000in}{0.000000in}}{%
\pgfpathmoveto{\pgfqpoint{-0.000000in}{0.000000in}}%
\pgfpathlineto{\pgfqpoint{-0.048611in}{0.000000in}}%
\pgfusepath{stroke,fill}%
}%
\begin{pgfscope}%
\pgfsys@transformshift{0.934256in}{2.711808in}%
\pgfsys@useobject{currentmarker}{}%
\end{pgfscope}%
\end{pgfscope}%
\begin{pgfscope}%
\definecolor{textcolor}{rgb}{0.000000,0.000000,0.000000}%
\pgfsetstrokecolor{textcolor}%
\pgfsetfillcolor{textcolor}%
\pgftext[x=0.333334in, y=2.642381in, left, base]{\color{textcolor}\rmfamily\fontsize{14.000000}{16.800000}\selectfont \(\displaystyle {\ensuremath{-}0.05}\)}%
\end{pgfscope}%
\begin{pgfscope}%
\pgfpathrectangle{\pgfqpoint{0.934256in}{0.608267in}}{\pgfqpoint{4.960000in}{3.696000in}}%
\pgfusepath{clip}%
\pgfsetrectcap%
\pgfsetroundjoin%
\pgfsetlinewidth{0.803000pt}%
\definecolor{currentstroke}{rgb}{0.690196,0.690196,0.690196}%
\pgfsetstrokecolor{currentstroke}%
\pgfsetdash{}{0pt}%
\pgfpathmoveto{\pgfqpoint{0.934256in}{3.210749in}}%
\pgfpathlineto{\pgfqpoint{5.894256in}{3.210749in}}%
\pgfusepath{stroke}%
\end{pgfscope}%
\begin{pgfscope}%
\pgfsetbuttcap%
\pgfsetroundjoin%
\definecolor{currentfill}{rgb}{0.000000,0.000000,0.000000}%
\pgfsetfillcolor{currentfill}%
\pgfsetlinewidth{0.803000pt}%
\definecolor{currentstroke}{rgb}{0.000000,0.000000,0.000000}%
\pgfsetstrokecolor{currentstroke}%
\pgfsetdash{}{0pt}%
\pgfsys@defobject{currentmarker}{\pgfqpoint{-0.048611in}{0.000000in}}{\pgfqpoint{-0.000000in}{0.000000in}}{%
\pgfpathmoveto{\pgfqpoint{-0.000000in}{0.000000in}}%
\pgfpathlineto{\pgfqpoint{-0.048611in}{0.000000in}}%
\pgfusepath{stroke,fill}%
}%
\begin{pgfscope}%
\pgfsys@transformshift{0.934256in}{3.210749in}%
\pgfsys@useobject{currentmarker}{}%
\end{pgfscope}%
\end{pgfscope}%
\begin{pgfscope}%
\definecolor{textcolor}{rgb}{0.000000,0.000000,0.000000}%
\pgfsetstrokecolor{textcolor}%
\pgfsetfillcolor{textcolor}%
\pgftext[x=0.488890in, y=3.141322in, left, base]{\color{textcolor}\rmfamily\fontsize{14.000000}{16.800000}\selectfont \(\displaystyle {0.00}\)}%
\end{pgfscope}%
\begin{pgfscope}%
\pgfpathrectangle{\pgfqpoint{0.934256in}{0.608267in}}{\pgfqpoint{4.960000in}{3.696000in}}%
\pgfusepath{clip}%
\pgfsetrectcap%
\pgfsetroundjoin%
\pgfsetlinewidth{0.803000pt}%
\definecolor{currentstroke}{rgb}{0.690196,0.690196,0.690196}%
\pgfsetstrokecolor{currentstroke}%
\pgfsetdash{}{0pt}%
\pgfpathmoveto{\pgfqpoint{0.934256in}{3.709690in}}%
\pgfpathlineto{\pgfqpoint{5.894256in}{3.709690in}}%
\pgfusepath{stroke}%
\end{pgfscope}%
\begin{pgfscope}%
\pgfsetbuttcap%
\pgfsetroundjoin%
\definecolor{currentfill}{rgb}{0.000000,0.000000,0.000000}%
\pgfsetfillcolor{currentfill}%
\pgfsetlinewidth{0.803000pt}%
\definecolor{currentstroke}{rgb}{0.000000,0.000000,0.000000}%
\pgfsetstrokecolor{currentstroke}%
\pgfsetdash{}{0pt}%
\pgfsys@defobject{currentmarker}{\pgfqpoint{-0.048611in}{0.000000in}}{\pgfqpoint{-0.000000in}{0.000000in}}{%
\pgfpathmoveto{\pgfqpoint{-0.000000in}{0.000000in}}%
\pgfpathlineto{\pgfqpoint{-0.048611in}{0.000000in}}%
\pgfusepath{stroke,fill}%
}%
\begin{pgfscope}%
\pgfsys@transformshift{0.934256in}{3.709690in}%
\pgfsys@useobject{currentmarker}{}%
\end{pgfscope}%
\end{pgfscope}%
\begin{pgfscope}%
\definecolor{textcolor}{rgb}{0.000000,0.000000,0.000000}%
\pgfsetstrokecolor{textcolor}%
\pgfsetfillcolor{textcolor}%
\pgftext[x=0.488890in, y=3.640263in, left, base]{\color{textcolor}\rmfamily\fontsize{14.000000}{16.800000}\selectfont \(\displaystyle {0.05}\)}%
\end{pgfscope}%
\begin{pgfscope}%
\pgfpathrectangle{\pgfqpoint{0.934256in}{0.608267in}}{\pgfqpoint{4.960000in}{3.696000in}}%
\pgfusepath{clip}%
\pgfsetrectcap%
\pgfsetroundjoin%
\pgfsetlinewidth{0.803000pt}%
\definecolor{currentstroke}{rgb}{0.690196,0.690196,0.690196}%
\pgfsetstrokecolor{currentstroke}%
\pgfsetdash{}{0pt}%
\pgfpathmoveto{\pgfqpoint{0.934256in}{4.208631in}}%
\pgfpathlineto{\pgfqpoint{5.894256in}{4.208631in}}%
\pgfusepath{stroke}%
\end{pgfscope}%
\begin{pgfscope}%
\pgfsetbuttcap%
\pgfsetroundjoin%
\definecolor{currentfill}{rgb}{0.000000,0.000000,0.000000}%
\pgfsetfillcolor{currentfill}%
\pgfsetlinewidth{0.803000pt}%
\definecolor{currentstroke}{rgb}{0.000000,0.000000,0.000000}%
\pgfsetstrokecolor{currentstroke}%
\pgfsetdash{}{0pt}%
\pgfsys@defobject{currentmarker}{\pgfqpoint{-0.048611in}{0.000000in}}{\pgfqpoint{-0.000000in}{0.000000in}}{%
\pgfpathmoveto{\pgfqpoint{-0.000000in}{0.000000in}}%
\pgfpathlineto{\pgfqpoint{-0.048611in}{0.000000in}}%
\pgfusepath{stroke,fill}%
}%
\begin{pgfscope}%
\pgfsys@transformshift{0.934256in}{4.208631in}%
\pgfsys@useobject{currentmarker}{}%
\end{pgfscope}%
\end{pgfscope}%
\begin{pgfscope}%
\definecolor{textcolor}{rgb}{0.000000,0.000000,0.000000}%
\pgfsetstrokecolor{textcolor}%
\pgfsetfillcolor{textcolor}%
\pgftext[x=0.488890in, y=4.139204in, left, base]{\color{textcolor}\rmfamily\fontsize{14.000000}{16.800000}\selectfont \(\displaystyle {0.10}\)}%
\end{pgfscope}%
\begin{pgfscope}%
\definecolor{textcolor}{rgb}{0.000000,0.000000,0.000000}%
\pgfsetstrokecolor{textcolor}%
\pgfsetfillcolor{textcolor}%
\pgftext[x=0.277779in,y=2.456267in,,bottom,rotate=90.000000]{\color{textcolor}\rmfamily\fontsize{14.000000}{16.800000}\selectfont \(\displaystyle \delta_{в},\ рад\)}%
\end{pgfscope}%
\begin{pgfscope}%
\pgfpathrectangle{\pgfqpoint{0.934256in}{0.608267in}}{\pgfqpoint{4.960000in}{3.696000in}}%
\pgfusepath{clip}%
\pgfsetrectcap%
\pgfsetroundjoin%
\pgfsetlinewidth{2.007500pt}%
\definecolor{currentstroke}{rgb}{0.121569,0.466667,0.705882}%
\pgfsetstrokecolor{currentstroke}%
\pgfsetdash{}{0pt}%
\pgfpathmoveto{\pgfqpoint{0.934256in}{3.210749in}}%
\pgfpathlineto{\pgfqpoint{1.099589in}{3.210749in}}%
\pgfpathlineto{\pgfqpoint{1.101408in}{3.183038in}}%
\pgfpathlineto{\pgfqpoint{1.188373in}{1.808890in}}%
\pgfpathlineto{\pgfqpoint{1.188869in}{1.816413in}}%
\pgfpathlineto{\pgfqpoint{1.335685in}{4.136267in}}%
\pgfpathlineto{\pgfqpoint{1.336181in}{4.128597in}}%
\pgfpathlineto{\pgfqpoint{1.452907in}{2.286437in}}%
\pgfpathlineto{\pgfqpoint{1.453237in}{2.291662in}}%
\pgfpathlineto{\pgfqpoint{1.507136in}{3.127222in}}%
\pgfpathlineto{\pgfqpoint{1.510443in}{3.132273in}}%
\pgfpathlineto{\pgfqpoint{1.512757in}{3.132533in}}%
\pgfpathlineto{\pgfqpoint{1.515237in}{3.130261in}}%
\pgfpathlineto{\pgfqpoint{1.518709in}{3.123589in}}%
\pgfpathlineto{\pgfqpoint{1.524661in}{3.106410in}}%
\pgfpathlineto{\pgfqpoint{1.539045in}{3.063907in}}%
\pgfpathlineto{\pgfqpoint{1.545824in}{3.051208in}}%
\pgfpathlineto{\pgfqpoint{1.551941in}{3.044214in}}%
\pgfpathlineto{\pgfqpoint{1.558059in}{3.040484in}}%
\pgfpathlineto{\pgfqpoint{1.565168in}{3.038820in}}%
\pgfpathlineto{\pgfqpoint{1.575749in}{3.038928in}}%
\pgfpathlineto{\pgfqpoint{1.610635in}{3.042778in}}%
\pgfpathlineto{\pgfqpoint{1.644032in}{3.048326in}}%
\pgfpathlineto{\pgfqpoint{1.700741in}{3.060396in}}%
\pgfpathlineto{\pgfqpoint{1.934853in}{3.112188in}}%
\pgfpathlineto{\pgfqpoint{2.020661in}{3.129983in}}%
\pgfpathlineto{\pgfqpoint{2.023968in}{3.135735in}}%
\pgfpathlineto{\pgfqpoint{2.028267in}{3.148228in}}%
\pgfpathlineto{\pgfqpoint{2.034880in}{3.175515in}}%
\pgfpathlineto{\pgfqpoint{2.050752in}{3.242802in}}%
\pgfpathlineto{\pgfqpoint{2.057531in}{3.261367in}}%
\pgfpathlineto{\pgfqpoint{2.063483in}{3.271791in}}%
\pgfpathlineto{\pgfqpoint{2.069269in}{3.277686in}}%
\pgfpathlineto{\pgfqpoint{2.075221in}{3.280669in}}%
\pgfpathlineto{\pgfqpoint{2.082661in}{3.281795in}}%
\pgfpathlineto{\pgfqpoint{2.097376in}{3.281148in}}%
\pgfpathlineto{\pgfqpoint{2.118704in}{3.280984in}}%
\pgfpathlineto{\pgfqpoint{2.143339in}{3.283221in}}%
\pgfpathlineto{\pgfqpoint{2.169627in}{3.287809in}}%
\pgfpathlineto{\pgfqpoint{2.198395in}{3.295176in}}%
\pgfpathlineto{\pgfqpoint{2.234107in}{3.306747in}}%
\pgfpathlineto{\pgfqpoint{2.285856in}{3.326090in}}%
\pgfpathlineto{\pgfqpoint{2.412005in}{3.373930in}}%
\pgfpathlineto{\pgfqpoint{2.466400in}{3.391718in}}%
\pgfpathlineto{\pgfqpoint{2.516331in}{3.405730in}}%
\pgfpathlineto{\pgfqpoint{2.564443in}{3.416932in}}%
\pgfpathlineto{\pgfqpoint{2.611728in}{3.425663in}}%
\pgfpathlineto{\pgfqpoint{2.658848in}{3.432107in}}%
\pgfpathlineto{\pgfqpoint{2.706464in}{3.436370in}}%
\pgfpathlineto{\pgfqpoint{2.754907in}{3.438464in}}%
\pgfpathlineto{\pgfqpoint{2.804507in}{3.438374in}}%
\pgfpathlineto{\pgfqpoint{2.855925in}{3.436039in}}%
\pgfpathlineto{\pgfqpoint{2.909493in}{3.431361in}}%
\pgfpathlineto{\pgfqpoint{2.966037in}{3.424164in}}%
\pgfpathlineto{\pgfqpoint{3.026384in}{3.414210in}}%
\pgfpathlineto{\pgfqpoint{3.092021in}{3.401089in}}%
\pgfpathlineto{\pgfqpoint{3.165264in}{3.384128in}}%
\pgfpathlineto{\pgfqpoint{3.251237in}{3.361854in}}%
\pgfpathlineto{\pgfqpoint{3.365483in}{3.329782in}}%
\pgfpathlineto{\pgfqpoint{3.644400in}{3.250629in}}%
\pgfpathlineto{\pgfqpoint{3.743765in}{3.225315in}}%
\pgfpathlineto{\pgfqpoint{3.832219in}{3.205048in}}%
\pgfpathlineto{\pgfqpoint{3.915051in}{3.188329in}}%
\pgfpathlineto{\pgfqpoint{3.994907in}{3.174469in}}%
\pgfpathlineto{\pgfqpoint{4.073109in}{3.163145in}}%
\pgfpathlineto{\pgfqpoint{4.150981in}{3.154112in}}%
\pgfpathlineto{\pgfqpoint{4.229349in}{3.147260in}}%
\pgfpathlineto{\pgfqpoint{4.309205in}{3.142514in}}%
\pgfpathlineto{\pgfqpoint{4.391707in}{3.139852in}}%
\pgfpathlineto{\pgfqpoint{4.478011in}{3.139312in}}%
\pgfpathlineto{\pgfqpoint{4.569771in}{3.140988in}}%
\pgfpathlineto{\pgfqpoint{4.669797in}{3.145074in}}%
\pgfpathlineto{\pgfqpoint{4.783051in}{3.151984in}}%
\pgfpathlineto{\pgfqpoint{4.921435in}{3.162761in}}%
\pgfpathlineto{\pgfqpoint{5.150091in}{3.183194in}}%
\pgfpathlineto{\pgfqpoint{5.386683in}{3.203560in}}%
\pgfpathlineto{\pgfqpoint{5.544245in}{3.214865in}}%
\pgfpathlineto{\pgfqpoint{5.687093in}{3.222898in}}%
\pgfpathlineto{\pgfqpoint{5.825643in}{3.228464in}}%
\pgfpathlineto{\pgfqpoint{5.894256in}{3.230401in}}%
\pgfpathlineto{\pgfqpoint{5.894256in}{3.230401in}}%
\pgfusepath{stroke}%
\end{pgfscope}%
\begin{pgfscope}%
\pgfpathrectangle{\pgfqpoint{0.934256in}{0.608267in}}{\pgfqpoint{4.960000in}{3.696000in}}%
\pgfusepath{clip}%
\pgfsetbuttcap%
\pgfsetroundjoin%
\pgfsetlinewidth{2.007500pt}%
\definecolor{currentstroke}{rgb}{1.000000,0.498039,0.054902}%
\pgfsetstrokecolor{currentstroke}%
\pgfsetdash{{12.800000pt}{3.200000pt}{2.000000pt}{3.200000pt}}{0.000000pt}%
\pgfpathmoveto{\pgfqpoint{0.934256in}{3.210749in}}%
\pgfpathlineto{\pgfqpoint{1.099589in}{3.210749in}}%
\pgfpathlineto{\pgfqpoint{1.100251in}{3.195441in}}%
\pgfpathlineto{\pgfqpoint{1.103227in}{3.012523in}}%
\pgfpathlineto{\pgfqpoint{1.138608in}{0.776267in}}%
\pgfpathlineto{\pgfqpoint{1.139104in}{0.802590in}}%
\pgfpathlineto{\pgfqpoint{1.184075in}{3.582506in}}%
\pgfpathlineto{\pgfqpoint{1.187712in}{3.619265in}}%
\pgfpathlineto{\pgfqpoint{1.190357in}{3.628317in}}%
\pgfpathlineto{\pgfqpoint{1.191680in}{3.627977in}}%
\pgfpathlineto{\pgfqpoint{1.193333in}{3.623572in}}%
\pgfpathlineto{\pgfqpoint{1.195979in}{3.608734in}}%
\pgfpathlineto{\pgfqpoint{1.200277in}{3.569708in}}%
\pgfpathlineto{\pgfqpoint{1.210197in}{3.452116in}}%
\pgfpathlineto{\pgfqpoint{1.219787in}{3.351463in}}%
\pgfpathlineto{\pgfqpoint{1.227227in}{3.295485in}}%
\pgfpathlineto{\pgfqpoint{1.234336in}{3.257943in}}%
\pgfpathlineto{\pgfqpoint{1.241941in}{3.229417in}}%
\pgfpathlineto{\pgfqpoint{1.251365in}{3.202923in}}%
\pgfpathlineto{\pgfqpoint{1.266576in}{3.167583in}}%
\pgfpathlineto{\pgfqpoint{1.284928in}{3.130110in}}%
\pgfpathlineto{\pgfqpoint{1.299973in}{3.104300in}}%
\pgfpathlineto{\pgfqpoint{1.314523in}{3.083694in}}%
\pgfpathlineto{\pgfqpoint{1.329568in}{3.066198in}}%
\pgfpathlineto{\pgfqpoint{1.345109in}{3.051472in}}%
\pgfpathlineto{\pgfqpoint{1.361147in}{3.039290in}}%
\pgfpathlineto{\pgfqpoint{1.377680in}{3.029488in}}%
\pgfpathlineto{\pgfqpoint{1.395040in}{3.021771in}}%
\pgfpathlineto{\pgfqpoint{1.413557in}{3.015997in}}%
\pgfpathlineto{\pgfqpoint{1.433397in}{3.012153in}}%
\pgfpathlineto{\pgfqpoint{1.455221in}{3.010214in}}%
\pgfpathlineto{\pgfqpoint{1.479691in}{3.010316in}}%
\pgfpathlineto{\pgfqpoint{1.507797in}{3.012714in}}%
\pgfpathlineto{\pgfqpoint{1.541360in}{3.017893in}}%
\pgfpathlineto{\pgfqpoint{1.583685in}{3.026791in}}%
\pgfpathlineto{\pgfqpoint{1.645520in}{3.042289in}}%
\pgfpathlineto{\pgfqpoint{1.833669in}{3.090619in}}%
\pgfpathlineto{\pgfqpoint{1.914683in}{3.108454in}}%
\pgfpathlineto{\pgfqpoint{1.995200in}{3.123895in}}%
\pgfpathlineto{\pgfqpoint{2.013387in}{3.128165in}}%
\pgfpathlineto{\pgfqpoint{2.016528in}{3.132314in}}%
\pgfpathlineto{\pgfqpoint{2.020331in}{3.141456in}}%
\pgfpathlineto{\pgfqpoint{2.025621in}{3.160704in}}%
\pgfpathlineto{\pgfqpoint{2.049429in}{3.255930in}}%
\pgfpathlineto{\pgfqpoint{2.055712in}{3.268732in}}%
\pgfpathlineto{\pgfqpoint{2.061499in}{3.275789in}}%
\pgfpathlineto{\pgfqpoint{2.067285in}{3.279513in}}%
\pgfpathlineto{\pgfqpoint{2.074064in}{3.281187in}}%
\pgfpathlineto{\pgfqpoint{2.084645in}{3.281171in}}%
\pgfpathlineto{\pgfqpoint{2.112421in}{3.280615in}}%
\pgfpathlineto{\pgfqpoint{2.137056in}{3.282866in}}%
\pgfpathlineto{\pgfqpoint{2.163344in}{3.287464in}}%
\pgfpathlineto{\pgfqpoint{2.192112in}{3.294839in}}%
\pgfpathlineto{\pgfqpoint{2.227824in}{3.306421in}}%
\pgfpathlineto{\pgfqpoint{2.279573in}{3.325782in}}%
\pgfpathlineto{\pgfqpoint{2.405888in}{3.373734in}}%
\pgfpathlineto{\pgfqpoint{2.460283in}{3.391541in}}%
\pgfpathlineto{\pgfqpoint{2.510213in}{3.405570in}}%
\pgfpathlineto{\pgfqpoint{2.558325in}{3.416788in}}%
\pgfpathlineto{\pgfqpoint{2.605611in}{3.425535in}}%
\pgfpathlineto{\pgfqpoint{2.652731in}{3.431994in}}%
\pgfpathlineto{\pgfqpoint{2.700347in}{3.436273in}}%
\pgfpathlineto{\pgfqpoint{2.748789in}{3.438383in}}%
\pgfpathlineto{\pgfqpoint{2.798389in}{3.438309in}}%
\pgfpathlineto{\pgfqpoint{2.849643in}{3.436001in}}%
\pgfpathlineto{\pgfqpoint{2.903211in}{3.431345in}}%
\pgfpathlineto{\pgfqpoint{2.959589in}{3.424195in}}%
\pgfpathlineto{\pgfqpoint{3.019936in}{3.414268in}}%
\pgfpathlineto{\pgfqpoint{3.085408in}{3.401210in}}%
\pgfpathlineto{\pgfqpoint{3.158651in}{3.384279in}}%
\pgfpathlineto{\pgfqpoint{3.244459in}{3.362078in}}%
\pgfpathlineto{\pgfqpoint{3.358208in}{3.330177in}}%
\pgfpathlineto{\pgfqpoint{3.640101in}{3.250225in}}%
\pgfpathlineto{\pgfqpoint{3.739301in}{3.224994in}}%
\pgfpathlineto{\pgfqpoint{3.827589in}{3.204804in}}%
\pgfpathlineto{\pgfqpoint{3.910421in}{3.188122in}}%
\pgfpathlineto{\pgfqpoint{3.990277in}{3.174300in}}%
\pgfpathlineto{\pgfqpoint{4.068480in}{3.163011in}}%
\pgfpathlineto{\pgfqpoint{4.146352in}{3.154014in}}%
\pgfpathlineto{\pgfqpoint{4.224885in}{3.147185in}}%
\pgfpathlineto{\pgfqpoint{4.304907in}{3.142469in}}%
\pgfpathlineto{\pgfqpoint{4.387408in}{3.139847in}}%
\pgfpathlineto{\pgfqpoint{4.473712in}{3.139345in}}%
\pgfpathlineto{\pgfqpoint{4.565637in}{3.141059in}}%
\pgfpathlineto{\pgfqpoint{4.665995in}{3.145197in}}%
\pgfpathlineto{\pgfqpoint{4.779579in}{3.152164in}}%
\pgfpathlineto{\pgfqpoint{4.918624in}{3.163027in}}%
\pgfpathlineto{\pgfqpoint{5.152736in}{3.183966in}}%
\pgfpathlineto{\pgfqpoint{5.383707in}{3.203782in}}%
\pgfpathlineto{\pgfqpoint{5.540939in}{3.215021in}}%
\pgfpathlineto{\pgfqpoint{5.683621in}{3.223004in}}%
\pgfpathlineto{\pgfqpoint{5.822171in}{3.228531in}}%
\pgfpathlineto{\pgfqpoint{5.894256in}{3.230533in}}%
\pgfpathlineto{\pgfqpoint{5.894256in}{3.230533in}}%
\pgfusepath{stroke}%
\end{pgfscope}%
\begin{pgfscope}%
\pgfsetrectcap%
\pgfsetmiterjoin%
\pgfsetlinewidth{0.803000pt}%
\definecolor{currentstroke}{rgb}{0.000000,0.000000,0.000000}%
\pgfsetstrokecolor{currentstroke}%
\pgfsetdash{}{0pt}%
\pgfpathmoveto{\pgfqpoint{0.934256in}{0.608267in}}%
\pgfpathlineto{\pgfqpoint{0.934256in}{4.304267in}}%
\pgfusepath{stroke}%
\end{pgfscope}%
\begin{pgfscope}%
\pgfsetrectcap%
\pgfsetmiterjoin%
\pgfsetlinewidth{0.803000pt}%
\definecolor{currentstroke}{rgb}{0.000000,0.000000,0.000000}%
\pgfsetstrokecolor{currentstroke}%
\pgfsetdash{}{0pt}%
\pgfpathmoveto{\pgfqpoint{5.894256in}{0.608267in}}%
\pgfpathlineto{\pgfqpoint{5.894256in}{4.304267in}}%
\pgfusepath{stroke}%
\end{pgfscope}%
\begin{pgfscope}%
\pgfsetrectcap%
\pgfsetmiterjoin%
\pgfsetlinewidth{0.803000pt}%
\definecolor{currentstroke}{rgb}{0.000000,0.000000,0.000000}%
\pgfsetstrokecolor{currentstroke}%
\pgfsetdash{}{0pt}%
\pgfpathmoveto{\pgfqpoint{0.934256in}{0.608267in}}%
\pgfpathlineto{\pgfqpoint{5.894256in}{0.608267in}}%
\pgfusepath{stroke}%
\end{pgfscope}%
\begin{pgfscope}%
\pgfsetrectcap%
\pgfsetmiterjoin%
\pgfsetlinewidth{0.803000pt}%
\definecolor{currentstroke}{rgb}{0.000000,0.000000,0.000000}%
\pgfsetstrokecolor{currentstroke}%
\pgfsetdash{}{0pt}%
\pgfpathmoveto{\pgfqpoint{0.934256in}{4.304267in}}%
\pgfpathlineto{\pgfqpoint{5.894256in}{4.304267in}}%
\pgfusepath{stroke}%
\end{pgfscope}%
\begin{pgfscope}%
\pgfsetbuttcap%
\pgfsetmiterjoin%
\definecolor{currentfill}{rgb}{1.000000,1.000000,1.000000}%
\pgfsetfillcolor{currentfill}%
\pgfsetfillopacity{0.800000}%
\pgfsetlinewidth{1.003750pt}%
\definecolor{currentstroke}{rgb}{0.800000,0.800000,0.800000}%
\pgfsetstrokecolor{currentstroke}%
\pgfsetstrokeopacity{0.800000}%
\pgfsetdash{}{0pt}%
\pgfpathmoveto{\pgfqpoint{3.805765in}{3.237126in}}%
\pgfpathlineto{\pgfqpoint{5.758145in}{3.237126in}}%
\pgfpathquadraticcurveto{\pgfqpoint{5.797034in}{3.237126in}}{\pgfqpoint{5.797034in}{3.276015in}}%
\pgfpathlineto{\pgfqpoint{5.797034in}{4.168156in}}%
\pgfpathquadraticcurveto{\pgfqpoint{5.797034in}{4.207045in}}{\pgfqpoint{5.758145in}{4.207045in}}%
\pgfpathlineto{\pgfqpoint{3.805765in}{4.207045in}}%
\pgfpathquadraticcurveto{\pgfqpoint{3.766876in}{4.207045in}}{\pgfqpoint{3.766876in}{4.168156in}}%
\pgfpathlineto{\pgfqpoint{3.766876in}{3.276015in}}%
\pgfpathquadraticcurveto{\pgfqpoint{3.766876in}{3.237126in}}{\pgfqpoint{3.805765in}{3.237126in}}%
\pgfpathclose%
\pgfusepath{stroke,fill}%
\end{pgfscope}%
\begin{pgfscope}%
\pgfsetrectcap%
\pgfsetroundjoin%
\pgfsetlinewidth{2.007500pt}%
\definecolor{currentstroke}{rgb}{0.121569,0.466667,0.705882}%
\pgfsetstrokecolor{currentstroke}%
\pgfsetdash{}{0pt}%
\pgfpathmoveto{\pgfqpoint{3.844654in}{3.975942in}}%
\pgfpathlineto{\pgfqpoint{4.233543in}{3.975942in}}%
\pgfusepath{stroke}%
\end{pgfscope}%
\begin{pgfscope}%
\definecolor{textcolor}{rgb}{0.000000,0.000000,0.000000}%
\pgfsetstrokecolor{textcolor}%
\pgfsetfillcolor{textcolor}%
\pgftext[x=4.389099in,y=3.907887in,left,base]{\color{textcolor}\rmfamily\fontsize{14.000000}{16.800000}\selectfont \(\displaystyle \dot{\delta}_{{в}_{max}}=15 \frac{град.}{сек.}\)}%
\end{pgfscope}%
\begin{pgfscope}%
\pgfsetbuttcap%
\pgfsetroundjoin%
\pgfsetlinewidth{2.007500pt}%
\definecolor{currentstroke}{rgb}{1.000000,0.498039,0.054902}%
\pgfsetstrokecolor{currentstroke}%
\pgfsetdash{{12.800000pt}{3.200000pt}{2.000000pt}{3.200000pt}}{0.000000pt}%
\pgfpathmoveto{\pgfqpoint{3.844654in}{3.520149in}}%
\pgfpathlineto{\pgfqpoint{4.233543in}{3.520149in}}%
\pgfusepath{stroke}%
\end{pgfscope}%
\begin{pgfscope}%
\definecolor{textcolor}{rgb}{0.000000,0.000000,0.000000}%
\pgfsetstrokecolor{textcolor}%
\pgfsetfillcolor{textcolor}%
\pgftext[x=4.389099in,y=3.452094in,left,base]{\color{textcolor}\rmfamily\fontsize{14.000000}{16.800000}\selectfont \(\displaystyle \dot{\delta}_{{в}_{max}}=60 \frac{град.}{сек.}\)}%
\end{pgfscope}%
\begin{pgfscope}%
\pgfsetbuttcap%
\pgfsetmiterjoin%
\definecolor{currentfill}{rgb}{1.000000,1.000000,1.000000}%
\pgfsetfillcolor{currentfill}%
\pgfsetlinewidth{0.000000pt}%
\definecolor{currentstroke}{rgb}{0.000000,0.000000,0.000000}%
\pgfsetstrokecolor{currentstroke}%
\pgfsetstrokeopacity{0.000000}%
\pgfsetdash{}{0pt}%
\pgfpathmoveto{\pgfqpoint{3.110256in}{1.040267in}}%
\pgfpathlineto{\pgfqpoint{5.670256in}{1.040267in}}%
\pgfpathlineto{\pgfqpoint{5.670256in}{2.480267in}}%
\pgfpathlineto{\pgfqpoint{3.110256in}{2.480267in}}%
\pgfpathclose%
\pgfusepath{fill}%
\end{pgfscope}%
\begin{pgfscope}%
\pgfpathrectangle{\pgfqpoint{3.110256in}{1.040267in}}{\pgfqpoint{2.560000in}{1.440000in}}%
\pgfusepath{clip}%
\pgfsetrectcap%
\pgfsetroundjoin%
\pgfsetlinewidth{0.803000pt}%
\definecolor{currentstroke}{rgb}{0.690196,0.690196,0.690196}%
\pgfsetstrokecolor{currentstroke}%
\pgfsetdash{}{0pt}%
\pgfpathmoveto{\pgfqpoint{3.110256in}{1.040267in}}%
\pgfpathlineto{\pgfqpoint{3.110256in}{2.480267in}}%
\pgfusepath{stroke}%
\end{pgfscope}%
\begin{pgfscope}%
\pgfsetbuttcap%
\pgfsetroundjoin%
\definecolor{currentfill}{rgb}{0.000000,0.000000,0.000000}%
\pgfsetfillcolor{currentfill}%
\pgfsetlinewidth{0.803000pt}%
\definecolor{currentstroke}{rgb}{0.000000,0.000000,0.000000}%
\pgfsetstrokecolor{currentstroke}%
\pgfsetdash{}{0pt}%
\pgfsys@defobject{currentmarker}{\pgfqpoint{0.000000in}{-0.048611in}}{\pgfqpoint{0.000000in}{0.000000in}}{%
\pgfpathmoveto{\pgfqpoint{0.000000in}{0.000000in}}%
\pgfpathlineto{\pgfqpoint{0.000000in}{-0.048611in}}%
\pgfusepath{stroke,fill}%
}%
\begin{pgfscope}%
\pgfsys@transformshift{3.110256in}{1.040267in}%
\pgfsys@useobject{currentmarker}{}%
\end{pgfscope}%
\end{pgfscope}%
\begin{pgfscope}%
\definecolor{textcolor}{rgb}{0.000000,0.000000,0.000000}%
\pgfsetstrokecolor{textcolor}%
\pgfsetfillcolor{textcolor}%
\pgftext[x=3.110256in,y=0.943045in,,top]{\color{textcolor}\rmfamily\fontsize{14.000000}{16.800000}\selectfont \(\displaystyle {0}\)}%
\end{pgfscope}%
\begin{pgfscope}%
\pgfpathrectangle{\pgfqpoint{3.110256in}{1.040267in}}{\pgfqpoint{2.560000in}{1.440000in}}%
\pgfusepath{clip}%
\pgfsetrectcap%
\pgfsetroundjoin%
\pgfsetlinewidth{0.803000pt}%
\definecolor{currentstroke}{rgb}{0.690196,0.690196,0.690196}%
\pgfsetstrokecolor{currentstroke}%
\pgfsetdash{}{0pt}%
\pgfpathmoveto{\pgfqpoint{3.963589in}{1.040267in}}%
\pgfpathlineto{\pgfqpoint{3.963589in}{2.480267in}}%
\pgfusepath{stroke}%
\end{pgfscope}%
\begin{pgfscope}%
\pgfsetbuttcap%
\pgfsetroundjoin%
\definecolor{currentfill}{rgb}{0.000000,0.000000,0.000000}%
\pgfsetfillcolor{currentfill}%
\pgfsetlinewidth{0.803000pt}%
\definecolor{currentstroke}{rgb}{0.000000,0.000000,0.000000}%
\pgfsetstrokecolor{currentstroke}%
\pgfsetdash{}{0pt}%
\pgfsys@defobject{currentmarker}{\pgfqpoint{0.000000in}{-0.048611in}}{\pgfqpoint{0.000000in}{0.000000in}}{%
\pgfpathmoveto{\pgfqpoint{0.000000in}{0.000000in}}%
\pgfpathlineto{\pgfqpoint{0.000000in}{-0.048611in}}%
\pgfusepath{stroke,fill}%
}%
\begin{pgfscope}%
\pgfsys@transformshift{3.963589in}{1.040267in}%
\pgfsys@useobject{currentmarker}{}%
\end{pgfscope}%
\end{pgfscope}%
\begin{pgfscope}%
\definecolor{textcolor}{rgb}{0.000000,0.000000,0.000000}%
\pgfsetstrokecolor{textcolor}%
\pgfsetfillcolor{textcolor}%
\pgftext[x=3.963589in,y=0.943045in,,top]{\color{textcolor}\rmfamily\fontsize{14.000000}{16.800000}\selectfont \(\displaystyle {1}\)}%
\end{pgfscope}%
\begin{pgfscope}%
\pgfpathrectangle{\pgfqpoint{3.110256in}{1.040267in}}{\pgfqpoint{2.560000in}{1.440000in}}%
\pgfusepath{clip}%
\pgfsetrectcap%
\pgfsetroundjoin%
\pgfsetlinewidth{0.803000pt}%
\definecolor{currentstroke}{rgb}{0.690196,0.690196,0.690196}%
\pgfsetstrokecolor{currentstroke}%
\pgfsetdash{}{0pt}%
\pgfpathmoveto{\pgfqpoint{4.816923in}{1.040267in}}%
\pgfpathlineto{\pgfqpoint{4.816923in}{2.480267in}}%
\pgfusepath{stroke}%
\end{pgfscope}%
\begin{pgfscope}%
\pgfsetbuttcap%
\pgfsetroundjoin%
\definecolor{currentfill}{rgb}{0.000000,0.000000,0.000000}%
\pgfsetfillcolor{currentfill}%
\pgfsetlinewidth{0.803000pt}%
\definecolor{currentstroke}{rgb}{0.000000,0.000000,0.000000}%
\pgfsetstrokecolor{currentstroke}%
\pgfsetdash{}{0pt}%
\pgfsys@defobject{currentmarker}{\pgfqpoint{0.000000in}{-0.048611in}}{\pgfqpoint{0.000000in}{0.000000in}}{%
\pgfpathmoveto{\pgfqpoint{0.000000in}{0.000000in}}%
\pgfpathlineto{\pgfqpoint{0.000000in}{-0.048611in}}%
\pgfusepath{stroke,fill}%
}%
\begin{pgfscope}%
\pgfsys@transformshift{4.816923in}{1.040267in}%
\pgfsys@useobject{currentmarker}{}%
\end{pgfscope}%
\end{pgfscope}%
\begin{pgfscope}%
\definecolor{textcolor}{rgb}{0.000000,0.000000,0.000000}%
\pgfsetstrokecolor{textcolor}%
\pgfsetfillcolor{textcolor}%
\pgftext[x=4.816923in,y=0.943045in,,top]{\color{textcolor}\rmfamily\fontsize{14.000000}{16.800000}\selectfont \(\displaystyle {2}\)}%
\end{pgfscope}%
\begin{pgfscope}%
\pgfpathrectangle{\pgfqpoint{3.110256in}{1.040267in}}{\pgfqpoint{2.560000in}{1.440000in}}%
\pgfusepath{clip}%
\pgfsetrectcap%
\pgfsetroundjoin%
\pgfsetlinewidth{0.803000pt}%
\definecolor{currentstroke}{rgb}{0.690196,0.690196,0.690196}%
\pgfsetstrokecolor{currentstroke}%
\pgfsetdash{}{0pt}%
\pgfpathmoveto{\pgfqpoint{5.670256in}{1.040267in}}%
\pgfpathlineto{\pgfqpoint{5.670256in}{2.480267in}}%
\pgfusepath{stroke}%
\end{pgfscope}%
\begin{pgfscope}%
\pgfsetbuttcap%
\pgfsetroundjoin%
\definecolor{currentfill}{rgb}{0.000000,0.000000,0.000000}%
\pgfsetfillcolor{currentfill}%
\pgfsetlinewidth{0.803000pt}%
\definecolor{currentstroke}{rgb}{0.000000,0.000000,0.000000}%
\pgfsetstrokecolor{currentstroke}%
\pgfsetdash{}{0pt}%
\pgfsys@defobject{currentmarker}{\pgfqpoint{0.000000in}{-0.048611in}}{\pgfqpoint{0.000000in}{0.000000in}}{%
\pgfpathmoveto{\pgfqpoint{0.000000in}{0.000000in}}%
\pgfpathlineto{\pgfqpoint{0.000000in}{-0.048611in}}%
\pgfusepath{stroke,fill}%
}%
\begin{pgfscope}%
\pgfsys@transformshift{5.670256in}{1.040267in}%
\pgfsys@useobject{currentmarker}{}%
\end{pgfscope}%
\end{pgfscope}%
\begin{pgfscope}%
\definecolor{textcolor}{rgb}{0.000000,0.000000,0.000000}%
\pgfsetstrokecolor{textcolor}%
\pgfsetfillcolor{textcolor}%
\pgftext[x=5.670256in,y=0.943045in,,top]{\color{textcolor}\rmfamily\fontsize{14.000000}{16.800000}\selectfont \(\displaystyle {3}\)}%
\end{pgfscope}%
\begin{pgfscope}%
\pgfpathrectangle{\pgfqpoint{3.110256in}{1.040267in}}{\pgfqpoint{2.560000in}{1.440000in}}%
\pgfusepath{clip}%
\pgfsetrectcap%
\pgfsetroundjoin%
\pgfsetlinewidth{0.803000pt}%
\definecolor{currentstroke}{rgb}{0.690196,0.690196,0.690196}%
\pgfsetstrokecolor{currentstroke}%
\pgfsetdash{}{0pt}%
\pgfpathmoveto{\pgfqpoint{3.110256in}{1.276651in}}%
\pgfpathlineto{\pgfqpoint{5.670256in}{1.276651in}}%
\pgfusepath{stroke}%
\end{pgfscope}%
\begin{pgfscope}%
\pgfsetbuttcap%
\pgfsetroundjoin%
\definecolor{currentfill}{rgb}{0.000000,0.000000,0.000000}%
\pgfsetfillcolor{currentfill}%
\pgfsetlinewidth{0.803000pt}%
\definecolor{currentstroke}{rgb}{0.000000,0.000000,0.000000}%
\pgfsetstrokecolor{currentstroke}%
\pgfsetdash{}{0pt}%
\pgfsys@defobject{currentmarker}{\pgfqpoint{-0.048611in}{0.000000in}}{\pgfqpoint{-0.000000in}{0.000000in}}{%
\pgfpathmoveto{\pgfqpoint{-0.000000in}{0.000000in}}%
\pgfpathlineto{\pgfqpoint{-0.048611in}{0.000000in}}%
\pgfusepath{stroke,fill}%
}%
\begin{pgfscope}%
\pgfsys@transformshift{3.110256in}{1.276651in}%
\pgfsys@useobject{currentmarker}{}%
\end{pgfscope}%
\end{pgfscope}%
\begin{pgfscope}%
\definecolor{textcolor}{rgb}{0.000000,0.000000,0.000000}%
\pgfsetstrokecolor{textcolor}%
\pgfsetfillcolor{textcolor}%
\pgftext[x=2.607250in, y=1.207224in, left, base]{\color{textcolor}\rmfamily\fontsize{14.000000}{16.800000}\selectfont \(\displaystyle {\ensuremath{-}0.2}\)}%
\end{pgfscope}%
\begin{pgfscope}%
\pgfpathrectangle{\pgfqpoint{3.110256in}{1.040267in}}{\pgfqpoint{2.560000in}{1.440000in}}%
\pgfusepath{clip}%
\pgfsetrectcap%
\pgfsetroundjoin%
\pgfsetlinewidth{0.803000pt}%
\definecolor{currentstroke}{rgb}{0.690196,0.690196,0.690196}%
\pgfsetstrokecolor{currentstroke}%
\pgfsetdash{}{0pt}%
\pgfpathmoveto{\pgfqpoint{3.110256in}{2.054221in}}%
\pgfpathlineto{\pgfqpoint{5.670256in}{2.054221in}}%
\pgfusepath{stroke}%
\end{pgfscope}%
\begin{pgfscope}%
\pgfsetbuttcap%
\pgfsetroundjoin%
\definecolor{currentfill}{rgb}{0.000000,0.000000,0.000000}%
\pgfsetfillcolor{currentfill}%
\pgfsetlinewidth{0.803000pt}%
\definecolor{currentstroke}{rgb}{0.000000,0.000000,0.000000}%
\pgfsetstrokecolor{currentstroke}%
\pgfsetdash{}{0pt}%
\pgfsys@defobject{currentmarker}{\pgfqpoint{-0.048611in}{0.000000in}}{\pgfqpoint{-0.000000in}{0.000000in}}{%
\pgfpathmoveto{\pgfqpoint{-0.000000in}{0.000000in}}%
\pgfpathlineto{\pgfqpoint{-0.048611in}{0.000000in}}%
\pgfusepath{stroke,fill}%
}%
\begin{pgfscope}%
\pgfsys@transformshift{3.110256in}{2.054221in}%
\pgfsys@useobject{currentmarker}{}%
\end{pgfscope}%
\end{pgfscope}%
\begin{pgfscope}%
\definecolor{textcolor}{rgb}{0.000000,0.000000,0.000000}%
\pgfsetstrokecolor{textcolor}%
\pgfsetfillcolor{textcolor}%
\pgftext[x=2.762806in, y=1.984794in, left, base]{\color{textcolor}\rmfamily\fontsize{14.000000}{16.800000}\selectfont \(\displaystyle {0.0}\)}%
\end{pgfscope}%
\begin{pgfscope}%
\pgfpathrectangle{\pgfqpoint{3.110256in}{1.040267in}}{\pgfqpoint{2.560000in}{1.440000in}}%
\pgfusepath{clip}%
\pgfsetrectcap%
\pgfsetroundjoin%
\pgfsetlinewidth{2.007500pt}%
\definecolor{currentstroke}{rgb}{0.121569,0.466667,0.705882}%
\pgfsetstrokecolor{currentstroke}%
\pgfsetdash{}{0pt}%
\pgfpathmoveto{\pgfqpoint{3.110256in}{2.054221in}}%
\pgfpathlineto{\pgfqpoint{3.964443in}{2.053603in}}%
\pgfpathlineto{\pgfqpoint{4.421829in}{1.508042in}}%
\pgfpathlineto{\pgfqpoint{4.433776in}{1.522169in}}%
\pgfpathlineto{\pgfqpoint{5.182149in}{2.414813in}}%
\pgfpathlineto{\pgfqpoint{5.204336in}{2.388414in}}%
\pgfpathlineto{\pgfqpoint{5.671109in}{1.831657in}}%
\pgfpathlineto{\pgfqpoint{5.671109in}{1.831657in}}%
\pgfusepath{stroke}%
\end{pgfscope}%
\begin{pgfscope}%
\pgfpathrectangle{\pgfqpoint{3.110256in}{1.040267in}}{\pgfqpoint{2.560000in}{1.440000in}}%
\pgfusepath{clip}%
\pgfsetbuttcap%
\pgfsetroundjoin%
\pgfsetlinewidth{2.007500pt}%
\definecolor{currentstroke}{rgb}{1.000000,0.498039,0.054902}%
\pgfsetstrokecolor{currentstroke}%
\pgfsetdash{{7.400000pt}{3.200000pt}}{0.000000pt}%
\pgfpathmoveto{\pgfqpoint{3.110256in}{2.054221in}}%
\pgfpathlineto{\pgfqpoint{3.964443in}{2.053603in}}%
\pgfpathlineto{\pgfqpoint{3.967003in}{2.048257in}}%
\pgfpathlineto{\pgfqpoint{3.970416in}{2.033989in}}%
\pgfpathlineto{\pgfqpoint{4.164976in}{1.105722in}}%
\pgfpathlineto{\pgfqpoint{4.166683in}{1.111906in}}%
\pgfpathlineto{\pgfqpoint{4.393669in}{2.191993in}}%
\pgfpathlineto{\pgfqpoint{4.403909in}{2.203352in}}%
\pgfpathlineto{\pgfqpoint{4.413296in}{2.210624in}}%
\pgfpathlineto{\pgfqpoint{4.422683in}{2.215065in}}%
\pgfpathlineto{\pgfqpoint{4.432069in}{2.216910in}}%
\pgfpathlineto{\pgfqpoint{4.441456in}{2.216435in}}%
\pgfpathlineto{\pgfqpoint{4.451696in}{2.213624in}}%
\pgfpathlineto{\pgfqpoint{4.463643in}{2.207844in}}%
\pgfpathlineto{\pgfqpoint{4.478149in}{2.198066in}}%
\pgfpathlineto{\pgfqpoint{4.496923in}{2.182504in}}%
\pgfpathlineto{\pgfqpoint{4.572016in}{2.117413in}}%
\pgfpathlineto{\pgfqpoint{4.595056in}{2.101961in}}%
\pgfpathlineto{\pgfqpoint{4.617243in}{2.089718in}}%
\pgfpathlineto{\pgfqpoint{4.641136in}{2.079111in}}%
\pgfpathlineto{\pgfqpoint{4.667589in}{2.069872in}}%
\pgfpathlineto{\pgfqpoint{4.699163in}{2.061288in}}%
\pgfpathlineto{\pgfqpoint{4.741829in}{2.052156in}}%
\pgfpathlineto{\pgfqpoint{4.813509in}{2.039392in}}%
\pgfpathlineto{\pgfqpoint{4.913349in}{2.023776in}}%
\pgfpathlineto{\pgfqpoint{4.998683in}{2.012647in}}%
\pgfpathlineto{\pgfqpoint{5.088283in}{2.003261in}}%
\pgfpathlineto{\pgfqpoint{5.188976in}{1.995005in}}%
\pgfpathlineto{\pgfqpoint{5.301616in}{1.988037in}}%
\pgfpathlineto{\pgfqpoint{5.427056in}{1.982552in}}%
\pgfpathlineto{\pgfqpoint{5.568709in}{1.978646in}}%
\pgfpathlineto{\pgfqpoint{5.671109in}{1.977017in}}%
\pgfpathlineto{\pgfqpoint{5.671109in}{1.977017in}}%
\pgfusepath{stroke}%
\end{pgfscope}%
\begin{pgfscope}%
\pgfsetrectcap%
\pgfsetmiterjoin%
\pgfsetlinewidth{0.803000pt}%
\definecolor{currentstroke}{rgb}{0.000000,0.000000,0.000000}%
\pgfsetstrokecolor{currentstroke}%
\pgfsetdash{}{0pt}%
\pgfpathmoveto{\pgfqpoint{3.110256in}{1.040267in}}%
\pgfpathlineto{\pgfqpoint{3.110256in}{2.480267in}}%
\pgfusepath{stroke}%
\end{pgfscope}%
\begin{pgfscope}%
\pgfsetrectcap%
\pgfsetmiterjoin%
\pgfsetlinewidth{0.803000pt}%
\definecolor{currentstroke}{rgb}{0.000000,0.000000,0.000000}%
\pgfsetstrokecolor{currentstroke}%
\pgfsetdash{}{0pt}%
\pgfpathmoveto{\pgfqpoint{5.670256in}{1.040267in}}%
\pgfpathlineto{\pgfqpoint{5.670256in}{2.480267in}}%
\pgfusepath{stroke}%
\end{pgfscope}%
\begin{pgfscope}%
\pgfsetrectcap%
\pgfsetmiterjoin%
\pgfsetlinewidth{0.803000pt}%
\definecolor{currentstroke}{rgb}{0.000000,0.000000,0.000000}%
\pgfsetstrokecolor{currentstroke}%
\pgfsetdash{}{0pt}%
\pgfpathmoveto{\pgfqpoint{3.110256in}{1.040267in}}%
\pgfpathlineto{\pgfqpoint{5.670256in}{1.040267in}}%
\pgfusepath{stroke}%
\end{pgfscope}%
\begin{pgfscope}%
\pgfsetrectcap%
\pgfsetmiterjoin%
\pgfsetlinewidth{0.803000pt}%
\definecolor{currentstroke}{rgb}{0.000000,0.000000,0.000000}%
\pgfsetstrokecolor{currentstroke}%
\pgfsetdash{}{0pt}%
\pgfpathmoveto{\pgfqpoint{3.110256in}{2.480267in}}%
\pgfpathlineto{\pgfqpoint{5.670256in}{2.480267in}}%
\pgfusepath{stroke}%
\end{pgfscope}%
\end{pgfpicture}%
\makeatother%
\endgroup%
}
    \caption{Изменение положения руля высоты для различных $\dot{\delta}_\text{в max}$}
    \label{fig:model_DD_delta_elevator}
    \end{minipage}
\end{figure}
\begin{figure}[H]
    \begin{minipage}{0.48\textwidth}
    \centering
    \resizebox{1.1\linewidth}{!}{%% Creator: Matplotlib, PGF backend
%%
%% To include the figure in your LaTeX document, write
%%   \input{<filename>.pgf}
%%
%% Make sure the required packages are loaded in your preamble
%%   \usepackage{pgf}
%%
%% Figures using additional raster images can only be included by \input if
%% they are in the same directory as the main LaTeX file. For loading figures
%% from other directories you can use the `import` package
%%   \usepackage{import}
%%
%% and then include the figures with
%%   \import{<path to file>}{<filename>.pgf}
%%
%% Matplotlib used the following preamble
%%   \usepackage[warn]{mathtext}
%%   \usepackage[T2A]{fontenc}
%%   \usepackage[utf8]{inputenc}
%%   \usepackage[english,russian]{babel}
%%
\begingroup%
\makeatletter%
\begin{pgfpicture}%
\pgfpathrectangle{\pgfpointorigin}{\pgfqpoint{6.400000in}{4.800000in}}%
\pgfusepath{use as bounding box, clip}%
\begin{pgfscope}%
\pgfsetbuttcap%
\pgfsetmiterjoin%
\definecolor{currentfill}{rgb}{1.000000,1.000000,1.000000}%
\pgfsetfillcolor{currentfill}%
\pgfsetlinewidth{0.000000pt}%
\definecolor{currentstroke}{rgb}{1.000000,1.000000,1.000000}%
\pgfsetstrokecolor{currentstroke}%
\pgfsetdash{}{0pt}%
\pgfpathmoveto{\pgfqpoint{0.000000in}{0.000000in}}%
\pgfpathlineto{\pgfqpoint{6.400000in}{0.000000in}}%
\pgfpathlineto{\pgfqpoint{6.400000in}{4.800000in}}%
\pgfpathlineto{\pgfqpoint{0.000000in}{4.800000in}}%
\pgfpathclose%
\pgfusepath{fill}%
\end{pgfscope}%
\begin{pgfscope}%
\pgfsetbuttcap%
\pgfsetmiterjoin%
\definecolor{currentfill}{rgb}{1.000000,1.000000,1.000000}%
\pgfsetfillcolor{currentfill}%
\pgfsetlinewidth{0.000000pt}%
\definecolor{currentstroke}{rgb}{0.000000,0.000000,0.000000}%
\pgfsetstrokecolor{currentstroke}%
\pgfsetstrokeopacity{0.000000}%
\pgfsetdash{}{0pt}%
\pgfpathmoveto{\pgfqpoint{0.800000in}{0.528000in}}%
\pgfpathlineto{\pgfqpoint{5.760000in}{0.528000in}}%
\pgfpathlineto{\pgfqpoint{5.760000in}{4.224000in}}%
\pgfpathlineto{\pgfqpoint{0.800000in}{4.224000in}}%
\pgfpathclose%
\pgfusepath{fill}%
\end{pgfscope}%
\begin{pgfscope}%
\pgfpathrectangle{\pgfqpoint{0.800000in}{0.528000in}}{\pgfqpoint{4.960000in}{3.696000in}}%
\pgfusepath{clip}%
\pgfsetrectcap%
\pgfsetroundjoin%
\pgfsetlinewidth{0.803000pt}%
\definecolor{currentstroke}{rgb}{0.690196,0.690196,0.690196}%
\pgfsetstrokecolor{currentstroke}%
\pgfsetdash{}{0pt}%
\pgfpathmoveto{\pgfqpoint{1.025455in}{0.528000in}}%
\pgfpathlineto{\pgfqpoint{1.025455in}{4.224000in}}%
\pgfusepath{stroke}%
\end{pgfscope}%
\begin{pgfscope}%
\pgfsetbuttcap%
\pgfsetroundjoin%
\definecolor{currentfill}{rgb}{0.000000,0.000000,0.000000}%
\pgfsetfillcolor{currentfill}%
\pgfsetlinewidth{0.803000pt}%
\definecolor{currentstroke}{rgb}{0.000000,0.000000,0.000000}%
\pgfsetstrokecolor{currentstroke}%
\pgfsetdash{}{0pt}%
\pgfsys@defobject{currentmarker}{\pgfqpoint{0.000000in}{-0.048611in}}{\pgfqpoint{0.000000in}{0.000000in}}{%
\pgfpathmoveto{\pgfqpoint{0.000000in}{0.000000in}}%
\pgfpathlineto{\pgfqpoint{0.000000in}{-0.048611in}}%
\pgfusepath{stroke,fill}%
}%
\begin{pgfscope}%
\pgfsys@transformshift{1.025455in}{0.528000in}%
\pgfsys@useobject{currentmarker}{}%
\end{pgfscope}%
\end{pgfscope}%
\begin{pgfscope}%
\definecolor{textcolor}{rgb}{0.000000,0.000000,0.000000}%
\pgfsetstrokecolor{textcolor}%
\pgfsetfillcolor{textcolor}%
\pgftext[x=1.025455in,y=0.430778in,,top]{\color{textcolor}\rmfamily\fontsize{10.000000}{12.000000}\selectfont \(\displaystyle {0}\)}%
\end{pgfscope}%
\begin{pgfscope}%
\pgfpathrectangle{\pgfqpoint{0.800000in}{0.528000in}}{\pgfqpoint{4.960000in}{3.696000in}}%
\pgfusepath{clip}%
\pgfsetrectcap%
\pgfsetroundjoin%
\pgfsetlinewidth{0.803000pt}%
\definecolor{currentstroke}{rgb}{0.690196,0.690196,0.690196}%
\pgfsetstrokecolor{currentstroke}%
\pgfsetdash{}{0pt}%
\pgfpathmoveto{\pgfqpoint{1.776970in}{0.528000in}}%
\pgfpathlineto{\pgfqpoint{1.776970in}{4.224000in}}%
\pgfusepath{stroke}%
\end{pgfscope}%
\begin{pgfscope}%
\pgfsetbuttcap%
\pgfsetroundjoin%
\definecolor{currentfill}{rgb}{0.000000,0.000000,0.000000}%
\pgfsetfillcolor{currentfill}%
\pgfsetlinewidth{0.803000pt}%
\definecolor{currentstroke}{rgb}{0.000000,0.000000,0.000000}%
\pgfsetstrokecolor{currentstroke}%
\pgfsetdash{}{0pt}%
\pgfsys@defobject{currentmarker}{\pgfqpoint{0.000000in}{-0.048611in}}{\pgfqpoint{0.000000in}{0.000000in}}{%
\pgfpathmoveto{\pgfqpoint{0.000000in}{0.000000in}}%
\pgfpathlineto{\pgfqpoint{0.000000in}{-0.048611in}}%
\pgfusepath{stroke,fill}%
}%
\begin{pgfscope}%
\pgfsys@transformshift{1.776970in}{0.528000in}%
\pgfsys@useobject{currentmarker}{}%
\end{pgfscope}%
\end{pgfscope}%
\begin{pgfscope}%
\definecolor{textcolor}{rgb}{0.000000,0.000000,0.000000}%
\pgfsetstrokecolor{textcolor}%
\pgfsetfillcolor{textcolor}%
\pgftext[x=1.776970in,y=0.430778in,,top]{\color{textcolor}\rmfamily\fontsize{10.000000}{12.000000}\selectfont \(\displaystyle {5}\)}%
\end{pgfscope}%
\begin{pgfscope}%
\pgfpathrectangle{\pgfqpoint{0.800000in}{0.528000in}}{\pgfqpoint{4.960000in}{3.696000in}}%
\pgfusepath{clip}%
\pgfsetrectcap%
\pgfsetroundjoin%
\pgfsetlinewidth{0.803000pt}%
\definecolor{currentstroke}{rgb}{0.690196,0.690196,0.690196}%
\pgfsetstrokecolor{currentstroke}%
\pgfsetdash{}{0pt}%
\pgfpathmoveto{\pgfqpoint{2.528485in}{0.528000in}}%
\pgfpathlineto{\pgfqpoint{2.528485in}{4.224000in}}%
\pgfusepath{stroke}%
\end{pgfscope}%
\begin{pgfscope}%
\pgfsetbuttcap%
\pgfsetroundjoin%
\definecolor{currentfill}{rgb}{0.000000,0.000000,0.000000}%
\pgfsetfillcolor{currentfill}%
\pgfsetlinewidth{0.803000pt}%
\definecolor{currentstroke}{rgb}{0.000000,0.000000,0.000000}%
\pgfsetstrokecolor{currentstroke}%
\pgfsetdash{}{0pt}%
\pgfsys@defobject{currentmarker}{\pgfqpoint{0.000000in}{-0.048611in}}{\pgfqpoint{0.000000in}{0.000000in}}{%
\pgfpathmoveto{\pgfqpoint{0.000000in}{0.000000in}}%
\pgfpathlineto{\pgfqpoint{0.000000in}{-0.048611in}}%
\pgfusepath{stroke,fill}%
}%
\begin{pgfscope}%
\pgfsys@transformshift{2.528485in}{0.528000in}%
\pgfsys@useobject{currentmarker}{}%
\end{pgfscope}%
\end{pgfscope}%
\begin{pgfscope}%
\definecolor{textcolor}{rgb}{0.000000,0.000000,0.000000}%
\pgfsetstrokecolor{textcolor}%
\pgfsetfillcolor{textcolor}%
\pgftext[x=2.528485in,y=0.430778in,,top]{\color{textcolor}\rmfamily\fontsize{10.000000}{12.000000}\selectfont \(\displaystyle {10}\)}%
\end{pgfscope}%
\begin{pgfscope}%
\pgfpathrectangle{\pgfqpoint{0.800000in}{0.528000in}}{\pgfqpoint{4.960000in}{3.696000in}}%
\pgfusepath{clip}%
\pgfsetrectcap%
\pgfsetroundjoin%
\pgfsetlinewidth{0.803000pt}%
\definecolor{currentstroke}{rgb}{0.690196,0.690196,0.690196}%
\pgfsetstrokecolor{currentstroke}%
\pgfsetdash{}{0pt}%
\pgfpathmoveto{\pgfqpoint{3.280000in}{0.528000in}}%
\pgfpathlineto{\pgfqpoint{3.280000in}{4.224000in}}%
\pgfusepath{stroke}%
\end{pgfscope}%
\begin{pgfscope}%
\pgfsetbuttcap%
\pgfsetroundjoin%
\definecolor{currentfill}{rgb}{0.000000,0.000000,0.000000}%
\pgfsetfillcolor{currentfill}%
\pgfsetlinewidth{0.803000pt}%
\definecolor{currentstroke}{rgb}{0.000000,0.000000,0.000000}%
\pgfsetstrokecolor{currentstroke}%
\pgfsetdash{}{0pt}%
\pgfsys@defobject{currentmarker}{\pgfqpoint{0.000000in}{-0.048611in}}{\pgfqpoint{0.000000in}{0.000000in}}{%
\pgfpathmoveto{\pgfqpoint{0.000000in}{0.000000in}}%
\pgfpathlineto{\pgfqpoint{0.000000in}{-0.048611in}}%
\pgfusepath{stroke,fill}%
}%
\begin{pgfscope}%
\pgfsys@transformshift{3.280000in}{0.528000in}%
\pgfsys@useobject{currentmarker}{}%
\end{pgfscope}%
\end{pgfscope}%
\begin{pgfscope}%
\definecolor{textcolor}{rgb}{0.000000,0.000000,0.000000}%
\pgfsetstrokecolor{textcolor}%
\pgfsetfillcolor{textcolor}%
\pgftext[x=3.280000in,y=0.430778in,,top]{\color{textcolor}\rmfamily\fontsize{10.000000}{12.000000}\selectfont \(\displaystyle {15}\)}%
\end{pgfscope}%
\begin{pgfscope}%
\pgfpathrectangle{\pgfqpoint{0.800000in}{0.528000in}}{\pgfqpoint{4.960000in}{3.696000in}}%
\pgfusepath{clip}%
\pgfsetrectcap%
\pgfsetroundjoin%
\pgfsetlinewidth{0.803000pt}%
\definecolor{currentstroke}{rgb}{0.690196,0.690196,0.690196}%
\pgfsetstrokecolor{currentstroke}%
\pgfsetdash{}{0pt}%
\pgfpathmoveto{\pgfqpoint{4.031515in}{0.528000in}}%
\pgfpathlineto{\pgfqpoint{4.031515in}{4.224000in}}%
\pgfusepath{stroke}%
\end{pgfscope}%
\begin{pgfscope}%
\pgfsetbuttcap%
\pgfsetroundjoin%
\definecolor{currentfill}{rgb}{0.000000,0.000000,0.000000}%
\pgfsetfillcolor{currentfill}%
\pgfsetlinewidth{0.803000pt}%
\definecolor{currentstroke}{rgb}{0.000000,0.000000,0.000000}%
\pgfsetstrokecolor{currentstroke}%
\pgfsetdash{}{0pt}%
\pgfsys@defobject{currentmarker}{\pgfqpoint{0.000000in}{-0.048611in}}{\pgfqpoint{0.000000in}{0.000000in}}{%
\pgfpathmoveto{\pgfqpoint{0.000000in}{0.000000in}}%
\pgfpathlineto{\pgfqpoint{0.000000in}{-0.048611in}}%
\pgfusepath{stroke,fill}%
}%
\begin{pgfscope}%
\pgfsys@transformshift{4.031515in}{0.528000in}%
\pgfsys@useobject{currentmarker}{}%
\end{pgfscope}%
\end{pgfscope}%
\begin{pgfscope}%
\definecolor{textcolor}{rgb}{0.000000,0.000000,0.000000}%
\pgfsetstrokecolor{textcolor}%
\pgfsetfillcolor{textcolor}%
\pgftext[x=4.031515in,y=0.430778in,,top]{\color{textcolor}\rmfamily\fontsize{10.000000}{12.000000}\selectfont \(\displaystyle {20}\)}%
\end{pgfscope}%
\begin{pgfscope}%
\pgfpathrectangle{\pgfqpoint{0.800000in}{0.528000in}}{\pgfqpoint{4.960000in}{3.696000in}}%
\pgfusepath{clip}%
\pgfsetrectcap%
\pgfsetroundjoin%
\pgfsetlinewidth{0.803000pt}%
\definecolor{currentstroke}{rgb}{0.690196,0.690196,0.690196}%
\pgfsetstrokecolor{currentstroke}%
\pgfsetdash{}{0pt}%
\pgfpathmoveto{\pgfqpoint{4.783030in}{0.528000in}}%
\pgfpathlineto{\pgfqpoint{4.783030in}{4.224000in}}%
\pgfusepath{stroke}%
\end{pgfscope}%
\begin{pgfscope}%
\pgfsetbuttcap%
\pgfsetroundjoin%
\definecolor{currentfill}{rgb}{0.000000,0.000000,0.000000}%
\pgfsetfillcolor{currentfill}%
\pgfsetlinewidth{0.803000pt}%
\definecolor{currentstroke}{rgb}{0.000000,0.000000,0.000000}%
\pgfsetstrokecolor{currentstroke}%
\pgfsetdash{}{0pt}%
\pgfsys@defobject{currentmarker}{\pgfqpoint{0.000000in}{-0.048611in}}{\pgfqpoint{0.000000in}{0.000000in}}{%
\pgfpathmoveto{\pgfqpoint{0.000000in}{0.000000in}}%
\pgfpathlineto{\pgfqpoint{0.000000in}{-0.048611in}}%
\pgfusepath{stroke,fill}%
}%
\begin{pgfscope}%
\pgfsys@transformshift{4.783030in}{0.528000in}%
\pgfsys@useobject{currentmarker}{}%
\end{pgfscope}%
\end{pgfscope}%
\begin{pgfscope}%
\definecolor{textcolor}{rgb}{0.000000,0.000000,0.000000}%
\pgfsetstrokecolor{textcolor}%
\pgfsetfillcolor{textcolor}%
\pgftext[x=4.783030in,y=0.430778in,,top]{\color{textcolor}\rmfamily\fontsize{10.000000}{12.000000}\selectfont \(\displaystyle {25}\)}%
\end{pgfscope}%
\begin{pgfscope}%
\pgfpathrectangle{\pgfqpoint{0.800000in}{0.528000in}}{\pgfqpoint{4.960000in}{3.696000in}}%
\pgfusepath{clip}%
\pgfsetrectcap%
\pgfsetroundjoin%
\pgfsetlinewidth{0.803000pt}%
\definecolor{currentstroke}{rgb}{0.690196,0.690196,0.690196}%
\pgfsetstrokecolor{currentstroke}%
\pgfsetdash{}{0pt}%
\pgfpathmoveto{\pgfqpoint{5.534545in}{0.528000in}}%
\pgfpathlineto{\pgfqpoint{5.534545in}{4.224000in}}%
\pgfusepath{stroke}%
\end{pgfscope}%
\begin{pgfscope}%
\pgfsetbuttcap%
\pgfsetroundjoin%
\definecolor{currentfill}{rgb}{0.000000,0.000000,0.000000}%
\pgfsetfillcolor{currentfill}%
\pgfsetlinewidth{0.803000pt}%
\definecolor{currentstroke}{rgb}{0.000000,0.000000,0.000000}%
\pgfsetstrokecolor{currentstroke}%
\pgfsetdash{}{0pt}%
\pgfsys@defobject{currentmarker}{\pgfqpoint{0.000000in}{-0.048611in}}{\pgfqpoint{0.000000in}{0.000000in}}{%
\pgfpathmoveto{\pgfqpoint{0.000000in}{0.000000in}}%
\pgfpathlineto{\pgfqpoint{0.000000in}{-0.048611in}}%
\pgfusepath{stroke,fill}%
}%
\begin{pgfscope}%
\pgfsys@transformshift{5.534545in}{0.528000in}%
\pgfsys@useobject{currentmarker}{}%
\end{pgfscope}%
\end{pgfscope}%
\begin{pgfscope}%
\definecolor{textcolor}{rgb}{0.000000,0.000000,0.000000}%
\pgfsetstrokecolor{textcolor}%
\pgfsetfillcolor{textcolor}%
\pgftext[x=5.534545in,y=0.430778in,,top]{\color{textcolor}\rmfamily\fontsize{10.000000}{12.000000}\selectfont \(\displaystyle {30}\)}%
\end{pgfscope}%
\begin{pgfscope}%
\definecolor{textcolor}{rgb}{0.000000,0.000000,0.000000}%
\pgfsetstrokecolor{textcolor}%
\pgfsetfillcolor{textcolor}%
\pgftext[x=3.280000in,y=0.251796in,,top]{\color{textcolor}\rmfamily\fontsize{10.000000}{12.000000}\selectfont \(\displaystyle t,\ с\)}%
\end{pgfscope}%
\begin{pgfscope}%
\pgfpathrectangle{\pgfqpoint{0.800000in}{0.528000in}}{\pgfqpoint{4.960000in}{3.696000in}}%
\pgfusepath{clip}%
\pgfsetrectcap%
\pgfsetroundjoin%
\pgfsetlinewidth{0.803000pt}%
\definecolor{currentstroke}{rgb}{0.690196,0.690196,0.690196}%
\pgfsetstrokecolor{currentstroke}%
\pgfsetdash{}{0pt}%
\pgfpathmoveto{\pgfqpoint{0.800000in}{0.979609in}}%
\pgfpathlineto{\pgfqpoint{5.760000in}{0.979609in}}%
\pgfusepath{stroke}%
\end{pgfscope}%
\begin{pgfscope}%
\pgfsetbuttcap%
\pgfsetroundjoin%
\definecolor{currentfill}{rgb}{0.000000,0.000000,0.000000}%
\pgfsetfillcolor{currentfill}%
\pgfsetlinewidth{0.803000pt}%
\definecolor{currentstroke}{rgb}{0.000000,0.000000,0.000000}%
\pgfsetstrokecolor{currentstroke}%
\pgfsetdash{}{0pt}%
\pgfsys@defobject{currentmarker}{\pgfqpoint{-0.048611in}{0.000000in}}{\pgfqpoint{-0.000000in}{0.000000in}}{%
\pgfpathmoveto{\pgfqpoint{-0.000000in}{0.000000in}}%
\pgfpathlineto{\pgfqpoint{-0.048611in}{0.000000in}}%
\pgfusepath{stroke,fill}%
}%
\begin{pgfscope}%
\pgfsys@transformshift{0.800000in}{0.979609in}%
\pgfsys@useobject{currentmarker}{}%
\end{pgfscope}%
\end{pgfscope}%
\begin{pgfscope}%
\definecolor{textcolor}{rgb}{0.000000,0.000000,0.000000}%
\pgfsetstrokecolor{textcolor}%
\pgfsetfillcolor{textcolor}%
\pgftext[x=0.347838in, y=0.931395in, left, base]{\color{textcolor}\rmfamily\fontsize{10.000000}{12.000000}\selectfont \(\displaystyle {\ensuremath{-}0.02}\)}%
\end{pgfscope}%
\begin{pgfscope}%
\pgfpathrectangle{\pgfqpoint{0.800000in}{0.528000in}}{\pgfqpoint{4.960000in}{3.696000in}}%
\pgfusepath{clip}%
\pgfsetrectcap%
\pgfsetroundjoin%
\pgfsetlinewidth{0.803000pt}%
\definecolor{currentstroke}{rgb}{0.690196,0.690196,0.690196}%
\pgfsetstrokecolor{currentstroke}%
\pgfsetdash{}{0pt}%
\pgfpathmoveto{\pgfqpoint{0.800000in}{1.435019in}}%
\pgfpathlineto{\pgfqpoint{5.760000in}{1.435019in}}%
\pgfusepath{stroke}%
\end{pgfscope}%
\begin{pgfscope}%
\pgfsetbuttcap%
\pgfsetroundjoin%
\definecolor{currentfill}{rgb}{0.000000,0.000000,0.000000}%
\pgfsetfillcolor{currentfill}%
\pgfsetlinewidth{0.803000pt}%
\definecolor{currentstroke}{rgb}{0.000000,0.000000,0.000000}%
\pgfsetstrokecolor{currentstroke}%
\pgfsetdash{}{0pt}%
\pgfsys@defobject{currentmarker}{\pgfqpoint{-0.048611in}{0.000000in}}{\pgfqpoint{-0.000000in}{0.000000in}}{%
\pgfpathmoveto{\pgfqpoint{-0.000000in}{0.000000in}}%
\pgfpathlineto{\pgfqpoint{-0.048611in}{0.000000in}}%
\pgfusepath{stroke,fill}%
}%
\begin{pgfscope}%
\pgfsys@transformshift{0.800000in}{1.435019in}%
\pgfsys@useobject{currentmarker}{}%
\end{pgfscope}%
\end{pgfscope}%
\begin{pgfscope}%
\definecolor{textcolor}{rgb}{0.000000,0.000000,0.000000}%
\pgfsetstrokecolor{textcolor}%
\pgfsetfillcolor{textcolor}%
\pgftext[x=0.455863in, y=1.386806in, left, base]{\color{textcolor}\rmfamily\fontsize{10.000000}{12.000000}\selectfont \(\displaystyle {0.00}\)}%
\end{pgfscope}%
\begin{pgfscope}%
\pgfpathrectangle{\pgfqpoint{0.800000in}{0.528000in}}{\pgfqpoint{4.960000in}{3.696000in}}%
\pgfusepath{clip}%
\pgfsetrectcap%
\pgfsetroundjoin%
\pgfsetlinewidth{0.803000pt}%
\definecolor{currentstroke}{rgb}{0.690196,0.690196,0.690196}%
\pgfsetstrokecolor{currentstroke}%
\pgfsetdash{}{0pt}%
\pgfpathmoveto{\pgfqpoint{0.800000in}{1.890430in}}%
\pgfpathlineto{\pgfqpoint{5.760000in}{1.890430in}}%
\pgfusepath{stroke}%
\end{pgfscope}%
\begin{pgfscope}%
\pgfsetbuttcap%
\pgfsetroundjoin%
\definecolor{currentfill}{rgb}{0.000000,0.000000,0.000000}%
\pgfsetfillcolor{currentfill}%
\pgfsetlinewidth{0.803000pt}%
\definecolor{currentstroke}{rgb}{0.000000,0.000000,0.000000}%
\pgfsetstrokecolor{currentstroke}%
\pgfsetdash{}{0pt}%
\pgfsys@defobject{currentmarker}{\pgfqpoint{-0.048611in}{0.000000in}}{\pgfqpoint{-0.000000in}{0.000000in}}{%
\pgfpathmoveto{\pgfqpoint{-0.000000in}{0.000000in}}%
\pgfpathlineto{\pgfqpoint{-0.048611in}{0.000000in}}%
\pgfusepath{stroke,fill}%
}%
\begin{pgfscope}%
\pgfsys@transformshift{0.800000in}{1.890430in}%
\pgfsys@useobject{currentmarker}{}%
\end{pgfscope}%
\end{pgfscope}%
\begin{pgfscope}%
\definecolor{textcolor}{rgb}{0.000000,0.000000,0.000000}%
\pgfsetstrokecolor{textcolor}%
\pgfsetfillcolor{textcolor}%
\pgftext[x=0.455863in, y=1.842216in, left, base]{\color{textcolor}\rmfamily\fontsize{10.000000}{12.000000}\selectfont \(\displaystyle {0.02}\)}%
\end{pgfscope}%
\begin{pgfscope}%
\pgfpathrectangle{\pgfqpoint{0.800000in}{0.528000in}}{\pgfqpoint{4.960000in}{3.696000in}}%
\pgfusepath{clip}%
\pgfsetrectcap%
\pgfsetroundjoin%
\pgfsetlinewidth{0.803000pt}%
\definecolor{currentstroke}{rgb}{0.690196,0.690196,0.690196}%
\pgfsetstrokecolor{currentstroke}%
\pgfsetdash{}{0pt}%
\pgfpathmoveto{\pgfqpoint{0.800000in}{2.345841in}}%
\pgfpathlineto{\pgfqpoint{5.760000in}{2.345841in}}%
\pgfusepath{stroke}%
\end{pgfscope}%
\begin{pgfscope}%
\pgfsetbuttcap%
\pgfsetroundjoin%
\definecolor{currentfill}{rgb}{0.000000,0.000000,0.000000}%
\pgfsetfillcolor{currentfill}%
\pgfsetlinewidth{0.803000pt}%
\definecolor{currentstroke}{rgb}{0.000000,0.000000,0.000000}%
\pgfsetstrokecolor{currentstroke}%
\pgfsetdash{}{0pt}%
\pgfsys@defobject{currentmarker}{\pgfqpoint{-0.048611in}{0.000000in}}{\pgfqpoint{-0.000000in}{0.000000in}}{%
\pgfpathmoveto{\pgfqpoint{-0.000000in}{0.000000in}}%
\pgfpathlineto{\pgfqpoint{-0.048611in}{0.000000in}}%
\pgfusepath{stroke,fill}%
}%
\begin{pgfscope}%
\pgfsys@transformshift{0.800000in}{2.345841in}%
\pgfsys@useobject{currentmarker}{}%
\end{pgfscope}%
\end{pgfscope}%
\begin{pgfscope}%
\definecolor{textcolor}{rgb}{0.000000,0.000000,0.000000}%
\pgfsetstrokecolor{textcolor}%
\pgfsetfillcolor{textcolor}%
\pgftext[x=0.455863in, y=2.297627in, left, base]{\color{textcolor}\rmfamily\fontsize{10.000000}{12.000000}\selectfont \(\displaystyle {0.04}\)}%
\end{pgfscope}%
\begin{pgfscope}%
\pgfpathrectangle{\pgfqpoint{0.800000in}{0.528000in}}{\pgfqpoint{4.960000in}{3.696000in}}%
\pgfusepath{clip}%
\pgfsetrectcap%
\pgfsetroundjoin%
\pgfsetlinewidth{0.803000pt}%
\definecolor{currentstroke}{rgb}{0.690196,0.690196,0.690196}%
\pgfsetstrokecolor{currentstroke}%
\pgfsetdash{}{0pt}%
\pgfpathmoveto{\pgfqpoint{0.800000in}{2.801251in}}%
\pgfpathlineto{\pgfqpoint{5.760000in}{2.801251in}}%
\pgfusepath{stroke}%
\end{pgfscope}%
\begin{pgfscope}%
\pgfsetbuttcap%
\pgfsetroundjoin%
\definecolor{currentfill}{rgb}{0.000000,0.000000,0.000000}%
\pgfsetfillcolor{currentfill}%
\pgfsetlinewidth{0.803000pt}%
\definecolor{currentstroke}{rgb}{0.000000,0.000000,0.000000}%
\pgfsetstrokecolor{currentstroke}%
\pgfsetdash{}{0pt}%
\pgfsys@defobject{currentmarker}{\pgfqpoint{-0.048611in}{0.000000in}}{\pgfqpoint{-0.000000in}{0.000000in}}{%
\pgfpathmoveto{\pgfqpoint{-0.000000in}{0.000000in}}%
\pgfpathlineto{\pgfqpoint{-0.048611in}{0.000000in}}%
\pgfusepath{stroke,fill}%
}%
\begin{pgfscope}%
\pgfsys@transformshift{0.800000in}{2.801251in}%
\pgfsys@useobject{currentmarker}{}%
\end{pgfscope}%
\end{pgfscope}%
\begin{pgfscope}%
\definecolor{textcolor}{rgb}{0.000000,0.000000,0.000000}%
\pgfsetstrokecolor{textcolor}%
\pgfsetfillcolor{textcolor}%
\pgftext[x=0.455863in, y=2.753038in, left, base]{\color{textcolor}\rmfamily\fontsize{10.000000}{12.000000}\selectfont \(\displaystyle {0.06}\)}%
\end{pgfscope}%
\begin{pgfscope}%
\pgfpathrectangle{\pgfqpoint{0.800000in}{0.528000in}}{\pgfqpoint{4.960000in}{3.696000in}}%
\pgfusepath{clip}%
\pgfsetrectcap%
\pgfsetroundjoin%
\pgfsetlinewidth{0.803000pt}%
\definecolor{currentstroke}{rgb}{0.690196,0.690196,0.690196}%
\pgfsetstrokecolor{currentstroke}%
\pgfsetdash{}{0pt}%
\pgfpathmoveto{\pgfqpoint{0.800000in}{3.256662in}}%
\pgfpathlineto{\pgfqpoint{5.760000in}{3.256662in}}%
\pgfusepath{stroke}%
\end{pgfscope}%
\begin{pgfscope}%
\pgfsetbuttcap%
\pgfsetroundjoin%
\definecolor{currentfill}{rgb}{0.000000,0.000000,0.000000}%
\pgfsetfillcolor{currentfill}%
\pgfsetlinewidth{0.803000pt}%
\definecolor{currentstroke}{rgb}{0.000000,0.000000,0.000000}%
\pgfsetstrokecolor{currentstroke}%
\pgfsetdash{}{0pt}%
\pgfsys@defobject{currentmarker}{\pgfqpoint{-0.048611in}{0.000000in}}{\pgfqpoint{-0.000000in}{0.000000in}}{%
\pgfpathmoveto{\pgfqpoint{-0.000000in}{0.000000in}}%
\pgfpathlineto{\pgfqpoint{-0.048611in}{0.000000in}}%
\pgfusepath{stroke,fill}%
}%
\begin{pgfscope}%
\pgfsys@transformshift{0.800000in}{3.256662in}%
\pgfsys@useobject{currentmarker}{}%
\end{pgfscope}%
\end{pgfscope}%
\begin{pgfscope}%
\definecolor{textcolor}{rgb}{0.000000,0.000000,0.000000}%
\pgfsetstrokecolor{textcolor}%
\pgfsetfillcolor{textcolor}%
\pgftext[x=0.455863in, y=3.208449in, left, base]{\color{textcolor}\rmfamily\fontsize{10.000000}{12.000000}\selectfont \(\displaystyle {0.08}\)}%
\end{pgfscope}%
\begin{pgfscope}%
\pgfpathrectangle{\pgfqpoint{0.800000in}{0.528000in}}{\pgfqpoint{4.960000in}{3.696000in}}%
\pgfusepath{clip}%
\pgfsetrectcap%
\pgfsetroundjoin%
\pgfsetlinewidth{0.803000pt}%
\definecolor{currentstroke}{rgb}{0.690196,0.690196,0.690196}%
\pgfsetstrokecolor{currentstroke}%
\pgfsetdash{}{0pt}%
\pgfpathmoveto{\pgfqpoint{0.800000in}{3.712073in}}%
\pgfpathlineto{\pgfqpoint{5.760000in}{3.712073in}}%
\pgfusepath{stroke}%
\end{pgfscope}%
\begin{pgfscope}%
\pgfsetbuttcap%
\pgfsetroundjoin%
\definecolor{currentfill}{rgb}{0.000000,0.000000,0.000000}%
\pgfsetfillcolor{currentfill}%
\pgfsetlinewidth{0.803000pt}%
\definecolor{currentstroke}{rgb}{0.000000,0.000000,0.000000}%
\pgfsetstrokecolor{currentstroke}%
\pgfsetdash{}{0pt}%
\pgfsys@defobject{currentmarker}{\pgfqpoint{-0.048611in}{0.000000in}}{\pgfqpoint{-0.000000in}{0.000000in}}{%
\pgfpathmoveto{\pgfqpoint{-0.000000in}{0.000000in}}%
\pgfpathlineto{\pgfqpoint{-0.048611in}{0.000000in}}%
\pgfusepath{stroke,fill}%
}%
\begin{pgfscope}%
\pgfsys@transformshift{0.800000in}{3.712073in}%
\pgfsys@useobject{currentmarker}{}%
\end{pgfscope}%
\end{pgfscope}%
\begin{pgfscope}%
\definecolor{textcolor}{rgb}{0.000000,0.000000,0.000000}%
\pgfsetstrokecolor{textcolor}%
\pgfsetfillcolor{textcolor}%
\pgftext[x=0.455863in, y=3.663859in, left, base]{\color{textcolor}\rmfamily\fontsize{10.000000}{12.000000}\selectfont \(\displaystyle {0.10}\)}%
\end{pgfscope}%
\begin{pgfscope}%
\pgfpathrectangle{\pgfqpoint{0.800000in}{0.528000in}}{\pgfqpoint{4.960000in}{3.696000in}}%
\pgfusepath{clip}%
\pgfsetrectcap%
\pgfsetroundjoin%
\pgfsetlinewidth{0.803000pt}%
\definecolor{currentstroke}{rgb}{0.690196,0.690196,0.690196}%
\pgfsetstrokecolor{currentstroke}%
\pgfsetdash{}{0pt}%
\pgfpathmoveto{\pgfqpoint{0.800000in}{4.167484in}}%
\pgfpathlineto{\pgfqpoint{5.760000in}{4.167484in}}%
\pgfusepath{stroke}%
\end{pgfscope}%
\begin{pgfscope}%
\pgfsetbuttcap%
\pgfsetroundjoin%
\definecolor{currentfill}{rgb}{0.000000,0.000000,0.000000}%
\pgfsetfillcolor{currentfill}%
\pgfsetlinewidth{0.803000pt}%
\definecolor{currentstroke}{rgb}{0.000000,0.000000,0.000000}%
\pgfsetstrokecolor{currentstroke}%
\pgfsetdash{}{0pt}%
\pgfsys@defobject{currentmarker}{\pgfqpoint{-0.048611in}{0.000000in}}{\pgfqpoint{-0.000000in}{0.000000in}}{%
\pgfpathmoveto{\pgfqpoint{-0.000000in}{0.000000in}}%
\pgfpathlineto{\pgfqpoint{-0.048611in}{0.000000in}}%
\pgfusepath{stroke,fill}%
}%
\begin{pgfscope}%
\pgfsys@transformshift{0.800000in}{4.167484in}%
\pgfsys@useobject{currentmarker}{}%
\end{pgfscope}%
\end{pgfscope}%
\begin{pgfscope}%
\definecolor{textcolor}{rgb}{0.000000,0.000000,0.000000}%
\pgfsetstrokecolor{textcolor}%
\pgfsetfillcolor{textcolor}%
\pgftext[x=0.455863in, y=4.119270in, left, base]{\color{textcolor}\rmfamily\fontsize{10.000000}{12.000000}\selectfont \(\displaystyle {0.12}\)}%
\end{pgfscope}%
\begin{pgfscope}%
\definecolor{textcolor}{rgb}{0.000000,0.000000,0.000000}%
\pgfsetstrokecolor{textcolor}%
\pgfsetfillcolor{textcolor}%
\pgftext[x=0.292283in,y=2.376000in,,bottom,rotate=90.000000]{\color{textcolor}\rmfamily\fontsize{10.000000}{12.000000}\selectfont \(\displaystyle \omega_z,\ рад/с\)}%
\end{pgfscope}%
\begin{pgfscope}%
\pgfpathrectangle{\pgfqpoint{0.800000in}{0.528000in}}{\pgfqpoint{4.960000in}{3.696000in}}%
\pgfusepath{clip}%
\pgfsetrectcap%
\pgfsetroundjoin%
\pgfsetlinewidth{1.505625pt}%
\definecolor{currentstroke}{rgb}{0.121569,0.466667,0.705882}%
\pgfsetstrokecolor{currentstroke}%
\pgfsetdash{}{0pt}%
\pgfpathmoveto{\pgfqpoint{1.025455in}{1.435019in}}%
\pgfpathlineto{\pgfqpoint{1.177862in}{1.436067in}}%
\pgfpathlineto{\pgfqpoint{1.180417in}{1.440311in}}%
\pgfpathlineto{\pgfqpoint{1.183874in}{1.451193in}}%
\pgfpathlineto{\pgfqpoint{1.188233in}{1.473244in}}%
\pgfpathlineto{\pgfqpoint{1.193644in}{1.513344in}}%
\pgfpathlineto{\pgfqpoint{1.200107in}{1.579352in}}%
\pgfpathlineto{\pgfqpoint{1.207622in}{1.680272in}}%
\pgfpathlineto{\pgfqpoint{1.216339in}{1.828870in}}%
\pgfpathlineto{\pgfqpoint{1.226410in}{2.041032in}}%
\pgfpathlineto{\pgfqpoint{1.237833in}{2.331695in}}%
\pgfpathlineto{\pgfqpoint{1.250608in}{2.715955in}}%
\pgfpathlineto{\pgfqpoint{1.279617in}{3.622128in}}%
\pgfpathlineto{\pgfqpoint{1.288635in}{3.807153in}}%
\pgfpathlineto{\pgfqpoint{1.296451in}{3.925426in}}%
\pgfpathlineto{\pgfqpoint{1.303215in}{3.996854in}}%
\pgfpathlineto{\pgfqpoint{1.308776in}{4.034488in}}%
\pgfpathlineto{\pgfqpoint{1.313135in}{4.050900in}}%
\pgfpathlineto{\pgfqpoint{1.316291in}{4.055706in}}%
\pgfpathlineto{\pgfqpoint{1.318395in}{4.055643in}}%
\pgfpathlineto{\pgfqpoint{1.320499in}{4.052988in}}%
\pgfpathlineto{\pgfqpoint{1.323355in}{4.045275in}}%
\pgfpathlineto{\pgfqpoint{1.327113in}{4.027991in}}%
\pgfpathlineto{\pgfqpoint{1.331922in}{3.994209in}}%
\pgfpathlineto{\pgfqpoint{1.337784in}{3.935676in}}%
\pgfpathlineto{\pgfqpoint{1.344698in}{3.842721in}}%
\pgfpathlineto{\pgfqpoint{1.354618in}{3.673720in}}%
\pgfpathlineto{\pgfqpoint{1.363185in}{3.551438in}}%
\pgfpathlineto{\pgfqpoint{1.370701in}{3.473279in}}%
\pgfpathlineto{\pgfqpoint{1.386633in}{3.321526in}}%
\pgfpathlineto{\pgfqpoint{1.397905in}{3.185344in}}%
\pgfpathlineto{\pgfqpoint{1.439840in}{2.656187in}}%
\pgfpathlineto{\pgfqpoint{1.457726in}{2.470153in}}%
\pgfpathlineto{\pgfqpoint{1.475913in}{2.303679in}}%
\pgfpathlineto{\pgfqpoint{1.494400in}{2.154571in}}%
\pgfpathlineto{\pgfqpoint{1.513188in}{2.020816in}}%
\pgfpathlineto{\pgfqpoint{1.532427in}{1.899519in}}%
\pgfpathlineto{\pgfqpoint{1.552267in}{1.788374in}}%
\pgfpathlineto{\pgfqpoint{1.572708in}{1.686283in}}%
\pgfpathlineto{\pgfqpoint{1.593901in}{1.591609in}}%
\pgfpathlineto{\pgfqpoint{1.615845in}{1.503644in}}%
\pgfpathlineto{\pgfqpoint{1.638541in}{1.421741in}}%
\pgfpathlineto{\pgfqpoint{1.661988in}{1.345330in}}%
\pgfpathlineto{\pgfqpoint{1.686337in}{1.273504in}}%
\pgfpathlineto{\pgfqpoint{1.711287in}{1.206746in}}%
\pgfpathlineto{\pgfqpoint{1.736839in}{1.144631in}}%
\pgfpathlineto{\pgfqpoint{1.762992in}{1.086838in}}%
\pgfpathlineto{\pgfqpoint{1.789445in}{1.033715in}}%
\pgfpathlineto{\pgfqpoint{1.816048in}{0.985208in}}%
\pgfpathlineto{\pgfqpoint{1.842652in}{0.941252in}}%
\pgfpathlineto{\pgfqpoint{1.869105in}{0.901759in}}%
\pgfpathlineto{\pgfqpoint{1.895408in}{0.866423in}}%
\pgfpathlineto{\pgfqpoint{1.921411in}{0.835162in}}%
\pgfpathlineto{\pgfqpoint{1.947113in}{0.807697in}}%
\pgfpathlineto{\pgfqpoint{1.972364in}{0.783917in}}%
\pgfpathlineto{\pgfqpoint{1.997314in}{0.763432in}}%
\pgfpathlineto{\pgfqpoint{2.021964in}{0.746038in}}%
\pgfpathlineto{\pgfqpoint{2.046313in}{0.731546in}}%
\pgfpathlineto{\pgfqpoint{2.070512in}{0.719712in}}%
\pgfpathlineto{\pgfqpoint{2.094560in}{0.710409in}}%
\pgfpathlineto{\pgfqpoint{2.118608in}{0.703478in}}%
\pgfpathlineto{\pgfqpoint{2.142807in}{0.698820in}}%
\pgfpathlineto{\pgfqpoint{2.167156in}{0.696398in}}%
\pgfpathlineto{\pgfqpoint{2.191806in}{0.696174in}}%
\pgfpathlineto{\pgfqpoint{2.217057in}{0.698178in}}%
\pgfpathlineto{\pgfqpoint{2.242909in}{0.702472in}}%
\pgfpathlineto{\pgfqpoint{2.269663in}{0.709193in}}%
\pgfpathlineto{\pgfqpoint{2.297319in}{0.718451in}}%
\pgfpathlineto{\pgfqpoint{2.326177in}{0.730473in}}%
\pgfpathlineto{\pgfqpoint{2.356538in}{0.745558in}}%
\pgfpathlineto{\pgfqpoint{2.388703in}{0.764067in}}%
\pgfpathlineto{\pgfqpoint{2.422822in}{0.786308in}}%
\pgfpathlineto{\pgfqpoint{2.459496in}{0.812916in}}%
\pgfpathlineto{\pgfqpoint{2.499326in}{0.844619in}}%
\pgfpathlineto{\pgfqpoint{2.543215in}{0.882457in}}%
\pgfpathlineto{\pgfqpoint{2.592965in}{0.928373in}}%
\pgfpathlineto{\pgfqpoint{2.651884in}{0.985898in}}%
\pgfpathlineto{\pgfqpoint{2.731244in}{1.066739in}}%
\pgfpathlineto{\pgfqpoint{2.914313in}{1.254189in}}%
\pgfpathlineto{\pgfqpoint{2.980596in}{1.318218in}}%
\pgfpathlineto{\pgfqpoint{3.038012in}{1.370654in}}%
\pgfpathlineto{\pgfqpoint{3.090468in}{1.415597in}}%
\pgfpathlineto{\pgfqpoint{3.139467in}{1.454696in}}%
\pgfpathlineto{\pgfqpoint{3.186061in}{1.489073in}}%
\pgfpathlineto{\pgfqpoint{3.230851in}{1.519389in}}%
\pgfpathlineto{\pgfqpoint{3.274288in}{1.546131in}}%
\pgfpathlineto{\pgfqpoint{3.316674in}{1.569635in}}%
\pgfpathlineto{\pgfqpoint{3.358308in}{1.590196in}}%
\pgfpathlineto{\pgfqpoint{3.399341in}{1.607996in}}%
\pgfpathlineto{\pgfqpoint{3.440073in}{1.623254in}}%
\pgfpathlineto{\pgfqpoint{3.480655in}{1.636093in}}%
\pgfpathlineto{\pgfqpoint{3.521387in}{1.646649in}}%
\pgfpathlineto{\pgfqpoint{3.562419in}{1.654983in}}%
\pgfpathlineto{\pgfqpoint{3.603903in}{1.661136in}}%
\pgfpathlineto{\pgfqpoint{3.646138in}{1.665147in}}%
\pgfpathlineto{\pgfqpoint{3.689425in}{1.667010in}}%
\pgfpathlineto{\pgfqpoint{3.733915in}{1.666689in}}%
\pgfpathlineto{\pgfqpoint{3.780058in}{1.664122in}}%
\pgfpathlineto{\pgfqpoint{3.828456in}{1.659179in}}%
\pgfpathlineto{\pgfqpoint{3.879559in}{1.651696in}}%
\pgfpathlineto{\pgfqpoint{3.934269in}{1.641403in}}%
\pgfpathlineto{\pgfqpoint{3.993789in}{1.627904in}}%
\pgfpathlineto{\pgfqpoint{4.060524in}{1.610434in}}%
\pgfpathlineto{\pgfqpoint{4.139132in}{1.587472in}}%
\pgfpathlineto{\pgfqpoint{4.244945in}{1.554061in}}%
\pgfpathlineto{\pgfqpoint{4.484528in}{1.477746in}}%
\pgfpathlineto{\pgfqpoint{4.575462in}{1.451596in}}%
\pgfpathlineto{\pgfqpoint{4.655874in}{1.430753in}}%
\pgfpathlineto{\pgfqpoint{4.731025in}{1.413550in}}%
\pgfpathlineto{\pgfqpoint{4.803171in}{1.399299in}}%
\pgfpathlineto{\pgfqpoint{4.873813in}{1.387600in}}%
\pgfpathlineto{\pgfqpoint{4.944005in}{1.378227in}}%
\pgfpathlineto{\pgfqpoint{5.014647in}{1.371041in}}%
\pgfpathlineto{\pgfqpoint{5.086492in}{1.365978in}}%
\pgfpathlineto{\pgfqpoint{5.160441in}{1.363010in}}%
\pgfpathlineto{\pgfqpoint{5.237547in}{1.362159in}}%
\pgfpathlineto{\pgfqpoint{5.319312in}{1.363503in}}%
\pgfpathlineto{\pgfqpoint{5.407990in}{1.367218in}}%
\pgfpathlineto{\pgfqpoint{5.507491in}{1.373664in}}%
\pgfpathlineto{\pgfqpoint{5.534545in}{1.375755in}}%
\pgfpathlineto{\pgfqpoint{5.534545in}{1.375755in}}%
\pgfusepath{stroke}%
\end{pgfscope}%
\begin{pgfscope}%
\pgfpathrectangle{\pgfqpoint{0.800000in}{0.528000in}}{\pgfqpoint{4.960000in}{3.696000in}}%
\pgfusepath{clip}%
\pgfsetbuttcap%
\pgfsetroundjoin%
\pgfsetlinewidth{1.505625pt}%
\definecolor{currentstroke}{rgb}{1.000000,0.498039,0.054902}%
\pgfsetstrokecolor{currentstroke}%
\pgfsetdash{{9.600000pt}{2.400000pt}{1.500000pt}{2.400000pt}}{0.000000pt}%
\pgfpathmoveto{\pgfqpoint{1.025455in}{1.435019in}}%
\pgfpathlineto{\pgfqpoint{1.177110in}{1.435918in}}%
\pgfpathlineto{\pgfqpoint{1.178613in}{1.440886in}}%
\pgfpathlineto{\pgfqpoint{1.181018in}{1.458198in}}%
\pgfpathlineto{\pgfqpoint{1.184325in}{1.500678in}}%
\pgfpathlineto{\pgfqpoint{1.188684in}{1.589341in}}%
\pgfpathlineto{\pgfqpoint{1.194095in}{1.750304in}}%
\pgfpathlineto{\pgfqpoint{1.200558in}{2.014989in}}%
\pgfpathlineto{\pgfqpoint{1.208073in}{2.419408in}}%
\pgfpathlineto{\pgfqpoint{1.223103in}{3.265906in}}%
\pgfpathlineto{\pgfqpoint{1.229566in}{3.486671in}}%
\pgfpathlineto{\pgfqpoint{1.234827in}{3.599398in}}%
\pgfpathlineto{\pgfqpoint{1.238885in}{3.645893in}}%
\pgfpathlineto{\pgfqpoint{1.241590in}{3.657552in}}%
\pgfpathlineto{\pgfqpoint{1.243695in}{3.658953in}}%
\pgfpathlineto{\pgfqpoint{1.245498in}{3.657346in}}%
\pgfpathlineto{\pgfqpoint{1.248204in}{3.651251in}}%
\pgfpathlineto{\pgfqpoint{1.253013in}{3.633723in}}%
\pgfpathlineto{\pgfqpoint{1.268795in}{3.572786in}}%
\pgfpathlineto{\pgfqpoint{1.276010in}{3.555119in}}%
\pgfpathlineto{\pgfqpoint{1.284276in}{3.540840in}}%
\pgfpathlineto{\pgfqpoint{1.303665in}{3.513782in}}%
\pgfpathlineto{\pgfqpoint{1.347554in}{3.449355in}}%
\pgfpathlineto{\pgfqpoint{1.351913in}{3.437206in}}%
\pgfpathlineto{\pgfqpoint{1.356422in}{3.417358in}}%
\pgfpathlineto{\pgfqpoint{1.361682in}{3.383004in}}%
\pgfpathlineto{\pgfqpoint{1.368145in}{3.324556in}}%
\pgfpathlineto{\pgfqpoint{1.377164in}{3.220095in}}%
\pgfpathlineto{\pgfqpoint{1.402114in}{2.891113in}}%
\pgfpathlineto{\pgfqpoint{1.420601in}{2.669465in}}%
\pgfpathlineto{\pgfqpoint{1.438788in}{2.477914in}}%
\pgfpathlineto{\pgfqpoint{1.457125in}{2.307916in}}%
\pgfpathlineto{\pgfqpoint{1.475462in}{2.158205in}}%
\pgfpathlineto{\pgfqpoint{1.494250in}{2.022871in}}%
\pgfpathlineto{\pgfqpoint{1.513488in}{1.900280in}}%
\pgfpathlineto{\pgfqpoint{1.533178in}{1.788889in}}%
\pgfpathlineto{\pgfqpoint{1.553619in}{1.685888in}}%
\pgfpathlineto{\pgfqpoint{1.574662in}{1.591147in}}%
\pgfpathlineto{\pgfqpoint{1.596456in}{1.503163in}}%
\pgfpathlineto{\pgfqpoint{1.619001in}{1.421292in}}%
\pgfpathlineto{\pgfqpoint{1.642298in}{1.344962in}}%
\pgfpathlineto{\pgfqpoint{1.666347in}{1.273683in}}%
\pgfpathlineto{\pgfqpoint{1.691147in}{1.207051in}}%
\pgfpathlineto{\pgfqpoint{1.716548in}{1.145096in}}%
\pgfpathlineto{\pgfqpoint{1.742550in}{1.087487in}}%
\pgfpathlineto{\pgfqpoint{1.768853in}{1.034562in}}%
\pgfpathlineto{\pgfqpoint{1.795307in}{0.986259in}}%
\pgfpathlineto{\pgfqpoint{1.821910in}{0.942265in}}%
\pgfpathlineto{\pgfqpoint{1.848364in}{0.902768in}}%
\pgfpathlineto{\pgfqpoint{1.874667in}{0.867452in}}%
\pgfpathlineto{\pgfqpoint{1.900669in}{0.836226in}}%
\pgfpathlineto{\pgfqpoint{1.926221in}{0.808958in}}%
\pgfpathlineto{\pgfqpoint{1.951472in}{0.785211in}}%
\pgfpathlineto{\pgfqpoint{1.976422in}{0.764761in}}%
\pgfpathlineto{\pgfqpoint{2.001072in}{0.747404in}}%
\pgfpathlineto{\pgfqpoint{2.025421in}{0.732949in}}%
\pgfpathlineto{\pgfqpoint{2.049619in}{0.721150in}}%
\pgfpathlineto{\pgfqpoint{2.073668in}{0.711879in}}%
\pgfpathlineto{\pgfqpoint{2.097716in}{0.704978in}}%
\pgfpathlineto{\pgfqpoint{2.121915in}{0.700347in}}%
\pgfpathlineto{\pgfqpoint{2.146264in}{0.697947in}}%
\pgfpathlineto{\pgfqpoint{2.171064in}{0.697747in}}%
\pgfpathlineto{\pgfqpoint{2.196315in}{0.699779in}}%
\pgfpathlineto{\pgfqpoint{2.222167in}{0.704097in}}%
\pgfpathlineto{\pgfqpoint{2.248921in}{0.710836in}}%
\pgfpathlineto{\pgfqpoint{2.276727in}{0.720164in}}%
\pgfpathlineto{\pgfqpoint{2.305736in}{0.732273in}}%
\pgfpathlineto{\pgfqpoint{2.336097in}{0.747382in}}%
\pgfpathlineto{\pgfqpoint{2.368262in}{0.765908in}}%
\pgfpathlineto{\pgfqpoint{2.402531in}{0.788259in}}%
\pgfpathlineto{\pgfqpoint{2.439355in}{0.814987in}}%
\pgfpathlineto{\pgfqpoint{2.479336in}{0.846817in}}%
\pgfpathlineto{\pgfqpoint{2.523375in}{0.884784in}}%
\pgfpathlineto{\pgfqpoint{2.573275in}{0.930822in}}%
\pgfpathlineto{\pgfqpoint{2.632645in}{0.988748in}}%
\pgfpathlineto{\pgfqpoint{2.713207in}{1.070724in}}%
\pgfpathlineto{\pgfqpoint{2.890715in}{1.252154in}}%
\pgfpathlineto{\pgfqpoint{2.957450in}{1.316595in}}%
\pgfpathlineto{\pgfqpoint{3.015166in}{1.369295in}}%
\pgfpathlineto{\pgfqpoint{3.067772in}{1.414366in}}%
\pgfpathlineto{\pgfqpoint{3.116921in}{1.453590in}}%
\pgfpathlineto{\pgfqpoint{3.163665in}{1.488085in}}%
\pgfpathlineto{\pgfqpoint{3.208606in}{1.518511in}}%
\pgfpathlineto{\pgfqpoint{3.252044in}{1.545264in}}%
\pgfpathlineto{\pgfqpoint{3.294429in}{1.568785in}}%
\pgfpathlineto{\pgfqpoint{3.336063in}{1.589368in}}%
\pgfpathlineto{\pgfqpoint{3.377096in}{1.607196in}}%
\pgfpathlineto{\pgfqpoint{3.417828in}{1.622486in}}%
\pgfpathlineto{\pgfqpoint{3.458410in}{1.635360in}}%
\pgfpathlineto{\pgfqpoint{3.499142in}{1.645956in}}%
\pgfpathlineto{\pgfqpoint{3.540175in}{1.654333in}}%
\pgfpathlineto{\pgfqpoint{3.581658in}{1.660532in}}%
\pgfpathlineto{\pgfqpoint{3.623893in}{1.664592in}}%
\pgfpathlineto{\pgfqpoint{3.667181in}{1.666507in}}%
\pgfpathlineto{\pgfqpoint{3.711670in}{1.666240in}}%
\pgfpathlineto{\pgfqpoint{3.757813in}{1.663730in}}%
\pgfpathlineto{\pgfqpoint{3.806061in}{1.658865in}}%
\pgfpathlineto{\pgfqpoint{3.857164in}{1.651450in}}%
\pgfpathlineto{\pgfqpoint{3.911724in}{1.641259in}}%
\pgfpathlineto{\pgfqpoint{3.971244in}{1.627840in}}%
\pgfpathlineto{\pgfqpoint{4.037828in}{1.610494in}}%
\pgfpathlineto{\pgfqpoint{4.116136in}{1.587713in}}%
\pgfpathlineto{\pgfqpoint{4.221047in}{1.554697in}}%
\pgfpathlineto{\pgfqpoint{4.465891in}{1.476903in}}%
\pgfpathlineto{\pgfqpoint{4.556524in}{1.450957in}}%
\pgfpathlineto{\pgfqpoint{4.636785in}{1.430259in}}%
\pgfpathlineto{\pgfqpoint{4.711937in}{1.413155in}}%
\pgfpathlineto{\pgfqpoint{4.784082in}{1.398999in}}%
\pgfpathlineto{\pgfqpoint{4.854875in}{1.387369in}}%
\pgfpathlineto{\pgfqpoint{4.925217in}{1.378068in}}%
\pgfpathlineto{\pgfqpoint{4.995859in}{1.370972in}}%
\pgfpathlineto{\pgfqpoint{5.067855in}{1.365984in}}%
\pgfpathlineto{\pgfqpoint{5.141954in}{1.363094in}}%
\pgfpathlineto{\pgfqpoint{5.219360in}{1.362321in}}%
\pgfpathlineto{\pgfqpoint{5.301425in}{1.363749in}}%
\pgfpathlineto{\pgfqpoint{5.390555in}{1.367560in}}%
\pgfpathlineto{\pgfqpoint{5.490807in}{1.374128in}}%
\pgfpathlineto{\pgfqpoint{5.534545in}{1.377595in}}%
\pgfpathlineto{\pgfqpoint{5.534545in}{1.377595in}}%
\pgfusepath{stroke}%
\end{pgfscope}%
\begin{pgfscope}%
\pgfsetrectcap%
\pgfsetmiterjoin%
\pgfsetlinewidth{0.803000pt}%
\definecolor{currentstroke}{rgb}{0.000000,0.000000,0.000000}%
\pgfsetstrokecolor{currentstroke}%
\pgfsetdash{}{0pt}%
\pgfpathmoveto{\pgfqpoint{0.800000in}{0.528000in}}%
\pgfpathlineto{\pgfqpoint{0.800000in}{4.224000in}}%
\pgfusepath{stroke}%
\end{pgfscope}%
\begin{pgfscope}%
\pgfsetrectcap%
\pgfsetmiterjoin%
\pgfsetlinewidth{0.803000pt}%
\definecolor{currentstroke}{rgb}{0.000000,0.000000,0.000000}%
\pgfsetstrokecolor{currentstroke}%
\pgfsetdash{}{0pt}%
\pgfpathmoveto{\pgfqpoint{5.760000in}{0.528000in}}%
\pgfpathlineto{\pgfqpoint{5.760000in}{4.224000in}}%
\pgfusepath{stroke}%
\end{pgfscope}%
\begin{pgfscope}%
\pgfsetrectcap%
\pgfsetmiterjoin%
\pgfsetlinewidth{0.803000pt}%
\definecolor{currentstroke}{rgb}{0.000000,0.000000,0.000000}%
\pgfsetstrokecolor{currentstroke}%
\pgfsetdash{}{0pt}%
\pgfpathmoveto{\pgfqpoint{0.800000in}{0.528000in}}%
\pgfpathlineto{\pgfqpoint{5.760000in}{0.528000in}}%
\pgfusepath{stroke}%
\end{pgfscope}%
\begin{pgfscope}%
\pgfsetrectcap%
\pgfsetmiterjoin%
\pgfsetlinewidth{0.803000pt}%
\definecolor{currentstroke}{rgb}{0.000000,0.000000,0.000000}%
\pgfsetstrokecolor{currentstroke}%
\pgfsetdash{}{0pt}%
\pgfpathmoveto{\pgfqpoint{0.800000in}{4.224000in}}%
\pgfpathlineto{\pgfqpoint{5.760000in}{4.224000in}}%
\pgfusepath{stroke}%
\end{pgfscope}%
\begin{pgfscope}%
\pgfsetbuttcap%
\pgfsetmiterjoin%
\definecolor{currentfill}{rgb}{1.000000,1.000000,1.000000}%
\pgfsetfillcolor{currentfill}%
\pgfsetfillopacity{0.800000}%
\pgfsetlinewidth{1.003750pt}%
\definecolor{currentstroke}{rgb}{0.800000,0.800000,0.800000}%
\pgfsetstrokecolor{currentstroke}%
\pgfsetstrokeopacity{0.800000}%
\pgfsetdash{}{0pt}%
\pgfpathmoveto{\pgfqpoint{3.409453in}{3.475970in}}%
\pgfpathlineto{\pgfqpoint{5.662778in}{3.475970in}}%
\pgfpathquadraticcurveto{\pgfqpoint{5.690556in}{3.475970in}}{\pgfqpoint{5.690556in}{3.503748in}}%
\pgfpathlineto{\pgfqpoint{5.690556in}{4.126778in}}%
\pgfpathquadraticcurveto{\pgfqpoint{5.690556in}{4.154556in}}{\pgfqpoint{5.662778in}{4.154556in}}%
\pgfpathlineto{\pgfqpoint{3.409453in}{4.154556in}}%
\pgfpathquadraticcurveto{\pgfqpoint{3.381676in}{4.154556in}}{\pgfqpoint{3.381676in}{4.126778in}}%
\pgfpathlineto{\pgfqpoint{3.381676in}{3.503748in}}%
\pgfpathquadraticcurveto{\pgfqpoint{3.381676in}{3.475970in}}{\pgfqpoint{3.409453in}{3.475970in}}%
\pgfpathclose%
\pgfusepath{stroke,fill}%
\end{pgfscope}%
\begin{pgfscope}%
\pgfsetrectcap%
\pgfsetroundjoin%
\pgfsetlinewidth{1.505625pt}%
\definecolor{currentstroke}{rgb}{0.121569,0.466667,0.705882}%
\pgfsetstrokecolor{currentstroke}%
\pgfsetdash{}{0pt}%
\pgfpathmoveto{\pgfqpoint{3.437231in}{3.993867in}}%
\pgfpathlineto{\pgfqpoint{3.715009in}{3.993867in}}%
\pgfusepath{stroke}%
\end{pgfscope}%
\begin{pgfscope}%
\definecolor{textcolor}{rgb}{0.000000,0.000000,0.000000}%
\pgfsetstrokecolor{textcolor}%
\pgfsetfillcolor{textcolor}%
\pgftext[x=3.826120in,y=3.945256in,left,base]{\color{textcolor}\rmfamily\fontsize{10.000000}{12.000000}\selectfont Модель при \(\displaystyle \dot{\delta}_{{в}_{max}}=15 \frac{град.}{сек.}\)}%
\end{pgfscope}%
\begin{pgfscope}%
\pgfsetbuttcap%
\pgfsetroundjoin%
\pgfsetlinewidth{1.505625pt}%
\definecolor{currentstroke}{rgb}{1.000000,0.498039,0.054902}%
\pgfsetstrokecolor{currentstroke}%
\pgfsetdash{{9.600000pt}{2.400000pt}{1.500000pt}{2.400000pt}}{0.000000pt}%
\pgfpathmoveto{\pgfqpoint{3.437231in}{3.675407in}}%
\pgfpathlineto{\pgfqpoint{3.715009in}{3.675407in}}%
\pgfusepath{stroke}%
\end{pgfscope}%
\begin{pgfscope}%
\definecolor{textcolor}{rgb}{0.000000,0.000000,0.000000}%
\pgfsetstrokecolor{textcolor}%
\pgfsetfillcolor{textcolor}%
\pgftext[x=3.826120in,y=3.626796in,left,base]{\color{textcolor}\rmfamily\fontsize{10.000000}{12.000000}\selectfont Модель при \(\displaystyle \dot{\delta}_{{в}_{max}}=60 \frac{град.}{сек.}\)}%
\end{pgfscope}%
\end{pgfpicture}%
\makeatother%
\endgroup%
}
    \caption{Изменение угловой скорости для различных $\dot{\delta}_\text{в max}$}
    \label{fig:model_DD_omega_z}
    \end{minipage}
    \hfill
    \begin{minipage}{0.48\textwidth}
    \centering
    \resizebox{1.1\linewidth}{!}{%% Creator: Matplotlib, PGF backend
%%
%% To include the figure in your LaTeX document, write
%%   \input{<filename>.pgf}
%%
%% Make sure the required packages are loaded in your preamble
%%   \usepackage{pgf}
%%
%% Figures using additional raster images can only be included by \input if
%% they are in the same directory as the main LaTeX file. For loading figures
%% from other directories you can use the `import` package
%%   \usepackage{import}
%%
%% and then include the figures with
%%   \import{<path to file>}{<filename>.pgf}
%%
%% Matplotlib used the following preamble
%%   \usepackage[warn]{mathtext}
%%   \usepackage[T2A]{fontenc}
%%   \usepackage[utf8]{inputenc}
%%   \usepackage[english,russian]{babel}
%%
\begingroup%
\makeatletter%
\begin{pgfpicture}%
\pgfpathrectangle{\pgfpointorigin}{\pgfqpoint{6.400000in}{4.800000in}}%
\pgfusepath{use as bounding box, clip}%
\begin{pgfscope}%
\pgfsetbuttcap%
\pgfsetmiterjoin%
\definecolor{currentfill}{rgb}{1.000000,1.000000,1.000000}%
\pgfsetfillcolor{currentfill}%
\pgfsetlinewidth{0.000000pt}%
\definecolor{currentstroke}{rgb}{1.000000,1.000000,1.000000}%
\pgfsetstrokecolor{currentstroke}%
\pgfsetdash{}{0pt}%
\pgfpathmoveto{\pgfqpoint{0.000000in}{0.000000in}}%
\pgfpathlineto{\pgfqpoint{6.400000in}{0.000000in}}%
\pgfpathlineto{\pgfqpoint{6.400000in}{4.800000in}}%
\pgfpathlineto{\pgfqpoint{0.000000in}{4.800000in}}%
\pgfpathclose%
\pgfusepath{fill}%
\end{pgfscope}%
\begin{pgfscope}%
\pgfsetbuttcap%
\pgfsetmiterjoin%
\definecolor{currentfill}{rgb}{1.000000,1.000000,1.000000}%
\pgfsetfillcolor{currentfill}%
\pgfsetlinewidth{0.000000pt}%
\definecolor{currentstroke}{rgb}{0.000000,0.000000,0.000000}%
\pgfsetstrokecolor{currentstroke}%
\pgfsetstrokeopacity{0.000000}%
\pgfsetdash{}{0pt}%
\pgfpathmoveto{\pgfqpoint{0.800000in}{0.528000in}}%
\pgfpathlineto{\pgfqpoint{5.760000in}{0.528000in}}%
\pgfpathlineto{\pgfqpoint{5.760000in}{4.224000in}}%
\pgfpathlineto{\pgfqpoint{0.800000in}{4.224000in}}%
\pgfpathclose%
\pgfusepath{fill}%
\end{pgfscope}%
\begin{pgfscope}%
\pgfpathrectangle{\pgfqpoint{0.800000in}{0.528000in}}{\pgfqpoint{4.960000in}{3.696000in}}%
\pgfusepath{clip}%
\pgfsetrectcap%
\pgfsetroundjoin%
\pgfsetlinewidth{0.803000pt}%
\definecolor{currentstroke}{rgb}{0.690196,0.690196,0.690196}%
\pgfsetstrokecolor{currentstroke}%
\pgfsetdash{}{0pt}%
\pgfpathmoveto{\pgfqpoint{1.025455in}{0.528000in}}%
\pgfpathlineto{\pgfqpoint{1.025455in}{4.224000in}}%
\pgfusepath{stroke}%
\end{pgfscope}%
\begin{pgfscope}%
\pgfsetbuttcap%
\pgfsetroundjoin%
\definecolor{currentfill}{rgb}{0.000000,0.000000,0.000000}%
\pgfsetfillcolor{currentfill}%
\pgfsetlinewidth{0.803000pt}%
\definecolor{currentstroke}{rgb}{0.000000,0.000000,0.000000}%
\pgfsetstrokecolor{currentstroke}%
\pgfsetdash{}{0pt}%
\pgfsys@defobject{currentmarker}{\pgfqpoint{0.000000in}{-0.048611in}}{\pgfqpoint{0.000000in}{0.000000in}}{%
\pgfpathmoveto{\pgfqpoint{0.000000in}{0.000000in}}%
\pgfpathlineto{\pgfqpoint{0.000000in}{-0.048611in}}%
\pgfusepath{stroke,fill}%
}%
\begin{pgfscope}%
\pgfsys@transformshift{1.025455in}{0.528000in}%
\pgfsys@useobject{currentmarker}{}%
\end{pgfscope}%
\end{pgfscope}%
\begin{pgfscope}%
\definecolor{textcolor}{rgb}{0.000000,0.000000,0.000000}%
\pgfsetstrokecolor{textcolor}%
\pgfsetfillcolor{textcolor}%
\pgftext[x=1.025455in,y=0.430778in,,top]{\color{textcolor}\rmfamily\fontsize{10.000000}{12.000000}\selectfont \(\displaystyle {0}\)}%
\end{pgfscope}%
\begin{pgfscope}%
\pgfpathrectangle{\pgfqpoint{0.800000in}{0.528000in}}{\pgfqpoint{4.960000in}{3.696000in}}%
\pgfusepath{clip}%
\pgfsetrectcap%
\pgfsetroundjoin%
\pgfsetlinewidth{0.803000pt}%
\definecolor{currentstroke}{rgb}{0.690196,0.690196,0.690196}%
\pgfsetstrokecolor{currentstroke}%
\pgfsetdash{}{0pt}%
\pgfpathmoveto{\pgfqpoint{1.776970in}{0.528000in}}%
\pgfpathlineto{\pgfqpoint{1.776970in}{4.224000in}}%
\pgfusepath{stroke}%
\end{pgfscope}%
\begin{pgfscope}%
\pgfsetbuttcap%
\pgfsetroundjoin%
\definecolor{currentfill}{rgb}{0.000000,0.000000,0.000000}%
\pgfsetfillcolor{currentfill}%
\pgfsetlinewidth{0.803000pt}%
\definecolor{currentstroke}{rgb}{0.000000,0.000000,0.000000}%
\pgfsetstrokecolor{currentstroke}%
\pgfsetdash{}{0pt}%
\pgfsys@defobject{currentmarker}{\pgfqpoint{0.000000in}{-0.048611in}}{\pgfqpoint{0.000000in}{0.000000in}}{%
\pgfpathmoveto{\pgfqpoint{0.000000in}{0.000000in}}%
\pgfpathlineto{\pgfqpoint{0.000000in}{-0.048611in}}%
\pgfusepath{stroke,fill}%
}%
\begin{pgfscope}%
\pgfsys@transformshift{1.776970in}{0.528000in}%
\pgfsys@useobject{currentmarker}{}%
\end{pgfscope}%
\end{pgfscope}%
\begin{pgfscope}%
\definecolor{textcolor}{rgb}{0.000000,0.000000,0.000000}%
\pgfsetstrokecolor{textcolor}%
\pgfsetfillcolor{textcolor}%
\pgftext[x=1.776970in,y=0.430778in,,top]{\color{textcolor}\rmfamily\fontsize{10.000000}{12.000000}\selectfont \(\displaystyle {5}\)}%
\end{pgfscope}%
\begin{pgfscope}%
\pgfpathrectangle{\pgfqpoint{0.800000in}{0.528000in}}{\pgfqpoint{4.960000in}{3.696000in}}%
\pgfusepath{clip}%
\pgfsetrectcap%
\pgfsetroundjoin%
\pgfsetlinewidth{0.803000pt}%
\definecolor{currentstroke}{rgb}{0.690196,0.690196,0.690196}%
\pgfsetstrokecolor{currentstroke}%
\pgfsetdash{}{0pt}%
\pgfpathmoveto{\pgfqpoint{2.528485in}{0.528000in}}%
\pgfpathlineto{\pgfqpoint{2.528485in}{4.224000in}}%
\pgfusepath{stroke}%
\end{pgfscope}%
\begin{pgfscope}%
\pgfsetbuttcap%
\pgfsetroundjoin%
\definecolor{currentfill}{rgb}{0.000000,0.000000,0.000000}%
\pgfsetfillcolor{currentfill}%
\pgfsetlinewidth{0.803000pt}%
\definecolor{currentstroke}{rgb}{0.000000,0.000000,0.000000}%
\pgfsetstrokecolor{currentstroke}%
\pgfsetdash{}{0pt}%
\pgfsys@defobject{currentmarker}{\pgfqpoint{0.000000in}{-0.048611in}}{\pgfqpoint{0.000000in}{0.000000in}}{%
\pgfpathmoveto{\pgfqpoint{0.000000in}{0.000000in}}%
\pgfpathlineto{\pgfqpoint{0.000000in}{-0.048611in}}%
\pgfusepath{stroke,fill}%
}%
\begin{pgfscope}%
\pgfsys@transformshift{2.528485in}{0.528000in}%
\pgfsys@useobject{currentmarker}{}%
\end{pgfscope}%
\end{pgfscope}%
\begin{pgfscope}%
\definecolor{textcolor}{rgb}{0.000000,0.000000,0.000000}%
\pgfsetstrokecolor{textcolor}%
\pgfsetfillcolor{textcolor}%
\pgftext[x=2.528485in,y=0.430778in,,top]{\color{textcolor}\rmfamily\fontsize{10.000000}{12.000000}\selectfont \(\displaystyle {10}\)}%
\end{pgfscope}%
\begin{pgfscope}%
\pgfpathrectangle{\pgfqpoint{0.800000in}{0.528000in}}{\pgfqpoint{4.960000in}{3.696000in}}%
\pgfusepath{clip}%
\pgfsetrectcap%
\pgfsetroundjoin%
\pgfsetlinewidth{0.803000pt}%
\definecolor{currentstroke}{rgb}{0.690196,0.690196,0.690196}%
\pgfsetstrokecolor{currentstroke}%
\pgfsetdash{}{0pt}%
\pgfpathmoveto{\pgfqpoint{3.280000in}{0.528000in}}%
\pgfpathlineto{\pgfqpoint{3.280000in}{4.224000in}}%
\pgfusepath{stroke}%
\end{pgfscope}%
\begin{pgfscope}%
\pgfsetbuttcap%
\pgfsetroundjoin%
\definecolor{currentfill}{rgb}{0.000000,0.000000,0.000000}%
\pgfsetfillcolor{currentfill}%
\pgfsetlinewidth{0.803000pt}%
\definecolor{currentstroke}{rgb}{0.000000,0.000000,0.000000}%
\pgfsetstrokecolor{currentstroke}%
\pgfsetdash{}{0pt}%
\pgfsys@defobject{currentmarker}{\pgfqpoint{0.000000in}{-0.048611in}}{\pgfqpoint{0.000000in}{0.000000in}}{%
\pgfpathmoveto{\pgfqpoint{0.000000in}{0.000000in}}%
\pgfpathlineto{\pgfqpoint{0.000000in}{-0.048611in}}%
\pgfusepath{stroke,fill}%
}%
\begin{pgfscope}%
\pgfsys@transformshift{3.280000in}{0.528000in}%
\pgfsys@useobject{currentmarker}{}%
\end{pgfscope}%
\end{pgfscope}%
\begin{pgfscope}%
\definecolor{textcolor}{rgb}{0.000000,0.000000,0.000000}%
\pgfsetstrokecolor{textcolor}%
\pgfsetfillcolor{textcolor}%
\pgftext[x=3.280000in,y=0.430778in,,top]{\color{textcolor}\rmfamily\fontsize{10.000000}{12.000000}\selectfont \(\displaystyle {15}\)}%
\end{pgfscope}%
\begin{pgfscope}%
\pgfpathrectangle{\pgfqpoint{0.800000in}{0.528000in}}{\pgfqpoint{4.960000in}{3.696000in}}%
\pgfusepath{clip}%
\pgfsetrectcap%
\pgfsetroundjoin%
\pgfsetlinewidth{0.803000pt}%
\definecolor{currentstroke}{rgb}{0.690196,0.690196,0.690196}%
\pgfsetstrokecolor{currentstroke}%
\pgfsetdash{}{0pt}%
\pgfpathmoveto{\pgfqpoint{4.031515in}{0.528000in}}%
\pgfpathlineto{\pgfqpoint{4.031515in}{4.224000in}}%
\pgfusepath{stroke}%
\end{pgfscope}%
\begin{pgfscope}%
\pgfsetbuttcap%
\pgfsetroundjoin%
\definecolor{currentfill}{rgb}{0.000000,0.000000,0.000000}%
\pgfsetfillcolor{currentfill}%
\pgfsetlinewidth{0.803000pt}%
\definecolor{currentstroke}{rgb}{0.000000,0.000000,0.000000}%
\pgfsetstrokecolor{currentstroke}%
\pgfsetdash{}{0pt}%
\pgfsys@defobject{currentmarker}{\pgfqpoint{0.000000in}{-0.048611in}}{\pgfqpoint{0.000000in}{0.000000in}}{%
\pgfpathmoveto{\pgfqpoint{0.000000in}{0.000000in}}%
\pgfpathlineto{\pgfqpoint{0.000000in}{-0.048611in}}%
\pgfusepath{stroke,fill}%
}%
\begin{pgfscope}%
\pgfsys@transformshift{4.031515in}{0.528000in}%
\pgfsys@useobject{currentmarker}{}%
\end{pgfscope}%
\end{pgfscope}%
\begin{pgfscope}%
\definecolor{textcolor}{rgb}{0.000000,0.000000,0.000000}%
\pgfsetstrokecolor{textcolor}%
\pgfsetfillcolor{textcolor}%
\pgftext[x=4.031515in,y=0.430778in,,top]{\color{textcolor}\rmfamily\fontsize{10.000000}{12.000000}\selectfont \(\displaystyle {20}\)}%
\end{pgfscope}%
\begin{pgfscope}%
\pgfpathrectangle{\pgfqpoint{0.800000in}{0.528000in}}{\pgfqpoint{4.960000in}{3.696000in}}%
\pgfusepath{clip}%
\pgfsetrectcap%
\pgfsetroundjoin%
\pgfsetlinewidth{0.803000pt}%
\definecolor{currentstroke}{rgb}{0.690196,0.690196,0.690196}%
\pgfsetstrokecolor{currentstroke}%
\pgfsetdash{}{0pt}%
\pgfpathmoveto{\pgfqpoint{4.783030in}{0.528000in}}%
\pgfpathlineto{\pgfqpoint{4.783030in}{4.224000in}}%
\pgfusepath{stroke}%
\end{pgfscope}%
\begin{pgfscope}%
\pgfsetbuttcap%
\pgfsetroundjoin%
\definecolor{currentfill}{rgb}{0.000000,0.000000,0.000000}%
\pgfsetfillcolor{currentfill}%
\pgfsetlinewidth{0.803000pt}%
\definecolor{currentstroke}{rgb}{0.000000,0.000000,0.000000}%
\pgfsetstrokecolor{currentstroke}%
\pgfsetdash{}{0pt}%
\pgfsys@defobject{currentmarker}{\pgfqpoint{0.000000in}{-0.048611in}}{\pgfqpoint{0.000000in}{0.000000in}}{%
\pgfpathmoveto{\pgfqpoint{0.000000in}{0.000000in}}%
\pgfpathlineto{\pgfqpoint{0.000000in}{-0.048611in}}%
\pgfusepath{stroke,fill}%
}%
\begin{pgfscope}%
\pgfsys@transformshift{4.783030in}{0.528000in}%
\pgfsys@useobject{currentmarker}{}%
\end{pgfscope}%
\end{pgfscope}%
\begin{pgfscope}%
\definecolor{textcolor}{rgb}{0.000000,0.000000,0.000000}%
\pgfsetstrokecolor{textcolor}%
\pgfsetfillcolor{textcolor}%
\pgftext[x=4.783030in,y=0.430778in,,top]{\color{textcolor}\rmfamily\fontsize{10.000000}{12.000000}\selectfont \(\displaystyle {25}\)}%
\end{pgfscope}%
\begin{pgfscope}%
\pgfpathrectangle{\pgfqpoint{0.800000in}{0.528000in}}{\pgfqpoint{4.960000in}{3.696000in}}%
\pgfusepath{clip}%
\pgfsetrectcap%
\pgfsetroundjoin%
\pgfsetlinewidth{0.803000pt}%
\definecolor{currentstroke}{rgb}{0.690196,0.690196,0.690196}%
\pgfsetstrokecolor{currentstroke}%
\pgfsetdash{}{0pt}%
\pgfpathmoveto{\pgfqpoint{5.534545in}{0.528000in}}%
\pgfpathlineto{\pgfqpoint{5.534545in}{4.224000in}}%
\pgfusepath{stroke}%
\end{pgfscope}%
\begin{pgfscope}%
\pgfsetbuttcap%
\pgfsetroundjoin%
\definecolor{currentfill}{rgb}{0.000000,0.000000,0.000000}%
\pgfsetfillcolor{currentfill}%
\pgfsetlinewidth{0.803000pt}%
\definecolor{currentstroke}{rgb}{0.000000,0.000000,0.000000}%
\pgfsetstrokecolor{currentstroke}%
\pgfsetdash{}{0pt}%
\pgfsys@defobject{currentmarker}{\pgfqpoint{0.000000in}{-0.048611in}}{\pgfqpoint{0.000000in}{0.000000in}}{%
\pgfpathmoveto{\pgfqpoint{0.000000in}{0.000000in}}%
\pgfpathlineto{\pgfqpoint{0.000000in}{-0.048611in}}%
\pgfusepath{stroke,fill}%
}%
\begin{pgfscope}%
\pgfsys@transformshift{5.534545in}{0.528000in}%
\pgfsys@useobject{currentmarker}{}%
\end{pgfscope}%
\end{pgfscope}%
\begin{pgfscope}%
\definecolor{textcolor}{rgb}{0.000000,0.000000,0.000000}%
\pgfsetstrokecolor{textcolor}%
\pgfsetfillcolor{textcolor}%
\pgftext[x=5.534545in,y=0.430778in,,top]{\color{textcolor}\rmfamily\fontsize{10.000000}{12.000000}\selectfont \(\displaystyle {30}\)}%
\end{pgfscope}%
\begin{pgfscope}%
\definecolor{textcolor}{rgb}{0.000000,0.000000,0.000000}%
\pgfsetstrokecolor{textcolor}%
\pgfsetfillcolor{textcolor}%
\pgftext[x=3.280000in,y=0.251796in,,top]{\color{textcolor}\rmfamily\fontsize{10.000000}{12.000000}\selectfont \(\displaystyle t,\ с\)}%
\end{pgfscope}%
\begin{pgfscope}%
\pgfpathrectangle{\pgfqpoint{0.800000in}{0.528000in}}{\pgfqpoint{4.960000in}{3.696000in}}%
\pgfusepath{clip}%
\pgfsetrectcap%
\pgfsetroundjoin%
\pgfsetlinewidth{0.803000pt}%
\definecolor{currentstroke}{rgb}{0.690196,0.690196,0.690196}%
\pgfsetstrokecolor{currentstroke}%
\pgfsetdash{}{0pt}%
\pgfpathmoveto{\pgfqpoint{0.800000in}{0.668490in}}%
\pgfpathlineto{\pgfqpoint{5.760000in}{0.668490in}}%
\pgfusepath{stroke}%
\end{pgfscope}%
\begin{pgfscope}%
\pgfsetbuttcap%
\pgfsetroundjoin%
\definecolor{currentfill}{rgb}{0.000000,0.000000,0.000000}%
\pgfsetfillcolor{currentfill}%
\pgfsetlinewidth{0.803000pt}%
\definecolor{currentstroke}{rgb}{0.000000,0.000000,0.000000}%
\pgfsetstrokecolor{currentstroke}%
\pgfsetdash{}{0pt}%
\pgfsys@defobject{currentmarker}{\pgfqpoint{-0.048611in}{0.000000in}}{\pgfqpoint{-0.000000in}{0.000000in}}{%
\pgfpathmoveto{\pgfqpoint{-0.000000in}{0.000000in}}%
\pgfpathlineto{\pgfqpoint{-0.048611in}{0.000000in}}%
\pgfusepath{stroke,fill}%
}%
\begin{pgfscope}%
\pgfsys@transformshift{0.800000in}{0.668490in}%
\pgfsys@useobject{currentmarker}{}%
\end{pgfscope}%
\end{pgfscope}%
\begin{pgfscope}%
\definecolor{textcolor}{rgb}{0.000000,0.000000,0.000000}%
\pgfsetstrokecolor{textcolor}%
\pgfsetfillcolor{textcolor}%
\pgftext[x=0.278394in, y=0.620276in, left, base]{\color{textcolor}\rmfamily\fontsize{10.000000}{12.000000}\selectfont \(\displaystyle {\ensuremath{-}0.050}\)}%
\end{pgfscope}%
\begin{pgfscope}%
\pgfpathrectangle{\pgfqpoint{0.800000in}{0.528000in}}{\pgfqpoint{4.960000in}{3.696000in}}%
\pgfusepath{clip}%
\pgfsetrectcap%
\pgfsetroundjoin%
\pgfsetlinewidth{0.803000pt}%
\definecolor{currentstroke}{rgb}{0.690196,0.690196,0.690196}%
\pgfsetstrokecolor{currentstroke}%
\pgfsetdash{}{0pt}%
\pgfpathmoveto{\pgfqpoint{0.800000in}{1.091440in}}%
\pgfpathlineto{\pgfqpoint{5.760000in}{1.091440in}}%
\pgfusepath{stroke}%
\end{pgfscope}%
\begin{pgfscope}%
\pgfsetbuttcap%
\pgfsetroundjoin%
\definecolor{currentfill}{rgb}{0.000000,0.000000,0.000000}%
\pgfsetfillcolor{currentfill}%
\pgfsetlinewidth{0.803000pt}%
\definecolor{currentstroke}{rgb}{0.000000,0.000000,0.000000}%
\pgfsetstrokecolor{currentstroke}%
\pgfsetdash{}{0pt}%
\pgfsys@defobject{currentmarker}{\pgfqpoint{-0.048611in}{0.000000in}}{\pgfqpoint{-0.000000in}{0.000000in}}{%
\pgfpathmoveto{\pgfqpoint{-0.000000in}{0.000000in}}%
\pgfpathlineto{\pgfqpoint{-0.048611in}{0.000000in}}%
\pgfusepath{stroke,fill}%
}%
\begin{pgfscope}%
\pgfsys@transformshift{0.800000in}{1.091440in}%
\pgfsys@useobject{currentmarker}{}%
\end{pgfscope}%
\end{pgfscope}%
\begin{pgfscope}%
\definecolor{textcolor}{rgb}{0.000000,0.000000,0.000000}%
\pgfsetstrokecolor{textcolor}%
\pgfsetfillcolor{textcolor}%
\pgftext[x=0.278394in, y=1.043226in, left, base]{\color{textcolor}\rmfamily\fontsize{10.000000}{12.000000}\selectfont \(\displaystyle {\ensuremath{-}0.025}\)}%
\end{pgfscope}%
\begin{pgfscope}%
\pgfpathrectangle{\pgfqpoint{0.800000in}{0.528000in}}{\pgfqpoint{4.960000in}{3.696000in}}%
\pgfusepath{clip}%
\pgfsetrectcap%
\pgfsetroundjoin%
\pgfsetlinewidth{0.803000pt}%
\definecolor{currentstroke}{rgb}{0.690196,0.690196,0.690196}%
\pgfsetstrokecolor{currentstroke}%
\pgfsetdash{}{0pt}%
\pgfpathmoveto{\pgfqpoint{0.800000in}{1.514389in}}%
\pgfpathlineto{\pgfqpoint{5.760000in}{1.514389in}}%
\pgfusepath{stroke}%
\end{pgfscope}%
\begin{pgfscope}%
\pgfsetbuttcap%
\pgfsetroundjoin%
\definecolor{currentfill}{rgb}{0.000000,0.000000,0.000000}%
\pgfsetfillcolor{currentfill}%
\pgfsetlinewidth{0.803000pt}%
\definecolor{currentstroke}{rgb}{0.000000,0.000000,0.000000}%
\pgfsetstrokecolor{currentstroke}%
\pgfsetdash{}{0pt}%
\pgfsys@defobject{currentmarker}{\pgfqpoint{-0.048611in}{0.000000in}}{\pgfqpoint{-0.000000in}{0.000000in}}{%
\pgfpathmoveto{\pgfqpoint{-0.000000in}{0.000000in}}%
\pgfpathlineto{\pgfqpoint{-0.048611in}{0.000000in}}%
\pgfusepath{stroke,fill}%
}%
\begin{pgfscope}%
\pgfsys@transformshift{0.800000in}{1.514389in}%
\pgfsys@useobject{currentmarker}{}%
\end{pgfscope}%
\end{pgfscope}%
\begin{pgfscope}%
\definecolor{textcolor}{rgb}{0.000000,0.000000,0.000000}%
\pgfsetstrokecolor{textcolor}%
\pgfsetfillcolor{textcolor}%
\pgftext[x=0.386419in, y=1.466176in, left, base]{\color{textcolor}\rmfamily\fontsize{10.000000}{12.000000}\selectfont \(\displaystyle {0.000}\)}%
\end{pgfscope}%
\begin{pgfscope}%
\pgfpathrectangle{\pgfqpoint{0.800000in}{0.528000in}}{\pgfqpoint{4.960000in}{3.696000in}}%
\pgfusepath{clip}%
\pgfsetrectcap%
\pgfsetroundjoin%
\pgfsetlinewidth{0.803000pt}%
\definecolor{currentstroke}{rgb}{0.690196,0.690196,0.690196}%
\pgfsetstrokecolor{currentstroke}%
\pgfsetdash{}{0pt}%
\pgfpathmoveto{\pgfqpoint{0.800000in}{1.937339in}}%
\pgfpathlineto{\pgfqpoint{5.760000in}{1.937339in}}%
\pgfusepath{stroke}%
\end{pgfscope}%
\begin{pgfscope}%
\pgfsetbuttcap%
\pgfsetroundjoin%
\definecolor{currentfill}{rgb}{0.000000,0.000000,0.000000}%
\pgfsetfillcolor{currentfill}%
\pgfsetlinewidth{0.803000pt}%
\definecolor{currentstroke}{rgb}{0.000000,0.000000,0.000000}%
\pgfsetstrokecolor{currentstroke}%
\pgfsetdash{}{0pt}%
\pgfsys@defobject{currentmarker}{\pgfqpoint{-0.048611in}{0.000000in}}{\pgfqpoint{-0.000000in}{0.000000in}}{%
\pgfpathmoveto{\pgfqpoint{-0.000000in}{0.000000in}}%
\pgfpathlineto{\pgfqpoint{-0.048611in}{0.000000in}}%
\pgfusepath{stroke,fill}%
}%
\begin{pgfscope}%
\pgfsys@transformshift{0.800000in}{1.937339in}%
\pgfsys@useobject{currentmarker}{}%
\end{pgfscope}%
\end{pgfscope}%
\begin{pgfscope}%
\definecolor{textcolor}{rgb}{0.000000,0.000000,0.000000}%
\pgfsetstrokecolor{textcolor}%
\pgfsetfillcolor{textcolor}%
\pgftext[x=0.386419in, y=1.889125in, left, base]{\color{textcolor}\rmfamily\fontsize{10.000000}{12.000000}\selectfont \(\displaystyle {0.025}\)}%
\end{pgfscope}%
\begin{pgfscope}%
\pgfpathrectangle{\pgfqpoint{0.800000in}{0.528000in}}{\pgfqpoint{4.960000in}{3.696000in}}%
\pgfusepath{clip}%
\pgfsetrectcap%
\pgfsetroundjoin%
\pgfsetlinewidth{0.803000pt}%
\definecolor{currentstroke}{rgb}{0.690196,0.690196,0.690196}%
\pgfsetstrokecolor{currentstroke}%
\pgfsetdash{}{0pt}%
\pgfpathmoveto{\pgfqpoint{0.800000in}{2.360288in}}%
\pgfpathlineto{\pgfqpoint{5.760000in}{2.360288in}}%
\pgfusepath{stroke}%
\end{pgfscope}%
\begin{pgfscope}%
\pgfsetbuttcap%
\pgfsetroundjoin%
\definecolor{currentfill}{rgb}{0.000000,0.000000,0.000000}%
\pgfsetfillcolor{currentfill}%
\pgfsetlinewidth{0.803000pt}%
\definecolor{currentstroke}{rgb}{0.000000,0.000000,0.000000}%
\pgfsetstrokecolor{currentstroke}%
\pgfsetdash{}{0pt}%
\pgfsys@defobject{currentmarker}{\pgfqpoint{-0.048611in}{0.000000in}}{\pgfqpoint{-0.000000in}{0.000000in}}{%
\pgfpathmoveto{\pgfqpoint{-0.000000in}{0.000000in}}%
\pgfpathlineto{\pgfqpoint{-0.048611in}{0.000000in}}%
\pgfusepath{stroke,fill}%
}%
\begin{pgfscope}%
\pgfsys@transformshift{0.800000in}{2.360288in}%
\pgfsys@useobject{currentmarker}{}%
\end{pgfscope}%
\end{pgfscope}%
\begin{pgfscope}%
\definecolor{textcolor}{rgb}{0.000000,0.000000,0.000000}%
\pgfsetstrokecolor{textcolor}%
\pgfsetfillcolor{textcolor}%
\pgftext[x=0.386419in, y=2.312075in, left, base]{\color{textcolor}\rmfamily\fontsize{10.000000}{12.000000}\selectfont \(\displaystyle {0.050}\)}%
\end{pgfscope}%
\begin{pgfscope}%
\pgfpathrectangle{\pgfqpoint{0.800000in}{0.528000in}}{\pgfqpoint{4.960000in}{3.696000in}}%
\pgfusepath{clip}%
\pgfsetrectcap%
\pgfsetroundjoin%
\pgfsetlinewidth{0.803000pt}%
\definecolor{currentstroke}{rgb}{0.690196,0.690196,0.690196}%
\pgfsetstrokecolor{currentstroke}%
\pgfsetdash{}{0pt}%
\pgfpathmoveto{\pgfqpoint{0.800000in}{2.783238in}}%
\pgfpathlineto{\pgfqpoint{5.760000in}{2.783238in}}%
\pgfusepath{stroke}%
\end{pgfscope}%
\begin{pgfscope}%
\pgfsetbuttcap%
\pgfsetroundjoin%
\definecolor{currentfill}{rgb}{0.000000,0.000000,0.000000}%
\pgfsetfillcolor{currentfill}%
\pgfsetlinewidth{0.803000pt}%
\definecolor{currentstroke}{rgb}{0.000000,0.000000,0.000000}%
\pgfsetstrokecolor{currentstroke}%
\pgfsetdash{}{0pt}%
\pgfsys@defobject{currentmarker}{\pgfqpoint{-0.048611in}{0.000000in}}{\pgfqpoint{-0.000000in}{0.000000in}}{%
\pgfpathmoveto{\pgfqpoint{-0.000000in}{0.000000in}}%
\pgfpathlineto{\pgfqpoint{-0.048611in}{0.000000in}}%
\pgfusepath{stroke,fill}%
}%
\begin{pgfscope}%
\pgfsys@transformshift{0.800000in}{2.783238in}%
\pgfsys@useobject{currentmarker}{}%
\end{pgfscope}%
\end{pgfscope}%
\begin{pgfscope}%
\definecolor{textcolor}{rgb}{0.000000,0.000000,0.000000}%
\pgfsetstrokecolor{textcolor}%
\pgfsetfillcolor{textcolor}%
\pgftext[x=0.386419in, y=2.735025in, left, base]{\color{textcolor}\rmfamily\fontsize{10.000000}{12.000000}\selectfont \(\displaystyle {0.075}\)}%
\end{pgfscope}%
\begin{pgfscope}%
\pgfpathrectangle{\pgfqpoint{0.800000in}{0.528000in}}{\pgfqpoint{4.960000in}{3.696000in}}%
\pgfusepath{clip}%
\pgfsetrectcap%
\pgfsetroundjoin%
\pgfsetlinewidth{0.803000pt}%
\definecolor{currentstroke}{rgb}{0.690196,0.690196,0.690196}%
\pgfsetstrokecolor{currentstroke}%
\pgfsetdash{}{0pt}%
\pgfpathmoveto{\pgfqpoint{0.800000in}{3.206188in}}%
\pgfpathlineto{\pgfqpoint{5.760000in}{3.206188in}}%
\pgfusepath{stroke}%
\end{pgfscope}%
\begin{pgfscope}%
\pgfsetbuttcap%
\pgfsetroundjoin%
\definecolor{currentfill}{rgb}{0.000000,0.000000,0.000000}%
\pgfsetfillcolor{currentfill}%
\pgfsetlinewidth{0.803000pt}%
\definecolor{currentstroke}{rgb}{0.000000,0.000000,0.000000}%
\pgfsetstrokecolor{currentstroke}%
\pgfsetdash{}{0pt}%
\pgfsys@defobject{currentmarker}{\pgfqpoint{-0.048611in}{0.000000in}}{\pgfqpoint{-0.000000in}{0.000000in}}{%
\pgfpathmoveto{\pgfqpoint{-0.000000in}{0.000000in}}%
\pgfpathlineto{\pgfqpoint{-0.048611in}{0.000000in}}%
\pgfusepath{stroke,fill}%
}%
\begin{pgfscope}%
\pgfsys@transformshift{0.800000in}{3.206188in}%
\pgfsys@useobject{currentmarker}{}%
\end{pgfscope}%
\end{pgfscope}%
\begin{pgfscope}%
\definecolor{textcolor}{rgb}{0.000000,0.000000,0.000000}%
\pgfsetstrokecolor{textcolor}%
\pgfsetfillcolor{textcolor}%
\pgftext[x=0.386419in, y=3.157974in, left, base]{\color{textcolor}\rmfamily\fontsize{10.000000}{12.000000}\selectfont \(\displaystyle {0.100}\)}%
\end{pgfscope}%
\begin{pgfscope}%
\pgfpathrectangle{\pgfqpoint{0.800000in}{0.528000in}}{\pgfqpoint{4.960000in}{3.696000in}}%
\pgfusepath{clip}%
\pgfsetrectcap%
\pgfsetroundjoin%
\pgfsetlinewidth{0.803000pt}%
\definecolor{currentstroke}{rgb}{0.690196,0.690196,0.690196}%
\pgfsetstrokecolor{currentstroke}%
\pgfsetdash{}{0pt}%
\pgfpathmoveto{\pgfqpoint{0.800000in}{3.629137in}}%
\pgfpathlineto{\pgfqpoint{5.760000in}{3.629137in}}%
\pgfusepath{stroke}%
\end{pgfscope}%
\begin{pgfscope}%
\pgfsetbuttcap%
\pgfsetroundjoin%
\definecolor{currentfill}{rgb}{0.000000,0.000000,0.000000}%
\pgfsetfillcolor{currentfill}%
\pgfsetlinewidth{0.803000pt}%
\definecolor{currentstroke}{rgb}{0.000000,0.000000,0.000000}%
\pgfsetstrokecolor{currentstroke}%
\pgfsetdash{}{0pt}%
\pgfsys@defobject{currentmarker}{\pgfqpoint{-0.048611in}{0.000000in}}{\pgfqpoint{-0.000000in}{0.000000in}}{%
\pgfpathmoveto{\pgfqpoint{-0.000000in}{0.000000in}}%
\pgfpathlineto{\pgfqpoint{-0.048611in}{0.000000in}}%
\pgfusepath{stroke,fill}%
}%
\begin{pgfscope}%
\pgfsys@transformshift{0.800000in}{3.629137in}%
\pgfsys@useobject{currentmarker}{}%
\end{pgfscope}%
\end{pgfscope}%
\begin{pgfscope}%
\definecolor{textcolor}{rgb}{0.000000,0.000000,0.000000}%
\pgfsetstrokecolor{textcolor}%
\pgfsetfillcolor{textcolor}%
\pgftext[x=0.386419in, y=3.580924in, left, base]{\color{textcolor}\rmfamily\fontsize{10.000000}{12.000000}\selectfont \(\displaystyle {0.125}\)}%
\end{pgfscope}%
\begin{pgfscope}%
\pgfpathrectangle{\pgfqpoint{0.800000in}{0.528000in}}{\pgfqpoint{4.960000in}{3.696000in}}%
\pgfusepath{clip}%
\pgfsetrectcap%
\pgfsetroundjoin%
\pgfsetlinewidth{0.803000pt}%
\definecolor{currentstroke}{rgb}{0.690196,0.690196,0.690196}%
\pgfsetstrokecolor{currentstroke}%
\pgfsetdash{}{0pt}%
\pgfpathmoveto{\pgfqpoint{0.800000in}{4.052087in}}%
\pgfpathlineto{\pgfqpoint{5.760000in}{4.052087in}}%
\pgfusepath{stroke}%
\end{pgfscope}%
\begin{pgfscope}%
\pgfsetbuttcap%
\pgfsetroundjoin%
\definecolor{currentfill}{rgb}{0.000000,0.000000,0.000000}%
\pgfsetfillcolor{currentfill}%
\pgfsetlinewidth{0.803000pt}%
\definecolor{currentstroke}{rgb}{0.000000,0.000000,0.000000}%
\pgfsetstrokecolor{currentstroke}%
\pgfsetdash{}{0pt}%
\pgfsys@defobject{currentmarker}{\pgfqpoint{-0.048611in}{0.000000in}}{\pgfqpoint{-0.000000in}{0.000000in}}{%
\pgfpathmoveto{\pgfqpoint{-0.000000in}{0.000000in}}%
\pgfpathlineto{\pgfqpoint{-0.048611in}{0.000000in}}%
\pgfusepath{stroke,fill}%
}%
\begin{pgfscope}%
\pgfsys@transformshift{0.800000in}{4.052087in}%
\pgfsys@useobject{currentmarker}{}%
\end{pgfscope}%
\end{pgfscope}%
\begin{pgfscope}%
\definecolor{textcolor}{rgb}{0.000000,0.000000,0.000000}%
\pgfsetstrokecolor{textcolor}%
\pgfsetfillcolor{textcolor}%
\pgftext[x=0.386419in, y=4.003873in, left, base]{\color{textcolor}\rmfamily\fontsize{10.000000}{12.000000}\selectfont \(\displaystyle {0.150}\)}%
\end{pgfscope}%
\begin{pgfscope}%
\definecolor{textcolor}{rgb}{0.000000,0.000000,0.000000}%
\pgfsetstrokecolor{textcolor}%
\pgfsetfillcolor{textcolor}%
\pgftext[x=0.222838in,y=2.376000in,,bottom,rotate=90.000000]{\color{textcolor}\rmfamily\fontsize{10.000000}{12.000000}\selectfont \(\displaystyle \vartheta,\ рад\)}%
\end{pgfscope}%
\begin{pgfscope}%
\pgfpathrectangle{\pgfqpoint{0.800000in}{0.528000in}}{\pgfqpoint{4.960000in}{3.696000in}}%
\pgfusepath{clip}%
\pgfsetrectcap%
\pgfsetroundjoin%
\pgfsetlinewidth{1.505625pt}%
\definecolor{currentstroke}{rgb}{0.121569,0.466667,0.705882}%
\pgfsetstrokecolor{currentstroke}%
\pgfsetdash{}{0pt}%
\pgfpathmoveto{\pgfqpoint{1.025455in}{1.514389in}}%
\pgfpathlineto{\pgfqpoint{1.189736in}{1.515493in}}%
\pgfpathlineto{\pgfqpoint{1.197251in}{1.518396in}}%
\pgfpathlineto{\pgfqpoint{1.204015in}{1.523459in}}%
\pgfpathlineto{\pgfqpoint{1.210628in}{1.531355in}}%
\pgfpathlineto{\pgfqpoint{1.217392in}{1.543114in}}%
\pgfpathlineto{\pgfqpoint{1.224456in}{1.560086in}}%
\pgfpathlineto{\pgfqpoint{1.231971in}{1.584211in}}%
\pgfpathlineto{\pgfqpoint{1.239937in}{1.617539in}}%
\pgfpathlineto{\pgfqpoint{1.248504in}{1.663378in}}%
\pgfpathlineto{\pgfqpoint{1.257673in}{1.725179in}}%
\pgfpathlineto{\pgfqpoint{1.267593in}{1.808138in}}%
\pgfpathlineto{\pgfqpoint{1.279617in}{1.928821in}}%
\pgfpathlineto{\pgfqpoint{1.295248in}{2.109938in}}%
\pgfpathlineto{\pgfqpoint{1.321552in}{2.445402in}}%
\pgfpathlineto{\pgfqpoint{1.349057in}{2.786735in}}%
\pgfpathlineto{\pgfqpoint{1.367093in}{2.982135in}}%
\pgfpathlineto{\pgfqpoint{1.389338in}{3.198592in}}%
\pgfpathlineto{\pgfqpoint{1.407525in}{3.355207in}}%
\pgfpathlineto{\pgfqpoint{1.424058in}{3.478897in}}%
\pgfpathlineto{\pgfqpoint{1.440291in}{3.583987in}}%
\pgfpathlineto{\pgfqpoint{1.456373in}{3.673872in}}%
\pgfpathlineto{\pgfqpoint{1.472155in}{3.749818in}}%
\pgfpathlineto{\pgfqpoint{1.487636in}{3.813756in}}%
\pgfpathlineto{\pgfqpoint{1.502817in}{3.867319in}}%
\pgfpathlineto{\pgfqpoint{1.517697in}{3.911894in}}%
\pgfpathlineto{\pgfqpoint{1.532126in}{3.948306in}}%
\pgfpathlineto{\pgfqpoint{1.546104in}{3.977730in}}%
\pgfpathlineto{\pgfqpoint{1.559632in}{4.001161in}}%
\pgfpathlineto{\pgfqpoint{1.572558in}{4.019259in}}%
\pgfpathlineto{\pgfqpoint{1.585033in}{4.033019in}}%
\pgfpathlineto{\pgfqpoint{1.597057in}{4.043058in}}%
\pgfpathlineto{\pgfqpoint{1.608630in}{4.049906in}}%
\pgfpathlineto{\pgfqpoint{1.620053in}{4.054111in}}%
\pgfpathlineto{\pgfqpoint{1.631326in}{4.055902in}}%
\pgfpathlineto{\pgfqpoint{1.642749in}{4.055448in}}%
\pgfpathlineto{\pgfqpoint{1.654473in}{4.052727in}}%
\pgfpathlineto{\pgfqpoint{1.666798in}{4.047524in}}%
\pgfpathlineto{\pgfqpoint{1.679874in}{4.039511in}}%
\pgfpathlineto{\pgfqpoint{1.693852in}{4.028255in}}%
\pgfpathlineto{\pgfqpoint{1.708882in}{4.013217in}}%
\pgfpathlineto{\pgfqpoint{1.725115in}{3.993752in}}%
\pgfpathlineto{\pgfqpoint{1.742701in}{3.969113in}}%
\pgfpathlineto{\pgfqpoint{1.761789in}{3.938449in}}%
\pgfpathlineto{\pgfqpoint{1.782531in}{3.900816in}}%
\pgfpathlineto{\pgfqpoint{1.805076in}{3.855186in}}%
\pgfpathlineto{\pgfqpoint{1.829576in}{3.800463in}}%
\pgfpathlineto{\pgfqpoint{1.856330in}{3.735123in}}%
\pgfpathlineto{\pgfqpoint{1.885639in}{3.657509in}}%
\pgfpathlineto{\pgfqpoint{1.917804in}{3.565882in}}%
\pgfpathlineto{\pgfqpoint{1.953576in}{3.457086in}}%
\pgfpathlineto{\pgfqpoint{1.994007in}{3.326772in}}%
\pgfpathlineto{\pgfqpoint{2.040902in}{3.167845in}}%
\pgfpathlineto{\pgfqpoint{2.098919in}{2.962949in}}%
\pgfpathlineto{\pgfqpoint{2.190904in}{2.628675in}}%
\pgfpathlineto{\pgfqpoint{2.296718in}{2.246399in}}%
\pgfpathlineto{\pgfqpoint{2.358642in}{2.031076in}}%
\pgfpathlineto{\pgfqpoint{2.410948in}{1.857051in}}%
\pgfpathlineto{\pgfqpoint{2.457842in}{1.708597in}}%
\pgfpathlineto{\pgfqpoint{2.500979in}{1.579279in}}%
\pgfpathlineto{\pgfqpoint{2.541261in}{1.465411in}}%
\pgfpathlineto{\pgfqpoint{2.579287in}{1.364455in}}%
\pgfpathlineto{\pgfqpoint{2.615360in}{1.274863in}}%
\pgfpathlineto{\pgfqpoint{2.649930in}{1.194859in}}%
\pgfpathlineto{\pgfqpoint{2.682996in}{1.123843in}}%
\pgfpathlineto{\pgfqpoint{2.714710in}{1.060890in}}%
\pgfpathlineto{\pgfqpoint{2.745372in}{1.004884in}}%
\pgfpathlineto{\pgfqpoint{2.774982in}{0.955363in}}%
\pgfpathlineto{\pgfqpoint{2.803539in}{0.911856in}}%
\pgfpathlineto{\pgfqpoint{2.831195in}{0.873697in}}%
\pgfpathlineto{\pgfqpoint{2.858099in}{0.840303in}}%
\pgfpathlineto{\pgfqpoint{2.884252in}{0.811336in}}%
\pgfpathlineto{\pgfqpoint{2.909653in}{0.786460in}}%
\pgfpathlineto{\pgfqpoint{2.934453in}{0.765226in}}%
\pgfpathlineto{\pgfqpoint{2.958802in}{0.747266in}}%
\pgfpathlineto{\pgfqpoint{2.982701in}{0.732367in}}%
\pgfpathlineto{\pgfqpoint{3.006298in}{0.720255in}}%
\pgfpathlineto{\pgfqpoint{3.029595in}{0.710771in}}%
\pgfpathlineto{\pgfqpoint{3.052742in}{0.703723in}}%
\pgfpathlineto{\pgfqpoint{3.075888in}{0.698977in}}%
\pgfpathlineto{\pgfqpoint{3.099185in}{0.696458in}}%
\pgfpathlineto{\pgfqpoint{3.122783in}{0.696140in}}%
\pgfpathlineto{\pgfqpoint{3.146832in}{0.698048in}}%
\pgfpathlineto{\pgfqpoint{3.171481in}{0.702254in}}%
\pgfpathlineto{\pgfqpoint{3.196882in}{0.708869in}}%
\pgfpathlineto{\pgfqpoint{3.223185in}{0.718046in}}%
\pgfpathlineto{\pgfqpoint{3.250541in}{0.729971in}}%
\pgfpathlineto{\pgfqpoint{3.279248in}{0.744943in}}%
\pgfpathlineto{\pgfqpoint{3.309459in}{0.763238in}}%
\pgfpathlineto{\pgfqpoint{3.341624in}{0.785362in}}%
\pgfpathlineto{\pgfqpoint{3.376044in}{0.811793in}}%
\pgfpathlineto{\pgfqpoint{3.413168in}{0.843167in}}%
\pgfpathlineto{\pgfqpoint{3.453750in}{0.880444in}}%
\pgfpathlineto{\pgfqpoint{3.498992in}{0.925106in}}%
\pgfpathlineto{\pgfqpoint{3.550996in}{0.979682in}}%
\pgfpathlineto{\pgfqpoint{3.614725in}{1.049949in}}%
\pgfpathlineto{\pgfqpoint{3.715578in}{1.164970in}}%
\pgfpathlineto{\pgfqpoint{3.830259in}{1.294842in}}%
\pgfpathlineto{\pgfqpoint{3.898196in}{1.368385in}}%
\pgfpathlineto{\pgfqpoint{3.955913in}{1.427662in}}%
\pgfpathlineto{\pgfqpoint{4.008218in}{1.478248in}}%
\pgfpathlineto{\pgfqpoint{4.056766in}{1.522157in}}%
\pgfpathlineto{\pgfqpoint{4.102759in}{1.560799in}}%
\pgfpathlineto{\pgfqpoint{4.146798in}{1.594928in}}%
\pgfpathlineto{\pgfqpoint{4.189333in}{1.625103in}}%
\pgfpathlineto{\pgfqpoint{4.230667in}{1.651720in}}%
\pgfpathlineto{\pgfqpoint{4.271098in}{1.675126in}}%
\pgfpathlineto{\pgfqpoint{4.310778in}{1.695546in}}%
\pgfpathlineto{\pgfqpoint{4.350007in}{1.713247in}}%
\pgfpathlineto{\pgfqpoint{4.388936in}{1.728383in}}%
\pgfpathlineto{\pgfqpoint{4.427714in}{1.741086in}}%
\pgfpathlineto{\pgfqpoint{4.466642in}{1.751501in}}%
\pgfpathlineto{\pgfqpoint{4.505721in}{1.759662in}}%
\pgfpathlineto{\pgfqpoint{4.545401in}{1.765673in}}%
\pgfpathlineto{\pgfqpoint{4.585682in}{1.769522in}}%
\pgfpathlineto{\pgfqpoint{4.626865in}{1.771219in}}%
\pgfpathlineto{\pgfqpoint{4.669251in}{1.770734in}}%
\pgfpathlineto{\pgfqpoint{4.713139in}{1.767999in}}%
\pgfpathlineto{\pgfqpoint{4.758982in}{1.762900in}}%
\pgfpathlineto{\pgfqpoint{4.807379in}{1.755254in}}%
\pgfpathlineto{\pgfqpoint{4.858933in}{1.744828in}}%
\pgfpathlineto{\pgfqpoint{4.914696in}{1.731248in}}%
\pgfpathlineto{\pgfqpoint{4.976621in}{1.713827in}}%
\pgfpathlineto{\pgfqpoint{5.047864in}{1.691401in}}%
\pgfpathlineto{\pgfqpoint{5.137445in}{1.660743in}}%
\pgfpathlineto{\pgfqpoint{5.465556in}{1.546236in}}%
\pgfpathlineto{\pgfqpoint{5.534545in}{1.525136in}}%
\pgfpathlineto{\pgfqpoint{5.534545in}{1.525136in}}%
\pgfusepath{stroke}%
\end{pgfscope}%
\begin{pgfscope}%
\pgfpathrectangle{\pgfqpoint{0.800000in}{0.528000in}}{\pgfqpoint{4.960000in}{3.696000in}}%
\pgfusepath{clip}%
\pgfsetbuttcap%
\pgfsetroundjoin%
\pgfsetlinewidth{1.505625pt}%
\definecolor{currentstroke}{rgb}{1.000000,0.498039,0.054902}%
\pgfsetstrokecolor{currentstroke}%
\pgfsetdash{{9.600000pt}{2.400000pt}{1.500000pt}{2.400000pt}}{0.000000pt}%
\pgfpathmoveto{\pgfqpoint{1.025455in}{1.514389in}}%
\pgfpathlineto{\pgfqpoint{1.184926in}{1.515481in}}%
\pgfpathlineto{\pgfqpoint{1.189736in}{1.518446in}}%
\pgfpathlineto{\pgfqpoint{1.194245in}{1.523966in}}%
\pgfpathlineto{\pgfqpoint{1.198904in}{1.533393in}}%
\pgfpathlineto{\pgfqpoint{1.204015in}{1.549183in}}%
\pgfpathlineto{\pgfqpoint{1.209576in}{1.574219in}}%
\pgfpathlineto{\pgfqpoint{1.215738in}{1.613091in}}%
\pgfpathlineto{\pgfqpoint{1.223704in}{1.678965in}}%
\pgfpathlineto{\pgfqpoint{1.235278in}{1.795529in}}%
\pgfpathlineto{\pgfqpoint{1.310279in}{2.588996in}}%
\pgfpathlineto{\pgfqpoint{1.362885in}{3.115665in}}%
\pgfpathlineto{\pgfqpoint{1.383176in}{3.299341in}}%
\pgfpathlineto{\pgfqpoint{1.400010in}{3.432017in}}%
\pgfpathlineto{\pgfqpoint{1.416393in}{3.543847in}}%
\pgfpathlineto{\pgfqpoint{1.432475in}{3.638677in}}%
\pgfpathlineto{\pgfqpoint{1.448257in}{3.718881in}}%
\pgfpathlineto{\pgfqpoint{1.463888in}{3.787087in}}%
\pgfpathlineto{\pgfqpoint{1.479219in}{3.844232in}}%
\pgfpathlineto{\pgfqpoint{1.494099in}{3.891377in}}%
\pgfpathlineto{\pgfqpoint{1.508679in}{3.930365in}}%
\pgfpathlineto{\pgfqpoint{1.522807in}{3.961941in}}%
\pgfpathlineto{\pgfqpoint{1.536335in}{3.986921in}}%
\pgfpathlineto{\pgfqpoint{1.549411in}{4.006555in}}%
\pgfpathlineto{\pgfqpoint{1.562036in}{4.021601in}}%
\pgfpathlineto{\pgfqpoint{1.574061in}{4.032592in}}%
\pgfpathlineto{\pgfqpoint{1.585784in}{4.040365in}}%
\pgfpathlineto{\pgfqpoint{1.597207in}{4.045303in}}%
\pgfpathlineto{\pgfqpoint{1.608480in}{4.047764in}}%
\pgfpathlineto{\pgfqpoint{1.619753in}{4.047953in}}%
\pgfpathlineto{\pgfqpoint{1.631326in}{4.045901in}}%
\pgfpathlineto{\pgfqpoint{1.643350in}{4.041481in}}%
\pgfpathlineto{\pgfqpoint{1.656126in}{4.034359in}}%
\pgfpathlineto{\pgfqpoint{1.669804in}{4.024106in}}%
\pgfpathlineto{\pgfqpoint{1.684533in}{4.010190in}}%
\pgfpathlineto{\pgfqpoint{1.700465in}{3.991973in}}%
\pgfpathlineto{\pgfqpoint{1.717750in}{3.968709in}}%
\pgfpathlineto{\pgfqpoint{1.736388in}{3.939800in}}%
\pgfpathlineto{\pgfqpoint{1.756679in}{3.904118in}}%
\pgfpathlineto{\pgfqpoint{1.778773in}{3.860633in}}%
\pgfpathlineto{\pgfqpoint{1.802822in}{3.808245in}}%
\pgfpathlineto{\pgfqpoint{1.828975in}{3.745801in}}%
\pgfpathlineto{\pgfqpoint{1.857532in}{3.671725in}}%
\pgfpathlineto{\pgfqpoint{1.888945in}{3.583909in}}%
\pgfpathlineto{\pgfqpoint{1.923816in}{3.479639in}}%
\pgfpathlineto{\pgfqpoint{1.963045in}{3.355091in}}%
\pgfpathlineto{\pgfqpoint{2.008136in}{3.204271in}}%
\pgfpathlineto{\pgfqpoint{2.062846in}{3.013135in}}%
\pgfpathlineto{\pgfqpoint{2.140553in}{2.732740in}}%
\pgfpathlineto{\pgfqpoint{2.279132in}{2.232252in}}%
\pgfpathlineto{\pgfqpoint{2.340305in}{2.020432in}}%
\pgfpathlineto{\pgfqpoint{2.392160in}{1.848668in}}%
\pgfpathlineto{\pgfqpoint{2.438754in}{1.701837in}}%
\pgfpathlineto{\pgfqpoint{2.481741in}{1.573576in}}%
\pgfpathlineto{\pgfqpoint{2.521872in}{1.460684in}}%
\pgfpathlineto{\pgfqpoint{2.559748in}{1.360627in}}%
\pgfpathlineto{\pgfqpoint{2.595821in}{1.271496in}}%
\pgfpathlineto{\pgfqpoint{2.630240in}{1.192265in}}%
\pgfpathlineto{\pgfqpoint{2.663307in}{1.121638in}}%
\pgfpathlineto{\pgfqpoint{2.695021in}{1.059051in}}%
\pgfpathlineto{\pgfqpoint{2.725532in}{1.003652in}}%
\pgfpathlineto{\pgfqpoint{2.754992in}{0.954672in}}%
\pgfpathlineto{\pgfqpoint{2.783549in}{0.911427in}}%
\pgfpathlineto{\pgfqpoint{2.811205in}{0.873511in}}%
\pgfpathlineto{\pgfqpoint{2.838109in}{0.840346in}}%
\pgfpathlineto{\pgfqpoint{2.864262in}{0.811591in}}%
\pgfpathlineto{\pgfqpoint{2.889663in}{0.786913in}}%
\pgfpathlineto{\pgfqpoint{2.914463in}{0.765865in}}%
\pgfpathlineto{\pgfqpoint{2.938812in}{0.748077in}}%
\pgfpathlineto{\pgfqpoint{2.962710in}{0.733340in}}%
\pgfpathlineto{\pgfqpoint{2.986308in}{0.721379in}}%
\pgfpathlineto{\pgfqpoint{3.009605in}{0.712036in}}%
\pgfpathlineto{\pgfqpoint{3.032902in}{0.705082in}}%
\pgfpathlineto{\pgfqpoint{3.056199in}{0.700453in}}%
\pgfpathlineto{\pgfqpoint{3.079646in}{0.698071in}}%
\pgfpathlineto{\pgfqpoint{3.103244in}{0.697904in}}%
\pgfpathlineto{\pgfqpoint{3.127292in}{0.699957in}}%
\pgfpathlineto{\pgfqpoint{3.151942in}{0.704300in}}%
\pgfpathlineto{\pgfqpoint{3.177343in}{0.711045in}}%
\pgfpathlineto{\pgfqpoint{3.203646in}{0.720345in}}%
\pgfpathlineto{\pgfqpoint{3.231152in}{0.732456in}}%
\pgfpathlineto{\pgfqpoint{3.260010in}{0.747632in}}%
\pgfpathlineto{\pgfqpoint{3.290371in}{0.766145in}}%
\pgfpathlineto{\pgfqpoint{3.322686in}{0.788499in}}%
\pgfpathlineto{\pgfqpoint{3.357256in}{0.815171in}}%
\pgfpathlineto{\pgfqpoint{3.394531in}{0.846791in}}%
\pgfpathlineto{\pgfqpoint{3.435413in}{0.884455in}}%
\pgfpathlineto{\pgfqpoint{3.480955in}{0.929514in}}%
\pgfpathlineto{\pgfqpoint{3.533411in}{0.984641in}}%
\pgfpathlineto{\pgfqpoint{3.598342in}{1.056278in}}%
\pgfpathlineto{\pgfqpoint{3.706560in}{1.179589in}}%
\pgfpathlineto{\pgfqpoint{3.812223in}{1.298744in}}%
\pgfpathlineto{\pgfqpoint{3.879559in}{1.371340in}}%
\pgfpathlineto{\pgfqpoint{3.937125in}{1.430198in}}%
\pgfpathlineto{\pgfqpoint{3.989130in}{1.480252in}}%
\pgfpathlineto{\pgfqpoint{4.037678in}{1.523939in}}%
\pgfpathlineto{\pgfqpoint{4.083520in}{1.562246in}}%
\pgfpathlineto{\pgfqpoint{4.127408in}{1.596067in}}%
\pgfpathlineto{\pgfqpoint{4.169794in}{1.625962in}}%
\pgfpathlineto{\pgfqpoint{4.211127in}{1.652418in}}%
\pgfpathlineto{\pgfqpoint{4.251559in}{1.675672in}}%
\pgfpathlineto{\pgfqpoint{4.291239in}{1.695946in}}%
\pgfpathlineto{\pgfqpoint{4.330468in}{1.713508in}}%
\pgfpathlineto{\pgfqpoint{4.369396in}{1.728513in}}%
\pgfpathlineto{\pgfqpoint{4.408175in}{1.741092in}}%
\pgfpathlineto{\pgfqpoint{4.447103in}{1.751389in}}%
\pgfpathlineto{\pgfqpoint{4.486332in}{1.759466in}}%
\pgfpathlineto{\pgfqpoint{4.526012in}{1.765365in}}%
\pgfpathlineto{\pgfqpoint{4.566444in}{1.769118in}}%
\pgfpathlineto{\pgfqpoint{4.607777in}{1.770711in}}%
\pgfpathlineto{\pgfqpoint{4.650313in}{1.770114in}}%
\pgfpathlineto{\pgfqpoint{4.694352in}{1.767261in}}%
\pgfpathlineto{\pgfqpoint{4.740344in}{1.762040in}}%
\pgfpathlineto{\pgfqpoint{4.788892in}{1.754268in}}%
\pgfpathlineto{\pgfqpoint{4.840747in}{1.743683in}}%
\pgfpathlineto{\pgfqpoint{4.896960in}{1.729895in}}%
\pgfpathlineto{\pgfqpoint{4.959336in}{1.712253in}}%
\pgfpathlineto{\pgfqpoint{5.031481in}{1.689460in}}%
\pgfpathlineto{\pgfqpoint{5.123016in}{1.658067in}}%
\pgfpathlineto{\pgfqpoint{5.433993in}{1.549489in}}%
\pgfpathlineto{\pgfqpoint{5.516208in}{1.524291in}}%
\pgfpathlineto{\pgfqpoint{5.534545in}{1.519018in}}%
\pgfpathlineto{\pgfqpoint{5.534545in}{1.519018in}}%
\pgfusepath{stroke}%
\end{pgfscope}%
\begin{pgfscope}%
\pgfsetrectcap%
\pgfsetmiterjoin%
\pgfsetlinewidth{0.803000pt}%
\definecolor{currentstroke}{rgb}{0.000000,0.000000,0.000000}%
\pgfsetstrokecolor{currentstroke}%
\pgfsetdash{}{0pt}%
\pgfpathmoveto{\pgfqpoint{0.800000in}{0.528000in}}%
\pgfpathlineto{\pgfqpoint{0.800000in}{4.224000in}}%
\pgfusepath{stroke}%
\end{pgfscope}%
\begin{pgfscope}%
\pgfsetrectcap%
\pgfsetmiterjoin%
\pgfsetlinewidth{0.803000pt}%
\definecolor{currentstroke}{rgb}{0.000000,0.000000,0.000000}%
\pgfsetstrokecolor{currentstroke}%
\pgfsetdash{}{0pt}%
\pgfpathmoveto{\pgfqpoint{5.760000in}{0.528000in}}%
\pgfpathlineto{\pgfqpoint{5.760000in}{4.224000in}}%
\pgfusepath{stroke}%
\end{pgfscope}%
\begin{pgfscope}%
\pgfsetrectcap%
\pgfsetmiterjoin%
\pgfsetlinewidth{0.803000pt}%
\definecolor{currentstroke}{rgb}{0.000000,0.000000,0.000000}%
\pgfsetstrokecolor{currentstroke}%
\pgfsetdash{}{0pt}%
\pgfpathmoveto{\pgfqpoint{0.800000in}{0.528000in}}%
\pgfpathlineto{\pgfqpoint{5.760000in}{0.528000in}}%
\pgfusepath{stroke}%
\end{pgfscope}%
\begin{pgfscope}%
\pgfsetrectcap%
\pgfsetmiterjoin%
\pgfsetlinewidth{0.803000pt}%
\definecolor{currentstroke}{rgb}{0.000000,0.000000,0.000000}%
\pgfsetstrokecolor{currentstroke}%
\pgfsetdash{}{0pt}%
\pgfpathmoveto{\pgfqpoint{0.800000in}{4.224000in}}%
\pgfpathlineto{\pgfqpoint{5.760000in}{4.224000in}}%
\pgfusepath{stroke}%
\end{pgfscope}%
\begin{pgfscope}%
\pgfsetbuttcap%
\pgfsetmiterjoin%
\definecolor{currentfill}{rgb}{1.000000,1.000000,1.000000}%
\pgfsetfillcolor{currentfill}%
\pgfsetfillopacity{0.800000}%
\pgfsetlinewidth{1.003750pt}%
\definecolor{currentstroke}{rgb}{0.800000,0.800000,0.800000}%
\pgfsetstrokecolor{currentstroke}%
\pgfsetstrokeopacity{0.800000}%
\pgfsetdash{}{0pt}%
\pgfpathmoveto{\pgfqpoint{3.409453in}{3.475970in}}%
\pgfpathlineto{\pgfqpoint{5.662778in}{3.475970in}}%
\pgfpathquadraticcurveto{\pgfqpoint{5.690556in}{3.475970in}}{\pgfqpoint{5.690556in}{3.503748in}}%
\pgfpathlineto{\pgfqpoint{5.690556in}{4.126778in}}%
\pgfpathquadraticcurveto{\pgfqpoint{5.690556in}{4.154556in}}{\pgfqpoint{5.662778in}{4.154556in}}%
\pgfpathlineto{\pgfqpoint{3.409453in}{4.154556in}}%
\pgfpathquadraticcurveto{\pgfqpoint{3.381676in}{4.154556in}}{\pgfqpoint{3.381676in}{4.126778in}}%
\pgfpathlineto{\pgfqpoint{3.381676in}{3.503748in}}%
\pgfpathquadraticcurveto{\pgfqpoint{3.381676in}{3.475970in}}{\pgfqpoint{3.409453in}{3.475970in}}%
\pgfpathclose%
\pgfusepath{stroke,fill}%
\end{pgfscope}%
\begin{pgfscope}%
\pgfsetrectcap%
\pgfsetroundjoin%
\pgfsetlinewidth{1.505625pt}%
\definecolor{currentstroke}{rgb}{0.121569,0.466667,0.705882}%
\pgfsetstrokecolor{currentstroke}%
\pgfsetdash{}{0pt}%
\pgfpathmoveto{\pgfqpoint{3.437231in}{3.993867in}}%
\pgfpathlineto{\pgfqpoint{3.715009in}{3.993867in}}%
\pgfusepath{stroke}%
\end{pgfscope}%
\begin{pgfscope}%
\definecolor{textcolor}{rgb}{0.000000,0.000000,0.000000}%
\pgfsetstrokecolor{textcolor}%
\pgfsetfillcolor{textcolor}%
\pgftext[x=3.826120in,y=3.945256in,left,base]{\color{textcolor}\rmfamily\fontsize{10.000000}{12.000000}\selectfont Модель при \(\displaystyle \dot{\delta}_{{в}_{max}}=15 \frac{град.}{сек.}\)}%
\end{pgfscope}%
\begin{pgfscope}%
\pgfsetbuttcap%
\pgfsetroundjoin%
\pgfsetlinewidth{1.505625pt}%
\definecolor{currentstroke}{rgb}{1.000000,0.498039,0.054902}%
\pgfsetstrokecolor{currentstroke}%
\pgfsetdash{{9.600000pt}{2.400000pt}{1.500000pt}{2.400000pt}}{0.000000pt}%
\pgfpathmoveto{\pgfqpoint{3.437231in}{3.675407in}}%
\pgfpathlineto{\pgfqpoint{3.715009in}{3.675407in}}%
\pgfusepath{stroke}%
\end{pgfscope}%
\begin{pgfscope}%
\definecolor{textcolor}{rgb}{0.000000,0.000000,0.000000}%
\pgfsetstrokecolor{textcolor}%
\pgfsetfillcolor{textcolor}%
\pgftext[x=3.826120in,y=3.626796in,left,base]{\color{textcolor}\rmfamily\fontsize{10.000000}{12.000000}\selectfont Модель при \(\displaystyle \dot{\delta}_{{в}_{max}}=60 \frac{град.}{сек.}\)}%
\end{pgfscope}%
\end{pgfpicture}%
\makeatother%
\endgroup%
}
    \caption{Изменение угла тангажа для различных $\dot{\delta}_\text{в max}$}
    \label{fig:model_DD_theta}
\end{minipage}
\end{figure}

\begin{table}[H]
    \centering
    \caption{Сравнение параметров переходного процесса $\Delta H(t)$ при различных $\dot{\delta}_{в}$}
    \label{tab:stat_DD}

    \begin{tabular}{|c|c|c|}
        \hline
        {} &  Модель при $\dot{\delta}_{{в}_{max}}=15 \frac{град.}{сек.}$ &  Модель при $\dot{\delta}_{{в}_{max}}=60 \frac{град.}{сек.}$ \\
        \hline
        $t_{рег},\ с$ & 26.70 & 26.69 \\
        \hline
        $\sigma,\ \%$ & 24.73 & 24.66 \\
        \hline
    \end{tabular}
\end{table}

\subsection{Сравнение линейной и нелинейной модели}

Графики изменения $\Delta H$, $\delta_{в}$, $\omega_z$, $\vartheta$ для
линейной и нелинейной модели представлены на рисунках \ref{fig:model_Delta_H},
\ref{fig:delta_elevator}, \ref{fig:model_omega_z}, \ref{fig:model_theta}.
Моделирование нелинейной модели проводилось при 
$\dot{\delta}_\text{в max} = 60\, \frac{\text{град.}}{\text{сек.}}$.

\begin{figure}[H]
    \begin{minipage}{0.48\textwidth}
    \centering
    \resizebox{1.1\linewidth}{!}{%% Creator: Matplotlib, PGF backend
%%
%% To include the figure in your LaTeX document, write
%%   \input{<filename>.pgf}
%%
%% Make sure the required packages are loaded in your preamble
%%   \usepackage{pgf}
%%
%% Figures using additional raster images can only be included by \input if
%% they are in the same directory as the main LaTeX file. For loading figures
%% from other directories you can use the `import` package
%%   \usepackage{import}
%%
%% and then include the figures with
%%   \import{<path to file>}{<filename>.pgf}
%%
%% Matplotlib used the following preamble
%%   \usepackage[warn]{mathtext}
%%   \usepackage[T2A]{fontenc}
%%   \usepackage[utf8]{inputenc}
%%   \usepackage[english,russian]{babel}
%%
\begingroup%
\makeatletter%
\begin{pgfpicture}%
\pgfpathrectangle{\pgfpointorigin}{\pgfqpoint{5.882184in}{4.404267in}}%
\pgfusepath{use as bounding box, clip}%
\begin{pgfscope}%
\pgfsetbuttcap%
\pgfsetmiterjoin%
\definecolor{currentfill}{rgb}{1.000000,1.000000,1.000000}%
\pgfsetfillcolor{currentfill}%
\pgfsetlinewidth{0.000000pt}%
\definecolor{currentstroke}{rgb}{1.000000,1.000000,1.000000}%
\pgfsetstrokecolor{currentstroke}%
\pgfsetdash{}{0pt}%
\pgfpathmoveto{\pgfqpoint{0.000000in}{0.000000in}}%
\pgfpathlineto{\pgfqpoint{5.882184in}{0.000000in}}%
\pgfpathlineto{\pgfqpoint{5.882184in}{4.404267in}}%
\pgfpathlineto{\pgfqpoint{0.000000in}{4.404267in}}%
\pgfpathclose%
\pgfusepath{fill}%
\end{pgfscope}%
\begin{pgfscope}%
\pgfsetbuttcap%
\pgfsetmiterjoin%
\definecolor{currentfill}{rgb}{1.000000,1.000000,1.000000}%
\pgfsetfillcolor{currentfill}%
\pgfsetlinewidth{0.000000pt}%
\definecolor{currentstroke}{rgb}{0.000000,0.000000,0.000000}%
\pgfsetstrokecolor{currentstroke}%
\pgfsetstrokeopacity{0.000000}%
\pgfsetdash{}{0pt}%
\pgfpathmoveto{\pgfqpoint{0.724269in}{0.608267in}}%
\pgfpathlineto{\pgfqpoint{5.684269in}{0.608267in}}%
\pgfpathlineto{\pgfqpoint{5.684269in}{4.304267in}}%
\pgfpathlineto{\pgfqpoint{0.724269in}{4.304267in}}%
\pgfpathclose%
\pgfusepath{fill}%
\end{pgfscope}%
\begin{pgfscope}%
\pgfpathrectangle{\pgfqpoint{0.724269in}{0.608267in}}{\pgfqpoint{4.960000in}{3.696000in}}%
\pgfusepath{clip}%
\pgfsetrectcap%
\pgfsetroundjoin%
\pgfsetlinewidth{0.803000pt}%
\definecolor{currentstroke}{rgb}{0.690196,0.690196,0.690196}%
\pgfsetstrokecolor{currentstroke}%
\pgfsetdash{}{0pt}%
\pgfpathmoveto{\pgfqpoint{0.724269in}{0.608267in}}%
\pgfpathlineto{\pgfqpoint{0.724269in}{4.304267in}}%
\pgfusepath{stroke}%
\end{pgfscope}%
\begin{pgfscope}%
\pgfsetbuttcap%
\pgfsetroundjoin%
\definecolor{currentfill}{rgb}{0.000000,0.000000,0.000000}%
\pgfsetfillcolor{currentfill}%
\pgfsetlinewidth{0.803000pt}%
\definecolor{currentstroke}{rgb}{0.000000,0.000000,0.000000}%
\pgfsetstrokecolor{currentstroke}%
\pgfsetdash{}{0pt}%
\pgfsys@defobject{currentmarker}{\pgfqpoint{0.000000in}{-0.048611in}}{\pgfqpoint{0.000000in}{0.000000in}}{%
\pgfpathmoveto{\pgfqpoint{0.000000in}{0.000000in}}%
\pgfpathlineto{\pgfqpoint{0.000000in}{-0.048611in}}%
\pgfusepath{stroke,fill}%
}%
\begin{pgfscope}%
\pgfsys@transformshift{0.724269in}{0.608267in}%
\pgfsys@useobject{currentmarker}{}%
\end{pgfscope}%
\end{pgfscope}%
\begin{pgfscope}%
\definecolor{textcolor}{rgb}{0.000000,0.000000,0.000000}%
\pgfsetstrokecolor{textcolor}%
\pgfsetfillcolor{textcolor}%
\pgftext[x=0.724269in,y=0.511045in,,top]{\color{textcolor}\rmfamily\fontsize{14.000000}{16.800000}\selectfont \(\displaystyle {0}\)}%
\end{pgfscope}%
\begin{pgfscope}%
\pgfpathrectangle{\pgfqpoint{0.724269in}{0.608267in}}{\pgfqpoint{4.960000in}{3.696000in}}%
\pgfusepath{clip}%
\pgfsetrectcap%
\pgfsetroundjoin%
\pgfsetlinewidth{0.803000pt}%
\definecolor{currentstroke}{rgb}{0.690196,0.690196,0.690196}%
\pgfsetstrokecolor{currentstroke}%
\pgfsetdash{}{0pt}%
\pgfpathmoveto{\pgfqpoint{1.550935in}{0.608267in}}%
\pgfpathlineto{\pgfqpoint{1.550935in}{4.304267in}}%
\pgfusepath{stroke}%
\end{pgfscope}%
\begin{pgfscope}%
\pgfsetbuttcap%
\pgfsetroundjoin%
\definecolor{currentfill}{rgb}{0.000000,0.000000,0.000000}%
\pgfsetfillcolor{currentfill}%
\pgfsetlinewidth{0.803000pt}%
\definecolor{currentstroke}{rgb}{0.000000,0.000000,0.000000}%
\pgfsetstrokecolor{currentstroke}%
\pgfsetdash{}{0pt}%
\pgfsys@defobject{currentmarker}{\pgfqpoint{0.000000in}{-0.048611in}}{\pgfqpoint{0.000000in}{0.000000in}}{%
\pgfpathmoveto{\pgfqpoint{0.000000in}{0.000000in}}%
\pgfpathlineto{\pgfqpoint{0.000000in}{-0.048611in}}%
\pgfusepath{stroke,fill}%
}%
\begin{pgfscope}%
\pgfsys@transformshift{1.550935in}{0.608267in}%
\pgfsys@useobject{currentmarker}{}%
\end{pgfscope}%
\end{pgfscope}%
\begin{pgfscope}%
\definecolor{textcolor}{rgb}{0.000000,0.000000,0.000000}%
\pgfsetstrokecolor{textcolor}%
\pgfsetfillcolor{textcolor}%
\pgftext[x=1.550935in,y=0.511045in,,top]{\color{textcolor}\rmfamily\fontsize{14.000000}{16.800000}\selectfont \(\displaystyle {5}\)}%
\end{pgfscope}%
\begin{pgfscope}%
\pgfpathrectangle{\pgfqpoint{0.724269in}{0.608267in}}{\pgfqpoint{4.960000in}{3.696000in}}%
\pgfusepath{clip}%
\pgfsetrectcap%
\pgfsetroundjoin%
\pgfsetlinewidth{0.803000pt}%
\definecolor{currentstroke}{rgb}{0.690196,0.690196,0.690196}%
\pgfsetstrokecolor{currentstroke}%
\pgfsetdash{}{0pt}%
\pgfpathmoveto{\pgfqpoint{2.377602in}{0.608267in}}%
\pgfpathlineto{\pgfqpoint{2.377602in}{4.304267in}}%
\pgfusepath{stroke}%
\end{pgfscope}%
\begin{pgfscope}%
\pgfsetbuttcap%
\pgfsetroundjoin%
\definecolor{currentfill}{rgb}{0.000000,0.000000,0.000000}%
\pgfsetfillcolor{currentfill}%
\pgfsetlinewidth{0.803000pt}%
\definecolor{currentstroke}{rgb}{0.000000,0.000000,0.000000}%
\pgfsetstrokecolor{currentstroke}%
\pgfsetdash{}{0pt}%
\pgfsys@defobject{currentmarker}{\pgfqpoint{0.000000in}{-0.048611in}}{\pgfqpoint{0.000000in}{0.000000in}}{%
\pgfpathmoveto{\pgfqpoint{0.000000in}{0.000000in}}%
\pgfpathlineto{\pgfqpoint{0.000000in}{-0.048611in}}%
\pgfusepath{stroke,fill}%
}%
\begin{pgfscope}%
\pgfsys@transformshift{2.377602in}{0.608267in}%
\pgfsys@useobject{currentmarker}{}%
\end{pgfscope}%
\end{pgfscope}%
\begin{pgfscope}%
\definecolor{textcolor}{rgb}{0.000000,0.000000,0.000000}%
\pgfsetstrokecolor{textcolor}%
\pgfsetfillcolor{textcolor}%
\pgftext[x=2.377602in,y=0.511045in,,top]{\color{textcolor}\rmfamily\fontsize{14.000000}{16.800000}\selectfont \(\displaystyle {10}\)}%
\end{pgfscope}%
\begin{pgfscope}%
\pgfpathrectangle{\pgfqpoint{0.724269in}{0.608267in}}{\pgfqpoint{4.960000in}{3.696000in}}%
\pgfusepath{clip}%
\pgfsetrectcap%
\pgfsetroundjoin%
\pgfsetlinewidth{0.803000pt}%
\definecolor{currentstroke}{rgb}{0.690196,0.690196,0.690196}%
\pgfsetstrokecolor{currentstroke}%
\pgfsetdash{}{0pt}%
\pgfpathmoveto{\pgfqpoint{3.204269in}{0.608267in}}%
\pgfpathlineto{\pgfqpoint{3.204269in}{4.304267in}}%
\pgfusepath{stroke}%
\end{pgfscope}%
\begin{pgfscope}%
\pgfsetbuttcap%
\pgfsetroundjoin%
\definecolor{currentfill}{rgb}{0.000000,0.000000,0.000000}%
\pgfsetfillcolor{currentfill}%
\pgfsetlinewidth{0.803000pt}%
\definecolor{currentstroke}{rgb}{0.000000,0.000000,0.000000}%
\pgfsetstrokecolor{currentstroke}%
\pgfsetdash{}{0pt}%
\pgfsys@defobject{currentmarker}{\pgfqpoint{0.000000in}{-0.048611in}}{\pgfqpoint{0.000000in}{0.000000in}}{%
\pgfpathmoveto{\pgfqpoint{0.000000in}{0.000000in}}%
\pgfpathlineto{\pgfqpoint{0.000000in}{-0.048611in}}%
\pgfusepath{stroke,fill}%
}%
\begin{pgfscope}%
\pgfsys@transformshift{3.204269in}{0.608267in}%
\pgfsys@useobject{currentmarker}{}%
\end{pgfscope}%
\end{pgfscope}%
\begin{pgfscope}%
\definecolor{textcolor}{rgb}{0.000000,0.000000,0.000000}%
\pgfsetstrokecolor{textcolor}%
\pgfsetfillcolor{textcolor}%
\pgftext[x=3.204269in,y=0.511045in,,top]{\color{textcolor}\rmfamily\fontsize{14.000000}{16.800000}\selectfont \(\displaystyle {15}\)}%
\end{pgfscope}%
\begin{pgfscope}%
\pgfpathrectangle{\pgfqpoint{0.724269in}{0.608267in}}{\pgfqpoint{4.960000in}{3.696000in}}%
\pgfusepath{clip}%
\pgfsetrectcap%
\pgfsetroundjoin%
\pgfsetlinewidth{0.803000pt}%
\definecolor{currentstroke}{rgb}{0.690196,0.690196,0.690196}%
\pgfsetstrokecolor{currentstroke}%
\pgfsetdash{}{0pt}%
\pgfpathmoveto{\pgfqpoint{4.030935in}{0.608267in}}%
\pgfpathlineto{\pgfqpoint{4.030935in}{4.304267in}}%
\pgfusepath{stroke}%
\end{pgfscope}%
\begin{pgfscope}%
\pgfsetbuttcap%
\pgfsetroundjoin%
\definecolor{currentfill}{rgb}{0.000000,0.000000,0.000000}%
\pgfsetfillcolor{currentfill}%
\pgfsetlinewidth{0.803000pt}%
\definecolor{currentstroke}{rgb}{0.000000,0.000000,0.000000}%
\pgfsetstrokecolor{currentstroke}%
\pgfsetdash{}{0pt}%
\pgfsys@defobject{currentmarker}{\pgfqpoint{0.000000in}{-0.048611in}}{\pgfqpoint{0.000000in}{0.000000in}}{%
\pgfpathmoveto{\pgfqpoint{0.000000in}{0.000000in}}%
\pgfpathlineto{\pgfqpoint{0.000000in}{-0.048611in}}%
\pgfusepath{stroke,fill}%
}%
\begin{pgfscope}%
\pgfsys@transformshift{4.030935in}{0.608267in}%
\pgfsys@useobject{currentmarker}{}%
\end{pgfscope}%
\end{pgfscope}%
\begin{pgfscope}%
\definecolor{textcolor}{rgb}{0.000000,0.000000,0.000000}%
\pgfsetstrokecolor{textcolor}%
\pgfsetfillcolor{textcolor}%
\pgftext[x=4.030935in,y=0.511045in,,top]{\color{textcolor}\rmfamily\fontsize{14.000000}{16.800000}\selectfont \(\displaystyle {20}\)}%
\end{pgfscope}%
\begin{pgfscope}%
\pgfpathrectangle{\pgfqpoint{0.724269in}{0.608267in}}{\pgfqpoint{4.960000in}{3.696000in}}%
\pgfusepath{clip}%
\pgfsetrectcap%
\pgfsetroundjoin%
\pgfsetlinewidth{0.803000pt}%
\definecolor{currentstroke}{rgb}{0.690196,0.690196,0.690196}%
\pgfsetstrokecolor{currentstroke}%
\pgfsetdash{}{0pt}%
\pgfpathmoveto{\pgfqpoint{4.857602in}{0.608267in}}%
\pgfpathlineto{\pgfqpoint{4.857602in}{4.304267in}}%
\pgfusepath{stroke}%
\end{pgfscope}%
\begin{pgfscope}%
\pgfsetbuttcap%
\pgfsetroundjoin%
\definecolor{currentfill}{rgb}{0.000000,0.000000,0.000000}%
\pgfsetfillcolor{currentfill}%
\pgfsetlinewidth{0.803000pt}%
\definecolor{currentstroke}{rgb}{0.000000,0.000000,0.000000}%
\pgfsetstrokecolor{currentstroke}%
\pgfsetdash{}{0pt}%
\pgfsys@defobject{currentmarker}{\pgfqpoint{0.000000in}{-0.048611in}}{\pgfqpoint{0.000000in}{0.000000in}}{%
\pgfpathmoveto{\pgfqpoint{0.000000in}{0.000000in}}%
\pgfpathlineto{\pgfqpoint{0.000000in}{-0.048611in}}%
\pgfusepath{stroke,fill}%
}%
\begin{pgfscope}%
\pgfsys@transformshift{4.857602in}{0.608267in}%
\pgfsys@useobject{currentmarker}{}%
\end{pgfscope}%
\end{pgfscope}%
\begin{pgfscope}%
\definecolor{textcolor}{rgb}{0.000000,0.000000,0.000000}%
\pgfsetstrokecolor{textcolor}%
\pgfsetfillcolor{textcolor}%
\pgftext[x=4.857602in,y=0.511045in,,top]{\color{textcolor}\rmfamily\fontsize{14.000000}{16.800000}\selectfont \(\displaystyle {25}\)}%
\end{pgfscope}%
\begin{pgfscope}%
\pgfpathrectangle{\pgfqpoint{0.724269in}{0.608267in}}{\pgfqpoint{4.960000in}{3.696000in}}%
\pgfusepath{clip}%
\pgfsetrectcap%
\pgfsetroundjoin%
\pgfsetlinewidth{0.803000pt}%
\definecolor{currentstroke}{rgb}{0.690196,0.690196,0.690196}%
\pgfsetstrokecolor{currentstroke}%
\pgfsetdash{}{0pt}%
\pgfpathmoveto{\pgfqpoint{5.684269in}{0.608267in}}%
\pgfpathlineto{\pgfqpoint{5.684269in}{4.304267in}}%
\pgfusepath{stroke}%
\end{pgfscope}%
\begin{pgfscope}%
\pgfsetbuttcap%
\pgfsetroundjoin%
\definecolor{currentfill}{rgb}{0.000000,0.000000,0.000000}%
\pgfsetfillcolor{currentfill}%
\pgfsetlinewidth{0.803000pt}%
\definecolor{currentstroke}{rgb}{0.000000,0.000000,0.000000}%
\pgfsetstrokecolor{currentstroke}%
\pgfsetdash{}{0pt}%
\pgfsys@defobject{currentmarker}{\pgfqpoint{0.000000in}{-0.048611in}}{\pgfqpoint{0.000000in}{0.000000in}}{%
\pgfpathmoveto{\pgfqpoint{0.000000in}{0.000000in}}%
\pgfpathlineto{\pgfqpoint{0.000000in}{-0.048611in}}%
\pgfusepath{stroke,fill}%
}%
\begin{pgfscope}%
\pgfsys@transformshift{5.684269in}{0.608267in}%
\pgfsys@useobject{currentmarker}{}%
\end{pgfscope}%
\end{pgfscope}%
\begin{pgfscope}%
\definecolor{textcolor}{rgb}{0.000000,0.000000,0.000000}%
\pgfsetstrokecolor{textcolor}%
\pgfsetfillcolor{textcolor}%
\pgftext[x=5.684269in,y=0.511045in,,top]{\color{textcolor}\rmfamily\fontsize{14.000000}{16.800000}\selectfont \(\displaystyle {30}\)}%
\end{pgfscope}%
\begin{pgfscope}%
\definecolor{textcolor}{rgb}{0.000000,0.000000,0.000000}%
\pgfsetstrokecolor{textcolor}%
\pgfsetfillcolor{textcolor}%
\pgftext[x=3.204269in,y=0.277745in,,top]{\color{textcolor}\rmfamily\fontsize{14.000000}{16.800000}\selectfont \(\displaystyle t,\ с\)}%
\end{pgfscope}%
\begin{pgfscope}%
\pgfpathrectangle{\pgfqpoint{0.724269in}{0.608267in}}{\pgfqpoint{4.960000in}{3.696000in}}%
\pgfusepath{clip}%
\pgfsetrectcap%
\pgfsetroundjoin%
\pgfsetlinewidth{0.803000pt}%
\definecolor{currentstroke}{rgb}{0.690196,0.690196,0.690196}%
\pgfsetstrokecolor{currentstroke}%
\pgfsetdash{}{0pt}%
\pgfpathmoveto{\pgfqpoint{0.724269in}{0.608267in}}%
\pgfpathlineto{\pgfqpoint{5.684269in}{0.608267in}}%
\pgfusepath{stroke}%
\end{pgfscope}%
\begin{pgfscope}%
\pgfsetbuttcap%
\pgfsetroundjoin%
\definecolor{currentfill}{rgb}{0.000000,0.000000,0.000000}%
\pgfsetfillcolor{currentfill}%
\pgfsetlinewidth{0.803000pt}%
\definecolor{currentstroke}{rgb}{0.000000,0.000000,0.000000}%
\pgfsetstrokecolor{currentstroke}%
\pgfsetdash{}{0pt}%
\pgfsys@defobject{currentmarker}{\pgfqpoint{-0.048611in}{0.000000in}}{\pgfqpoint{-0.000000in}{0.000000in}}{%
\pgfpathmoveto{\pgfqpoint{-0.000000in}{0.000000in}}%
\pgfpathlineto{\pgfqpoint{-0.048611in}{0.000000in}}%
\pgfusepath{stroke,fill}%
}%
\begin{pgfscope}%
\pgfsys@transformshift{0.724269in}{0.608267in}%
\pgfsys@useobject{currentmarker}{}%
\end{pgfscope}%
\end{pgfscope}%
\begin{pgfscope}%
\definecolor{textcolor}{rgb}{0.000000,0.000000,0.000000}%
\pgfsetstrokecolor{textcolor}%
\pgfsetfillcolor{textcolor}%
\pgftext[x=0.529131in, y=0.538840in, left, base]{\color{textcolor}\rmfamily\fontsize{14.000000}{16.800000}\selectfont \(\displaystyle {0}\)}%
\end{pgfscope}%
\begin{pgfscope}%
\pgfpathrectangle{\pgfqpoint{0.724269in}{0.608267in}}{\pgfqpoint{4.960000in}{3.696000in}}%
\pgfusepath{clip}%
\pgfsetrectcap%
\pgfsetroundjoin%
\pgfsetlinewidth{0.803000pt}%
\definecolor{currentstroke}{rgb}{0.690196,0.690196,0.690196}%
\pgfsetstrokecolor{currentstroke}%
\pgfsetdash{}{0pt}%
\pgfpathmoveto{\pgfqpoint{0.724269in}{1.147367in}}%
\pgfpathlineto{\pgfqpoint{5.684269in}{1.147367in}}%
\pgfusepath{stroke}%
\end{pgfscope}%
\begin{pgfscope}%
\pgfsetbuttcap%
\pgfsetroundjoin%
\definecolor{currentfill}{rgb}{0.000000,0.000000,0.000000}%
\pgfsetfillcolor{currentfill}%
\pgfsetlinewidth{0.803000pt}%
\definecolor{currentstroke}{rgb}{0.000000,0.000000,0.000000}%
\pgfsetstrokecolor{currentstroke}%
\pgfsetdash{}{0pt}%
\pgfsys@defobject{currentmarker}{\pgfqpoint{-0.048611in}{0.000000in}}{\pgfqpoint{-0.000000in}{0.000000in}}{%
\pgfpathmoveto{\pgfqpoint{-0.000000in}{0.000000in}}%
\pgfpathlineto{\pgfqpoint{-0.048611in}{0.000000in}}%
\pgfusepath{stroke,fill}%
}%
\begin{pgfscope}%
\pgfsys@transformshift{0.724269in}{1.147367in}%
\pgfsys@useobject{currentmarker}{}%
\end{pgfscope}%
\end{pgfscope}%
\begin{pgfscope}%
\definecolor{textcolor}{rgb}{0.000000,0.000000,0.000000}%
\pgfsetstrokecolor{textcolor}%
\pgfsetfillcolor{textcolor}%
\pgftext[x=0.431216in, y=1.077939in, left, base]{\color{textcolor}\rmfamily\fontsize{14.000000}{16.800000}\selectfont \(\displaystyle {20}\)}%
\end{pgfscope}%
\begin{pgfscope}%
\pgfpathrectangle{\pgfqpoint{0.724269in}{0.608267in}}{\pgfqpoint{4.960000in}{3.696000in}}%
\pgfusepath{clip}%
\pgfsetrectcap%
\pgfsetroundjoin%
\pgfsetlinewidth{0.803000pt}%
\definecolor{currentstroke}{rgb}{0.690196,0.690196,0.690196}%
\pgfsetstrokecolor{currentstroke}%
\pgfsetdash{}{0pt}%
\pgfpathmoveto{\pgfqpoint{0.724269in}{1.686466in}}%
\pgfpathlineto{\pgfqpoint{5.684269in}{1.686466in}}%
\pgfusepath{stroke}%
\end{pgfscope}%
\begin{pgfscope}%
\pgfsetbuttcap%
\pgfsetroundjoin%
\definecolor{currentfill}{rgb}{0.000000,0.000000,0.000000}%
\pgfsetfillcolor{currentfill}%
\pgfsetlinewidth{0.803000pt}%
\definecolor{currentstroke}{rgb}{0.000000,0.000000,0.000000}%
\pgfsetstrokecolor{currentstroke}%
\pgfsetdash{}{0pt}%
\pgfsys@defobject{currentmarker}{\pgfqpoint{-0.048611in}{0.000000in}}{\pgfqpoint{-0.000000in}{0.000000in}}{%
\pgfpathmoveto{\pgfqpoint{-0.000000in}{0.000000in}}%
\pgfpathlineto{\pgfqpoint{-0.048611in}{0.000000in}}%
\pgfusepath{stroke,fill}%
}%
\begin{pgfscope}%
\pgfsys@transformshift{0.724269in}{1.686466in}%
\pgfsys@useobject{currentmarker}{}%
\end{pgfscope}%
\end{pgfscope}%
\begin{pgfscope}%
\definecolor{textcolor}{rgb}{0.000000,0.000000,0.000000}%
\pgfsetstrokecolor{textcolor}%
\pgfsetfillcolor{textcolor}%
\pgftext[x=0.431216in, y=1.617039in, left, base]{\color{textcolor}\rmfamily\fontsize{14.000000}{16.800000}\selectfont \(\displaystyle {40}\)}%
\end{pgfscope}%
\begin{pgfscope}%
\pgfpathrectangle{\pgfqpoint{0.724269in}{0.608267in}}{\pgfqpoint{4.960000in}{3.696000in}}%
\pgfusepath{clip}%
\pgfsetrectcap%
\pgfsetroundjoin%
\pgfsetlinewidth{0.803000pt}%
\definecolor{currentstroke}{rgb}{0.690196,0.690196,0.690196}%
\pgfsetstrokecolor{currentstroke}%
\pgfsetdash{}{0pt}%
\pgfpathmoveto{\pgfqpoint{0.724269in}{2.225566in}}%
\pgfpathlineto{\pgfqpoint{5.684269in}{2.225566in}}%
\pgfusepath{stroke}%
\end{pgfscope}%
\begin{pgfscope}%
\pgfsetbuttcap%
\pgfsetroundjoin%
\definecolor{currentfill}{rgb}{0.000000,0.000000,0.000000}%
\pgfsetfillcolor{currentfill}%
\pgfsetlinewidth{0.803000pt}%
\definecolor{currentstroke}{rgb}{0.000000,0.000000,0.000000}%
\pgfsetstrokecolor{currentstroke}%
\pgfsetdash{}{0pt}%
\pgfsys@defobject{currentmarker}{\pgfqpoint{-0.048611in}{0.000000in}}{\pgfqpoint{-0.000000in}{0.000000in}}{%
\pgfpathmoveto{\pgfqpoint{-0.000000in}{0.000000in}}%
\pgfpathlineto{\pgfqpoint{-0.048611in}{0.000000in}}%
\pgfusepath{stroke,fill}%
}%
\begin{pgfscope}%
\pgfsys@transformshift{0.724269in}{2.225566in}%
\pgfsys@useobject{currentmarker}{}%
\end{pgfscope}%
\end{pgfscope}%
\begin{pgfscope}%
\definecolor{textcolor}{rgb}{0.000000,0.000000,0.000000}%
\pgfsetstrokecolor{textcolor}%
\pgfsetfillcolor{textcolor}%
\pgftext[x=0.431216in, y=2.156138in, left, base]{\color{textcolor}\rmfamily\fontsize{14.000000}{16.800000}\selectfont \(\displaystyle {60}\)}%
\end{pgfscope}%
\begin{pgfscope}%
\pgfpathrectangle{\pgfqpoint{0.724269in}{0.608267in}}{\pgfqpoint{4.960000in}{3.696000in}}%
\pgfusepath{clip}%
\pgfsetrectcap%
\pgfsetroundjoin%
\pgfsetlinewidth{0.803000pt}%
\definecolor{currentstroke}{rgb}{0.690196,0.690196,0.690196}%
\pgfsetstrokecolor{currentstroke}%
\pgfsetdash{}{0pt}%
\pgfpathmoveto{\pgfqpoint{0.724269in}{2.764665in}}%
\pgfpathlineto{\pgfqpoint{5.684269in}{2.764665in}}%
\pgfusepath{stroke}%
\end{pgfscope}%
\begin{pgfscope}%
\pgfsetbuttcap%
\pgfsetroundjoin%
\definecolor{currentfill}{rgb}{0.000000,0.000000,0.000000}%
\pgfsetfillcolor{currentfill}%
\pgfsetlinewidth{0.803000pt}%
\definecolor{currentstroke}{rgb}{0.000000,0.000000,0.000000}%
\pgfsetstrokecolor{currentstroke}%
\pgfsetdash{}{0pt}%
\pgfsys@defobject{currentmarker}{\pgfqpoint{-0.048611in}{0.000000in}}{\pgfqpoint{-0.000000in}{0.000000in}}{%
\pgfpathmoveto{\pgfqpoint{-0.000000in}{0.000000in}}%
\pgfpathlineto{\pgfqpoint{-0.048611in}{0.000000in}}%
\pgfusepath{stroke,fill}%
}%
\begin{pgfscope}%
\pgfsys@transformshift{0.724269in}{2.764665in}%
\pgfsys@useobject{currentmarker}{}%
\end{pgfscope}%
\end{pgfscope}%
\begin{pgfscope}%
\definecolor{textcolor}{rgb}{0.000000,0.000000,0.000000}%
\pgfsetstrokecolor{textcolor}%
\pgfsetfillcolor{textcolor}%
\pgftext[x=0.431216in, y=2.695238in, left, base]{\color{textcolor}\rmfamily\fontsize{14.000000}{16.800000}\selectfont \(\displaystyle {80}\)}%
\end{pgfscope}%
\begin{pgfscope}%
\pgfpathrectangle{\pgfqpoint{0.724269in}{0.608267in}}{\pgfqpoint{4.960000in}{3.696000in}}%
\pgfusepath{clip}%
\pgfsetrectcap%
\pgfsetroundjoin%
\pgfsetlinewidth{0.803000pt}%
\definecolor{currentstroke}{rgb}{0.690196,0.690196,0.690196}%
\pgfsetstrokecolor{currentstroke}%
\pgfsetdash{}{0pt}%
\pgfpathmoveto{\pgfqpoint{0.724269in}{3.303764in}}%
\pgfpathlineto{\pgfqpoint{5.684269in}{3.303764in}}%
\pgfusepath{stroke}%
\end{pgfscope}%
\begin{pgfscope}%
\pgfsetbuttcap%
\pgfsetroundjoin%
\definecolor{currentfill}{rgb}{0.000000,0.000000,0.000000}%
\pgfsetfillcolor{currentfill}%
\pgfsetlinewidth{0.803000pt}%
\definecolor{currentstroke}{rgb}{0.000000,0.000000,0.000000}%
\pgfsetstrokecolor{currentstroke}%
\pgfsetdash{}{0pt}%
\pgfsys@defobject{currentmarker}{\pgfqpoint{-0.048611in}{0.000000in}}{\pgfqpoint{-0.000000in}{0.000000in}}{%
\pgfpathmoveto{\pgfqpoint{-0.000000in}{0.000000in}}%
\pgfpathlineto{\pgfqpoint{-0.048611in}{0.000000in}}%
\pgfusepath{stroke,fill}%
}%
\begin{pgfscope}%
\pgfsys@transformshift{0.724269in}{3.303764in}%
\pgfsys@useobject{currentmarker}{}%
\end{pgfscope}%
\end{pgfscope}%
\begin{pgfscope}%
\definecolor{textcolor}{rgb}{0.000000,0.000000,0.000000}%
\pgfsetstrokecolor{textcolor}%
\pgfsetfillcolor{textcolor}%
\pgftext[x=0.333300in, y=3.234337in, left, base]{\color{textcolor}\rmfamily\fontsize{14.000000}{16.800000}\selectfont \(\displaystyle {100}\)}%
\end{pgfscope}%
\begin{pgfscope}%
\pgfpathrectangle{\pgfqpoint{0.724269in}{0.608267in}}{\pgfqpoint{4.960000in}{3.696000in}}%
\pgfusepath{clip}%
\pgfsetrectcap%
\pgfsetroundjoin%
\pgfsetlinewidth{0.803000pt}%
\definecolor{currentstroke}{rgb}{0.690196,0.690196,0.690196}%
\pgfsetstrokecolor{currentstroke}%
\pgfsetdash{}{0pt}%
\pgfpathmoveto{\pgfqpoint{0.724269in}{3.842864in}}%
\pgfpathlineto{\pgfqpoint{5.684269in}{3.842864in}}%
\pgfusepath{stroke}%
\end{pgfscope}%
\begin{pgfscope}%
\pgfsetbuttcap%
\pgfsetroundjoin%
\definecolor{currentfill}{rgb}{0.000000,0.000000,0.000000}%
\pgfsetfillcolor{currentfill}%
\pgfsetlinewidth{0.803000pt}%
\definecolor{currentstroke}{rgb}{0.000000,0.000000,0.000000}%
\pgfsetstrokecolor{currentstroke}%
\pgfsetdash{}{0pt}%
\pgfsys@defobject{currentmarker}{\pgfqpoint{-0.048611in}{0.000000in}}{\pgfqpoint{-0.000000in}{0.000000in}}{%
\pgfpathmoveto{\pgfqpoint{-0.000000in}{0.000000in}}%
\pgfpathlineto{\pgfqpoint{-0.048611in}{0.000000in}}%
\pgfusepath{stroke,fill}%
}%
\begin{pgfscope}%
\pgfsys@transformshift{0.724269in}{3.842864in}%
\pgfsys@useobject{currentmarker}{}%
\end{pgfscope}%
\end{pgfscope}%
\begin{pgfscope}%
\definecolor{textcolor}{rgb}{0.000000,0.000000,0.000000}%
\pgfsetstrokecolor{textcolor}%
\pgfsetfillcolor{textcolor}%
\pgftext[x=0.333300in, y=3.773436in, left, base]{\color{textcolor}\rmfamily\fontsize{14.000000}{16.800000}\selectfont \(\displaystyle {120}\)}%
\end{pgfscope}%
\begin{pgfscope}%
\definecolor{textcolor}{rgb}{0.000000,0.000000,0.000000}%
\pgfsetstrokecolor{textcolor}%
\pgfsetfillcolor{textcolor}%
\pgftext[x=0.277745in,y=2.456267in,,bottom,rotate=90.000000]{\color{textcolor}\rmfamily\fontsize{14.000000}{16.800000}\selectfont \(\displaystyle \Delta H,\ м\)}%
\end{pgfscope}%
\begin{pgfscope}%
\pgfpathrectangle{\pgfqpoint{0.724269in}{0.608267in}}{\pgfqpoint{4.960000in}{3.696000in}}%
\pgfusepath{clip}%
\pgfsetrectcap%
\pgfsetroundjoin%
\pgfsetlinewidth{2.007500pt}%
\definecolor{currentstroke}{rgb}{0.121569,0.466667,0.705882}%
\pgfsetstrokecolor{currentstroke}%
\pgfsetdash{}{0pt}%
\pgfpathmoveto{\pgfqpoint{0.724269in}{0.608267in}}%
\pgfpathlineto{\pgfqpoint{0.976402in}{0.609378in}}%
\pgfpathlineto{\pgfqpoint{1.006989in}{0.612233in}}%
\pgfpathlineto{\pgfqpoint{1.033773in}{0.616944in}}%
\pgfpathlineto{\pgfqpoint{1.059069in}{0.623642in}}%
\pgfpathlineto{\pgfqpoint{1.083703in}{0.632471in}}%
\pgfpathlineto{\pgfqpoint{1.108173in}{0.643619in}}%
\pgfpathlineto{\pgfqpoint{1.132807in}{0.657311in}}%
\pgfpathlineto{\pgfqpoint{1.157773in}{0.673747in}}%
\pgfpathlineto{\pgfqpoint{1.183399in}{0.693300in}}%
\pgfpathlineto{\pgfqpoint{1.209853in}{0.716306in}}%
\pgfpathlineto{\pgfqpoint{1.237463in}{0.743320in}}%
\pgfpathlineto{\pgfqpoint{1.266231in}{0.774640in}}%
\pgfpathlineto{\pgfqpoint{1.296322in}{0.810749in}}%
\pgfpathlineto{\pgfqpoint{1.327901in}{0.852180in}}%
\pgfpathlineto{\pgfqpoint{1.361133in}{0.899507in}}%
\pgfpathlineto{\pgfqpoint{1.396183in}{0.953349in}}%
\pgfpathlineto{\pgfqpoint{1.433383in}{1.014640in}}%
\pgfpathlineto{\pgfqpoint{1.472898in}{1.084118in}}%
\pgfpathlineto{\pgfqpoint{1.514893in}{1.162543in}}%
\pgfpathlineto{\pgfqpoint{1.559698in}{1.251021in}}%
\pgfpathlineto{\pgfqpoint{1.607479in}{1.350378in}}%
\pgfpathlineto{\pgfqpoint{1.658733in}{1.462170in}}%
\pgfpathlineto{\pgfqpoint{1.713954in}{1.588059in}}%
\pgfpathlineto{\pgfqpoint{1.773474in}{1.729391in}}%
\pgfpathlineto{\pgfqpoint{1.837954in}{1.888340in}}%
\pgfpathlineto{\pgfqpoint{1.909874in}{2.071727in}}%
\pgfpathlineto{\pgfqpoint{2.010727in}{2.335692in}}%
\pgfpathlineto{\pgfqpoint{2.143490in}{2.682247in}}%
\pgfpathlineto{\pgfqpoint{2.212103in}{2.854931in}}%
\pgfpathlineto{\pgfqpoint{2.269639in}{2.993875in}}%
\pgfpathlineto{\pgfqpoint{2.321058in}{3.112375in}}%
\pgfpathlineto{\pgfqpoint{2.368343in}{3.215886in}}%
\pgfpathlineto{\pgfqpoint{2.412653in}{3.307623in}}%
\pgfpathlineto{\pgfqpoint{2.454482in}{3.389198in}}%
\pgfpathlineto{\pgfqpoint{2.494327in}{3.462102in}}%
\pgfpathlineto{\pgfqpoint{2.532354in}{3.527126in}}%
\pgfpathlineto{\pgfqpoint{2.568893in}{3.585288in}}%
\pgfpathlineto{\pgfqpoint{2.604274in}{3.637486in}}%
\pgfpathlineto{\pgfqpoint{2.638498in}{3.684059in}}%
\pgfpathlineto{\pgfqpoint{2.671565in}{3.725365in}}%
\pgfpathlineto{\pgfqpoint{2.703805in}{3.762133in}}%
\pgfpathlineto{\pgfqpoint{2.735218in}{3.794630in}}%
\pgfpathlineto{\pgfqpoint{2.765970in}{3.823274in}}%
\pgfpathlineto{\pgfqpoint{2.796061in}{3.848292in}}%
\pgfpathlineto{\pgfqpoint{2.825655in}{3.870025in}}%
\pgfpathlineto{\pgfqpoint{2.854754in}{3.888655in}}%
\pgfpathlineto{\pgfqpoint{2.883522in}{3.904449in}}%
\pgfpathlineto{\pgfqpoint{2.911959in}{3.917547in}}%
\pgfpathlineto{\pgfqpoint{2.940231in}{3.928144in}}%
\pgfpathlineto{\pgfqpoint{2.968503in}{3.936384in}}%
\pgfpathlineto{\pgfqpoint{2.996941in}{3.942359in}}%
\pgfpathlineto{\pgfqpoint{3.025543in}{3.946100in}}%
\pgfpathlineto{\pgfqpoint{3.054477in}{3.947646in}}%
\pgfpathlineto{\pgfqpoint{3.083906in}{3.946997in}}%
\pgfpathlineto{\pgfqpoint{3.113997in}{3.944110in}}%
\pgfpathlineto{\pgfqpoint{3.145079in}{3.938875in}}%
\pgfpathlineto{\pgfqpoint{3.177154in}{3.931195in}}%
\pgfpathlineto{\pgfqpoint{3.210386in}{3.920935in}}%
\pgfpathlineto{\pgfqpoint{3.245106in}{3.907874in}}%
\pgfpathlineto{\pgfqpoint{3.281645in}{3.891732in}}%
\pgfpathlineto{\pgfqpoint{3.320498in}{3.872096in}}%
\pgfpathlineto{\pgfqpoint{3.361997in}{3.848586in}}%
\pgfpathlineto{\pgfqpoint{3.406802in}{3.820602in}}%
\pgfpathlineto{\pgfqpoint{3.456071in}{3.787155in}}%
\pgfpathlineto{\pgfqpoint{3.511623in}{3.746688in}}%
\pgfpathlineto{\pgfqpoint{3.577426in}{3.695894in}}%
\pgfpathlineto{\pgfqpoint{3.665879in}{3.624581in}}%
\pgfpathlineto{\pgfqpoint{3.868743in}{3.460171in}}%
\pgfpathlineto{\pgfqpoint{3.942647in}{3.403728in}}%
\pgfpathlineto{\pgfqpoint{4.006962in}{3.357359in}}%
\pgfpathlineto{\pgfqpoint{4.065821in}{3.317628in}}%
\pgfpathlineto{\pgfqpoint{4.121042in}{3.283002in}}%
\pgfpathlineto{\pgfqpoint{4.173783in}{3.252531in}}%
\pgfpathlineto{\pgfqpoint{4.224706in}{3.225661in}}%
\pgfpathlineto{\pgfqpoint{4.274141in}{3.202068in}}%
\pgfpathlineto{\pgfqpoint{4.322749in}{3.181321in}}%
\pgfpathlineto{\pgfqpoint{4.370695in}{3.163270in}}%
\pgfpathlineto{\pgfqpoint{4.418146in}{3.147771in}}%
\pgfpathlineto{\pgfqpoint{4.465431in}{3.134652in}}%
\pgfpathlineto{\pgfqpoint{4.512882in}{3.123788in}}%
\pgfpathlineto{\pgfqpoint{4.560663in}{3.115124in}}%
\pgfpathlineto{\pgfqpoint{4.609106in}{3.108601in}}%
\pgfpathlineto{\pgfqpoint{4.658375in}{3.104209in}}%
\pgfpathlineto{\pgfqpoint{4.708967in}{3.101942in}}%
\pgfpathlineto{\pgfqpoint{4.761047in}{3.101843in}}%
\pgfpathlineto{\pgfqpoint{4.815277in}{3.103982in}}%
\pgfpathlineto{\pgfqpoint{4.871986in}{3.108463in}}%
\pgfpathlineto{\pgfqpoint{4.932167in}{3.115476in}}%
\pgfpathlineto{\pgfqpoint{4.996813in}{3.125281in}}%
\pgfpathlineto{\pgfqpoint{5.067741in}{3.138331in}}%
\pgfpathlineto{\pgfqpoint{5.148258in}{3.155468in}}%
\pgfpathlineto{\pgfqpoint{5.245970in}{3.178641in}}%
\pgfpathlineto{\pgfqpoint{5.398077in}{3.217331in}}%
\pgfpathlineto{\pgfqpoint{5.580770in}{3.263246in}}%
\pgfpathlineto{\pgfqpoint{5.684269in}{3.287069in}}%
\pgfpathlineto{\pgfqpoint{5.684269in}{3.287069in}}%
\pgfusepath{stroke}%
\end{pgfscope}%
\begin{pgfscope}%
\pgfpathrectangle{\pgfqpoint{0.724269in}{0.608267in}}{\pgfqpoint{4.960000in}{3.696000in}}%
\pgfusepath{clip}%
\pgfsetbuttcap%
\pgfsetroundjoin%
\pgfsetlinewidth{2.007500pt}%
\definecolor{currentstroke}{rgb}{1.000000,0.498039,0.054902}%
\pgfsetstrokecolor{currentstroke}%
\pgfsetdash{{7.400000pt}{3.200000pt}}{0.000000pt}%
\pgfpathmoveto{\pgfqpoint{0.724269in}{0.608267in}}%
\pgfpathlineto{\pgfqpoint{0.950994in}{0.608998in}}%
\pgfpathlineto{\pgfqpoint{0.977070in}{0.611418in}}%
\pgfpathlineto{\pgfqpoint{0.991583in}{0.613919in}}%
\pgfpathlineto{\pgfqpoint{1.011325in}{0.618979in}}%
\pgfpathlineto{\pgfqpoint{1.031963in}{0.626603in}}%
\pgfpathlineto{\pgfqpoint{1.056723in}{0.639255in}}%
\pgfpathlineto{\pgfqpoint{1.091973in}{0.664432in}}%
\pgfpathlineto{\pgfqpoint{1.130581in}{0.702185in}}%
\pgfpathlineto{\pgfqpoint{1.165698in}{0.745930in}}%
\pgfpathlineto{\pgfqpoint{1.198501in}{0.794792in}}%
\pgfpathlineto{\pgfqpoint{1.228562in}{0.846150in}}%
\pgfpathlineto{\pgfqpoint{1.258354in}{0.903000in}}%
\pgfpathlineto{\pgfqpoint{1.290457in}{0.970537in}}%
\pgfpathlineto{\pgfqpoint{1.331264in}{1.065117in}}%
\pgfpathlineto{\pgfqpoint{1.360986in}{1.139645in}}%
\pgfpathlineto{\pgfqpoint{1.390708in}{1.218483in}}%
\pgfpathlineto{\pgfqpoint{1.424577in}{1.313054in}}%
\pgfpathlineto{\pgfqpoint{1.494878in}{1.522872in}}%
\pgfpathlineto{\pgfqpoint{1.557362in}{1.721042in}}%
\pgfpathlineto{\pgfqpoint{1.655804in}{2.046194in}}%
\pgfpathlineto{\pgfqpoint{1.787018in}{2.483368in}}%
\pgfpathlineto{\pgfqpoint{1.858515in}{2.714174in}}%
\pgfpathlineto{\pgfqpoint{1.926651in}{2.924597in}}%
\pgfpathlineto{\pgfqpoint{1.993917in}{3.120351in}}%
\pgfpathlineto{\pgfqpoint{2.019729in}{3.191805in}}%
\pgfpathlineto{\pgfqpoint{2.053452in}{3.281807in}}%
\pgfpathlineto{\pgfqpoint{2.096499in}{3.390854in}}%
\pgfpathlineto{\pgfqpoint{2.150388in}{3.517565in}}%
\pgfpathlineto{\pgfqpoint{2.193867in}{3.611458in}}%
\pgfpathlineto{\pgfqpoint{2.223400in}{3.670840in}}%
\pgfpathlineto{\pgfqpoint{2.252933in}{3.726596in}}%
\pgfpathlineto{\pgfqpoint{2.290485in}{3.792187in}}%
\pgfpathlineto{\pgfqpoint{2.325784in}{3.848378in}}%
\pgfpathlineto{\pgfqpoint{2.361082in}{3.899254in}}%
\pgfpathlineto{\pgfqpoint{2.393648in}{3.941482in}}%
\pgfpathlineto{\pgfqpoint{2.423738in}{3.976505in}}%
\pgfpathlineto{\pgfqpoint{2.458800in}{4.012515in}}%
\pgfpathlineto{\pgfqpoint{2.496583in}{4.045622in}}%
\pgfpathlineto{\pgfqpoint{2.530534in}{4.070422in}}%
\pgfpathlineto{\pgfqpoint{2.557994in}{4.087125in}}%
\pgfpathlineto{\pgfqpoint{2.586864in}{4.101525in}}%
\pgfpathlineto{\pgfqpoint{2.624066in}{4.115431in}}%
\pgfpathlineto{\pgfqpoint{2.664079in}{4.124743in}}%
\pgfpathlineto{\pgfqpoint{2.696909in}{4.128190in}}%
\pgfpathlineto{\pgfqpoint{2.724905in}{4.128267in}}%
\pgfpathlineto{\pgfqpoint{2.756721in}{4.125289in}}%
\pgfpathlineto{\pgfqpoint{2.791777in}{4.118406in}}%
\pgfpathlineto{\pgfqpoint{2.826439in}{4.108089in}}%
\pgfpathlineto{\pgfqpoint{2.860334in}{4.094828in}}%
\pgfpathlineto{\pgfqpoint{2.897109in}{4.077130in}}%
\pgfpathlineto{\pgfqpoint{2.929172in}{4.059089in}}%
\pgfpathlineto{\pgfqpoint{2.956664in}{4.041823in}}%
\pgfpathlineto{\pgfqpoint{2.985556in}{4.022023in}}%
\pgfpathlineto{\pgfqpoint{3.026887in}{3.991005in}}%
\pgfpathlineto{\pgfqpoint{3.086763in}{3.941157in}}%
\pgfpathlineto{\pgfqpoint{3.159147in}{3.874624in}}%
\pgfpathlineto{\pgfqpoint{3.227207in}{3.807470in}}%
\pgfpathlineto{\pgfqpoint{3.328949in}{3.702251in}}%
\pgfpathlineto{\pgfqpoint{3.491625in}{3.532722in}}%
\pgfpathlineto{\pgfqpoint{3.559805in}{3.464634in}}%
\pgfpathlineto{\pgfqpoint{3.630333in}{3.397663in}}%
\pgfpathlineto{\pgfqpoint{3.685014in}{3.348759in}}%
\pgfpathlineto{\pgfqpoint{3.755949in}{3.289899in}}%
\pgfpathlineto{\pgfqpoint{3.787896in}{3.265240in}}%
\pgfpathlineto{\pgfqpoint{3.825294in}{3.237924in}}%
\pgfpathlineto{\pgfqpoint{3.858039in}{3.215427in}}%
\pgfpathlineto{\pgfqpoint{3.890785in}{3.194296in}}%
\pgfpathlineto{\pgfqpoint{3.923794in}{3.174403in}}%
\pgfpathlineto{\pgfqpoint{3.957333in}{3.155663in}}%
\pgfpathlineto{\pgfqpoint{3.995251in}{3.136280in}}%
\pgfpathlineto{\pgfqpoint{4.031417in}{3.119586in}}%
\pgfpathlineto{\pgfqpoint{4.086398in}{3.097557in}}%
\pgfpathlineto{\pgfqpoint{4.119044in}{3.086377in}}%
\pgfpathlineto{\pgfqpoint{4.159221in}{3.074531in}}%
\pgfpathlineto{\pgfqpoint{4.217509in}{3.061020in}}%
\pgfpathlineto{\pgfqpoint{4.259948in}{3.053837in}}%
\pgfpathlineto{\pgfqpoint{4.319918in}{3.047336in}}%
\pgfpathlineto{\pgfqpoint{4.354889in}{3.045428in}}%
\pgfpathlineto{\pgfqpoint{4.425416in}{3.045534in}}%
\pgfpathlineto{\pgfqpoint{4.484501in}{3.049388in}}%
\pgfpathlineto{\pgfqpoint{4.553999in}{3.057857in}}%
\pgfpathlineto{\pgfqpoint{4.623394in}{3.070040in}}%
\pgfpathlineto{\pgfqpoint{4.683827in}{3.083242in}}%
\pgfpathlineto{\pgfqpoint{4.756045in}{3.101637in}}%
\pgfpathlineto{\pgfqpoint{4.854698in}{3.130272in}}%
\pgfpathlineto{\pgfqpoint{4.986538in}{3.172379in}}%
\pgfpathlineto{\pgfqpoint{5.221020in}{3.248483in}}%
\pgfpathlineto{\pgfqpoint{5.330618in}{3.281015in}}%
\pgfpathlineto{\pgfqpoint{5.415530in}{3.303813in}}%
\pgfpathlineto{\pgfqpoint{5.512149in}{3.326666in}}%
\pgfpathlineto{\pgfqpoint{5.584808in}{3.341460in}}%
\pgfpathlineto{\pgfqpoint{5.652662in}{3.353320in}}%
\pgfpathlineto{\pgfqpoint{5.684269in}{3.358187in}}%
\pgfpathlineto{\pgfqpoint{5.684269in}{3.358187in}}%
\pgfusepath{stroke}%
\end{pgfscope}%
\begin{pgfscope}%
\pgfpathrectangle{\pgfqpoint{0.724269in}{0.608267in}}{\pgfqpoint{4.960000in}{3.696000in}}%
\pgfusepath{clip}%
\pgfsetbuttcap%
\pgfsetroundjoin%
\pgfsetlinewidth{4.015000pt}%
\definecolor{currentstroke}{rgb}{0.172549,0.627451,0.172549}%
\pgfsetstrokecolor{currentstroke}%
\pgfsetdash{{4.000000pt}{6.600000pt}}{0.000000pt}%
\pgfpathmoveto{\pgfqpoint{0.724269in}{0.608267in}}%
\pgfpathlineto{\pgfqpoint{0.889437in}{0.608267in}}%
\pgfpathlineto{\pgfqpoint{0.890925in}{3.303764in}}%
\pgfpathlineto{\pgfqpoint{5.684269in}{3.303764in}}%
\pgfpathlineto{\pgfqpoint{5.684269in}{3.303764in}}%
\pgfusepath{stroke}%
\end{pgfscope}%
\begin{pgfscope}%
\pgfsetrectcap%
\pgfsetmiterjoin%
\pgfsetlinewidth{0.803000pt}%
\definecolor{currentstroke}{rgb}{0.000000,0.000000,0.000000}%
\pgfsetstrokecolor{currentstroke}%
\pgfsetdash{}{0pt}%
\pgfpathmoveto{\pgfqpoint{0.724269in}{0.608267in}}%
\pgfpathlineto{\pgfqpoint{0.724269in}{4.304267in}}%
\pgfusepath{stroke}%
\end{pgfscope}%
\begin{pgfscope}%
\pgfsetrectcap%
\pgfsetmiterjoin%
\pgfsetlinewidth{0.803000pt}%
\definecolor{currentstroke}{rgb}{0.000000,0.000000,0.000000}%
\pgfsetstrokecolor{currentstroke}%
\pgfsetdash{}{0pt}%
\pgfpathmoveto{\pgfqpoint{5.684269in}{0.608267in}}%
\pgfpathlineto{\pgfqpoint{5.684269in}{4.304267in}}%
\pgfusepath{stroke}%
\end{pgfscope}%
\begin{pgfscope}%
\pgfsetrectcap%
\pgfsetmiterjoin%
\pgfsetlinewidth{0.803000pt}%
\definecolor{currentstroke}{rgb}{0.000000,0.000000,0.000000}%
\pgfsetstrokecolor{currentstroke}%
\pgfsetdash{}{0pt}%
\pgfpathmoveto{\pgfqpoint{0.724269in}{0.608267in}}%
\pgfpathlineto{\pgfqpoint{5.684269in}{0.608267in}}%
\pgfusepath{stroke}%
\end{pgfscope}%
\begin{pgfscope}%
\pgfsetrectcap%
\pgfsetmiterjoin%
\pgfsetlinewidth{0.803000pt}%
\definecolor{currentstroke}{rgb}{0.000000,0.000000,0.000000}%
\pgfsetstrokecolor{currentstroke}%
\pgfsetdash{}{0pt}%
\pgfpathmoveto{\pgfqpoint{0.724269in}{4.304267in}}%
\pgfpathlineto{\pgfqpoint{5.684269in}{4.304267in}}%
\pgfusepath{stroke}%
\end{pgfscope}%
\begin{pgfscope}%
\pgfsetbuttcap%
\pgfsetmiterjoin%
\definecolor{currentfill}{rgb}{1.000000,1.000000,1.000000}%
\pgfsetfillcolor{currentfill}%
\pgfsetfillopacity{0.800000}%
\pgfsetlinewidth{1.003750pt}%
\definecolor{currentstroke}{rgb}{0.800000,0.800000,0.800000}%
\pgfsetstrokecolor{currentstroke}%
\pgfsetstrokeopacity{0.800000}%
\pgfsetdash{}{0pt}%
\pgfpathmoveto{\pgfqpoint{2.271676in}{0.705489in}}%
\pgfpathlineto{\pgfqpoint{5.548158in}{0.705489in}}%
\pgfpathquadraticcurveto{\pgfqpoint{5.587047in}{0.705489in}}{\pgfqpoint{5.587047in}{0.744378in}}%
\pgfpathlineto{\pgfqpoint{5.587047in}{1.563444in}}%
\pgfpathquadraticcurveto{\pgfqpoint{5.587047in}{1.602333in}}{\pgfqpoint{5.548158in}{1.602333in}}%
\pgfpathlineto{\pgfqpoint{2.271676in}{1.602333in}}%
\pgfpathquadraticcurveto{\pgfqpoint{2.232787in}{1.602333in}}{\pgfqpoint{2.232787in}{1.563444in}}%
\pgfpathlineto{\pgfqpoint{2.232787in}{0.744378in}}%
\pgfpathquadraticcurveto{\pgfqpoint{2.232787in}{0.705489in}}{\pgfqpoint{2.271676in}{0.705489in}}%
\pgfpathclose%
\pgfusepath{stroke,fill}%
\end{pgfscope}%
\begin{pgfscope}%
\pgfsetrectcap%
\pgfsetroundjoin%
\pgfsetlinewidth{2.007500pt}%
\definecolor{currentstroke}{rgb}{0.121569,0.466667,0.705882}%
\pgfsetstrokecolor{currentstroke}%
\pgfsetdash{}{0pt}%
\pgfpathmoveto{\pgfqpoint{2.310565in}{1.453756in}}%
\pgfpathlineto{\pgfqpoint{2.699454in}{1.453756in}}%
\pgfusepath{stroke}%
\end{pgfscope}%
\begin{pgfscope}%
\definecolor{textcolor}{rgb}{0.000000,0.000000,0.000000}%
\pgfsetstrokecolor{textcolor}%
\pgfsetfillcolor{textcolor}%
\pgftext[x=2.855009in,y=1.385700in,left,base]{\color{textcolor}\rmfamily\fontsize{14.000000}{16.800000}\selectfont Нелинейная модель, \(\displaystyle M=0.61\)}%
\end{pgfscope}%
\begin{pgfscope}%
\pgfsetbuttcap%
\pgfsetroundjoin%
\pgfsetlinewidth{2.007500pt}%
\definecolor{currentstroke}{rgb}{1.000000,0.498039,0.054902}%
\pgfsetstrokecolor{currentstroke}%
\pgfsetdash{{7.400000pt}{3.200000pt}}{0.000000pt}%
\pgfpathmoveto{\pgfqpoint{2.310565in}{1.178789in}}%
\pgfpathlineto{\pgfqpoint{2.699454in}{1.178789in}}%
\pgfusepath{stroke}%
\end{pgfscope}%
\begin{pgfscope}%
\definecolor{textcolor}{rgb}{0.000000,0.000000,0.000000}%
\pgfsetstrokecolor{textcolor}%
\pgfsetfillcolor{textcolor}%
\pgftext[x=2.855009in,y=1.110733in,left,base]{\color{textcolor}\rmfamily\fontsize{14.000000}{16.800000}\selectfont Линейная модель, \(\displaystyle M=0.61\)}%
\end{pgfscope}%
\begin{pgfscope}%
\pgfsetbuttcap%
\pgfsetroundjoin%
\pgfsetlinewidth{4.015000pt}%
\definecolor{currentstroke}{rgb}{0.172549,0.627451,0.172549}%
\pgfsetstrokecolor{currentstroke}%
\pgfsetdash{{4.000000pt}{6.600000pt}}{0.000000pt}%
\pgfpathmoveto{\pgfqpoint{2.310565in}{0.903822in}}%
\pgfpathlineto{\pgfqpoint{2.699454in}{0.903822in}}%
\pgfusepath{stroke}%
\end{pgfscope}%
\begin{pgfscope}%
\definecolor{textcolor}{rgb}{0.000000,0.000000,0.000000}%
\pgfsetstrokecolor{textcolor}%
\pgfsetfillcolor{textcolor}%
\pgftext[x=2.855009in,y=0.835767in,left,base]{\color{textcolor}\rmfamily\fontsize{14.000000}{16.800000}\selectfont \(\displaystyle \Delta H_{зад}\)}%
\end{pgfscope}%
\end{pgfpicture}%
\makeatother%
\endgroup%
}
    \caption{Изменение высоты для линейной и нелинейной модели}
    \label{fig:model_Delta_H}
    \end{minipage}
    \hfill
    \begin{minipage}{0.48\textwidth}
    \centering
    \resizebox{1.1\linewidth}{!}{%% Creator: Matplotlib, PGF backend
%%
%% To include the figure in your LaTeX document, write
%%   \input{<filename>.pgf}
%%
%% Make sure the required packages are loaded in your preamble
%%   \usepackage{pgf}
%%
%% Figures using additional raster images can only be included by \input if
%% they are in the same directory as the main LaTeX file. For loading figures
%% from other directories you can use the `import` package
%%   \usepackage{import}
%%
%% and then include the figures with
%%   \import{<path to file>}{<filename>.pgf}
%%
%% Matplotlib used the following preamble
%%   \usepackage[warn]{mathtext}
%%   \usepackage[T2A]{fontenc}
%%   \usepackage[utf8]{inputenc}
%%   \usepackage[english,russian]{babel}
%%
\begingroup%
\makeatletter%
\begin{pgfpicture}%
\pgfpathrectangle{\pgfpointorigin}{\pgfqpoint{6.400000in}{4.800000in}}%
\pgfusepath{use as bounding box, clip}%
\begin{pgfscope}%
\pgfsetbuttcap%
\pgfsetmiterjoin%
\definecolor{currentfill}{rgb}{1.000000,1.000000,1.000000}%
\pgfsetfillcolor{currentfill}%
\pgfsetlinewidth{0.000000pt}%
\definecolor{currentstroke}{rgb}{1.000000,1.000000,1.000000}%
\pgfsetstrokecolor{currentstroke}%
\pgfsetdash{}{0pt}%
\pgfpathmoveto{\pgfqpoint{0.000000in}{0.000000in}}%
\pgfpathlineto{\pgfqpoint{6.400000in}{0.000000in}}%
\pgfpathlineto{\pgfqpoint{6.400000in}{4.800000in}}%
\pgfpathlineto{\pgfqpoint{0.000000in}{4.800000in}}%
\pgfpathclose%
\pgfusepath{fill}%
\end{pgfscope}%
\begin{pgfscope}%
\pgfsetbuttcap%
\pgfsetmiterjoin%
\definecolor{currentfill}{rgb}{1.000000,1.000000,1.000000}%
\pgfsetfillcolor{currentfill}%
\pgfsetlinewidth{0.000000pt}%
\definecolor{currentstroke}{rgb}{0.000000,0.000000,0.000000}%
\pgfsetstrokecolor{currentstroke}%
\pgfsetstrokeopacity{0.000000}%
\pgfsetdash{}{0pt}%
\pgfpathmoveto{\pgfqpoint{0.800000in}{0.528000in}}%
\pgfpathlineto{\pgfqpoint{5.760000in}{0.528000in}}%
\pgfpathlineto{\pgfqpoint{5.760000in}{4.224000in}}%
\pgfpathlineto{\pgfqpoint{0.800000in}{4.224000in}}%
\pgfpathclose%
\pgfusepath{fill}%
\end{pgfscope}%
\begin{pgfscope}%
\pgfpathrectangle{\pgfqpoint{0.800000in}{0.528000in}}{\pgfqpoint{4.960000in}{3.696000in}}%
\pgfusepath{clip}%
\pgfsetrectcap%
\pgfsetroundjoin%
\pgfsetlinewidth{0.803000pt}%
\definecolor{currentstroke}{rgb}{0.690196,0.690196,0.690196}%
\pgfsetstrokecolor{currentstroke}%
\pgfsetdash{}{0pt}%
\pgfpathmoveto{\pgfqpoint{0.800000in}{0.528000in}}%
\pgfpathlineto{\pgfqpoint{0.800000in}{4.224000in}}%
\pgfusepath{stroke}%
\end{pgfscope}%
\begin{pgfscope}%
\pgfsetbuttcap%
\pgfsetroundjoin%
\definecolor{currentfill}{rgb}{0.000000,0.000000,0.000000}%
\pgfsetfillcolor{currentfill}%
\pgfsetlinewidth{0.803000pt}%
\definecolor{currentstroke}{rgb}{0.000000,0.000000,0.000000}%
\pgfsetstrokecolor{currentstroke}%
\pgfsetdash{}{0pt}%
\pgfsys@defobject{currentmarker}{\pgfqpoint{0.000000in}{-0.048611in}}{\pgfqpoint{0.000000in}{0.000000in}}{%
\pgfpathmoveto{\pgfqpoint{0.000000in}{0.000000in}}%
\pgfpathlineto{\pgfqpoint{0.000000in}{-0.048611in}}%
\pgfusepath{stroke,fill}%
}%
\begin{pgfscope}%
\pgfsys@transformshift{0.800000in}{0.528000in}%
\pgfsys@useobject{currentmarker}{}%
\end{pgfscope}%
\end{pgfscope}%
\begin{pgfscope}%
\definecolor{textcolor}{rgb}{0.000000,0.000000,0.000000}%
\pgfsetstrokecolor{textcolor}%
\pgfsetfillcolor{textcolor}%
\pgftext[x=0.800000in,y=0.430778in,,top]{\color{textcolor}\rmfamily\fontsize{10.000000}{12.000000}\selectfont \(\displaystyle {0}\)}%
\end{pgfscope}%
\begin{pgfscope}%
\pgfpathrectangle{\pgfqpoint{0.800000in}{0.528000in}}{\pgfqpoint{4.960000in}{3.696000in}}%
\pgfusepath{clip}%
\pgfsetrectcap%
\pgfsetroundjoin%
\pgfsetlinewidth{0.803000pt}%
\definecolor{currentstroke}{rgb}{0.690196,0.690196,0.690196}%
\pgfsetstrokecolor{currentstroke}%
\pgfsetdash{}{0pt}%
\pgfpathmoveto{\pgfqpoint{1.626667in}{0.528000in}}%
\pgfpathlineto{\pgfqpoint{1.626667in}{4.224000in}}%
\pgfusepath{stroke}%
\end{pgfscope}%
\begin{pgfscope}%
\pgfsetbuttcap%
\pgfsetroundjoin%
\definecolor{currentfill}{rgb}{0.000000,0.000000,0.000000}%
\pgfsetfillcolor{currentfill}%
\pgfsetlinewidth{0.803000pt}%
\definecolor{currentstroke}{rgb}{0.000000,0.000000,0.000000}%
\pgfsetstrokecolor{currentstroke}%
\pgfsetdash{}{0pt}%
\pgfsys@defobject{currentmarker}{\pgfqpoint{0.000000in}{-0.048611in}}{\pgfqpoint{0.000000in}{0.000000in}}{%
\pgfpathmoveto{\pgfqpoint{0.000000in}{0.000000in}}%
\pgfpathlineto{\pgfqpoint{0.000000in}{-0.048611in}}%
\pgfusepath{stroke,fill}%
}%
\begin{pgfscope}%
\pgfsys@transformshift{1.626667in}{0.528000in}%
\pgfsys@useobject{currentmarker}{}%
\end{pgfscope}%
\end{pgfscope}%
\begin{pgfscope}%
\definecolor{textcolor}{rgb}{0.000000,0.000000,0.000000}%
\pgfsetstrokecolor{textcolor}%
\pgfsetfillcolor{textcolor}%
\pgftext[x=1.626667in,y=0.430778in,,top]{\color{textcolor}\rmfamily\fontsize{10.000000}{12.000000}\selectfont \(\displaystyle {5}\)}%
\end{pgfscope}%
\begin{pgfscope}%
\pgfpathrectangle{\pgfqpoint{0.800000in}{0.528000in}}{\pgfqpoint{4.960000in}{3.696000in}}%
\pgfusepath{clip}%
\pgfsetrectcap%
\pgfsetroundjoin%
\pgfsetlinewidth{0.803000pt}%
\definecolor{currentstroke}{rgb}{0.690196,0.690196,0.690196}%
\pgfsetstrokecolor{currentstroke}%
\pgfsetdash{}{0pt}%
\pgfpathmoveto{\pgfqpoint{2.453333in}{0.528000in}}%
\pgfpathlineto{\pgfqpoint{2.453333in}{4.224000in}}%
\pgfusepath{stroke}%
\end{pgfscope}%
\begin{pgfscope}%
\pgfsetbuttcap%
\pgfsetroundjoin%
\definecolor{currentfill}{rgb}{0.000000,0.000000,0.000000}%
\pgfsetfillcolor{currentfill}%
\pgfsetlinewidth{0.803000pt}%
\definecolor{currentstroke}{rgb}{0.000000,0.000000,0.000000}%
\pgfsetstrokecolor{currentstroke}%
\pgfsetdash{}{0pt}%
\pgfsys@defobject{currentmarker}{\pgfqpoint{0.000000in}{-0.048611in}}{\pgfqpoint{0.000000in}{0.000000in}}{%
\pgfpathmoveto{\pgfqpoint{0.000000in}{0.000000in}}%
\pgfpathlineto{\pgfqpoint{0.000000in}{-0.048611in}}%
\pgfusepath{stroke,fill}%
}%
\begin{pgfscope}%
\pgfsys@transformshift{2.453333in}{0.528000in}%
\pgfsys@useobject{currentmarker}{}%
\end{pgfscope}%
\end{pgfscope}%
\begin{pgfscope}%
\definecolor{textcolor}{rgb}{0.000000,0.000000,0.000000}%
\pgfsetstrokecolor{textcolor}%
\pgfsetfillcolor{textcolor}%
\pgftext[x=2.453333in,y=0.430778in,,top]{\color{textcolor}\rmfamily\fontsize{10.000000}{12.000000}\selectfont \(\displaystyle {10}\)}%
\end{pgfscope}%
\begin{pgfscope}%
\pgfpathrectangle{\pgfqpoint{0.800000in}{0.528000in}}{\pgfqpoint{4.960000in}{3.696000in}}%
\pgfusepath{clip}%
\pgfsetrectcap%
\pgfsetroundjoin%
\pgfsetlinewidth{0.803000pt}%
\definecolor{currentstroke}{rgb}{0.690196,0.690196,0.690196}%
\pgfsetstrokecolor{currentstroke}%
\pgfsetdash{}{0pt}%
\pgfpathmoveto{\pgfqpoint{3.280000in}{0.528000in}}%
\pgfpathlineto{\pgfqpoint{3.280000in}{4.224000in}}%
\pgfusepath{stroke}%
\end{pgfscope}%
\begin{pgfscope}%
\pgfsetbuttcap%
\pgfsetroundjoin%
\definecolor{currentfill}{rgb}{0.000000,0.000000,0.000000}%
\pgfsetfillcolor{currentfill}%
\pgfsetlinewidth{0.803000pt}%
\definecolor{currentstroke}{rgb}{0.000000,0.000000,0.000000}%
\pgfsetstrokecolor{currentstroke}%
\pgfsetdash{}{0pt}%
\pgfsys@defobject{currentmarker}{\pgfqpoint{0.000000in}{-0.048611in}}{\pgfqpoint{0.000000in}{0.000000in}}{%
\pgfpathmoveto{\pgfqpoint{0.000000in}{0.000000in}}%
\pgfpathlineto{\pgfqpoint{0.000000in}{-0.048611in}}%
\pgfusepath{stroke,fill}%
}%
\begin{pgfscope}%
\pgfsys@transformshift{3.280000in}{0.528000in}%
\pgfsys@useobject{currentmarker}{}%
\end{pgfscope}%
\end{pgfscope}%
\begin{pgfscope}%
\definecolor{textcolor}{rgb}{0.000000,0.000000,0.000000}%
\pgfsetstrokecolor{textcolor}%
\pgfsetfillcolor{textcolor}%
\pgftext[x=3.280000in,y=0.430778in,,top]{\color{textcolor}\rmfamily\fontsize{10.000000}{12.000000}\selectfont \(\displaystyle {15}\)}%
\end{pgfscope}%
\begin{pgfscope}%
\pgfpathrectangle{\pgfqpoint{0.800000in}{0.528000in}}{\pgfqpoint{4.960000in}{3.696000in}}%
\pgfusepath{clip}%
\pgfsetrectcap%
\pgfsetroundjoin%
\pgfsetlinewidth{0.803000pt}%
\definecolor{currentstroke}{rgb}{0.690196,0.690196,0.690196}%
\pgfsetstrokecolor{currentstroke}%
\pgfsetdash{}{0pt}%
\pgfpathmoveto{\pgfqpoint{4.106667in}{0.528000in}}%
\pgfpathlineto{\pgfqpoint{4.106667in}{4.224000in}}%
\pgfusepath{stroke}%
\end{pgfscope}%
\begin{pgfscope}%
\pgfsetbuttcap%
\pgfsetroundjoin%
\definecolor{currentfill}{rgb}{0.000000,0.000000,0.000000}%
\pgfsetfillcolor{currentfill}%
\pgfsetlinewidth{0.803000pt}%
\definecolor{currentstroke}{rgb}{0.000000,0.000000,0.000000}%
\pgfsetstrokecolor{currentstroke}%
\pgfsetdash{}{0pt}%
\pgfsys@defobject{currentmarker}{\pgfqpoint{0.000000in}{-0.048611in}}{\pgfqpoint{0.000000in}{0.000000in}}{%
\pgfpathmoveto{\pgfqpoint{0.000000in}{0.000000in}}%
\pgfpathlineto{\pgfqpoint{0.000000in}{-0.048611in}}%
\pgfusepath{stroke,fill}%
}%
\begin{pgfscope}%
\pgfsys@transformshift{4.106667in}{0.528000in}%
\pgfsys@useobject{currentmarker}{}%
\end{pgfscope}%
\end{pgfscope}%
\begin{pgfscope}%
\definecolor{textcolor}{rgb}{0.000000,0.000000,0.000000}%
\pgfsetstrokecolor{textcolor}%
\pgfsetfillcolor{textcolor}%
\pgftext[x=4.106667in,y=0.430778in,,top]{\color{textcolor}\rmfamily\fontsize{10.000000}{12.000000}\selectfont \(\displaystyle {20}\)}%
\end{pgfscope}%
\begin{pgfscope}%
\pgfpathrectangle{\pgfqpoint{0.800000in}{0.528000in}}{\pgfqpoint{4.960000in}{3.696000in}}%
\pgfusepath{clip}%
\pgfsetrectcap%
\pgfsetroundjoin%
\pgfsetlinewidth{0.803000pt}%
\definecolor{currentstroke}{rgb}{0.690196,0.690196,0.690196}%
\pgfsetstrokecolor{currentstroke}%
\pgfsetdash{}{0pt}%
\pgfpathmoveto{\pgfqpoint{4.933333in}{0.528000in}}%
\pgfpathlineto{\pgfqpoint{4.933333in}{4.224000in}}%
\pgfusepath{stroke}%
\end{pgfscope}%
\begin{pgfscope}%
\pgfsetbuttcap%
\pgfsetroundjoin%
\definecolor{currentfill}{rgb}{0.000000,0.000000,0.000000}%
\pgfsetfillcolor{currentfill}%
\pgfsetlinewidth{0.803000pt}%
\definecolor{currentstroke}{rgb}{0.000000,0.000000,0.000000}%
\pgfsetstrokecolor{currentstroke}%
\pgfsetdash{}{0pt}%
\pgfsys@defobject{currentmarker}{\pgfqpoint{0.000000in}{-0.048611in}}{\pgfqpoint{0.000000in}{0.000000in}}{%
\pgfpathmoveto{\pgfqpoint{0.000000in}{0.000000in}}%
\pgfpathlineto{\pgfqpoint{0.000000in}{-0.048611in}}%
\pgfusepath{stroke,fill}%
}%
\begin{pgfscope}%
\pgfsys@transformshift{4.933333in}{0.528000in}%
\pgfsys@useobject{currentmarker}{}%
\end{pgfscope}%
\end{pgfscope}%
\begin{pgfscope}%
\definecolor{textcolor}{rgb}{0.000000,0.000000,0.000000}%
\pgfsetstrokecolor{textcolor}%
\pgfsetfillcolor{textcolor}%
\pgftext[x=4.933333in,y=0.430778in,,top]{\color{textcolor}\rmfamily\fontsize{10.000000}{12.000000}\selectfont \(\displaystyle {25}\)}%
\end{pgfscope}%
\begin{pgfscope}%
\pgfpathrectangle{\pgfqpoint{0.800000in}{0.528000in}}{\pgfqpoint{4.960000in}{3.696000in}}%
\pgfusepath{clip}%
\pgfsetrectcap%
\pgfsetroundjoin%
\pgfsetlinewidth{0.803000pt}%
\definecolor{currentstroke}{rgb}{0.690196,0.690196,0.690196}%
\pgfsetstrokecolor{currentstroke}%
\pgfsetdash{}{0pt}%
\pgfpathmoveto{\pgfqpoint{5.760000in}{0.528000in}}%
\pgfpathlineto{\pgfqpoint{5.760000in}{4.224000in}}%
\pgfusepath{stroke}%
\end{pgfscope}%
\begin{pgfscope}%
\pgfsetbuttcap%
\pgfsetroundjoin%
\definecolor{currentfill}{rgb}{0.000000,0.000000,0.000000}%
\pgfsetfillcolor{currentfill}%
\pgfsetlinewidth{0.803000pt}%
\definecolor{currentstroke}{rgb}{0.000000,0.000000,0.000000}%
\pgfsetstrokecolor{currentstroke}%
\pgfsetdash{}{0pt}%
\pgfsys@defobject{currentmarker}{\pgfqpoint{0.000000in}{-0.048611in}}{\pgfqpoint{0.000000in}{0.000000in}}{%
\pgfpathmoveto{\pgfqpoint{0.000000in}{0.000000in}}%
\pgfpathlineto{\pgfqpoint{0.000000in}{-0.048611in}}%
\pgfusepath{stroke,fill}%
}%
\begin{pgfscope}%
\pgfsys@transformshift{5.760000in}{0.528000in}%
\pgfsys@useobject{currentmarker}{}%
\end{pgfscope}%
\end{pgfscope}%
\begin{pgfscope}%
\definecolor{textcolor}{rgb}{0.000000,0.000000,0.000000}%
\pgfsetstrokecolor{textcolor}%
\pgfsetfillcolor{textcolor}%
\pgftext[x=5.760000in,y=0.430778in,,top]{\color{textcolor}\rmfamily\fontsize{10.000000}{12.000000}\selectfont \(\displaystyle {30}\)}%
\end{pgfscope}%
\begin{pgfscope}%
\definecolor{textcolor}{rgb}{0.000000,0.000000,0.000000}%
\pgfsetstrokecolor{textcolor}%
\pgfsetfillcolor{textcolor}%
\pgftext[x=3.280000in,y=0.251796in,,top]{\color{textcolor}\rmfamily\fontsize{10.000000}{12.000000}\selectfont \(\displaystyle t,\ с\)}%
\end{pgfscope}%
\begin{pgfscope}%
\pgfpathrectangle{\pgfqpoint{0.800000in}{0.528000in}}{\pgfqpoint{4.960000in}{3.696000in}}%
\pgfusepath{clip}%
\pgfsetrectcap%
\pgfsetroundjoin%
\pgfsetlinewidth{0.803000pt}%
\definecolor{currentstroke}{rgb}{0.690196,0.690196,0.690196}%
\pgfsetstrokecolor{currentstroke}%
\pgfsetdash{}{0pt}%
\pgfpathmoveto{\pgfqpoint{0.800000in}{0.533413in}}%
\pgfpathlineto{\pgfqpoint{5.760000in}{0.533413in}}%
\pgfusepath{stroke}%
\end{pgfscope}%
\begin{pgfscope}%
\pgfsetbuttcap%
\pgfsetroundjoin%
\definecolor{currentfill}{rgb}{0.000000,0.000000,0.000000}%
\pgfsetfillcolor{currentfill}%
\pgfsetlinewidth{0.803000pt}%
\definecolor{currentstroke}{rgb}{0.000000,0.000000,0.000000}%
\pgfsetstrokecolor{currentstroke}%
\pgfsetdash{}{0pt}%
\pgfsys@defobject{currentmarker}{\pgfqpoint{-0.048611in}{0.000000in}}{\pgfqpoint{-0.000000in}{0.000000in}}{%
\pgfpathmoveto{\pgfqpoint{-0.000000in}{0.000000in}}%
\pgfpathlineto{\pgfqpoint{-0.048611in}{0.000000in}}%
\pgfusepath{stroke,fill}%
}%
\begin{pgfscope}%
\pgfsys@transformshift{0.800000in}{0.533413in}%
\pgfsys@useobject{currentmarker}{}%
\end{pgfscope}%
\end{pgfscope}%
\begin{pgfscope}%
\definecolor{textcolor}{rgb}{0.000000,0.000000,0.000000}%
\pgfsetstrokecolor{textcolor}%
\pgfsetfillcolor{textcolor}%
\pgftext[x=0.417283in, y=0.485200in, left, base]{\color{textcolor}\rmfamily\fontsize{10.000000}{12.000000}\selectfont \(\displaystyle {\ensuremath{-}0.5}\)}%
\end{pgfscope}%
\begin{pgfscope}%
\pgfpathrectangle{\pgfqpoint{0.800000in}{0.528000in}}{\pgfqpoint{4.960000in}{3.696000in}}%
\pgfusepath{clip}%
\pgfsetrectcap%
\pgfsetroundjoin%
\pgfsetlinewidth{0.803000pt}%
\definecolor{currentstroke}{rgb}{0.690196,0.690196,0.690196}%
\pgfsetstrokecolor{currentstroke}%
\pgfsetdash{}{0pt}%
\pgfpathmoveto{\pgfqpoint{0.800000in}{1.183495in}}%
\pgfpathlineto{\pgfqpoint{5.760000in}{1.183495in}}%
\pgfusepath{stroke}%
\end{pgfscope}%
\begin{pgfscope}%
\pgfsetbuttcap%
\pgfsetroundjoin%
\definecolor{currentfill}{rgb}{0.000000,0.000000,0.000000}%
\pgfsetfillcolor{currentfill}%
\pgfsetlinewidth{0.803000pt}%
\definecolor{currentstroke}{rgb}{0.000000,0.000000,0.000000}%
\pgfsetstrokecolor{currentstroke}%
\pgfsetdash{}{0pt}%
\pgfsys@defobject{currentmarker}{\pgfqpoint{-0.048611in}{0.000000in}}{\pgfqpoint{-0.000000in}{0.000000in}}{%
\pgfpathmoveto{\pgfqpoint{-0.000000in}{0.000000in}}%
\pgfpathlineto{\pgfqpoint{-0.048611in}{0.000000in}}%
\pgfusepath{stroke,fill}%
}%
\begin{pgfscope}%
\pgfsys@transformshift{0.800000in}{1.183495in}%
\pgfsys@useobject{currentmarker}{}%
\end{pgfscope}%
\end{pgfscope}%
\begin{pgfscope}%
\definecolor{textcolor}{rgb}{0.000000,0.000000,0.000000}%
\pgfsetstrokecolor{textcolor}%
\pgfsetfillcolor{textcolor}%
\pgftext[x=0.417283in, y=1.135282in, left, base]{\color{textcolor}\rmfamily\fontsize{10.000000}{12.000000}\selectfont \(\displaystyle {\ensuremath{-}0.4}\)}%
\end{pgfscope}%
\begin{pgfscope}%
\pgfpathrectangle{\pgfqpoint{0.800000in}{0.528000in}}{\pgfqpoint{4.960000in}{3.696000in}}%
\pgfusepath{clip}%
\pgfsetrectcap%
\pgfsetroundjoin%
\pgfsetlinewidth{0.803000pt}%
\definecolor{currentstroke}{rgb}{0.690196,0.690196,0.690196}%
\pgfsetstrokecolor{currentstroke}%
\pgfsetdash{}{0pt}%
\pgfpathmoveto{\pgfqpoint{0.800000in}{1.833577in}}%
\pgfpathlineto{\pgfqpoint{5.760000in}{1.833577in}}%
\pgfusepath{stroke}%
\end{pgfscope}%
\begin{pgfscope}%
\pgfsetbuttcap%
\pgfsetroundjoin%
\definecolor{currentfill}{rgb}{0.000000,0.000000,0.000000}%
\pgfsetfillcolor{currentfill}%
\pgfsetlinewidth{0.803000pt}%
\definecolor{currentstroke}{rgb}{0.000000,0.000000,0.000000}%
\pgfsetstrokecolor{currentstroke}%
\pgfsetdash{}{0pt}%
\pgfsys@defobject{currentmarker}{\pgfqpoint{-0.048611in}{0.000000in}}{\pgfqpoint{-0.000000in}{0.000000in}}{%
\pgfpathmoveto{\pgfqpoint{-0.000000in}{0.000000in}}%
\pgfpathlineto{\pgfqpoint{-0.048611in}{0.000000in}}%
\pgfusepath{stroke,fill}%
}%
\begin{pgfscope}%
\pgfsys@transformshift{0.800000in}{1.833577in}%
\pgfsys@useobject{currentmarker}{}%
\end{pgfscope}%
\end{pgfscope}%
\begin{pgfscope}%
\definecolor{textcolor}{rgb}{0.000000,0.000000,0.000000}%
\pgfsetstrokecolor{textcolor}%
\pgfsetfillcolor{textcolor}%
\pgftext[x=0.417283in, y=1.785364in, left, base]{\color{textcolor}\rmfamily\fontsize{10.000000}{12.000000}\selectfont \(\displaystyle {\ensuremath{-}0.3}\)}%
\end{pgfscope}%
\begin{pgfscope}%
\pgfpathrectangle{\pgfqpoint{0.800000in}{0.528000in}}{\pgfqpoint{4.960000in}{3.696000in}}%
\pgfusepath{clip}%
\pgfsetrectcap%
\pgfsetroundjoin%
\pgfsetlinewidth{0.803000pt}%
\definecolor{currentstroke}{rgb}{0.690196,0.690196,0.690196}%
\pgfsetstrokecolor{currentstroke}%
\pgfsetdash{}{0pt}%
\pgfpathmoveto{\pgfqpoint{0.800000in}{2.483659in}}%
\pgfpathlineto{\pgfqpoint{5.760000in}{2.483659in}}%
\pgfusepath{stroke}%
\end{pgfscope}%
\begin{pgfscope}%
\pgfsetbuttcap%
\pgfsetroundjoin%
\definecolor{currentfill}{rgb}{0.000000,0.000000,0.000000}%
\pgfsetfillcolor{currentfill}%
\pgfsetlinewidth{0.803000pt}%
\definecolor{currentstroke}{rgb}{0.000000,0.000000,0.000000}%
\pgfsetstrokecolor{currentstroke}%
\pgfsetdash{}{0pt}%
\pgfsys@defobject{currentmarker}{\pgfqpoint{-0.048611in}{0.000000in}}{\pgfqpoint{-0.000000in}{0.000000in}}{%
\pgfpathmoveto{\pgfqpoint{-0.000000in}{0.000000in}}%
\pgfpathlineto{\pgfqpoint{-0.048611in}{0.000000in}}%
\pgfusepath{stroke,fill}%
}%
\begin{pgfscope}%
\pgfsys@transformshift{0.800000in}{2.483659in}%
\pgfsys@useobject{currentmarker}{}%
\end{pgfscope}%
\end{pgfscope}%
\begin{pgfscope}%
\definecolor{textcolor}{rgb}{0.000000,0.000000,0.000000}%
\pgfsetstrokecolor{textcolor}%
\pgfsetfillcolor{textcolor}%
\pgftext[x=0.417283in, y=2.435446in, left, base]{\color{textcolor}\rmfamily\fontsize{10.000000}{12.000000}\selectfont \(\displaystyle {\ensuremath{-}0.2}\)}%
\end{pgfscope}%
\begin{pgfscope}%
\pgfpathrectangle{\pgfqpoint{0.800000in}{0.528000in}}{\pgfqpoint{4.960000in}{3.696000in}}%
\pgfusepath{clip}%
\pgfsetrectcap%
\pgfsetroundjoin%
\pgfsetlinewidth{0.803000pt}%
\definecolor{currentstroke}{rgb}{0.690196,0.690196,0.690196}%
\pgfsetstrokecolor{currentstroke}%
\pgfsetdash{}{0pt}%
\pgfpathmoveto{\pgfqpoint{0.800000in}{3.133741in}}%
\pgfpathlineto{\pgfqpoint{5.760000in}{3.133741in}}%
\pgfusepath{stroke}%
\end{pgfscope}%
\begin{pgfscope}%
\pgfsetbuttcap%
\pgfsetroundjoin%
\definecolor{currentfill}{rgb}{0.000000,0.000000,0.000000}%
\pgfsetfillcolor{currentfill}%
\pgfsetlinewidth{0.803000pt}%
\definecolor{currentstroke}{rgb}{0.000000,0.000000,0.000000}%
\pgfsetstrokecolor{currentstroke}%
\pgfsetdash{}{0pt}%
\pgfsys@defobject{currentmarker}{\pgfqpoint{-0.048611in}{0.000000in}}{\pgfqpoint{-0.000000in}{0.000000in}}{%
\pgfpathmoveto{\pgfqpoint{-0.000000in}{0.000000in}}%
\pgfpathlineto{\pgfqpoint{-0.048611in}{0.000000in}}%
\pgfusepath{stroke,fill}%
}%
\begin{pgfscope}%
\pgfsys@transformshift{0.800000in}{3.133741in}%
\pgfsys@useobject{currentmarker}{}%
\end{pgfscope}%
\end{pgfscope}%
\begin{pgfscope}%
\definecolor{textcolor}{rgb}{0.000000,0.000000,0.000000}%
\pgfsetstrokecolor{textcolor}%
\pgfsetfillcolor{textcolor}%
\pgftext[x=0.417283in, y=3.085528in, left, base]{\color{textcolor}\rmfamily\fontsize{10.000000}{12.000000}\selectfont \(\displaystyle {\ensuremath{-}0.1}\)}%
\end{pgfscope}%
\begin{pgfscope}%
\pgfpathrectangle{\pgfqpoint{0.800000in}{0.528000in}}{\pgfqpoint{4.960000in}{3.696000in}}%
\pgfusepath{clip}%
\pgfsetrectcap%
\pgfsetroundjoin%
\pgfsetlinewidth{0.803000pt}%
\definecolor{currentstroke}{rgb}{0.690196,0.690196,0.690196}%
\pgfsetstrokecolor{currentstroke}%
\pgfsetdash{}{0pt}%
\pgfpathmoveto{\pgfqpoint{0.800000in}{3.783823in}}%
\pgfpathlineto{\pgfqpoint{5.760000in}{3.783823in}}%
\pgfusepath{stroke}%
\end{pgfscope}%
\begin{pgfscope}%
\pgfsetbuttcap%
\pgfsetroundjoin%
\definecolor{currentfill}{rgb}{0.000000,0.000000,0.000000}%
\pgfsetfillcolor{currentfill}%
\pgfsetlinewidth{0.803000pt}%
\definecolor{currentstroke}{rgb}{0.000000,0.000000,0.000000}%
\pgfsetstrokecolor{currentstroke}%
\pgfsetdash{}{0pt}%
\pgfsys@defobject{currentmarker}{\pgfqpoint{-0.048611in}{0.000000in}}{\pgfqpoint{-0.000000in}{0.000000in}}{%
\pgfpathmoveto{\pgfqpoint{-0.000000in}{0.000000in}}%
\pgfpathlineto{\pgfqpoint{-0.048611in}{0.000000in}}%
\pgfusepath{stroke,fill}%
}%
\begin{pgfscope}%
\pgfsys@transformshift{0.800000in}{3.783823in}%
\pgfsys@useobject{currentmarker}{}%
\end{pgfscope}%
\end{pgfscope}%
\begin{pgfscope}%
\definecolor{textcolor}{rgb}{0.000000,0.000000,0.000000}%
\pgfsetstrokecolor{textcolor}%
\pgfsetfillcolor{textcolor}%
\pgftext[x=0.525308in, y=3.735610in, left, base]{\color{textcolor}\rmfamily\fontsize{10.000000}{12.000000}\selectfont \(\displaystyle {0.0}\)}%
\end{pgfscope}%
\begin{pgfscope}%
\definecolor{textcolor}{rgb}{0.000000,0.000000,0.000000}%
\pgfsetstrokecolor{textcolor}%
\pgfsetfillcolor{textcolor}%
\pgftext[x=0.361727in,y=2.376000in,,bottom,rotate=90.000000]{\color{textcolor}\rmfamily\fontsize{10.000000}{12.000000}\selectfont \(\displaystyle \delta_{в},\ рад\)}%
\end{pgfscope}%
\begin{pgfscope}%
\pgfpathrectangle{\pgfqpoint{0.800000in}{0.528000in}}{\pgfqpoint{4.960000in}{3.696000in}}%
\pgfusepath{clip}%
\pgfsetrectcap%
\pgfsetroundjoin%
\pgfsetlinewidth{1.505625pt}%
\definecolor{currentstroke}{rgb}{0.121569,0.466667,0.705882}%
\pgfsetstrokecolor{currentstroke}%
\pgfsetdash{}{0pt}%
\pgfpathmoveto{\pgfqpoint{0.800000in}{3.783823in}}%
\pgfpathlineto{\pgfqpoint{0.965499in}{3.782789in}}%
\pgfpathlineto{\pgfqpoint{0.966325in}{3.763336in}}%
\pgfpathlineto{\pgfqpoint{0.992944in}{2.667579in}}%
\pgfpathlineto{\pgfqpoint{1.004352in}{2.197851in}}%
\pgfpathlineto{\pgfqpoint{1.004848in}{2.214999in}}%
\pgfpathlineto{\pgfqpoint{1.049819in}{4.026008in}}%
\pgfpathlineto{\pgfqpoint{1.053456in}{4.049956in}}%
\pgfpathlineto{\pgfqpoint{1.056101in}{4.055853in}}%
\pgfpathlineto{\pgfqpoint{1.057589in}{4.055469in}}%
\pgfpathlineto{\pgfqpoint{1.059408in}{4.051873in}}%
\pgfpathlineto{\pgfqpoint{1.062384in}{4.039858in}}%
\pgfpathlineto{\pgfqpoint{1.067179in}{4.009443in}}%
\pgfpathlineto{\pgfqpoint{1.089499in}{3.854478in}}%
\pgfpathlineto{\pgfqpoint{1.096773in}{3.824880in}}%
\pgfpathlineto{\pgfqpoint{1.104213in}{3.803737in}}%
\pgfpathlineto{\pgfqpoint{1.112811in}{3.786092in}}%
\pgfpathlineto{\pgfqpoint{1.125376in}{3.765834in}}%
\pgfpathlineto{\pgfqpoint{1.145381in}{3.737930in}}%
\pgfpathlineto{\pgfqpoint{1.161915in}{3.718432in}}%
\pgfpathlineto{\pgfqpoint{1.177621in}{3.703303in}}%
\pgfpathlineto{\pgfqpoint{1.194155in}{3.690452in}}%
\pgfpathlineto{\pgfqpoint{1.211845in}{3.679514in}}%
\pgfpathlineto{\pgfqpoint{1.230528in}{3.670572in}}%
\pgfpathlineto{\pgfqpoint{1.250533in}{3.663487in}}%
\pgfpathlineto{\pgfqpoint{1.272192in}{3.658216in}}%
\pgfpathlineto{\pgfqpoint{1.296000in}{3.654748in}}%
\pgfpathlineto{\pgfqpoint{1.322784in}{3.653138in}}%
\pgfpathlineto{\pgfqpoint{1.353701in}{3.653563in}}%
\pgfpathlineto{\pgfqpoint{1.390901in}{3.656386in}}%
\pgfpathlineto{\pgfqpoint{1.438352in}{3.662342in}}%
\pgfpathlineto{\pgfqpoint{1.508949in}{3.673682in}}%
\pgfpathlineto{\pgfqpoint{1.720907in}{3.708791in}}%
\pgfpathlineto{\pgfqpoint{1.819776in}{3.722281in}}%
\pgfpathlineto{\pgfqpoint{1.880453in}{3.730925in}}%
\pgfpathlineto{\pgfqpoint{1.884256in}{3.735469in}}%
\pgfpathlineto{\pgfqpoint{1.889051in}{3.745229in}}%
\pgfpathlineto{\pgfqpoint{1.896821in}{3.767157in}}%
\pgfpathlineto{\pgfqpoint{1.909552in}{3.802310in}}%
\pgfpathlineto{\pgfqpoint{1.916661in}{3.815590in}}%
\pgfpathlineto{\pgfqpoint{1.923109in}{3.823178in}}%
\pgfpathlineto{\pgfqpoint{1.929557in}{3.827381in}}%
\pgfpathlineto{\pgfqpoint{1.936832in}{3.829402in}}%
\pgfpathlineto{\pgfqpoint{1.947579in}{3.829796in}}%
\pgfpathlineto{\pgfqpoint{1.988085in}{3.829764in}}%
\pgfpathlineto{\pgfqpoint{2.020325in}{3.832641in}}%
\pgfpathlineto{\pgfqpoint{2.054549in}{3.837984in}}%
\pgfpathlineto{\pgfqpoint{2.097867in}{3.847141in}}%
\pgfpathlineto{\pgfqpoint{2.168960in}{3.864790in}}%
\pgfpathlineto{\pgfqpoint{2.275269in}{3.890826in}}%
\pgfpathlineto{\pgfqpoint{2.340576in}{3.904423in}}%
\pgfpathlineto{\pgfqpoint{2.400757in}{3.914697in}}%
\pgfpathlineto{\pgfqpoint{2.459120in}{3.922416in}}%
\pgfpathlineto{\pgfqpoint{2.517317in}{3.927870in}}%
\pgfpathlineto{\pgfqpoint{2.576176in}{3.931149in}}%
\pgfpathlineto{\pgfqpoint{2.636523in}{3.932273in}}%
\pgfpathlineto{\pgfqpoint{2.699019in}{3.931201in}}%
\pgfpathlineto{\pgfqpoint{2.764656in}{3.927829in}}%
\pgfpathlineto{\pgfqpoint{2.834427in}{3.921994in}}%
\pgfpathlineto{\pgfqpoint{2.910149in}{3.913397in}}%
\pgfpathlineto{\pgfqpoint{2.994800in}{3.901498in}}%
\pgfpathlineto{\pgfqpoint{3.094496in}{3.885155in}}%
\pgfpathlineto{\pgfqpoint{3.229573in}{3.860571in}}%
\pgfpathlineto{\pgfqpoint{3.524859in}{3.806271in}}%
\pgfpathlineto{\pgfqpoint{3.642245in}{3.787376in}}%
\pgfpathlineto{\pgfqpoint{3.747067in}{3.772728in}}%
\pgfpathlineto{\pgfqpoint{3.846267in}{3.761098in}}%
\pgfpathlineto{\pgfqpoint{3.942987in}{3.751992in}}%
\pgfpathlineto{\pgfqpoint{4.039376in}{3.745153in}}%
\pgfpathlineto{\pgfqpoint{4.137088in}{3.740455in}}%
\pgfpathlineto{\pgfqpoint{4.238107in}{3.737839in}}%
\pgfpathlineto{\pgfqpoint{4.344581in}{3.737330in}}%
\pgfpathlineto{\pgfqpoint{4.459653in}{3.739038in}}%
\pgfpathlineto{\pgfqpoint{4.589109in}{3.743239in}}%
\pgfpathlineto{\pgfqpoint{4.746011in}{3.750656in}}%
\pgfpathlineto{\pgfqpoint{4.996491in}{3.765074in}}%
\pgfpathlineto{\pgfqpoint{5.286485in}{3.781143in}}%
\pgfpathlineto{\pgfqpoint{5.473808in}{3.789237in}}%
\pgfpathlineto{\pgfqpoint{5.646912in}{3.794493in}}%
\pgfpathlineto{\pgfqpoint{5.760000in}{3.796712in}}%
\pgfpathlineto{\pgfqpoint{5.760000in}{3.796712in}}%
\pgfusepath{stroke}%
\end{pgfscope}%
\begin{pgfscope}%
\pgfpathrectangle{\pgfqpoint{0.800000in}{0.528000in}}{\pgfqpoint{4.960000in}{3.696000in}}%
\pgfusepath{clip}%
\pgfsetbuttcap%
\pgfsetroundjoin%
\pgfsetlinewidth{1.505625pt}%
\definecolor{currentstroke}{rgb}{1.000000,0.498039,0.054902}%
\pgfsetstrokecolor{currentstroke}%
\pgfsetdash{{5.550000pt}{2.400000pt}}{0.000000pt}%
\pgfpathmoveto{\pgfqpoint{0.800000in}{3.783823in}}%
\pgfpathlineto{\pgfqpoint{0.965333in}{3.783823in}}%
\pgfpathlineto{\pgfqpoint{0.968029in}{3.576154in}}%
\pgfpathlineto{\pgfqpoint{0.974145in}{2.326486in}}%
\pgfpathlineto{\pgfqpoint{0.978686in}{1.398102in}}%
\pgfpathlineto{\pgfqpoint{0.984449in}{0.706208in}}%
\pgfpathlineto{\pgfqpoint{0.991632in}{0.696000in}}%
\pgfpathlineto{\pgfqpoint{0.999384in}{1.386744in}}%
\pgfpathlineto{\pgfqpoint{1.007499in}{2.338795in}}%
\pgfpathlineto{\pgfqpoint{1.016323in}{3.194967in}}%
\pgfpathlineto{\pgfqpoint{1.026725in}{3.767319in}}%
\pgfpathlineto{\pgfqpoint{1.039513in}{3.979932in}}%
\pgfpathlineto{\pgfqpoint{1.052801in}{3.962401in}}%
\pgfpathlineto{\pgfqpoint{1.067314in}{3.902745in}}%
\pgfpathlineto{\pgfqpoint{1.087056in}{3.843505in}}%
\pgfpathlineto{\pgfqpoint{1.107694in}{3.789699in}}%
\pgfpathlineto{\pgfqpoint{1.132454in}{3.730782in}}%
\pgfpathlineto{\pgfqpoint{1.167704in}{3.670525in}}%
\pgfpathlineto{\pgfqpoint{1.206313in}{3.632291in}}%
\pgfpathlineto{\pgfqpoint{1.241429in}{3.614252in}}%
\pgfpathlineto{\pgfqpoint{1.274232in}{3.607908in}}%
\pgfpathlineto{\pgfqpoint{1.304293in}{3.610540in}}%
\pgfpathlineto{\pgfqpoint{1.334086in}{3.618693in}}%
\pgfpathlineto{\pgfqpoint{1.366188in}{3.631017in}}%
\pgfpathlineto{\pgfqpoint{1.406996in}{3.651176in}}%
\pgfpathlineto{\pgfqpoint{1.466440in}{3.683563in}}%
\pgfpathlineto{\pgfqpoint{1.633093in}{3.782220in}}%
\pgfpathlineto{\pgfqpoint{1.677893in}{3.808693in}}%
\pgfpathlineto{\pgfqpoint{1.731536in}{3.835599in}}%
\pgfpathlineto{\pgfqpoint{1.765963in}{3.852235in}}%
\pgfpathlineto{\pgfqpoint{1.812963in}{3.870649in}}%
\pgfpathlineto{\pgfqpoint{1.837856in}{3.883381in}}%
\pgfpathlineto{\pgfqpoint{1.862749in}{3.894306in}}%
\pgfpathlineto{\pgfqpoint{1.901527in}{3.909221in}}%
\pgfpathlineto{\pgfqpoint{1.934246in}{3.920145in}}%
\pgfpathlineto{\pgfqpoint{1.966965in}{3.929711in}}%
\pgfpathlineto{\pgfqpoint{2.002382in}{3.938864in}}%
\pgfpathlineto{\pgfqpoint{2.043835in}{3.946069in}}%
\pgfpathlineto{\pgfqpoint{2.095461in}{3.958266in}}%
\pgfpathlineto{\pgfqpoint{2.172231in}{3.970479in}}%
\pgfpathlineto{\pgfqpoint{2.328664in}{3.973181in}}%
\pgfpathlineto{\pgfqpoint{2.401515in}{3.970275in}}%
\pgfpathlineto{\pgfqpoint{2.436813in}{3.967895in}}%
\pgfpathlineto{\pgfqpoint{2.499470in}{3.959000in}}%
\pgfpathlineto{\pgfqpoint{2.534531in}{3.953391in}}%
\pgfpathlineto{\pgfqpoint{2.606266in}{3.939441in}}%
\pgfpathlineto{\pgfqpoint{2.633725in}{3.936774in}}%
\pgfpathlineto{\pgfqpoint{2.739810in}{3.917514in}}%
\pgfpathlineto{\pgfqpoint{2.832452in}{3.892948in}}%
\pgfpathlineto{\pgfqpoint{2.902170in}{3.876336in}}%
\pgfpathlineto{\pgfqpoint{3.004903in}{3.854396in}}%
\pgfpathlineto{\pgfqpoint{3.032396in}{3.845960in}}%
\pgfpathlineto{\pgfqpoint{3.102619in}{3.827880in}}%
\pgfpathlineto{\pgfqpoint{3.234878in}{3.800923in}}%
\pgfpathlineto{\pgfqpoint{3.302938in}{3.787781in}}%
\pgfpathlineto{\pgfqpoint{3.332403in}{3.781097in}}%
\pgfpathlineto{\pgfqpoint{3.404680in}{3.766628in}}%
\pgfpathlineto{\pgfqpoint{3.500764in}{3.754924in}}%
\pgfpathlineto{\pgfqpoint{3.706065in}{3.730421in}}%
\pgfpathlineto{\pgfqpoint{3.732573in}{3.731087in}}%
\pgfpathlineto{\pgfqpoint{3.933770in}{3.723455in}}%
\pgfpathlineto{\pgfqpoint{3.999525in}{3.723659in}}%
\pgfpathlineto{\pgfqpoint{4.070982in}{3.726429in}}%
\pgfpathlineto{\pgfqpoint{4.107149in}{3.729785in}}%
\pgfpathlineto{\pgfqpoint{4.162130in}{3.728174in}}%
\pgfpathlineto{\pgfqpoint{4.264096in}{3.732619in}}%
\pgfpathlineto{\pgfqpoint{4.395650in}{3.741497in}}%
\pgfpathlineto{\pgfqpoint{4.430621in}{3.743655in}}%
\pgfpathlineto{\pgfqpoint{4.476228in}{3.744211in}}%
\pgfpathlineto{\pgfqpoint{4.526067in}{3.750667in}}%
\pgfpathlineto{\pgfqpoint{4.560233in}{3.753510in}}%
\pgfpathlineto{\pgfqpoint{4.602587in}{3.758637in}}%
\pgfpathlineto{\pgfqpoint{4.629731in}{3.758864in}}%
\pgfpathlineto{\pgfqpoint{4.729342in}{3.765333in}}%
\pgfpathlineto{\pgfqpoint{4.866010in}{3.776964in}}%
\pgfpathlineto{\pgfqpoint{4.966701in}{3.781440in}}%
\pgfpathlineto{\pgfqpoint{5.003797in}{3.781671in}}%
\pgfpathlineto{\pgfqpoint{5.035639in}{3.784076in}}%
\pgfpathlineto{\pgfqpoint{5.062269in}{3.787582in}}%
\pgfpathlineto{\pgfqpoint{5.162623in}{3.793059in}}%
\pgfpathlineto{\pgfqpoint{5.365094in}{3.799355in}}%
\pgfpathlineto{\pgfqpoint{5.406349in}{3.802972in}}%
\pgfpathlineto{\pgfqpoint{5.457201in}{3.801099in}}%
\pgfpathlineto{\pgfqpoint{5.491261in}{3.801302in}}%
\pgfpathlineto{\pgfqpoint{5.533747in}{3.799826in}}%
\pgfpathlineto{\pgfqpoint{5.560814in}{3.801835in}}%
\pgfpathlineto{\pgfqpoint{5.630698in}{3.803733in}}%
\pgfpathlineto{\pgfqpoint{5.760000in}{3.801557in}}%
\pgfpathlineto{\pgfqpoint{5.760000in}{3.801557in}}%
\pgfusepath{stroke}%
\end{pgfscope}%
\begin{pgfscope}%
\pgfsetrectcap%
\pgfsetmiterjoin%
\pgfsetlinewidth{0.803000pt}%
\definecolor{currentstroke}{rgb}{0.000000,0.000000,0.000000}%
\pgfsetstrokecolor{currentstroke}%
\pgfsetdash{}{0pt}%
\pgfpathmoveto{\pgfqpoint{0.800000in}{0.528000in}}%
\pgfpathlineto{\pgfqpoint{0.800000in}{4.224000in}}%
\pgfusepath{stroke}%
\end{pgfscope}%
\begin{pgfscope}%
\pgfsetrectcap%
\pgfsetmiterjoin%
\pgfsetlinewidth{0.803000pt}%
\definecolor{currentstroke}{rgb}{0.000000,0.000000,0.000000}%
\pgfsetstrokecolor{currentstroke}%
\pgfsetdash{}{0pt}%
\pgfpathmoveto{\pgfqpoint{5.760000in}{0.528000in}}%
\pgfpathlineto{\pgfqpoint{5.760000in}{4.224000in}}%
\pgfusepath{stroke}%
\end{pgfscope}%
\begin{pgfscope}%
\pgfsetrectcap%
\pgfsetmiterjoin%
\pgfsetlinewidth{0.803000pt}%
\definecolor{currentstroke}{rgb}{0.000000,0.000000,0.000000}%
\pgfsetstrokecolor{currentstroke}%
\pgfsetdash{}{0pt}%
\pgfpathmoveto{\pgfqpoint{0.800000in}{0.528000in}}%
\pgfpathlineto{\pgfqpoint{5.760000in}{0.528000in}}%
\pgfusepath{stroke}%
\end{pgfscope}%
\begin{pgfscope}%
\pgfsetrectcap%
\pgfsetmiterjoin%
\pgfsetlinewidth{0.803000pt}%
\definecolor{currentstroke}{rgb}{0.000000,0.000000,0.000000}%
\pgfsetstrokecolor{currentstroke}%
\pgfsetdash{}{0pt}%
\pgfpathmoveto{\pgfqpoint{0.800000in}{4.224000in}}%
\pgfpathlineto{\pgfqpoint{5.760000in}{4.224000in}}%
\pgfusepath{stroke}%
\end{pgfscope}%
\begin{pgfscope}%
\pgfsetbuttcap%
\pgfsetmiterjoin%
\definecolor{currentfill}{rgb}{1.000000,1.000000,1.000000}%
\pgfsetfillcolor{currentfill}%
\pgfsetfillopacity{0.800000}%
\pgfsetlinewidth{1.003750pt}%
\definecolor{currentstroke}{rgb}{0.800000,0.800000,0.800000}%
\pgfsetstrokecolor{currentstroke}%
\pgfsetstrokeopacity{0.800000}%
\pgfsetdash{}{0pt}%
\pgfpathmoveto{\pgfqpoint{3.299872in}{0.597444in}}%
\pgfpathlineto{\pgfqpoint{5.662778in}{0.597444in}}%
\pgfpathquadraticcurveto{\pgfqpoint{5.690556in}{0.597444in}}{\pgfqpoint{5.690556in}{0.625222in}}%
\pgfpathlineto{\pgfqpoint{5.690556in}{0.998666in}}%
\pgfpathquadraticcurveto{\pgfqpoint{5.690556in}{1.026444in}}{\pgfqpoint{5.662778in}{1.026444in}}%
\pgfpathlineto{\pgfqpoint{3.299872in}{1.026444in}}%
\pgfpathquadraticcurveto{\pgfqpoint{3.272094in}{1.026444in}}{\pgfqpoint{3.272094in}{0.998666in}}%
\pgfpathlineto{\pgfqpoint{3.272094in}{0.625222in}}%
\pgfpathquadraticcurveto{\pgfqpoint{3.272094in}{0.597444in}}{\pgfqpoint{3.299872in}{0.597444in}}%
\pgfpathclose%
\pgfusepath{stroke,fill}%
\end{pgfscope}%
\begin{pgfscope}%
\pgfsetrectcap%
\pgfsetroundjoin%
\pgfsetlinewidth{1.505625pt}%
\definecolor{currentstroke}{rgb}{0.121569,0.466667,0.705882}%
\pgfsetstrokecolor{currentstroke}%
\pgfsetdash{}{0pt}%
\pgfpathmoveto{\pgfqpoint{3.327650in}{0.922277in}}%
\pgfpathlineto{\pgfqpoint{3.605428in}{0.922277in}}%
\pgfusepath{stroke}%
\end{pgfscope}%
\begin{pgfscope}%
\definecolor{textcolor}{rgb}{0.000000,0.000000,0.000000}%
\pgfsetstrokecolor{textcolor}%
\pgfsetfillcolor{textcolor}%
\pgftext[x=3.716539in,y=0.873666in,left,base]{\color{textcolor}\rmfamily\fontsize{10.000000}{12.000000}\selectfont Нелинейная модель, \(\displaystyle M=0.61\)}%
\end{pgfscope}%
\begin{pgfscope}%
\pgfsetbuttcap%
\pgfsetroundjoin%
\pgfsetlinewidth{1.505625pt}%
\definecolor{currentstroke}{rgb}{1.000000,0.498039,0.054902}%
\pgfsetstrokecolor{currentstroke}%
\pgfsetdash{{5.550000pt}{2.400000pt}}{0.000000pt}%
\pgfpathmoveto{\pgfqpoint{3.327650in}{0.728611in}}%
\pgfpathlineto{\pgfqpoint{3.605428in}{0.728611in}}%
\pgfusepath{stroke}%
\end{pgfscope}%
\begin{pgfscope}%
\definecolor{textcolor}{rgb}{0.000000,0.000000,0.000000}%
\pgfsetstrokecolor{textcolor}%
\pgfsetfillcolor{textcolor}%
\pgftext[x=3.716539in,y=0.680000in,left,base]{\color{textcolor}\rmfamily\fontsize{10.000000}{12.000000}\selectfont Линейная модель, \(\displaystyle M=0.61\)}%
\end{pgfscope}%
\begin{pgfscope}%
\pgfsetbuttcap%
\pgfsetmiterjoin%
\definecolor{currentfill}{rgb}{1.000000,1.000000,1.000000}%
\pgfsetfillcolor{currentfill}%
\pgfsetlinewidth{0.000000pt}%
\definecolor{currentstroke}{rgb}{0.000000,0.000000,0.000000}%
\pgfsetstrokecolor{currentstroke}%
\pgfsetstrokeopacity{0.000000}%
\pgfsetdash{}{0pt}%
\pgfpathmoveto{\pgfqpoint{2.880000in}{1.440000in}}%
\pgfpathlineto{\pgfqpoint{5.440000in}{1.440000in}}%
\pgfpathlineto{\pgfqpoint{5.440000in}{2.880000in}}%
\pgfpathlineto{\pgfqpoint{2.880000in}{2.880000in}}%
\pgfpathclose%
\pgfusepath{fill}%
\end{pgfscope}%
\begin{pgfscope}%
\pgfpathrectangle{\pgfqpoint{2.880000in}{1.440000in}}{\pgfqpoint{2.560000in}{1.440000in}}%
\pgfusepath{clip}%
\pgfsetrectcap%
\pgfsetroundjoin%
\pgfsetlinewidth{0.803000pt}%
\definecolor{currentstroke}{rgb}{0.690196,0.690196,0.690196}%
\pgfsetstrokecolor{currentstroke}%
\pgfsetdash{}{0pt}%
\pgfpathmoveto{\pgfqpoint{2.880000in}{1.440000in}}%
\pgfpathlineto{\pgfqpoint{2.880000in}{2.880000in}}%
\pgfusepath{stroke}%
\end{pgfscope}%
\begin{pgfscope}%
\pgfsetbuttcap%
\pgfsetroundjoin%
\definecolor{currentfill}{rgb}{0.000000,0.000000,0.000000}%
\pgfsetfillcolor{currentfill}%
\pgfsetlinewidth{0.803000pt}%
\definecolor{currentstroke}{rgb}{0.000000,0.000000,0.000000}%
\pgfsetstrokecolor{currentstroke}%
\pgfsetdash{}{0pt}%
\pgfsys@defobject{currentmarker}{\pgfqpoint{0.000000in}{-0.048611in}}{\pgfqpoint{0.000000in}{0.000000in}}{%
\pgfpathmoveto{\pgfqpoint{0.000000in}{0.000000in}}%
\pgfpathlineto{\pgfqpoint{0.000000in}{-0.048611in}}%
\pgfusepath{stroke,fill}%
}%
\begin{pgfscope}%
\pgfsys@transformshift{2.880000in}{1.440000in}%
\pgfsys@useobject{currentmarker}{}%
\end{pgfscope}%
\end{pgfscope}%
\begin{pgfscope}%
\definecolor{textcolor}{rgb}{0.000000,0.000000,0.000000}%
\pgfsetstrokecolor{textcolor}%
\pgfsetfillcolor{textcolor}%
\pgftext[x=2.880000in,y=1.342778in,,top]{\color{textcolor}\rmfamily\fontsize{10.000000}{12.000000}\selectfont \(\displaystyle {0.0}\)}%
\end{pgfscope}%
\begin{pgfscope}%
\pgfpathrectangle{\pgfqpoint{2.880000in}{1.440000in}}{\pgfqpoint{2.560000in}{1.440000in}}%
\pgfusepath{clip}%
\pgfsetrectcap%
\pgfsetroundjoin%
\pgfsetlinewidth{0.803000pt}%
\definecolor{currentstroke}{rgb}{0.690196,0.690196,0.690196}%
\pgfsetstrokecolor{currentstroke}%
\pgfsetdash{}{0pt}%
\pgfpathmoveto{\pgfqpoint{3.306667in}{1.440000in}}%
\pgfpathlineto{\pgfqpoint{3.306667in}{2.880000in}}%
\pgfusepath{stroke}%
\end{pgfscope}%
\begin{pgfscope}%
\pgfsetbuttcap%
\pgfsetroundjoin%
\definecolor{currentfill}{rgb}{0.000000,0.000000,0.000000}%
\pgfsetfillcolor{currentfill}%
\pgfsetlinewidth{0.803000pt}%
\definecolor{currentstroke}{rgb}{0.000000,0.000000,0.000000}%
\pgfsetstrokecolor{currentstroke}%
\pgfsetdash{}{0pt}%
\pgfsys@defobject{currentmarker}{\pgfqpoint{0.000000in}{-0.048611in}}{\pgfqpoint{0.000000in}{0.000000in}}{%
\pgfpathmoveto{\pgfqpoint{0.000000in}{0.000000in}}%
\pgfpathlineto{\pgfqpoint{0.000000in}{-0.048611in}}%
\pgfusepath{stroke,fill}%
}%
\begin{pgfscope}%
\pgfsys@transformshift{3.306667in}{1.440000in}%
\pgfsys@useobject{currentmarker}{}%
\end{pgfscope}%
\end{pgfscope}%
\begin{pgfscope}%
\definecolor{textcolor}{rgb}{0.000000,0.000000,0.000000}%
\pgfsetstrokecolor{textcolor}%
\pgfsetfillcolor{textcolor}%
\pgftext[x=3.306667in,y=1.342778in,,top]{\color{textcolor}\rmfamily\fontsize{10.000000}{12.000000}\selectfont \(\displaystyle {0.5}\)}%
\end{pgfscope}%
\begin{pgfscope}%
\pgfpathrectangle{\pgfqpoint{2.880000in}{1.440000in}}{\pgfqpoint{2.560000in}{1.440000in}}%
\pgfusepath{clip}%
\pgfsetrectcap%
\pgfsetroundjoin%
\pgfsetlinewidth{0.803000pt}%
\definecolor{currentstroke}{rgb}{0.690196,0.690196,0.690196}%
\pgfsetstrokecolor{currentstroke}%
\pgfsetdash{}{0pt}%
\pgfpathmoveto{\pgfqpoint{3.733333in}{1.440000in}}%
\pgfpathlineto{\pgfqpoint{3.733333in}{2.880000in}}%
\pgfusepath{stroke}%
\end{pgfscope}%
\begin{pgfscope}%
\pgfsetbuttcap%
\pgfsetroundjoin%
\definecolor{currentfill}{rgb}{0.000000,0.000000,0.000000}%
\pgfsetfillcolor{currentfill}%
\pgfsetlinewidth{0.803000pt}%
\definecolor{currentstroke}{rgb}{0.000000,0.000000,0.000000}%
\pgfsetstrokecolor{currentstroke}%
\pgfsetdash{}{0pt}%
\pgfsys@defobject{currentmarker}{\pgfqpoint{0.000000in}{-0.048611in}}{\pgfqpoint{0.000000in}{0.000000in}}{%
\pgfpathmoveto{\pgfqpoint{0.000000in}{0.000000in}}%
\pgfpathlineto{\pgfqpoint{0.000000in}{-0.048611in}}%
\pgfusepath{stroke,fill}%
}%
\begin{pgfscope}%
\pgfsys@transformshift{3.733333in}{1.440000in}%
\pgfsys@useobject{currentmarker}{}%
\end{pgfscope}%
\end{pgfscope}%
\begin{pgfscope}%
\definecolor{textcolor}{rgb}{0.000000,0.000000,0.000000}%
\pgfsetstrokecolor{textcolor}%
\pgfsetfillcolor{textcolor}%
\pgftext[x=3.733333in,y=1.342778in,,top]{\color{textcolor}\rmfamily\fontsize{10.000000}{12.000000}\selectfont \(\displaystyle {1.0}\)}%
\end{pgfscope}%
\begin{pgfscope}%
\pgfpathrectangle{\pgfqpoint{2.880000in}{1.440000in}}{\pgfqpoint{2.560000in}{1.440000in}}%
\pgfusepath{clip}%
\pgfsetrectcap%
\pgfsetroundjoin%
\pgfsetlinewidth{0.803000pt}%
\definecolor{currentstroke}{rgb}{0.690196,0.690196,0.690196}%
\pgfsetstrokecolor{currentstroke}%
\pgfsetdash{}{0pt}%
\pgfpathmoveto{\pgfqpoint{4.160000in}{1.440000in}}%
\pgfpathlineto{\pgfqpoint{4.160000in}{2.880000in}}%
\pgfusepath{stroke}%
\end{pgfscope}%
\begin{pgfscope}%
\pgfsetbuttcap%
\pgfsetroundjoin%
\definecolor{currentfill}{rgb}{0.000000,0.000000,0.000000}%
\pgfsetfillcolor{currentfill}%
\pgfsetlinewidth{0.803000pt}%
\definecolor{currentstroke}{rgb}{0.000000,0.000000,0.000000}%
\pgfsetstrokecolor{currentstroke}%
\pgfsetdash{}{0pt}%
\pgfsys@defobject{currentmarker}{\pgfqpoint{0.000000in}{-0.048611in}}{\pgfqpoint{0.000000in}{0.000000in}}{%
\pgfpathmoveto{\pgfqpoint{0.000000in}{0.000000in}}%
\pgfpathlineto{\pgfqpoint{0.000000in}{-0.048611in}}%
\pgfusepath{stroke,fill}%
}%
\begin{pgfscope}%
\pgfsys@transformshift{4.160000in}{1.440000in}%
\pgfsys@useobject{currentmarker}{}%
\end{pgfscope}%
\end{pgfscope}%
\begin{pgfscope}%
\definecolor{textcolor}{rgb}{0.000000,0.000000,0.000000}%
\pgfsetstrokecolor{textcolor}%
\pgfsetfillcolor{textcolor}%
\pgftext[x=4.160000in,y=1.342778in,,top]{\color{textcolor}\rmfamily\fontsize{10.000000}{12.000000}\selectfont \(\displaystyle {1.5}\)}%
\end{pgfscope}%
\begin{pgfscope}%
\pgfpathrectangle{\pgfqpoint{2.880000in}{1.440000in}}{\pgfqpoint{2.560000in}{1.440000in}}%
\pgfusepath{clip}%
\pgfsetrectcap%
\pgfsetroundjoin%
\pgfsetlinewidth{0.803000pt}%
\definecolor{currentstroke}{rgb}{0.690196,0.690196,0.690196}%
\pgfsetstrokecolor{currentstroke}%
\pgfsetdash{}{0pt}%
\pgfpathmoveto{\pgfqpoint{4.586667in}{1.440000in}}%
\pgfpathlineto{\pgfqpoint{4.586667in}{2.880000in}}%
\pgfusepath{stroke}%
\end{pgfscope}%
\begin{pgfscope}%
\pgfsetbuttcap%
\pgfsetroundjoin%
\definecolor{currentfill}{rgb}{0.000000,0.000000,0.000000}%
\pgfsetfillcolor{currentfill}%
\pgfsetlinewidth{0.803000pt}%
\definecolor{currentstroke}{rgb}{0.000000,0.000000,0.000000}%
\pgfsetstrokecolor{currentstroke}%
\pgfsetdash{}{0pt}%
\pgfsys@defobject{currentmarker}{\pgfqpoint{0.000000in}{-0.048611in}}{\pgfqpoint{0.000000in}{0.000000in}}{%
\pgfpathmoveto{\pgfqpoint{0.000000in}{0.000000in}}%
\pgfpathlineto{\pgfqpoint{0.000000in}{-0.048611in}}%
\pgfusepath{stroke,fill}%
}%
\begin{pgfscope}%
\pgfsys@transformshift{4.586667in}{1.440000in}%
\pgfsys@useobject{currentmarker}{}%
\end{pgfscope}%
\end{pgfscope}%
\begin{pgfscope}%
\definecolor{textcolor}{rgb}{0.000000,0.000000,0.000000}%
\pgfsetstrokecolor{textcolor}%
\pgfsetfillcolor{textcolor}%
\pgftext[x=4.586667in,y=1.342778in,,top]{\color{textcolor}\rmfamily\fontsize{10.000000}{12.000000}\selectfont \(\displaystyle {2.0}\)}%
\end{pgfscope}%
\begin{pgfscope}%
\pgfpathrectangle{\pgfqpoint{2.880000in}{1.440000in}}{\pgfqpoint{2.560000in}{1.440000in}}%
\pgfusepath{clip}%
\pgfsetrectcap%
\pgfsetroundjoin%
\pgfsetlinewidth{0.803000pt}%
\definecolor{currentstroke}{rgb}{0.690196,0.690196,0.690196}%
\pgfsetstrokecolor{currentstroke}%
\pgfsetdash{}{0pt}%
\pgfpathmoveto{\pgfqpoint{5.013333in}{1.440000in}}%
\pgfpathlineto{\pgfqpoint{5.013333in}{2.880000in}}%
\pgfusepath{stroke}%
\end{pgfscope}%
\begin{pgfscope}%
\pgfsetbuttcap%
\pgfsetroundjoin%
\definecolor{currentfill}{rgb}{0.000000,0.000000,0.000000}%
\pgfsetfillcolor{currentfill}%
\pgfsetlinewidth{0.803000pt}%
\definecolor{currentstroke}{rgb}{0.000000,0.000000,0.000000}%
\pgfsetstrokecolor{currentstroke}%
\pgfsetdash{}{0pt}%
\pgfsys@defobject{currentmarker}{\pgfqpoint{0.000000in}{-0.048611in}}{\pgfqpoint{0.000000in}{0.000000in}}{%
\pgfpathmoveto{\pgfqpoint{0.000000in}{0.000000in}}%
\pgfpathlineto{\pgfqpoint{0.000000in}{-0.048611in}}%
\pgfusepath{stroke,fill}%
}%
\begin{pgfscope}%
\pgfsys@transformshift{5.013333in}{1.440000in}%
\pgfsys@useobject{currentmarker}{}%
\end{pgfscope}%
\end{pgfscope}%
\begin{pgfscope}%
\definecolor{textcolor}{rgb}{0.000000,0.000000,0.000000}%
\pgfsetstrokecolor{textcolor}%
\pgfsetfillcolor{textcolor}%
\pgftext[x=5.013333in,y=1.342778in,,top]{\color{textcolor}\rmfamily\fontsize{10.000000}{12.000000}\selectfont \(\displaystyle {2.5}\)}%
\end{pgfscope}%
\begin{pgfscope}%
\pgfpathrectangle{\pgfqpoint{2.880000in}{1.440000in}}{\pgfqpoint{2.560000in}{1.440000in}}%
\pgfusepath{clip}%
\pgfsetrectcap%
\pgfsetroundjoin%
\pgfsetlinewidth{0.803000pt}%
\definecolor{currentstroke}{rgb}{0.690196,0.690196,0.690196}%
\pgfsetstrokecolor{currentstroke}%
\pgfsetdash{}{0pt}%
\pgfpathmoveto{\pgfqpoint{5.440000in}{1.440000in}}%
\pgfpathlineto{\pgfqpoint{5.440000in}{2.880000in}}%
\pgfusepath{stroke}%
\end{pgfscope}%
\begin{pgfscope}%
\pgfsetbuttcap%
\pgfsetroundjoin%
\definecolor{currentfill}{rgb}{0.000000,0.000000,0.000000}%
\pgfsetfillcolor{currentfill}%
\pgfsetlinewidth{0.803000pt}%
\definecolor{currentstroke}{rgb}{0.000000,0.000000,0.000000}%
\pgfsetstrokecolor{currentstroke}%
\pgfsetdash{}{0pt}%
\pgfsys@defobject{currentmarker}{\pgfqpoint{0.000000in}{-0.048611in}}{\pgfqpoint{0.000000in}{0.000000in}}{%
\pgfpathmoveto{\pgfqpoint{0.000000in}{0.000000in}}%
\pgfpathlineto{\pgfqpoint{0.000000in}{-0.048611in}}%
\pgfusepath{stroke,fill}%
}%
\begin{pgfscope}%
\pgfsys@transformshift{5.440000in}{1.440000in}%
\pgfsys@useobject{currentmarker}{}%
\end{pgfscope}%
\end{pgfscope}%
\begin{pgfscope}%
\definecolor{textcolor}{rgb}{0.000000,0.000000,0.000000}%
\pgfsetstrokecolor{textcolor}%
\pgfsetfillcolor{textcolor}%
\pgftext[x=5.440000in,y=1.342778in,,top]{\color{textcolor}\rmfamily\fontsize{10.000000}{12.000000}\selectfont \(\displaystyle {3.0}\)}%
\end{pgfscope}%
\begin{pgfscope}%
\pgfpathrectangle{\pgfqpoint{2.880000in}{1.440000in}}{\pgfqpoint{2.560000in}{1.440000in}}%
\pgfusepath{clip}%
\pgfsetrectcap%
\pgfsetroundjoin%
\pgfsetlinewidth{0.803000pt}%
\definecolor{currentstroke}{rgb}{0.690196,0.690196,0.690196}%
\pgfsetstrokecolor{currentstroke}%
\pgfsetdash{}{0pt}%
\pgfpathmoveto{\pgfqpoint{2.880000in}{1.695388in}}%
\pgfpathlineto{\pgfqpoint{5.440000in}{1.695388in}}%
\pgfusepath{stroke}%
\end{pgfscope}%
\begin{pgfscope}%
\pgfsetbuttcap%
\pgfsetroundjoin%
\definecolor{currentfill}{rgb}{0.000000,0.000000,0.000000}%
\pgfsetfillcolor{currentfill}%
\pgfsetlinewidth{0.803000pt}%
\definecolor{currentstroke}{rgb}{0.000000,0.000000,0.000000}%
\pgfsetstrokecolor{currentstroke}%
\pgfsetdash{}{0pt}%
\pgfsys@defobject{currentmarker}{\pgfqpoint{-0.048611in}{0.000000in}}{\pgfqpoint{-0.000000in}{0.000000in}}{%
\pgfpathmoveto{\pgfqpoint{-0.000000in}{0.000000in}}%
\pgfpathlineto{\pgfqpoint{-0.048611in}{0.000000in}}%
\pgfusepath{stroke,fill}%
}%
\begin{pgfscope}%
\pgfsys@transformshift{2.880000in}{1.695388in}%
\pgfsys@useobject{currentmarker}{}%
\end{pgfscope}%
\end{pgfscope}%
\begin{pgfscope}%
\definecolor{textcolor}{rgb}{0.000000,0.000000,0.000000}%
\pgfsetstrokecolor{textcolor}%
\pgfsetfillcolor{textcolor}%
\pgftext[x=2.497283in, y=1.647174in, left, base]{\color{textcolor}\rmfamily\fontsize{10.000000}{12.000000}\selectfont \(\displaystyle {\ensuremath{-}0.4}\)}%
\end{pgfscope}%
\begin{pgfscope}%
\pgfpathrectangle{\pgfqpoint{2.880000in}{1.440000in}}{\pgfqpoint{2.560000in}{1.440000in}}%
\pgfusepath{clip}%
\pgfsetrectcap%
\pgfsetroundjoin%
\pgfsetlinewidth{0.803000pt}%
\definecolor{currentstroke}{rgb}{0.690196,0.690196,0.690196}%
\pgfsetstrokecolor{currentstroke}%
\pgfsetdash{}{0pt}%
\pgfpathmoveto{\pgfqpoint{2.880000in}{2.201945in}}%
\pgfpathlineto{\pgfqpoint{5.440000in}{2.201945in}}%
\pgfusepath{stroke}%
\end{pgfscope}%
\begin{pgfscope}%
\pgfsetbuttcap%
\pgfsetroundjoin%
\definecolor{currentfill}{rgb}{0.000000,0.000000,0.000000}%
\pgfsetfillcolor{currentfill}%
\pgfsetlinewidth{0.803000pt}%
\definecolor{currentstroke}{rgb}{0.000000,0.000000,0.000000}%
\pgfsetstrokecolor{currentstroke}%
\pgfsetdash{}{0pt}%
\pgfsys@defobject{currentmarker}{\pgfqpoint{-0.048611in}{0.000000in}}{\pgfqpoint{-0.000000in}{0.000000in}}{%
\pgfpathmoveto{\pgfqpoint{-0.000000in}{0.000000in}}%
\pgfpathlineto{\pgfqpoint{-0.048611in}{0.000000in}}%
\pgfusepath{stroke,fill}%
}%
\begin{pgfscope}%
\pgfsys@transformshift{2.880000in}{2.201945in}%
\pgfsys@useobject{currentmarker}{}%
\end{pgfscope}%
\end{pgfscope}%
\begin{pgfscope}%
\definecolor{textcolor}{rgb}{0.000000,0.000000,0.000000}%
\pgfsetstrokecolor{textcolor}%
\pgfsetfillcolor{textcolor}%
\pgftext[x=2.497283in, y=2.153732in, left, base]{\color{textcolor}\rmfamily\fontsize{10.000000}{12.000000}\selectfont \(\displaystyle {\ensuremath{-}0.2}\)}%
\end{pgfscope}%
\begin{pgfscope}%
\pgfpathrectangle{\pgfqpoint{2.880000in}{1.440000in}}{\pgfqpoint{2.560000in}{1.440000in}}%
\pgfusepath{clip}%
\pgfsetrectcap%
\pgfsetroundjoin%
\pgfsetlinewidth{0.803000pt}%
\definecolor{currentstroke}{rgb}{0.690196,0.690196,0.690196}%
\pgfsetstrokecolor{currentstroke}%
\pgfsetdash{}{0pt}%
\pgfpathmoveto{\pgfqpoint{2.880000in}{2.708503in}}%
\pgfpathlineto{\pgfqpoint{5.440000in}{2.708503in}}%
\pgfusepath{stroke}%
\end{pgfscope}%
\begin{pgfscope}%
\pgfsetbuttcap%
\pgfsetroundjoin%
\definecolor{currentfill}{rgb}{0.000000,0.000000,0.000000}%
\pgfsetfillcolor{currentfill}%
\pgfsetlinewidth{0.803000pt}%
\definecolor{currentstroke}{rgb}{0.000000,0.000000,0.000000}%
\pgfsetstrokecolor{currentstroke}%
\pgfsetdash{}{0pt}%
\pgfsys@defobject{currentmarker}{\pgfqpoint{-0.048611in}{0.000000in}}{\pgfqpoint{-0.000000in}{0.000000in}}{%
\pgfpathmoveto{\pgfqpoint{-0.000000in}{0.000000in}}%
\pgfpathlineto{\pgfqpoint{-0.048611in}{0.000000in}}%
\pgfusepath{stroke,fill}%
}%
\begin{pgfscope}%
\pgfsys@transformshift{2.880000in}{2.708503in}%
\pgfsys@useobject{currentmarker}{}%
\end{pgfscope}%
\end{pgfscope}%
\begin{pgfscope}%
\definecolor{textcolor}{rgb}{0.000000,0.000000,0.000000}%
\pgfsetstrokecolor{textcolor}%
\pgfsetfillcolor{textcolor}%
\pgftext[x=2.605308in, y=2.660289in, left, base]{\color{textcolor}\rmfamily\fontsize{10.000000}{12.000000}\selectfont \(\displaystyle {0.0}\)}%
\end{pgfscope}%
\begin{pgfscope}%
\pgfpathrectangle{\pgfqpoint{2.880000in}{1.440000in}}{\pgfqpoint{2.560000in}{1.440000in}}%
\pgfusepath{clip}%
\pgfsetrectcap%
\pgfsetroundjoin%
\pgfsetlinewidth{1.505625pt}%
\definecolor{currentstroke}{rgb}{0.121569,0.466667,0.705882}%
\pgfsetstrokecolor{currentstroke}%
\pgfsetdash{}{0pt}%
\pgfpathmoveto{\pgfqpoint{2.880000in}{2.708503in}}%
\pgfpathlineto{\pgfqpoint{3.734187in}{2.708100in}}%
\pgfpathlineto{\pgfqpoint{3.736747in}{2.704617in}}%
\pgfpathlineto{\pgfqpoint{3.741013in}{2.692670in}}%
\pgfpathlineto{\pgfqpoint{3.934720in}{2.090591in}}%
\pgfpathlineto{\pgfqpoint{3.936427in}{2.094620in}}%
\pgfpathlineto{\pgfqpoint{4.163413in}{2.798255in}}%
\pgfpathlineto{\pgfqpoint{4.174507in}{2.806165in}}%
\pgfpathlineto{\pgfqpoint{4.185600in}{2.811359in}}%
\pgfpathlineto{\pgfqpoint{4.196693in}{2.814032in}}%
\pgfpathlineto{\pgfqpoint{4.207787in}{2.814454in}}%
\pgfpathlineto{\pgfqpoint{4.219733in}{2.812750in}}%
\pgfpathlineto{\pgfqpoint{4.233387in}{2.808582in}}%
\pgfpathlineto{\pgfqpoint{4.250453in}{2.800931in}}%
\pgfpathlineto{\pgfqpoint{4.275200in}{2.787072in}}%
\pgfpathlineto{\pgfqpoint{4.332373in}{2.754297in}}%
\pgfpathlineto{\pgfqpoint{4.359680in}{2.741680in}}%
\pgfpathlineto{\pgfqpoint{4.386133in}{2.731905in}}%
\pgfpathlineto{\pgfqpoint{4.415147in}{2.723642in}}%
\pgfpathlineto{\pgfqpoint{4.448427in}{2.716571in}}%
\pgfpathlineto{\pgfqpoint{4.491947in}{2.709735in}}%
\pgfpathlineto{\pgfqpoint{4.564480in}{2.700918in}}%
\pgfpathlineto{\pgfqpoint{4.687360in}{2.688272in}}%
\pgfpathlineto{\pgfqpoint{4.793173in}{2.679586in}}%
\pgfpathlineto{\pgfqpoint{4.908373in}{2.672440in}}%
\pgfpathlineto{\pgfqpoint{5.041493in}{2.666455in}}%
\pgfpathlineto{\pgfqpoint{5.193387in}{2.661892in}}%
\pgfpathlineto{\pgfqpoint{5.370027in}{2.658879in}}%
\pgfpathlineto{\pgfqpoint{5.440853in}{2.658207in}}%
\pgfpathlineto{\pgfqpoint{5.440853in}{2.658207in}}%
\pgfusepath{stroke}%
\end{pgfscope}%
\begin{pgfscope}%
\pgfpathrectangle{\pgfqpoint{2.880000in}{1.440000in}}{\pgfqpoint{2.560000in}{1.440000in}}%
\pgfusepath{clip}%
\pgfsetbuttcap%
\pgfsetroundjoin%
\pgfsetlinewidth{1.505625pt}%
\definecolor{currentstroke}{rgb}{1.000000,0.498039,0.054902}%
\pgfsetstrokecolor{currentstroke}%
\pgfsetdash{{5.550000pt}{2.400000pt}}{0.000000pt}%
\pgfpathmoveto{\pgfqpoint{2.880000in}{2.708503in}}%
\pgfpathlineto{\pgfqpoint{3.733333in}{2.708503in}}%
\pgfpathlineto{\pgfqpoint{3.747246in}{2.627592in}}%
\pgfpathlineto{\pgfqpoint{3.761158in}{2.438305in}}%
\pgfpathlineto{\pgfqpoint{3.778815in}{2.140709in}}%
\pgfpathlineto{\pgfqpoint{3.802248in}{1.779001in}}%
\pgfpathlineto{\pgfqpoint{3.831997in}{1.509432in}}%
\pgfpathlineto{\pgfqpoint{3.869071in}{1.505455in}}%
\pgfpathlineto{\pgfqpoint{3.909078in}{1.774575in}}%
\pgfpathlineto{\pgfqpoint{3.950961in}{2.145505in}}%
\pgfpathlineto{\pgfqpoint{3.996504in}{2.479078in}}%
\pgfpathlineto{\pgfqpoint{4.050194in}{2.702073in}}%
\pgfpathlineto{\pgfqpoint{4.116199in}{2.784909in}}%
\pgfpathlineto{\pgfqpoint{4.184779in}{2.778078in}}%
\pgfpathlineto{\pgfqpoint{4.259688in}{2.754836in}}%
\pgfpathlineto{\pgfqpoint{4.361580in}{2.731755in}}%
\pgfpathlineto{\pgfqpoint{4.468099in}{2.710792in}}%
\pgfpathlineto{\pgfqpoint{4.595891in}{2.687837in}}%
\pgfpathlineto{\pgfqpoint{4.777827in}{2.664360in}}%
\pgfpathlineto{\pgfqpoint{4.977097in}{2.649464in}}%
\pgfpathlineto{\pgfqpoint{5.158344in}{2.642436in}}%
\pgfpathlineto{\pgfqpoint{5.327650in}{2.639964in}}%
\pgfpathlineto{\pgfqpoint{5.450000in}{2.640773in}}%
\pgfpathlineto{\pgfqpoint{5.450000in}{2.640773in}}%
\pgfusepath{stroke}%
\end{pgfscope}%
\begin{pgfscope}%
\pgfsetrectcap%
\pgfsetmiterjoin%
\pgfsetlinewidth{0.803000pt}%
\definecolor{currentstroke}{rgb}{0.000000,0.000000,0.000000}%
\pgfsetstrokecolor{currentstroke}%
\pgfsetdash{}{0pt}%
\pgfpathmoveto{\pgfqpoint{2.880000in}{1.440000in}}%
\pgfpathlineto{\pgfqpoint{2.880000in}{2.880000in}}%
\pgfusepath{stroke}%
\end{pgfscope}%
\begin{pgfscope}%
\pgfsetrectcap%
\pgfsetmiterjoin%
\pgfsetlinewidth{0.803000pt}%
\definecolor{currentstroke}{rgb}{0.000000,0.000000,0.000000}%
\pgfsetstrokecolor{currentstroke}%
\pgfsetdash{}{0pt}%
\pgfpathmoveto{\pgfqpoint{5.440000in}{1.440000in}}%
\pgfpathlineto{\pgfqpoint{5.440000in}{2.880000in}}%
\pgfusepath{stroke}%
\end{pgfscope}%
\begin{pgfscope}%
\pgfsetrectcap%
\pgfsetmiterjoin%
\pgfsetlinewidth{0.803000pt}%
\definecolor{currentstroke}{rgb}{0.000000,0.000000,0.000000}%
\pgfsetstrokecolor{currentstroke}%
\pgfsetdash{}{0pt}%
\pgfpathmoveto{\pgfqpoint{2.880000in}{1.440000in}}%
\pgfpathlineto{\pgfqpoint{5.440000in}{1.440000in}}%
\pgfusepath{stroke}%
\end{pgfscope}%
\begin{pgfscope}%
\pgfsetrectcap%
\pgfsetmiterjoin%
\pgfsetlinewidth{0.803000pt}%
\definecolor{currentstroke}{rgb}{0.000000,0.000000,0.000000}%
\pgfsetstrokecolor{currentstroke}%
\pgfsetdash{}{0pt}%
\pgfpathmoveto{\pgfqpoint{2.880000in}{2.880000in}}%
\pgfpathlineto{\pgfqpoint{5.440000in}{2.880000in}}%
\pgfusepath{stroke}%
\end{pgfscope}%
\end{pgfpicture}%
\makeatother%
\endgroup%
}
    \caption{Изменение положения руля высоты для линейной и нелинейной модели}
    \label{fig:delta_elevator}
    \end{minipage}
\end{figure}

\begin{figure}[H]
    \begin{minipage}{0.48\textwidth}
    \centering
    \resizebox{1.1\linewidth}{!}{%% Creator: Matplotlib, PGF backend
%%
%% To include the figure in your LaTeX document, write
%%   \input{<filename>.pgf}
%%
%% Make sure the required packages are loaded in your preamble
%%   \usepackage{pgf}
%%
%% Figures using additional raster images can only be included by \input if
%% they are in the same directory as the main LaTeX file. For loading figures
%% from other directories you can use the `import` package
%%   \usepackage{import}
%%
%% and then include the figures with
%%   \import{<path to file>}{<filename>.pgf}
%%
%% Matplotlib used the following preamble
%%   \usepackage[warn]{mathtext}
%%   \usepackage[T2A]{fontenc}
%%   \usepackage[utf8]{inputenc}
%%   \usepackage[english,russian]{babel}
%%
\begingroup%
\makeatletter%
\begin{pgfpicture}%
\pgfpathrectangle{\pgfpointorigin}{\pgfqpoint{6.400000in}{4.800000in}}%
\pgfusepath{use as bounding box, clip}%
\begin{pgfscope}%
\pgfsetbuttcap%
\pgfsetmiterjoin%
\definecolor{currentfill}{rgb}{1.000000,1.000000,1.000000}%
\pgfsetfillcolor{currentfill}%
\pgfsetlinewidth{0.000000pt}%
\definecolor{currentstroke}{rgb}{1.000000,1.000000,1.000000}%
\pgfsetstrokecolor{currentstroke}%
\pgfsetdash{}{0pt}%
\pgfpathmoveto{\pgfqpoint{0.000000in}{0.000000in}}%
\pgfpathlineto{\pgfqpoint{6.400000in}{0.000000in}}%
\pgfpathlineto{\pgfqpoint{6.400000in}{4.800000in}}%
\pgfpathlineto{\pgfqpoint{0.000000in}{4.800000in}}%
\pgfpathclose%
\pgfusepath{fill}%
\end{pgfscope}%
\begin{pgfscope}%
\pgfsetbuttcap%
\pgfsetmiterjoin%
\definecolor{currentfill}{rgb}{1.000000,1.000000,1.000000}%
\pgfsetfillcolor{currentfill}%
\pgfsetlinewidth{0.000000pt}%
\definecolor{currentstroke}{rgb}{0.000000,0.000000,0.000000}%
\pgfsetstrokecolor{currentstroke}%
\pgfsetstrokeopacity{0.000000}%
\pgfsetdash{}{0pt}%
\pgfpathmoveto{\pgfqpoint{0.800000in}{0.528000in}}%
\pgfpathlineto{\pgfqpoint{5.760000in}{0.528000in}}%
\pgfpathlineto{\pgfqpoint{5.760000in}{4.224000in}}%
\pgfpathlineto{\pgfqpoint{0.800000in}{4.224000in}}%
\pgfpathclose%
\pgfusepath{fill}%
\end{pgfscope}%
\begin{pgfscope}%
\pgfpathrectangle{\pgfqpoint{0.800000in}{0.528000in}}{\pgfqpoint{4.960000in}{3.696000in}}%
\pgfusepath{clip}%
\pgfsetrectcap%
\pgfsetroundjoin%
\pgfsetlinewidth{0.803000pt}%
\definecolor{currentstroke}{rgb}{0.690196,0.690196,0.690196}%
\pgfsetstrokecolor{currentstroke}%
\pgfsetdash{}{0pt}%
\pgfpathmoveto{\pgfqpoint{1.025455in}{0.528000in}}%
\pgfpathlineto{\pgfqpoint{1.025455in}{4.224000in}}%
\pgfusepath{stroke}%
\end{pgfscope}%
\begin{pgfscope}%
\pgfsetbuttcap%
\pgfsetroundjoin%
\definecolor{currentfill}{rgb}{0.000000,0.000000,0.000000}%
\pgfsetfillcolor{currentfill}%
\pgfsetlinewidth{0.803000pt}%
\definecolor{currentstroke}{rgb}{0.000000,0.000000,0.000000}%
\pgfsetstrokecolor{currentstroke}%
\pgfsetdash{}{0pt}%
\pgfsys@defobject{currentmarker}{\pgfqpoint{0.000000in}{-0.048611in}}{\pgfqpoint{0.000000in}{0.000000in}}{%
\pgfpathmoveto{\pgfqpoint{0.000000in}{0.000000in}}%
\pgfpathlineto{\pgfqpoint{0.000000in}{-0.048611in}}%
\pgfusepath{stroke,fill}%
}%
\begin{pgfscope}%
\pgfsys@transformshift{1.025455in}{0.528000in}%
\pgfsys@useobject{currentmarker}{}%
\end{pgfscope}%
\end{pgfscope}%
\begin{pgfscope}%
\definecolor{textcolor}{rgb}{0.000000,0.000000,0.000000}%
\pgfsetstrokecolor{textcolor}%
\pgfsetfillcolor{textcolor}%
\pgftext[x=1.025455in,y=0.430778in,,top]{\color{textcolor}\rmfamily\fontsize{10.000000}{12.000000}\selectfont \(\displaystyle {0}\)}%
\end{pgfscope}%
\begin{pgfscope}%
\pgfpathrectangle{\pgfqpoint{0.800000in}{0.528000in}}{\pgfqpoint{4.960000in}{3.696000in}}%
\pgfusepath{clip}%
\pgfsetrectcap%
\pgfsetroundjoin%
\pgfsetlinewidth{0.803000pt}%
\definecolor{currentstroke}{rgb}{0.690196,0.690196,0.690196}%
\pgfsetstrokecolor{currentstroke}%
\pgfsetdash{}{0pt}%
\pgfpathmoveto{\pgfqpoint{1.776970in}{0.528000in}}%
\pgfpathlineto{\pgfqpoint{1.776970in}{4.224000in}}%
\pgfusepath{stroke}%
\end{pgfscope}%
\begin{pgfscope}%
\pgfsetbuttcap%
\pgfsetroundjoin%
\definecolor{currentfill}{rgb}{0.000000,0.000000,0.000000}%
\pgfsetfillcolor{currentfill}%
\pgfsetlinewidth{0.803000pt}%
\definecolor{currentstroke}{rgb}{0.000000,0.000000,0.000000}%
\pgfsetstrokecolor{currentstroke}%
\pgfsetdash{}{0pt}%
\pgfsys@defobject{currentmarker}{\pgfqpoint{0.000000in}{-0.048611in}}{\pgfqpoint{0.000000in}{0.000000in}}{%
\pgfpathmoveto{\pgfqpoint{0.000000in}{0.000000in}}%
\pgfpathlineto{\pgfqpoint{0.000000in}{-0.048611in}}%
\pgfusepath{stroke,fill}%
}%
\begin{pgfscope}%
\pgfsys@transformshift{1.776970in}{0.528000in}%
\pgfsys@useobject{currentmarker}{}%
\end{pgfscope}%
\end{pgfscope}%
\begin{pgfscope}%
\definecolor{textcolor}{rgb}{0.000000,0.000000,0.000000}%
\pgfsetstrokecolor{textcolor}%
\pgfsetfillcolor{textcolor}%
\pgftext[x=1.776970in,y=0.430778in,,top]{\color{textcolor}\rmfamily\fontsize{10.000000}{12.000000}\selectfont \(\displaystyle {5}\)}%
\end{pgfscope}%
\begin{pgfscope}%
\pgfpathrectangle{\pgfqpoint{0.800000in}{0.528000in}}{\pgfqpoint{4.960000in}{3.696000in}}%
\pgfusepath{clip}%
\pgfsetrectcap%
\pgfsetroundjoin%
\pgfsetlinewidth{0.803000pt}%
\definecolor{currentstroke}{rgb}{0.690196,0.690196,0.690196}%
\pgfsetstrokecolor{currentstroke}%
\pgfsetdash{}{0pt}%
\pgfpathmoveto{\pgfqpoint{2.528485in}{0.528000in}}%
\pgfpathlineto{\pgfqpoint{2.528485in}{4.224000in}}%
\pgfusepath{stroke}%
\end{pgfscope}%
\begin{pgfscope}%
\pgfsetbuttcap%
\pgfsetroundjoin%
\definecolor{currentfill}{rgb}{0.000000,0.000000,0.000000}%
\pgfsetfillcolor{currentfill}%
\pgfsetlinewidth{0.803000pt}%
\definecolor{currentstroke}{rgb}{0.000000,0.000000,0.000000}%
\pgfsetstrokecolor{currentstroke}%
\pgfsetdash{}{0pt}%
\pgfsys@defobject{currentmarker}{\pgfqpoint{0.000000in}{-0.048611in}}{\pgfqpoint{0.000000in}{0.000000in}}{%
\pgfpathmoveto{\pgfqpoint{0.000000in}{0.000000in}}%
\pgfpathlineto{\pgfqpoint{0.000000in}{-0.048611in}}%
\pgfusepath{stroke,fill}%
}%
\begin{pgfscope}%
\pgfsys@transformshift{2.528485in}{0.528000in}%
\pgfsys@useobject{currentmarker}{}%
\end{pgfscope}%
\end{pgfscope}%
\begin{pgfscope}%
\definecolor{textcolor}{rgb}{0.000000,0.000000,0.000000}%
\pgfsetstrokecolor{textcolor}%
\pgfsetfillcolor{textcolor}%
\pgftext[x=2.528485in,y=0.430778in,,top]{\color{textcolor}\rmfamily\fontsize{10.000000}{12.000000}\selectfont \(\displaystyle {10}\)}%
\end{pgfscope}%
\begin{pgfscope}%
\pgfpathrectangle{\pgfqpoint{0.800000in}{0.528000in}}{\pgfqpoint{4.960000in}{3.696000in}}%
\pgfusepath{clip}%
\pgfsetrectcap%
\pgfsetroundjoin%
\pgfsetlinewidth{0.803000pt}%
\definecolor{currentstroke}{rgb}{0.690196,0.690196,0.690196}%
\pgfsetstrokecolor{currentstroke}%
\pgfsetdash{}{0pt}%
\pgfpathmoveto{\pgfqpoint{3.280000in}{0.528000in}}%
\pgfpathlineto{\pgfqpoint{3.280000in}{4.224000in}}%
\pgfusepath{stroke}%
\end{pgfscope}%
\begin{pgfscope}%
\pgfsetbuttcap%
\pgfsetroundjoin%
\definecolor{currentfill}{rgb}{0.000000,0.000000,0.000000}%
\pgfsetfillcolor{currentfill}%
\pgfsetlinewidth{0.803000pt}%
\definecolor{currentstroke}{rgb}{0.000000,0.000000,0.000000}%
\pgfsetstrokecolor{currentstroke}%
\pgfsetdash{}{0pt}%
\pgfsys@defobject{currentmarker}{\pgfqpoint{0.000000in}{-0.048611in}}{\pgfqpoint{0.000000in}{0.000000in}}{%
\pgfpathmoveto{\pgfqpoint{0.000000in}{0.000000in}}%
\pgfpathlineto{\pgfqpoint{0.000000in}{-0.048611in}}%
\pgfusepath{stroke,fill}%
}%
\begin{pgfscope}%
\pgfsys@transformshift{3.280000in}{0.528000in}%
\pgfsys@useobject{currentmarker}{}%
\end{pgfscope}%
\end{pgfscope}%
\begin{pgfscope}%
\definecolor{textcolor}{rgb}{0.000000,0.000000,0.000000}%
\pgfsetstrokecolor{textcolor}%
\pgfsetfillcolor{textcolor}%
\pgftext[x=3.280000in,y=0.430778in,,top]{\color{textcolor}\rmfamily\fontsize{10.000000}{12.000000}\selectfont \(\displaystyle {15}\)}%
\end{pgfscope}%
\begin{pgfscope}%
\pgfpathrectangle{\pgfqpoint{0.800000in}{0.528000in}}{\pgfqpoint{4.960000in}{3.696000in}}%
\pgfusepath{clip}%
\pgfsetrectcap%
\pgfsetroundjoin%
\pgfsetlinewidth{0.803000pt}%
\definecolor{currentstroke}{rgb}{0.690196,0.690196,0.690196}%
\pgfsetstrokecolor{currentstroke}%
\pgfsetdash{}{0pt}%
\pgfpathmoveto{\pgfqpoint{4.031515in}{0.528000in}}%
\pgfpathlineto{\pgfqpoint{4.031515in}{4.224000in}}%
\pgfusepath{stroke}%
\end{pgfscope}%
\begin{pgfscope}%
\pgfsetbuttcap%
\pgfsetroundjoin%
\definecolor{currentfill}{rgb}{0.000000,0.000000,0.000000}%
\pgfsetfillcolor{currentfill}%
\pgfsetlinewidth{0.803000pt}%
\definecolor{currentstroke}{rgb}{0.000000,0.000000,0.000000}%
\pgfsetstrokecolor{currentstroke}%
\pgfsetdash{}{0pt}%
\pgfsys@defobject{currentmarker}{\pgfqpoint{0.000000in}{-0.048611in}}{\pgfqpoint{0.000000in}{0.000000in}}{%
\pgfpathmoveto{\pgfqpoint{0.000000in}{0.000000in}}%
\pgfpathlineto{\pgfqpoint{0.000000in}{-0.048611in}}%
\pgfusepath{stroke,fill}%
}%
\begin{pgfscope}%
\pgfsys@transformshift{4.031515in}{0.528000in}%
\pgfsys@useobject{currentmarker}{}%
\end{pgfscope}%
\end{pgfscope}%
\begin{pgfscope}%
\definecolor{textcolor}{rgb}{0.000000,0.000000,0.000000}%
\pgfsetstrokecolor{textcolor}%
\pgfsetfillcolor{textcolor}%
\pgftext[x=4.031515in,y=0.430778in,,top]{\color{textcolor}\rmfamily\fontsize{10.000000}{12.000000}\selectfont \(\displaystyle {20}\)}%
\end{pgfscope}%
\begin{pgfscope}%
\pgfpathrectangle{\pgfqpoint{0.800000in}{0.528000in}}{\pgfqpoint{4.960000in}{3.696000in}}%
\pgfusepath{clip}%
\pgfsetrectcap%
\pgfsetroundjoin%
\pgfsetlinewidth{0.803000pt}%
\definecolor{currentstroke}{rgb}{0.690196,0.690196,0.690196}%
\pgfsetstrokecolor{currentstroke}%
\pgfsetdash{}{0pt}%
\pgfpathmoveto{\pgfqpoint{4.783030in}{0.528000in}}%
\pgfpathlineto{\pgfqpoint{4.783030in}{4.224000in}}%
\pgfusepath{stroke}%
\end{pgfscope}%
\begin{pgfscope}%
\pgfsetbuttcap%
\pgfsetroundjoin%
\definecolor{currentfill}{rgb}{0.000000,0.000000,0.000000}%
\pgfsetfillcolor{currentfill}%
\pgfsetlinewidth{0.803000pt}%
\definecolor{currentstroke}{rgb}{0.000000,0.000000,0.000000}%
\pgfsetstrokecolor{currentstroke}%
\pgfsetdash{}{0pt}%
\pgfsys@defobject{currentmarker}{\pgfqpoint{0.000000in}{-0.048611in}}{\pgfqpoint{0.000000in}{0.000000in}}{%
\pgfpathmoveto{\pgfqpoint{0.000000in}{0.000000in}}%
\pgfpathlineto{\pgfqpoint{0.000000in}{-0.048611in}}%
\pgfusepath{stroke,fill}%
}%
\begin{pgfscope}%
\pgfsys@transformshift{4.783030in}{0.528000in}%
\pgfsys@useobject{currentmarker}{}%
\end{pgfscope}%
\end{pgfscope}%
\begin{pgfscope}%
\definecolor{textcolor}{rgb}{0.000000,0.000000,0.000000}%
\pgfsetstrokecolor{textcolor}%
\pgfsetfillcolor{textcolor}%
\pgftext[x=4.783030in,y=0.430778in,,top]{\color{textcolor}\rmfamily\fontsize{10.000000}{12.000000}\selectfont \(\displaystyle {25}\)}%
\end{pgfscope}%
\begin{pgfscope}%
\pgfpathrectangle{\pgfqpoint{0.800000in}{0.528000in}}{\pgfqpoint{4.960000in}{3.696000in}}%
\pgfusepath{clip}%
\pgfsetrectcap%
\pgfsetroundjoin%
\pgfsetlinewidth{0.803000pt}%
\definecolor{currentstroke}{rgb}{0.690196,0.690196,0.690196}%
\pgfsetstrokecolor{currentstroke}%
\pgfsetdash{}{0pt}%
\pgfpathmoveto{\pgfqpoint{5.534545in}{0.528000in}}%
\pgfpathlineto{\pgfqpoint{5.534545in}{4.224000in}}%
\pgfusepath{stroke}%
\end{pgfscope}%
\begin{pgfscope}%
\pgfsetbuttcap%
\pgfsetroundjoin%
\definecolor{currentfill}{rgb}{0.000000,0.000000,0.000000}%
\pgfsetfillcolor{currentfill}%
\pgfsetlinewidth{0.803000pt}%
\definecolor{currentstroke}{rgb}{0.000000,0.000000,0.000000}%
\pgfsetstrokecolor{currentstroke}%
\pgfsetdash{}{0pt}%
\pgfsys@defobject{currentmarker}{\pgfqpoint{0.000000in}{-0.048611in}}{\pgfqpoint{0.000000in}{0.000000in}}{%
\pgfpathmoveto{\pgfqpoint{0.000000in}{0.000000in}}%
\pgfpathlineto{\pgfqpoint{0.000000in}{-0.048611in}}%
\pgfusepath{stroke,fill}%
}%
\begin{pgfscope}%
\pgfsys@transformshift{5.534545in}{0.528000in}%
\pgfsys@useobject{currentmarker}{}%
\end{pgfscope}%
\end{pgfscope}%
\begin{pgfscope}%
\definecolor{textcolor}{rgb}{0.000000,0.000000,0.000000}%
\pgfsetstrokecolor{textcolor}%
\pgfsetfillcolor{textcolor}%
\pgftext[x=5.534545in,y=0.430778in,,top]{\color{textcolor}\rmfamily\fontsize{10.000000}{12.000000}\selectfont \(\displaystyle {30}\)}%
\end{pgfscope}%
\begin{pgfscope}%
\definecolor{textcolor}{rgb}{0.000000,0.000000,0.000000}%
\pgfsetstrokecolor{textcolor}%
\pgfsetfillcolor{textcolor}%
\pgftext[x=3.280000in,y=0.251796in,,top]{\color{textcolor}\rmfamily\fontsize{10.000000}{12.000000}\selectfont \(\displaystyle t,\ с\)}%
\end{pgfscope}%
\begin{pgfscope}%
\pgfpathrectangle{\pgfqpoint{0.800000in}{0.528000in}}{\pgfqpoint{4.960000in}{3.696000in}}%
\pgfusepath{clip}%
\pgfsetrectcap%
\pgfsetroundjoin%
\pgfsetlinewidth{0.803000pt}%
\definecolor{currentstroke}{rgb}{0.690196,0.690196,0.690196}%
\pgfsetstrokecolor{currentstroke}%
\pgfsetdash{}{0pt}%
\pgfpathmoveto{\pgfqpoint{0.800000in}{0.828223in}}%
\pgfpathlineto{\pgfqpoint{5.760000in}{0.828223in}}%
\pgfusepath{stroke}%
\end{pgfscope}%
\begin{pgfscope}%
\pgfsetbuttcap%
\pgfsetroundjoin%
\definecolor{currentfill}{rgb}{0.000000,0.000000,0.000000}%
\pgfsetfillcolor{currentfill}%
\pgfsetlinewidth{0.803000pt}%
\definecolor{currentstroke}{rgb}{0.000000,0.000000,0.000000}%
\pgfsetstrokecolor{currentstroke}%
\pgfsetdash{}{0pt}%
\pgfsys@defobject{currentmarker}{\pgfqpoint{-0.048611in}{0.000000in}}{\pgfqpoint{-0.000000in}{0.000000in}}{%
\pgfpathmoveto{\pgfqpoint{-0.000000in}{0.000000in}}%
\pgfpathlineto{\pgfqpoint{-0.048611in}{0.000000in}}%
\pgfusepath{stroke,fill}%
}%
\begin{pgfscope}%
\pgfsys@transformshift{0.800000in}{0.828223in}%
\pgfsys@useobject{currentmarker}{}%
\end{pgfscope}%
\end{pgfscope}%
\begin{pgfscope}%
\definecolor{textcolor}{rgb}{0.000000,0.000000,0.000000}%
\pgfsetstrokecolor{textcolor}%
\pgfsetfillcolor{textcolor}%
\pgftext[x=0.278394in, y=0.780009in, left, base]{\color{textcolor}\rmfamily\fontsize{10.000000}{12.000000}\selectfont \(\displaystyle {\ensuremath{-}0.025}\)}%
\end{pgfscope}%
\begin{pgfscope}%
\pgfpathrectangle{\pgfqpoint{0.800000in}{0.528000in}}{\pgfqpoint{4.960000in}{3.696000in}}%
\pgfusepath{clip}%
\pgfsetrectcap%
\pgfsetroundjoin%
\pgfsetlinewidth{0.803000pt}%
\definecolor{currentstroke}{rgb}{0.690196,0.690196,0.690196}%
\pgfsetstrokecolor{currentstroke}%
\pgfsetdash{}{0pt}%
\pgfpathmoveto{\pgfqpoint{0.800000in}{1.262883in}}%
\pgfpathlineto{\pgfqpoint{5.760000in}{1.262883in}}%
\pgfusepath{stroke}%
\end{pgfscope}%
\begin{pgfscope}%
\pgfsetbuttcap%
\pgfsetroundjoin%
\definecolor{currentfill}{rgb}{0.000000,0.000000,0.000000}%
\pgfsetfillcolor{currentfill}%
\pgfsetlinewidth{0.803000pt}%
\definecolor{currentstroke}{rgb}{0.000000,0.000000,0.000000}%
\pgfsetstrokecolor{currentstroke}%
\pgfsetdash{}{0pt}%
\pgfsys@defobject{currentmarker}{\pgfqpoint{-0.048611in}{0.000000in}}{\pgfqpoint{-0.000000in}{0.000000in}}{%
\pgfpathmoveto{\pgfqpoint{-0.000000in}{0.000000in}}%
\pgfpathlineto{\pgfqpoint{-0.048611in}{0.000000in}}%
\pgfusepath{stroke,fill}%
}%
\begin{pgfscope}%
\pgfsys@transformshift{0.800000in}{1.262883in}%
\pgfsys@useobject{currentmarker}{}%
\end{pgfscope}%
\end{pgfscope}%
\begin{pgfscope}%
\definecolor{textcolor}{rgb}{0.000000,0.000000,0.000000}%
\pgfsetstrokecolor{textcolor}%
\pgfsetfillcolor{textcolor}%
\pgftext[x=0.386419in, y=1.214670in, left, base]{\color{textcolor}\rmfamily\fontsize{10.000000}{12.000000}\selectfont \(\displaystyle {0.000}\)}%
\end{pgfscope}%
\begin{pgfscope}%
\pgfpathrectangle{\pgfqpoint{0.800000in}{0.528000in}}{\pgfqpoint{4.960000in}{3.696000in}}%
\pgfusepath{clip}%
\pgfsetrectcap%
\pgfsetroundjoin%
\pgfsetlinewidth{0.803000pt}%
\definecolor{currentstroke}{rgb}{0.690196,0.690196,0.690196}%
\pgfsetstrokecolor{currentstroke}%
\pgfsetdash{}{0pt}%
\pgfpathmoveto{\pgfqpoint{0.800000in}{1.697544in}}%
\pgfpathlineto{\pgfqpoint{5.760000in}{1.697544in}}%
\pgfusepath{stroke}%
\end{pgfscope}%
\begin{pgfscope}%
\pgfsetbuttcap%
\pgfsetroundjoin%
\definecolor{currentfill}{rgb}{0.000000,0.000000,0.000000}%
\pgfsetfillcolor{currentfill}%
\pgfsetlinewidth{0.803000pt}%
\definecolor{currentstroke}{rgb}{0.000000,0.000000,0.000000}%
\pgfsetstrokecolor{currentstroke}%
\pgfsetdash{}{0pt}%
\pgfsys@defobject{currentmarker}{\pgfqpoint{-0.048611in}{0.000000in}}{\pgfqpoint{-0.000000in}{0.000000in}}{%
\pgfpathmoveto{\pgfqpoint{-0.000000in}{0.000000in}}%
\pgfpathlineto{\pgfqpoint{-0.048611in}{0.000000in}}%
\pgfusepath{stroke,fill}%
}%
\begin{pgfscope}%
\pgfsys@transformshift{0.800000in}{1.697544in}%
\pgfsys@useobject{currentmarker}{}%
\end{pgfscope}%
\end{pgfscope}%
\begin{pgfscope}%
\definecolor{textcolor}{rgb}{0.000000,0.000000,0.000000}%
\pgfsetstrokecolor{textcolor}%
\pgfsetfillcolor{textcolor}%
\pgftext[x=0.386419in, y=1.649330in, left, base]{\color{textcolor}\rmfamily\fontsize{10.000000}{12.000000}\selectfont \(\displaystyle {0.025}\)}%
\end{pgfscope}%
\begin{pgfscope}%
\pgfpathrectangle{\pgfqpoint{0.800000in}{0.528000in}}{\pgfqpoint{4.960000in}{3.696000in}}%
\pgfusepath{clip}%
\pgfsetrectcap%
\pgfsetroundjoin%
\pgfsetlinewidth{0.803000pt}%
\definecolor{currentstroke}{rgb}{0.690196,0.690196,0.690196}%
\pgfsetstrokecolor{currentstroke}%
\pgfsetdash{}{0pt}%
\pgfpathmoveto{\pgfqpoint{0.800000in}{2.132204in}}%
\pgfpathlineto{\pgfqpoint{5.760000in}{2.132204in}}%
\pgfusepath{stroke}%
\end{pgfscope}%
\begin{pgfscope}%
\pgfsetbuttcap%
\pgfsetroundjoin%
\definecolor{currentfill}{rgb}{0.000000,0.000000,0.000000}%
\pgfsetfillcolor{currentfill}%
\pgfsetlinewidth{0.803000pt}%
\definecolor{currentstroke}{rgb}{0.000000,0.000000,0.000000}%
\pgfsetstrokecolor{currentstroke}%
\pgfsetdash{}{0pt}%
\pgfsys@defobject{currentmarker}{\pgfqpoint{-0.048611in}{0.000000in}}{\pgfqpoint{-0.000000in}{0.000000in}}{%
\pgfpathmoveto{\pgfqpoint{-0.000000in}{0.000000in}}%
\pgfpathlineto{\pgfqpoint{-0.048611in}{0.000000in}}%
\pgfusepath{stroke,fill}%
}%
\begin{pgfscope}%
\pgfsys@transformshift{0.800000in}{2.132204in}%
\pgfsys@useobject{currentmarker}{}%
\end{pgfscope}%
\end{pgfscope}%
\begin{pgfscope}%
\definecolor{textcolor}{rgb}{0.000000,0.000000,0.000000}%
\pgfsetstrokecolor{textcolor}%
\pgfsetfillcolor{textcolor}%
\pgftext[x=0.386419in, y=2.083991in, left, base]{\color{textcolor}\rmfamily\fontsize{10.000000}{12.000000}\selectfont \(\displaystyle {0.050}\)}%
\end{pgfscope}%
\begin{pgfscope}%
\pgfpathrectangle{\pgfqpoint{0.800000in}{0.528000in}}{\pgfqpoint{4.960000in}{3.696000in}}%
\pgfusepath{clip}%
\pgfsetrectcap%
\pgfsetroundjoin%
\pgfsetlinewidth{0.803000pt}%
\definecolor{currentstroke}{rgb}{0.690196,0.690196,0.690196}%
\pgfsetstrokecolor{currentstroke}%
\pgfsetdash{}{0pt}%
\pgfpathmoveto{\pgfqpoint{0.800000in}{2.566865in}}%
\pgfpathlineto{\pgfqpoint{5.760000in}{2.566865in}}%
\pgfusepath{stroke}%
\end{pgfscope}%
\begin{pgfscope}%
\pgfsetbuttcap%
\pgfsetroundjoin%
\definecolor{currentfill}{rgb}{0.000000,0.000000,0.000000}%
\pgfsetfillcolor{currentfill}%
\pgfsetlinewidth{0.803000pt}%
\definecolor{currentstroke}{rgb}{0.000000,0.000000,0.000000}%
\pgfsetstrokecolor{currentstroke}%
\pgfsetdash{}{0pt}%
\pgfsys@defobject{currentmarker}{\pgfqpoint{-0.048611in}{0.000000in}}{\pgfqpoint{-0.000000in}{0.000000in}}{%
\pgfpathmoveto{\pgfqpoint{-0.000000in}{0.000000in}}%
\pgfpathlineto{\pgfqpoint{-0.048611in}{0.000000in}}%
\pgfusepath{stroke,fill}%
}%
\begin{pgfscope}%
\pgfsys@transformshift{0.800000in}{2.566865in}%
\pgfsys@useobject{currentmarker}{}%
\end{pgfscope}%
\end{pgfscope}%
\begin{pgfscope}%
\definecolor{textcolor}{rgb}{0.000000,0.000000,0.000000}%
\pgfsetstrokecolor{textcolor}%
\pgfsetfillcolor{textcolor}%
\pgftext[x=0.386419in, y=2.518651in, left, base]{\color{textcolor}\rmfamily\fontsize{10.000000}{12.000000}\selectfont \(\displaystyle {0.075}\)}%
\end{pgfscope}%
\begin{pgfscope}%
\pgfpathrectangle{\pgfqpoint{0.800000in}{0.528000in}}{\pgfqpoint{4.960000in}{3.696000in}}%
\pgfusepath{clip}%
\pgfsetrectcap%
\pgfsetroundjoin%
\pgfsetlinewidth{0.803000pt}%
\definecolor{currentstroke}{rgb}{0.690196,0.690196,0.690196}%
\pgfsetstrokecolor{currentstroke}%
\pgfsetdash{}{0pt}%
\pgfpathmoveto{\pgfqpoint{0.800000in}{3.001525in}}%
\pgfpathlineto{\pgfqpoint{5.760000in}{3.001525in}}%
\pgfusepath{stroke}%
\end{pgfscope}%
\begin{pgfscope}%
\pgfsetbuttcap%
\pgfsetroundjoin%
\definecolor{currentfill}{rgb}{0.000000,0.000000,0.000000}%
\pgfsetfillcolor{currentfill}%
\pgfsetlinewidth{0.803000pt}%
\definecolor{currentstroke}{rgb}{0.000000,0.000000,0.000000}%
\pgfsetstrokecolor{currentstroke}%
\pgfsetdash{}{0pt}%
\pgfsys@defobject{currentmarker}{\pgfqpoint{-0.048611in}{0.000000in}}{\pgfqpoint{-0.000000in}{0.000000in}}{%
\pgfpathmoveto{\pgfqpoint{-0.000000in}{0.000000in}}%
\pgfpathlineto{\pgfqpoint{-0.048611in}{0.000000in}}%
\pgfusepath{stroke,fill}%
}%
\begin{pgfscope}%
\pgfsys@transformshift{0.800000in}{3.001525in}%
\pgfsys@useobject{currentmarker}{}%
\end{pgfscope}%
\end{pgfscope}%
\begin{pgfscope}%
\definecolor{textcolor}{rgb}{0.000000,0.000000,0.000000}%
\pgfsetstrokecolor{textcolor}%
\pgfsetfillcolor{textcolor}%
\pgftext[x=0.386419in, y=2.953312in, left, base]{\color{textcolor}\rmfamily\fontsize{10.000000}{12.000000}\selectfont \(\displaystyle {0.100}\)}%
\end{pgfscope}%
\begin{pgfscope}%
\pgfpathrectangle{\pgfqpoint{0.800000in}{0.528000in}}{\pgfqpoint{4.960000in}{3.696000in}}%
\pgfusepath{clip}%
\pgfsetrectcap%
\pgfsetroundjoin%
\pgfsetlinewidth{0.803000pt}%
\definecolor{currentstroke}{rgb}{0.690196,0.690196,0.690196}%
\pgfsetstrokecolor{currentstroke}%
\pgfsetdash{}{0pt}%
\pgfpathmoveto{\pgfqpoint{0.800000in}{3.436186in}}%
\pgfpathlineto{\pgfqpoint{5.760000in}{3.436186in}}%
\pgfusepath{stroke}%
\end{pgfscope}%
\begin{pgfscope}%
\pgfsetbuttcap%
\pgfsetroundjoin%
\definecolor{currentfill}{rgb}{0.000000,0.000000,0.000000}%
\pgfsetfillcolor{currentfill}%
\pgfsetlinewidth{0.803000pt}%
\definecolor{currentstroke}{rgb}{0.000000,0.000000,0.000000}%
\pgfsetstrokecolor{currentstroke}%
\pgfsetdash{}{0pt}%
\pgfsys@defobject{currentmarker}{\pgfqpoint{-0.048611in}{0.000000in}}{\pgfqpoint{-0.000000in}{0.000000in}}{%
\pgfpathmoveto{\pgfqpoint{-0.000000in}{0.000000in}}%
\pgfpathlineto{\pgfqpoint{-0.048611in}{0.000000in}}%
\pgfusepath{stroke,fill}%
}%
\begin{pgfscope}%
\pgfsys@transformshift{0.800000in}{3.436186in}%
\pgfsys@useobject{currentmarker}{}%
\end{pgfscope}%
\end{pgfscope}%
\begin{pgfscope}%
\definecolor{textcolor}{rgb}{0.000000,0.000000,0.000000}%
\pgfsetstrokecolor{textcolor}%
\pgfsetfillcolor{textcolor}%
\pgftext[x=0.386419in, y=3.387972in, left, base]{\color{textcolor}\rmfamily\fontsize{10.000000}{12.000000}\selectfont \(\displaystyle {0.125}\)}%
\end{pgfscope}%
\begin{pgfscope}%
\pgfpathrectangle{\pgfqpoint{0.800000in}{0.528000in}}{\pgfqpoint{4.960000in}{3.696000in}}%
\pgfusepath{clip}%
\pgfsetrectcap%
\pgfsetroundjoin%
\pgfsetlinewidth{0.803000pt}%
\definecolor{currentstroke}{rgb}{0.690196,0.690196,0.690196}%
\pgfsetstrokecolor{currentstroke}%
\pgfsetdash{}{0pt}%
\pgfpathmoveto{\pgfqpoint{0.800000in}{3.870846in}}%
\pgfpathlineto{\pgfqpoint{5.760000in}{3.870846in}}%
\pgfusepath{stroke}%
\end{pgfscope}%
\begin{pgfscope}%
\pgfsetbuttcap%
\pgfsetroundjoin%
\definecolor{currentfill}{rgb}{0.000000,0.000000,0.000000}%
\pgfsetfillcolor{currentfill}%
\pgfsetlinewidth{0.803000pt}%
\definecolor{currentstroke}{rgb}{0.000000,0.000000,0.000000}%
\pgfsetstrokecolor{currentstroke}%
\pgfsetdash{}{0pt}%
\pgfsys@defobject{currentmarker}{\pgfqpoint{-0.048611in}{0.000000in}}{\pgfqpoint{-0.000000in}{0.000000in}}{%
\pgfpathmoveto{\pgfqpoint{-0.000000in}{0.000000in}}%
\pgfpathlineto{\pgfqpoint{-0.048611in}{0.000000in}}%
\pgfusepath{stroke,fill}%
}%
\begin{pgfscope}%
\pgfsys@transformshift{0.800000in}{3.870846in}%
\pgfsys@useobject{currentmarker}{}%
\end{pgfscope}%
\end{pgfscope}%
\begin{pgfscope}%
\definecolor{textcolor}{rgb}{0.000000,0.000000,0.000000}%
\pgfsetstrokecolor{textcolor}%
\pgfsetfillcolor{textcolor}%
\pgftext[x=0.386419in, y=3.822633in, left, base]{\color{textcolor}\rmfamily\fontsize{10.000000}{12.000000}\selectfont \(\displaystyle {0.150}\)}%
\end{pgfscope}%
\begin{pgfscope}%
\definecolor{textcolor}{rgb}{0.000000,0.000000,0.000000}%
\pgfsetstrokecolor{textcolor}%
\pgfsetfillcolor{textcolor}%
\pgftext[x=0.222838in,y=2.376000in,,bottom,rotate=90.000000]{\color{textcolor}\rmfamily\fontsize{10.000000}{12.000000}\selectfont \(\displaystyle \omega_z,\ рад/с\)}%
\end{pgfscope}%
\begin{pgfscope}%
\pgfpathrectangle{\pgfqpoint{0.800000in}{0.528000in}}{\pgfqpoint{4.960000in}{3.696000in}}%
\pgfusepath{clip}%
\pgfsetrectcap%
\pgfsetroundjoin%
\pgfsetlinewidth{1.505625pt}%
\definecolor{currentstroke}{rgb}{0.121569,0.466667,0.705882}%
\pgfsetstrokecolor{currentstroke}%
\pgfsetdash{}{0pt}%
\pgfpathmoveto{\pgfqpoint{1.025455in}{1.262883in}}%
\pgfpathlineto{\pgfqpoint{1.177261in}{1.263794in}}%
\pgfpathlineto{\pgfqpoint{1.178914in}{1.268534in}}%
\pgfpathlineto{\pgfqpoint{1.181469in}{1.284035in}}%
\pgfpathlineto{\pgfqpoint{1.184926in}{1.320676in}}%
\pgfpathlineto{\pgfqpoint{1.189435in}{1.395227in}}%
\pgfpathlineto{\pgfqpoint{1.194996in}{1.528234in}}%
\pgfpathlineto{\pgfqpoint{1.201610in}{1.744222in}}%
\pgfpathlineto{\pgfqpoint{1.209425in}{2.078344in}}%
\pgfpathlineto{\pgfqpoint{1.220698in}{2.570073in}}%
\pgfpathlineto{\pgfqpoint{1.227462in}{2.764800in}}%
\pgfpathlineto{\pgfqpoint{1.233023in}{2.867902in}}%
\pgfpathlineto{\pgfqpoint{1.237382in}{2.913290in}}%
\pgfpathlineto{\pgfqpoint{1.240388in}{2.926682in}}%
\pgfpathlineto{\pgfqpoint{1.242041in}{2.927873in}}%
\pgfpathlineto{\pgfqpoint{1.242192in}{2.927765in}}%
\pgfpathlineto{\pgfqpoint{1.243544in}{2.925177in}}%
\pgfpathlineto{\pgfqpoint{1.245648in}{2.915393in}}%
\pgfpathlineto{\pgfqpoint{1.248655in}{2.889340in}}%
\pgfpathlineto{\pgfqpoint{1.252863in}{2.830536in}}%
\pgfpathlineto{\pgfqpoint{1.278865in}{2.436692in}}%
\pgfpathlineto{\pgfqpoint{1.289387in}{2.323548in}}%
\pgfpathlineto{\pgfqpoint{1.301862in}{2.213878in}}%
\pgfpathlineto{\pgfqpoint{1.317343in}{2.097683in}}%
\pgfpathlineto{\pgfqpoint{1.333726in}{1.990955in}}%
\pgfpathlineto{\pgfqpoint{1.349658in}{1.901333in}}%
\pgfpathlineto{\pgfqpoint{1.365741in}{1.823334in}}%
\pgfpathlineto{\pgfqpoint{1.381973in}{1.755385in}}%
\pgfpathlineto{\pgfqpoint{1.398356in}{1.696143in}}%
\pgfpathlineto{\pgfqpoint{1.414890in}{1.644506in}}%
\pgfpathlineto{\pgfqpoint{1.431573in}{1.599521in}}%
\pgfpathlineto{\pgfqpoint{1.448558in}{1.560021in}}%
\pgfpathlineto{\pgfqpoint{1.465842in}{1.525395in}}%
\pgfpathlineto{\pgfqpoint{1.483428in}{1.495090in}}%
\pgfpathlineto{\pgfqpoint{1.501464in}{1.468394in}}%
\pgfpathlineto{\pgfqpoint{1.519952in}{1.444941in}}%
\pgfpathlineto{\pgfqpoint{1.539190in}{1.424082in}}%
\pgfpathlineto{\pgfqpoint{1.559331in}{1.405501in}}%
\pgfpathlineto{\pgfqpoint{1.580674in}{1.388837in}}%
\pgfpathlineto{\pgfqpoint{1.603370in}{1.373939in}}%
\pgfpathlineto{\pgfqpoint{1.627869in}{1.360521in}}%
\pgfpathlineto{\pgfqpoint{1.654773in}{1.348349in}}%
\pgfpathlineto{\pgfqpoint{1.684533in}{1.337355in}}%
\pgfpathlineto{\pgfqpoint{1.718201in}{1.327335in}}%
\pgfpathlineto{\pgfqpoint{1.756829in}{1.318216in}}%
\pgfpathlineto{\pgfqpoint{1.802070in}{1.309885in}}%
\pgfpathlineto{\pgfqpoint{1.856330in}{1.302231in}}%
\pgfpathlineto{\pgfqpoint{1.923064in}{1.295149in}}%
\pgfpathlineto{\pgfqpoint{2.016553in}{1.285477in}}%
\pgfpathlineto{\pgfqpoint{2.021813in}{1.280697in}}%
\pgfpathlineto{\pgfqpoint{2.027525in}{1.272287in}}%
\pgfpathlineto{\pgfqpoint{2.034589in}{1.257615in}}%
\pgfpathlineto{\pgfqpoint{2.044960in}{1.230276in}}%
\pgfpathlineto{\pgfqpoint{2.084640in}{1.121067in}}%
\pgfpathlineto{\pgfqpoint{2.103428in}{1.078048in}}%
\pgfpathlineto{\pgfqpoint{2.122216in}{1.040444in}}%
\pgfpathlineto{\pgfqpoint{2.141154in}{1.007422in}}%
\pgfpathlineto{\pgfqpoint{2.160393in}{0.978274in}}%
\pgfpathlineto{\pgfqpoint{2.180082in}{0.952413in}}%
\pgfpathlineto{\pgfqpoint{2.200373in}{0.929395in}}%
\pgfpathlineto{\pgfqpoint{2.221265in}{0.909038in}}%
\pgfpathlineto{\pgfqpoint{2.242759in}{0.891176in}}%
\pgfpathlineto{\pgfqpoint{2.265004in}{0.875568in}}%
\pgfpathlineto{\pgfqpoint{2.288000in}{0.862138in}}%
\pgfpathlineto{\pgfqpoint{2.311898in}{0.850760in}}%
\pgfpathlineto{\pgfqpoint{2.336698in}{0.841427in}}%
\pgfpathlineto{\pgfqpoint{2.362400in}{0.834135in}}%
\pgfpathlineto{\pgfqpoint{2.389154in}{0.828864in}}%
\pgfpathlineto{\pgfqpoint{2.416960in}{0.825656in}}%
\pgfpathlineto{\pgfqpoint{2.445968in}{0.824556in}}%
\pgfpathlineto{\pgfqpoint{2.476330in}{0.825645in}}%
\pgfpathlineto{\pgfqpoint{2.508044in}{0.829019in}}%
\pgfpathlineto{\pgfqpoint{2.541411in}{0.834823in}}%
\pgfpathlineto{\pgfqpoint{2.576582in}{0.843219in}}%
\pgfpathlineto{\pgfqpoint{2.613857in}{0.854426in}}%
\pgfpathlineto{\pgfqpoint{2.653687in}{0.868754in}}%
\pgfpathlineto{\pgfqpoint{2.696674in}{0.886625in}}%
\pgfpathlineto{\pgfqpoint{2.743568in}{0.908582in}}%
\pgfpathlineto{\pgfqpoint{2.795724in}{0.935512in}}%
\pgfpathlineto{\pgfqpoint{2.855995in}{0.969215in}}%
\pgfpathlineto{\pgfqpoint{2.931147in}{1.013924in}}%
\pgfpathlineto{\pgfqpoint{3.078444in}{1.104880in}}%
\pgfpathlineto{\pgfqpoint{3.178846in}{1.165350in}}%
\pgfpathlineto{\pgfqpoint{3.252795in}{1.207336in}}%
\pgfpathlineto{\pgfqpoint{3.317576in}{1.241609in}}%
\pgfpathlineto{\pgfqpoint{3.377396in}{1.270776in}}%
\pgfpathlineto{\pgfqpoint{3.433910in}{1.295886in}}%
\pgfpathlineto{\pgfqpoint{3.488320in}{1.317649in}}%
\pgfpathlineto{\pgfqpoint{3.541227in}{1.336436in}}%
\pgfpathlineto{\pgfqpoint{3.593232in}{1.352559in}}%
\pgfpathlineto{\pgfqpoint{3.644785in}{1.366223in}}%
\pgfpathlineto{\pgfqpoint{3.696189in}{1.377553in}}%
\pgfpathlineto{\pgfqpoint{3.747893in}{1.386674in}}%
\pgfpathlineto{\pgfqpoint{3.800199in}{1.393639in}}%
\pgfpathlineto{\pgfqpoint{3.853406in}{1.398477in}}%
\pgfpathlineto{\pgfqpoint{3.908116in}{1.401208in}}%
\pgfpathlineto{\pgfqpoint{3.964630in}{1.401791in}}%
\pgfpathlineto{\pgfqpoint{4.023699in}{1.400162in}}%
\pgfpathlineto{\pgfqpoint{4.086225in}{1.396191in}}%
\pgfpathlineto{\pgfqpoint{4.153411in}{1.389663in}}%
\pgfpathlineto{\pgfqpoint{4.227210in}{1.380217in}}%
\pgfpathlineto{\pgfqpoint{4.311530in}{1.367117in}}%
\pgfpathlineto{\pgfqpoint{4.415539in}{1.348594in}}%
\pgfpathlineto{\pgfqpoint{4.603268in}{1.312383in}}%
\pgfpathlineto{\pgfqpoint{4.757479in}{1.283731in}}%
\pgfpathlineto{\pgfqpoint{4.867501in}{1.265540in}}%
\pgfpathlineto{\pgfqpoint{4.965648in}{1.251542in}}%
\pgfpathlineto{\pgfqpoint{5.058536in}{1.240528in}}%
\pgfpathlineto{\pgfqpoint{5.149319in}{1.232000in}}%
\pgfpathlineto{\pgfqpoint{5.239952in}{1.225723in}}%
\pgfpathlineto{\pgfqpoint{5.332238in}{1.221571in}}%
\pgfpathlineto{\pgfqpoint{5.428131in}{1.219501in}}%
\pgfpathlineto{\pgfqpoint{5.529886in}{1.219558in}}%
\pgfpathlineto{\pgfqpoint{5.534545in}{1.219611in}}%
\pgfpathlineto{\pgfqpoint{5.534545in}{1.219611in}}%
\pgfusepath{stroke}%
\end{pgfscope}%
\begin{pgfscope}%
\pgfpathrectangle{\pgfqpoint{0.800000in}{0.528000in}}{\pgfqpoint{4.960000in}{3.696000in}}%
\pgfusepath{clip}%
\pgfsetbuttcap%
\pgfsetroundjoin%
\pgfsetlinewidth{1.505625pt}%
\definecolor{currentstroke}{rgb}{1.000000,0.498039,0.054902}%
\pgfsetstrokecolor{currentstroke}%
\pgfsetdash{{5.550000pt}{2.400000pt}}{0.000000pt}%
\pgfpathmoveto{\pgfqpoint{1.025455in}{1.262883in}}%
\pgfpathlineto{\pgfqpoint{1.175758in}{1.262883in}}%
\pgfpathlineto{\pgfqpoint{1.178208in}{1.268880in}}%
\pgfpathlineto{\pgfqpoint{1.180659in}{1.304850in}}%
\pgfpathlineto{\pgfqpoint{1.183768in}{1.416820in}}%
\pgfpathlineto{\pgfqpoint{1.187896in}{1.684714in}}%
\pgfpathlineto{\pgfqpoint{1.206713in}{3.455891in}}%
\pgfpathlineto{\pgfqpoint{1.214090in}{3.870869in}}%
\pgfpathlineto{\pgfqpoint{1.222112in}{4.056000in}}%
\pgfpathlineto{\pgfqpoint{1.231568in}{4.030882in}}%
\pgfpathlineto{\pgfqpoint{1.243194in}{3.840338in}}%
\pgfpathlineto{\pgfqpoint{1.268468in}{3.371383in}}%
\pgfpathlineto{\pgfqpoint{1.286415in}{3.083563in}}%
\pgfpathlineto{\pgfqpoint{1.305177in}{2.816674in}}%
\pgfpathlineto{\pgfqpoint{1.327685in}{2.542111in}}%
\pgfpathlineto{\pgfqpoint{1.359731in}{2.226893in}}%
\pgfpathlineto{\pgfqpoint{1.394830in}{1.959249in}}%
\pgfpathlineto{\pgfqpoint{1.426754in}{1.766993in}}%
\pgfpathlineto{\pgfqpoint{1.456575in}{1.620335in}}%
\pgfpathlineto{\pgfqpoint{1.483903in}{1.507601in}}%
\pgfpathlineto{\pgfqpoint{1.510987in}{1.411249in}}%
\pgfpathlineto{\pgfqpoint{1.540171in}{1.320918in}}%
\pgfpathlineto{\pgfqpoint{1.577269in}{1.222569in}}%
\pgfpathlineto{\pgfqpoint{1.604289in}{1.159348in}}%
\pgfpathlineto{\pgfqpoint{1.631309in}{1.102567in}}%
\pgfpathlineto{\pgfqpoint{1.662098in}{1.044597in}}%
\pgfpathlineto{\pgfqpoint{1.700127in}{0.980838in}}%
\pgfpathlineto{\pgfqpoint{1.726008in}{0.942957in}}%
\pgfpathlineto{\pgfqpoint{1.751889in}{0.908246in}}%
\pgfpathlineto{\pgfqpoint{1.782812in}{0.870706in}}%
\pgfpathlineto{\pgfqpoint{1.823539in}{0.828405in}}%
\pgfpathlineto{\pgfqpoint{1.847922in}{0.805084in}}%
\pgfpathlineto{\pgfqpoint{1.872305in}{0.784638in}}%
\pgfpathlineto{\pgfqpoint{1.903602in}{0.761935in}}%
\pgfpathlineto{\pgfqpoint{1.946330in}{0.735709in}}%
\pgfpathlineto{\pgfqpoint{1.991590in}{0.717235in}}%
\pgfpathlineto{\pgfqpoint{2.026843in}{0.706663in}}%
\pgfpathlineto{\pgfqpoint{2.056587in}{0.700502in}}%
\pgfpathlineto{\pgfqpoint{2.086332in}{0.696979in}}%
\pgfpathlineto{\pgfqpoint{2.118529in}{0.696000in}}%
\pgfpathlineto{\pgfqpoint{2.156214in}{0.697538in}}%
\pgfpathlineto{\pgfqpoint{2.203146in}{0.707378in}}%
\pgfpathlineto{\pgfqpoint{2.233803in}{0.715686in}}%
\pgfpathlineto{\pgfqpoint{2.297432in}{0.739077in}}%
\pgfpathlineto{\pgfqpoint{2.321926in}{0.750206in}}%
\pgfpathlineto{\pgfqpoint{2.361453in}{0.770053in}}%
\pgfpathlineto{\pgfqpoint{2.415149in}{0.801798in}}%
\pgfpathlineto{\pgfqpoint{2.481377in}{0.845655in}}%
\pgfpathlineto{\pgfqpoint{2.543072in}{0.889739in}}%
\pgfpathlineto{\pgfqpoint{2.602301in}{0.934496in}}%
\pgfpathlineto{\pgfqpoint{2.667514in}{0.985762in}}%
\pgfpathlineto{\pgfqpoint{2.752543in}{1.053734in}}%
\pgfpathlineto{\pgfqpoint{2.844215in}{1.124420in}}%
\pgfpathlineto{\pgfqpoint{2.905007in}{1.169754in}}%
\pgfpathlineto{\pgfqpoint{2.967332in}{1.213991in}}%
\pgfpathlineto{\pgfqpoint{3.000764in}{1.236708in}}%
\pgfpathlineto{\pgfqpoint{3.054905in}{1.269664in}}%
\pgfpathlineto{\pgfqpoint{3.081170in}{1.285131in}}%
\pgfpathlineto{\pgfqpoint{3.145961in}{1.320390in}}%
\pgfpathlineto{\pgfqpoint{3.206661in}{1.349690in}}%
\pgfpathlineto{\pgfqpoint{3.238980in}{1.363645in}}%
\pgfpathlineto{\pgfqpoint{3.271299in}{1.376328in}}%
\pgfpathlineto{\pgfqpoint{3.327639in}{1.395062in}}%
\pgfpathlineto{\pgfqpoint{3.358530in}{1.404048in}}%
\pgfpathlineto{\pgfqpoint{3.418907in}{1.419012in}}%
\pgfpathlineto{\pgfqpoint{3.480694in}{1.430320in}}%
\pgfpathlineto{\pgfqpoint{3.541233in}{1.437034in}}%
\pgfpathlineto{\pgfqpoint{3.603215in}{1.440073in}}%
\pgfpathlineto{\pgfqpoint{3.691430in}{1.439666in}}%
\pgfpathlineto{\pgfqpoint{3.752485in}{1.435492in}}%
\pgfpathlineto{\pgfqpoint{3.810570in}{1.428716in}}%
\pgfpathlineto{\pgfqpoint{3.904105in}{1.414094in}}%
\pgfpathlineto{\pgfqpoint{3.999075in}{1.396079in}}%
\pgfpathlineto{\pgfqpoint{4.031953in}{1.389231in}}%
\pgfpathlineto{\pgfqpoint{4.081936in}{1.376452in}}%
\pgfpathlineto{\pgfqpoint{4.266968in}{1.331198in}}%
\pgfpathlineto{\pgfqpoint{4.367480in}{1.304988in}}%
\pgfpathlineto{\pgfqpoint{4.412788in}{1.295980in}}%
\pgfpathlineto{\pgfqpoint{4.531704in}{1.269949in}}%
\pgfpathlineto{\pgfqpoint{4.597584in}{1.257248in}}%
\pgfpathlineto{\pgfqpoint{4.690706in}{1.241913in}}%
\pgfpathlineto{\pgfqpoint{4.780391in}{1.229023in}}%
\pgfpathlineto{\pgfqpoint{4.847088in}{1.220992in}}%
\pgfpathlineto{\pgfqpoint{4.991475in}{1.211082in}}%
\pgfpathlineto{\pgfqpoint{5.084117in}{1.207500in}}%
\pgfpathlineto{\pgfqpoint{5.175540in}{1.207326in}}%
\pgfpathlineto{\pgfqpoint{5.213045in}{1.208852in}}%
\pgfpathlineto{\pgfqpoint{5.236159in}{1.207907in}}%
\pgfpathlineto{\pgfqpoint{5.328861in}{1.210008in}}%
\pgfpathlineto{\pgfqpoint{5.416998in}{1.215107in}}%
\pgfpathlineto{\pgfqpoint{5.505812in}{1.219868in}}%
\pgfpathlineto{\pgfqpoint{5.534545in}{1.221876in}}%
\pgfpathlineto{\pgfqpoint{5.534545in}{1.221876in}}%
\pgfusepath{stroke}%
\end{pgfscope}%
\begin{pgfscope}%
\pgfsetrectcap%
\pgfsetmiterjoin%
\pgfsetlinewidth{0.803000pt}%
\definecolor{currentstroke}{rgb}{0.000000,0.000000,0.000000}%
\pgfsetstrokecolor{currentstroke}%
\pgfsetdash{}{0pt}%
\pgfpathmoveto{\pgfqpoint{0.800000in}{0.528000in}}%
\pgfpathlineto{\pgfqpoint{0.800000in}{4.224000in}}%
\pgfusepath{stroke}%
\end{pgfscope}%
\begin{pgfscope}%
\pgfsetrectcap%
\pgfsetmiterjoin%
\pgfsetlinewidth{0.803000pt}%
\definecolor{currentstroke}{rgb}{0.000000,0.000000,0.000000}%
\pgfsetstrokecolor{currentstroke}%
\pgfsetdash{}{0pt}%
\pgfpathmoveto{\pgfqpoint{5.760000in}{0.528000in}}%
\pgfpathlineto{\pgfqpoint{5.760000in}{4.224000in}}%
\pgfusepath{stroke}%
\end{pgfscope}%
\begin{pgfscope}%
\pgfsetrectcap%
\pgfsetmiterjoin%
\pgfsetlinewidth{0.803000pt}%
\definecolor{currentstroke}{rgb}{0.000000,0.000000,0.000000}%
\pgfsetstrokecolor{currentstroke}%
\pgfsetdash{}{0pt}%
\pgfpathmoveto{\pgfqpoint{0.800000in}{0.528000in}}%
\pgfpathlineto{\pgfqpoint{5.760000in}{0.528000in}}%
\pgfusepath{stroke}%
\end{pgfscope}%
\begin{pgfscope}%
\pgfsetrectcap%
\pgfsetmiterjoin%
\pgfsetlinewidth{0.803000pt}%
\definecolor{currentstroke}{rgb}{0.000000,0.000000,0.000000}%
\pgfsetstrokecolor{currentstroke}%
\pgfsetdash{}{0pt}%
\pgfpathmoveto{\pgfqpoint{0.800000in}{4.224000in}}%
\pgfpathlineto{\pgfqpoint{5.760000in}{4.224000in}}%
\pgfusepath{stroke}%
\end{pgfscope}%
\begin{pgfscope}%
\pgfsetbuttcap%
\pgfsetmiterjoin%
\definecolor{currentfill}{rgb}{1.000000,1.000000,1.000000}%
\pgfsetfillcolor{currentfill}%
\pgfsetfillopacity{0.800000}%
\pgfsetlinewidth{1.003750pt}%
\definecolor{currentstroke}{rgb}{0.800000,0.800000,0.800000}%
\pgfsetstrokecolor{currentstroke}%
\pgfsetstrokeopacity{0.800000}%
\pgfsetdash{}{0pt}%
\pgfpathmoveto{\pgfqpoint{3.299872in}{3.725556in}}%
\pgfpathlineto{\pgfqpoint{5.662778in}{3.725556in}}%
\pgfpathquadraticcurveto{\pgfqpoint{5.690556in}{3.725556in}}{\pgfqpoint{5.690556in}{3.753334in}}%
\pgfpathlineto{\pgfqpoint{5.690556in}{4.126778in}}%
\pgfpathquadraticcurveto{\pgfqpoint{5.690556in}{4.154556in}}{\pgfqpoint{5.662778in}{4.154556in}}%
\pgfpathlineto{\pgfqpoint{3.299872in}{4.154556in}}%
\pgfpathquadraticcurveto{\pgfqpoint{3.272094in}{4.154556in}}{\pgfqpoint{3.272094in}{4.126778in}}%
\pgfpathlineto{\pgfqpoint{3.272094in}{3.753334in}}%
\pgfpathquadraticcurveto{\pgfqpoint{3.272094in}{3.725556in}}{\pgfqpoint{3.299872in}{3.725556in}}%
\pgfpathclose%
\pgfusepath{stroke,fill}%
\end{pgfscope}%
\begin{pgfscope}%
\pgfsetrectcap%
\pgfsetroundjoin%
\pgfsetlinewidth{1.505625pt}%
\definecolor{currentstroke}{rgb}{0.121569,0.466667,0.705882}%
\pgfsetstrokecolor{currentstroke}%
\pgfsetdash{}{0pt}%
\pgfpathmoveto{\pgfqpoint{3.327650in}{4.050389in}}%
\pgfpathlineto{\pgfqpoint{3.605428in}{4.050389in}}%
\pgfusepath{stroke}%
\end{pgfscope}%
\begin{pgfscope}%
\definecolor{textcolor}{rgb}{0.000000,0.000000,0.000000}%
\pgfsetstrokecolor{textcolor}%
\pgfsetfillcolor{textcolor}%
\pgftext[x=3.716539in,y=4.001778in,left,base]{\color{textcolor}\rmfamily\fontsize{10.000000}{12.000000}\selectfont Нелинейная модель, \(\displaystyle M=0.61\)}%
\end{pgfscope}%
\begin{pgfscope}%
\pgfsetbuttcap%
\pgfsetroundjoin%
\pgfsetlinewidth{1.505625pt}%
\definecolor{currentstroke}{rgb}{1.000000,0.498039,0.054902}%
\pgfsetstrokecolor{currentstroke}%
\pgfsetdash{{5.550000pt}{2.400000pt}}{0.000000pt}%
\pgfpathmoveto{\pgfqpoint{3.327650in}{3.856723in}}%
\pgfpathlineto{\pgfqpoint{3.605428in}{3.856723in}}%
\pgfusepath{stroke}%
\end{pgfscope}%
\begin{pgfscope}%
\definecolor{textcolor}{rgb}{0.000000,0.000000,0.000000}%
\pgfsetstrokecolor{textcolor}%
\pgfsetfillcolor{textcolor}%
\pgftext[x=3.716539in,y=3.808112in,left,base]{\color{textcolor}\rmfamily\fontsize{10.000000}{12.000000}\selectfont Линейная модель, \(\displaystyle M=0.61\)}%
\end{pgfscope}%
\end{pgfpicture}%
\makeatother%
\endgroup%
}
    \caption{Изменение угловой скорости для линейной и нелинейной модели}
    \label{fig:model_omega_z}
    \end{minipage}
    \hfill
    \begin{minipage}{0.48\textwidth}
    \centering
    \resizebox{1.1\linewidth}{!}{%% Creator: Matplotlib, PGF backend
%%
%% To include the figure in your LaTeX document, write
%%   \input{<filename>.pgf}
%%
%% Make sure the required packages are loaded in your preamble
%%   \usepackage{pgf}
%%
%% Figures using additional raster images can only be included by \input if
%% they are in the same directory as the main LaTeX file. For loading figures
%% from other directories you can use the `import` package
%%   \usepackage{import}
%%
%% and then include the figures with
%%   \import{<path to file>}{<filename>.pgf}
%%
%% Matplotlib used the following preamble
%%   \usepackage[warn]{mathtext}
%%   \usepackage[T2A]{fontenc}
%%   \usepackage[utf8]{inputenc}
%%   \usepackage[english,russian]{babel}
%%
\begingroup%
\makeatletter%
\begin{pgfpicture}%
\pgfpathrectangle{\pgfpointorigin}{\pgfqpoint{6.400000in}{4.800000in}}%
\pgfusepath{use as bounding box, clip}%
\begin{pgfscope}%
\pgfsetbuttcap%
\pgfsetmiterjoin%
\definecolor{currentfill}{rgb}{1.000000,1.000000,1.000000}%
\pgfsetfillcolor{currentfill}%
\pgfsetlinewidth{0.000000pt}%
\definecolor{currentstroke}{rgb}{1.000000,1.000000,1.000000}%
\pgfsetstrokecolor{currentstroke}%
\pgfsetdash{}{0pt}%
\pgfpathmoveto{\pgfqpoint{0.000000in}{0.000000in}}%
\pgfpathlineto{\pgfqpoint{6.400000in}{0.000000in}}%
\pgfpathlineto{\pgfqpoint{6.400000in}{4.800000in}}%
\pgfpathlineto{\pgfqpoint{0.000000in}{4.800000in}}%
\pgfpathclose%
\pgfusepath{fill}%
\end{pgfscope}%
\begin{pgfscope}%
\pgfsetbuttcap%
\pgfsetmiterjoin%
\definecolor{currentfill}{rgb}{1.000000,1.000000,1.000000}%
\pgfsetfillcolor{currentfill}%
\pgfsetlinewidth{0.000000pt}%
\definecolor{currentstroke}{rgb}{0.000000,0.000000,0.000000}%
\pgfsetstrokecolor{currentstroke}%
\pgfsetstrokeopacity{0.000000}%
\pgfsetdash{}{0pt}%
\pgfpathmoveto{\pgfqpoint{0.800000in}{0.528000in}}%
\pgfpathlineto{\pgfqpoint{5.760000in}{0.528000in}}%
\pgfpathlineto{\pgfqpoint{5.760000in}{4.224000in}}%
\pgfpathlineto{\pgfqpoint{0.800000in}{4.224000in}}%
\pgfpathclose%
\pgfusepath{fill}%
\end{pgfscope}%
\begin{pgfscope}%
\pgfpathrectangle{\pgfqpoint{0.800000in}{0.528000in}}{\pgfqpoint{4.960000in}{3.696000in}}%
\pgfusepath{clip}%
\pgfsetrectcap%
\pgfsetroundjoin%
\pgfsetlinewidth{0.803000pt}%
\definecolor{currentstroke}{rgb}{0.690196,0.690196,0.690196}%
\pgfsetstrokecolor{currentstroke}%
\pgfsetdash{}{0pt}%
\pgfpathmoveto{\pgfqpoint{1.025455in}{0.528000in}}%
\pgfpathlineto{\pgfqpoint{1.025455in}{4.224000in}}%
\pgfusepath{stroke}%
\end{pgfscope}%
\begin{pgfscope}%
\pgfsetbuttcap%
\pgfsetroundjoin%
\definecolor{currentfill}{rgb}{0.000000,0.000000,0.000000}%
\pgfsetfillcolor{currentfill}%
\pgfsetlinewidth{0.803000pt}%
\definecolor{currentstroke}{rgb}{0.000000,0.000000,0.000000}%
\pgfsetstrokecolor{currentstroke}%
\pgfsetdash{}{0pt}%
\pgfsys@defobject{currentmarker}{\pgfqpoint{0.000000in}{-0.048611in}}{\pgfqpoint{0.000000in}{0.000000in}}{%
\pgfpathmoveto{\pgfqpoint{0.000000in}{0.000000in}}%
\pgfpathlineto{\pgfqpoint{0.000000in}{-0.048611in}}%
\pgfusepath{stroke,fill}%
}%
\begin{pgfscope}%
\pgfsys@transformshift{1.025455in}{0.528000in}%
\pgfsys@useobject{currentmarker}{}%
\end{pgfscope}%
\end{pgfscope}%
\begin{pgfscope}%
\definecolor{textcolor}{rgb}{0.000000,0.000000,0.000000}%
\pgfsetstrokecolor{textcolor}%
\pgfsetfillcolor{textcolor}%
\pgftext[x=1.025455in,y=0.430778in,,top]{\color{textcolor}\rmfamily\fontsize{10.000000}{12.000000}\selectfont \(\displaystyle {0}\)}%
\end{pgfscope}%
\begin{pgfscope}%
\pgfpathrectangle{\pgfqpoint{0.800000in}{0.528000in}}{\pgfqpoint{4.960000in}{3.696000in}}%
\pgfusepath{clip}%
\pgfsetrectcap%
\pgfsetroundjoin%
\pgfsetlinewidth{0.803000pt}%
\definecolor{currentstroke}{rgb}{0.690196,0.690196,0.690196}%
\pgfsetstrokecolor{currentstroke}%
\pgfsetdash{}{0pt}%
\pgfpathmoveto{\pgfqpoint{1.776970in}{0.528000in}}%
\pgfpathlineto{\pgfqpoint{1.776970in}{4.224000in}}%
\pgfusepath{stroke}%
\end{pgfscope}%
\begin{pgfscope}%
\pgfsetbuttcap%
\pgfsetroundjoin%
\definecolor{currentfill}{rgb}{0.000000,0.000000,0.000000}%
\pgfsetfillcolor{currentfill}%
\pgfsetlinewidth{0.803000pt}%
\definecolor{currentstroke}{rgb}{0.000000,0.000000,0.000000}%
\pgfsetstrokecolor{currentstroke}%
\pgfsetdash{}{0pt}%
\pgfsys@defobject{currentmarker}{\pgfqpoint{0.000000in}{-0.048611in}}{\pgfqpoint{0.000000in}{0.000000in}}{%
\pgfpathmoveto{\pgfqpoint{0.000000in}{0.000000in}}%
\pgfpathlineto{\pgfqpoint{0.000000in}{-0.048611in}}%
\pgfusepath{stroke,fill}%
}%
\begin{pgfscope}%
\pgfsys@transformshift{1.776970in}{0.528000in}%
\pgfsys@useobject{currentmarker}{}%
\end{pgfscope}%
\end{pgfscope}%
\begin{pgfscope}%
\definecolor{textcolor}{rgb}{0.000000,0.000000,0.000000}%
\pgfsetstrokecolor{textcolor}%
\pgfsetfillcolor{textcolor}%
\pgftext[x=1.776970in,y=0.430778in,,top]{\color{textcolor}\rmfamily\fontsize{10.000000}{12.000000}\selectfont \(\displaystyle {5}\)}%
\end{pgfscope}%
\begin{pgfscope}%
\pgfpathrectangle{\pgfqpoint{0.800000in}{0.528000in}}{\pgfqpoint{4.960000in}{3.696000in}}%
\pgfusepath{clip}%
\pgfsetrectcap%
\pgfsetroundjoin%
\pgfsetlinewidth{0.803000pt}%
\definecolor{currentstroke}{rgb}{0.690196,0.690196,0.690196}%
\pgfsetstrokecolor{currentstroke}%
\pgfsetdash{}{0pt}%
\pgfpathmoveto{\pgfqpoint{2.528485in}{0.528000in}}%
\pgfpathlineto{\pgfqpoint{2.528485in}{4.224000in}}%
\pgfusepath{stroke}%
\end{pgfscope}%
\begin{pgfscope}%
\pgfsetbuttcap%
\pgfsetroundjoin%
\definecolor{currentfill}{rgb}{0.000000,0.000000,0.000000}%
\pgfsetfillcolor{currentfill}%
\pgfsetlinewidth{0.803000pt}%
\definecolor{currentstroke}{rgb}{0.000000,0.000000,0.000000}%
\pgfsetstrokecolor{currentstroke}%
\pgfsetdash{}{0pt}%
\pgfsys@defobject{currentmarker}{\pgfqpoint{0.000000in}{-0.048611in}}{\pgfqpoint{0.000000in}{0.000000in}}{%
\pgfpathmoveto{\pgfqpoint{0.000000in}{0.000000in}}%
\pgfpathlineto{\pgfqpoint{0.000000in}{-0.048611in}}%
\pgfusepath{stroke,fill}%
}%
\begin{pgfscope}%
\pgfsys@transformshift{2.528485in}{0.528000in}%
\pgfsys@useobject{currentmarker}{}%
\end{pgfscope}%
\end{pgfscope}%
\begin{pgfscope}%
\definecolor{textcolor}{rgb}{0.000000,0.000000,0.000000}%
\pgfsetstrokecolor{textcolor}%
\pgfsetfillcolor{textcolor}%
\pgftext[x=2.528485in,y=0.430778in,,top]{\color{textcolor}\rmfamily\fontsize{10.000000}{12.000000}\selectfont \(\displaystyle {10}\)}%
\end{pgfscope}%
\begin{pgfscope}%
\pgfpathrectangle{\pgfqpoint{0.800000in}{0.528000in}}{\pgfqpoint{4.960000in}{3.696000in}}%
\pgfusepath{clip}%
\pgfsetrectcap%
\pgfsetroundjoin%
\pgfsetlinewidth{0.803000pt}%
\definecolor{currentstroke}{rgb}{0.690196,0.690196,0.690196}%
\pgfsetstrokecolor{currentstroke}%
\pgfsetdash{}{0pt}%
\pgfpathmoveto{\pgfqpoint{3.280000in}{0.528000in}}%
\pgfpathlineto{\pgfqpoint{3.280000in}{4.224000in}}%
\pgfusepath{stroke}%
\end{pgfscope}%
\begin{pgfscope}%
\pgfsetbuttcap%
\pgfsetroundjoin%
\definecolor{currentfill}{rgb}{0.000000,0.000000,0.000000}%
\pgfsetfillcolor{currentfill}%
\pgfsetlinewidth{0.803000pt}%
\definecolor{currentstroke}{rgb}{0.000000,0.000000,0.000000}%
\pgfsetstrokecolor{currentstroke}%
\pgfsetdash{}{0pt}%
\pgfsys@defobject{currentmarker}{\pgfqpoint{0.000000in}{-0.048611in}}{\pgfqpoint{0.000000in}{0.000000in}}{%
\pgfpathmoveto{\pgfqpoint{0.000000in}{0.000000in}}%
\pgfpathlineto{\pgfqpoint{0.000000in}{-0.048611in}}%
\pgfusepath{stroke,fill}%
}%
\begin{pgfscope}%
\pgfsys@transformshift{3.280000in}{0.528000in}%
\pgfsys@useobject{currentmarker}{}%
\end{pgfscope}%
\end{pgfscope}%
\begin{pgfscope}%
\definecolor{textcolor}{rgb}{0.000000,0.000000,0.000000}%
\pgfsetstrokecolor{textcolor}%
\pgfsetfillcolor{textcolor}%
\pgftext[x=3.280000in,y=0.430778in,,top]{\color{textcolor}\rmfamily\fontsize{10.000000}{12.000000}\selectfont \(\displaystyle {15}\)}%
\end{pgfscope}%
\begin{pgfscope}%
\pgfpathrectangle{\pgfqpoint{0.800000in}{0.528000in}}{\pgfqpoint{4.960000in}{3.696000in}}%
\pgfusepath{clip}%
\pgfsetrectcap%
\pgfsetroundjoin%
\pgfsetlinewidth{0.803000pt}%
\definecolor{currentstroke}{rgb}{0.690196,0.690196,0.690196}%
\pgfsetstrokecolor{currentstroke}%
\pgfsetdash{}{0pt}%
\pgfpathmoveto{\pgfqpoint{4.031515in}{0.528000in}}%
\pgfpathlineto{\pgfqpoint{4.031515in}{4.224000in}}%
\pgfusepath{stroke}%
\end{pgfscope}%
\begin{pgfscope}%
\pgfsetbuttcap%
\pgfsetroundjoin%
\definecolor{currentfill}{rgb}{0.000000,0.000000,0.000000}%
\pgfsetfillcolor{currentfill}%
\pgfsetlinewidth{0.803000pt}%
\definecolor{currentstroke}{rgb}{0.000000,0.000000,0.000000}%
\pgfsetstrokecolor{currentstroke}%
\pgfsetdash{}{0pt}%
\pgfsys@defobject{currentmarker}{\pgfqpoint{0.000000in}{-0.048611in}}{\pgfqpoint{0.000000in}{0.000000in}}{%
\pgfpathmoveto{\pgfqpoint{0.000000in}{0.000000in}}%
\pgfpathlineto{\pgfqpoint{0.000000in}{-0.048611in}}%
\pgfusepath{stroke,fill}%
}%
\begin{pgfscope}%
\pgfsys@transformshift{4.031515in}{0.528000in}%
\pgfsys@useobject{currentmarker}{}%
\end{pgfscope}%
\end{pgfscope}%
\begin{pgfscope}%
\definecolor{textcolor}{rgb}{0.000000,0.000000,0.000000}%
\pgfsetstrokecolor{textcolor}%
\pgfsetfillcolor{textcolor}%
\pgftext[x=4.031515in,y=0.430778in,,top]{\color{textcolor}\rmfamily\fontsize{10.000000}{12.000000}\selectfont \(\displaystyle {20}\)}%
\end{pgfscope}%
\begin{pgfscope}%
\pgfpathrectangle{\pgfqpoint{0.800000in}{0.528000in}}{\pgfqpoint{4.960000in}{3.696000in}}%
\pgfusepath{clip}%
\pgfsetrectcap%
\pgfsetroundjoin%
\pgfsetlinewidth{0.803000pt}%
\definecolor{currentstroke}{rgb}{0.690196,0.690196,0.690196}%
\pgfsetstrokecolor{currentstroke}%
\pgfsetdash{}{0pt}%
\pgfpathmoveto{\pgfqpoint{4.783030in}{0.528000in}}%
\pgfpathlineto{\pgfqpoint{4.783030in}{4.224000in}}%
\pgfusepath{stroke}%
\end{pgfscope}%
\begin{pgfscope}%
\pgfsetbuttcap%
\pgfsetroundjoin%
\definecolor{currentfill}{rgb}{0.000000,0.000000,0.000000}%
\pgfsetfillcolor{currentfill}%
\pgfsetlinewidth{0.803000pt}%
\definecolor{currentstroke}{rgb}{0.000000,0.000000,0.000000}%
\pgfsetstrokecolor{currentstroke}%
\pgfsetdash{}{0pt}%
\pgfsys@defobject{currentmarker}{\pgfqpoint{0.000000in}{-0.048611in}}{\pgfqpoint{0.000000in}{0.000000in}}{%
\pgfpathmoveto{\pgfqpoint{0.000000in}{0.000000in}}%
\pgfpathlineto{\pgfqpoint{0.000000in}{-0.048611in}}%
\pgfusepath{stroke,fill}%
}%
\begin{pgfscope}%
\pgfsys@transformshift{4.783030in}{0.528000in}%
\pgfsys@useobject{currentmarker}{}%
\end{pgfscope}%
\end{pgfscope}%
\begin{pgfscope}%
\definecolor{textcolor}{rgb}{0.000000,0.000000,0.000000}%
\pgfsetstrokecolor{textcolor}%
\pgfsetfillcolor{textcolor}%
\pgftext[x=4.783030in,y=0.430778in,,top]{\color{textcolor}\rmfamily\fontsize{10.000000}{12.000000}\selectfont \(\displaystyle {25}\)}%
\end{pgfscope}%
\begin{pgfscope}%
\pgfpathrectangle{\pgfqpoint{0.800000in}{0.528000in}}{\pgfqpoint{4.960000in}{3.696000in}}%
\pgfusepath{clip}%
\pgfsetrectcap%
\pgfsetroundjoin%
\pgfsetlinewidth{0.803000pt}%
\definecolor{currentstroke}{rgb}{0.690196,0.690196,0.690196}%
\pgfsetstrokecolor{currentstroke}%
\pgfsetdash{}{0pt}%
\pgfpathmoveto{\pgfqpoint{5.534545in}{0.528000in}}%
\pgfpathlineto{\pgfqpoint{5.534545in}{4.224000in}}%
\pgfusepath{stroke}%
\end{pgfscope}%
\begin{pgfscope}%
\pgfsetbuttcap%
\pgfsetroundjoin%
\definecolor{currentfill}{rgb}{0.000000,0.000000,0.000000}%
\pgfsetfillcolor{currentfill}%
\pgfsetlinewidth{0.803000pt}%
\definecolor{currentstroke}{rgb}{0.000000,0.000000,0.000000}%
\pgfsetstrokecolor{currentstroke}%
\pgfsetdash{}{0pt}%
\pgfsys@defobject{currentmarker}{\pgfqpoint{0.000000in}{-0.048611in}}{\pgfqpoint{0.000000in}{0.000000in}}{%
\pgfpathmoveto{\pgfqpoint{0.000000in}{0.000000in}}%
\pgfpathlineto{\pgfqpoint{0.000000in}{-0.048611in}}%
\pgfusepath{stroke,fill}%
}%
\begin{pgfscope}%
\pgfsys@transformshift{5.534545in}{0.528000in}%
\pgfsys@useobject{currentmarker}{}%
\end{pgfscope}%
\end{pgfscope}%
\begin{pgfscope}%
\definecolor{textcolor}{rgb}{0.000000,0.000000,0.000000}%
\pgfsetstrokecolor{textcolor}%
\pgfsetfillcolor{textcolor}%
\pgftext[x=5.534545in,y=0.430778in,,top]{\color{textcolor}\rmfamily\fontsize{10.000000}{12.000000}\selectfont \(\displaystyle {30}\)}%
\end{pgfscope}%
\begin{pgfscope}%
\definecolor{textcolor}{rgb}{0.000000,0.000000,0.000000}%
\pgfsetstrokecolor{textcolor}%
\pgfsetfillcolor{textcolor}%
\pgftext[x=3.280000in,y=0.251796in,,top]{\color{textcolor}\rmfamily\fontsize{10.000000}{12.000000}\selectfont \(\displaystyle t,\ с\)}%
\end{pgfscope}%
\begin{pgfscope}%
\pgfpathrectangle{\pgfqpoint{0.800000in}{0.528000in}}{\pgfqpoint{4.960000in}{3.696000in}}%
\pgfusepath{clip}%
\pgfsetrectcap%
\pgfsetroundjoin%
\pgfsetlinewidth{0.803000pt}%
\definecolor{currentstroke}{rgb}{0.690196,0.690196,0.690196}%
\pgfsetstrokecolor{currentstroke}%
\pgfsetdash{}{0pt}%
\pgfpathmoveto{\pgfqpoint{0.800000in}{0.671828in}}%
\pgfpathlineto{\pgfqpoint{5.760000in}{0.671828in}}%
\pgfusepath{stroke}%
\end{pgfscope}%
\begin{pgfscope}%
\pgfsetbuttcap%
\pgfsetroundjoin%
\definecolor{currentfill}{rgb}{0.000000,0.000000,0.000000}%
\pgfsetfillcolor{currentfill}%
\pgfsetlinewidth{0.803000pt}%
\definecolor{currentstroke}{rgb}{0.000000,0.000000,0.000000}%
\pgfsetstrokecolor{currentstroke}%
\pgfsetdash{}{0pt}%
\pgfsys@defobject{currentmarker}{\pgfqpoint{-0.048611in}{0.000000in}}{\pgfqpoint{-0.000000in}{0.000000in}}{%
\pgfpathmoveto{\pgfqpoint{-0.000000in}{0.000000in}}%
\pgfpathlineto{\pgfqpoint{-0.048611in}{0.000000in}}%
\pgfusepath{stroke,fill}%
}%
\begin{pgfscope}%
\pgfsys@transformshift{0.800000in}{0.671828in}%
\pgfsys@useobject{currentmarker}{}%
\end{pgfscope}%
\end{pgfscope}%
\begin{pgfscope}%
\definecolor{textcolor}{rgb}{0.000000,0.000000,0.000000}%
\pgfsetstrokecolor{textcolor}%
\pgfsetfillcolor{textcolor}%
\pgftext[x=0.278394in, y=0.623615in, left, base]{\color{textcolor}\rmfamily\fontsize{10.000000}{12.000000}\selectfont \(\displaystyle {\ensuremath{-}0.050}\)}%
\end{pgfscope}%
\begin{pgfscope}%
\pgfpathrectangle{\pgfqpoint{0.800000in}{0.528000in}}{\pgfqpoint{4.960000in}{3.696000in}}%
\pgfusepath{clip}%
\pgfsetrectcap%
\pgfsetroundjoin%
\pgfsetlinewidth{0.803000pt}%
\definecolor{currentstroke}{rgb}{0.690196,0.690196,0.690196}%
\pgfsetstrokecolor{currentstroke}%
\pgfsetdash{}{0pt}%
\pgfpathmoveto{\pgfqpoint{0.800000in}{1.092711in}}%
\pgfpathlineto{\pgfqpoint{5.760000in}{1.092711in}}%
\pgfusepath{stroke}%
\end{pgfscope}%
\begin{pgfscope}%
\pgfsetbuttcap%
\pgfsetroundjoin%
\definecolor{currentfill}{rgb}{0.000000,0.000000,0.000000}%
\pgfsetfillcolor{currentfill}%
\pgfsetlinewidth{0.803000pt}%
\definecolor{currentstroke}{rgb}{0.000000,0.000000,0.000000}%
\pgfsetstrokecolor{currentstroke}%
\pgfsetdash{}{0pt}%
\pgfsys@defobject{currentmarker}{\pgfqpoint{-0.048611in}{0.000000in}}{\pgfqpoint{-0.000000in}{0.000000in}}{%
\pgfpathmoveto{\pgfqpoint{-0.000000in}{0.000000in}}%
\pgfpathlineto{\pgfqpoint{-0.048611in}{0.000000in}}%
\pgfusepath{stroke,fill}%
}%
\begin{pgfscope}%
\pgfsys@transformshift{0.800000in}{1.092711in}%
\pgfsys@useobject{currentmarker}{}%
\end{pgfscope}%
\end{pgfscope}%
\begin{pgfscope}%
\definecolor{textcolor}{rgb}{0.000000,0.000000,0.000000}%
\pgfsetstrokecolor{textcolor}%
\pgfsetfillcolor{textcolor}%
\pgftext[x=0.278394in, y=1.044498in, left, base]{\color{textcolor}\rmfamily\fontsize{10.000000}{12.000000}\selectfont \(\displaystyle {\ensuremath{-}0.025}\)}%
\end{pgfscope}%
\begin{pgfscope}%
\pgfpathrectangle{\pgfqpoint{0.800000in}{0.528000in}}{\pgfqpoint{4.960000in}{3.696000in}}%
\pgfusepath{clip}%
\pgfsetrectcap%
\pgfsetroundjoin%
\pgfsetlinewidth{0.803000pt}%
\definecolor{currentstroke}{rgb}{0.690196,0.690196,0.690196}%
\pgfsetstrokecolor{currentstroke}%
\pgfsetdash{}{0pt}%
\pgfpathmoveto{\pgfqpoint{0.800000in}{1.513595in}}%
\pgfpathlineto{\pgfqpoint{5.760000in}{1.513595in}}%
\pgfusepath{stroke}%
\end{pgfscope}%
\begin{pgfscope}%
\pgfsetbuttcap%
\pgfsetroundjoin%
\definecolor{currentfill}{rgb}{0.000000,0.000000,0.000000}%
\pgfsetfillcolor{currentfill}%
\pgfsetlinewidth{0.803000pt}%
\definecolor{currentstroke}{rgb}{0.000000,0.000000,0.000000}%
\pgfsetstrokecolor{currentstroke}%
\pgfsetdash{}{0pt}%
\pgfsys@defobject{currentmarker}{\pgfqpoint{-0.048611in}{0.000000in}}{\pgfqpoint{-0.000000in}{0.000000in}}{%
\pgfpathmoveto{\pgfqpoint{-0.000000in}{0.000000in}}%
\pgfpathlineto{\pgfqpoint{-0.048611in}{0.000000in}}%
\pgfusepath{stroke,fill}%
}%
\begin{pgfscope}%
\pgfsys@transformshift{0.800000in}{1.513595in}%
\pgfsys@useobject{currentmarker}{}%
\end{pgfscope}%
\end{pgfscope}%
\begin{pgfscope}%
\definecolor{textcolor}{rgb}{0.000000,0.000000,0.000000}%
\pgfsetstrokecolor{textcolor}%
\pgfsetfillcolor{textcolor}%
\pgftext[x=0.386419in, y=1.465381in, left, base]{\color{textcolor}\rmfamily\fontsize{10.000000}{12.000000}\selectfont \(\displaystyle {0.000}\)}%
\end{pgfscope}%
\begin{pgfscope}%
\pgfpathrectangle{\pgfqpoint{0.800000in}{0.528000in}}{\pgfqpoint{4.960000in}{3.696000in}}%
\pgfusepath{clip}%
\pgfsetrectcap%
\pgfsetroundjoin%
\pgfsetlinewidth{0.803000pt}%
\definecolor{currentstroke}{rgb}{0.690196,0.690196,0.690196}%
\pgfsetstrokecolor{currentstroke}%
\pgfsetdash{}{0pt}%
\pgfpathmoveto{\pgfqpoint{0.800000in}{1.934478in}}%
\pgfpathlineto{\pgfqpoint{5.760000in}{1.934478in}}%
\pgfusepath{stroke}%
\end{pgfscope}%
\begin{pgfscope}%
\pgfsetbuttcap%
\pgfsetroundjoin%
\definecolor{currentfill}{rgb}{0.000000,0.000000,0.000000}%
\pgfsetfillcolor{currentfill}%
\pgfsetlinewidth{0.803000pt}%
\definecolor{currentstroke}{rgb}{0.000000,0.000000,0.000000}%
\pgfsetstrokecolor{currentstroke}%
\pgfsetdash{}{0pt}%
\pgfsys@defobject{currentmarker}{\pgfqpoint{-0.048611in}{0.000000in}}{\pgfqpoint{-0.000000in}{0.000000in}}{%
\pgfpathmoveto{\pgfqpoint{-0.000000in}{0.000000in}}%
\pgfpathlineto{\pgfqpoint{-0.048611in}{0.000000in}}%
\pgfusepath{stroke,fill}%
}%
\begin{pgfscope}%
\pgfsys@transformshift{0.800000in}{1.934478in}%
\pgfsys@useobject{currentmarker}{}%
\end{pgfscope}%
\end{pgfscope}%
\begin{pgfscope}%
\definecolor{textcolor}{rgb}{0.000000,0.000000,0.000000}%
\pgfsetstrokecolor{textcolor}%
\pgfsetfillcolor{textcolor}%
\pgftext[x=0.386419in, y=1.886264in, left, base]{\color{textcolor}\rmfamily\fontsize{10.000000}{12.000000}\selectfont \(\displaystyle {0.025}\)}%
\end{pgfscope}%
\begin{pgfscope}%
\pgfpathrectangle{\pgfqpoint{0.800000in}{0.528000in}}{\pgfqpoint{4.960000in}{3.696000in}}%
\pgfusepath{clip}%
\pgfsetrectcap%
\pgfsetroundjoin%
\pgfsetlinewidth{0.803000pt}%
\definecolor{currentstroke}{rgb}{0.690196,0.690196,0.690196}%
\pgfsetstrokecolor{currentstroke}%
\pgfsetdash{}{0pt}%
\pgfpathmoveto{\pgfqpoint{0.800000in}{2.355361in}}%
\pgfpathlineto{\pgfqpoint{5.760000in}{2.355361in}}%
\pgfusepath{stroke}%
\end{pgfscope}%
\begin{pgfscope}%
\pgfsetbuttcap%
\pgfsetroundjoin%
\definecolor{currentfill}{rgb}{0.000000,0.000000,0.000000}%
\pgfsetfillcolor{currentfill}%
\pgfsetlinewidth{0.803000pt}%
\definecolor{currentstroke}{rgb}{0.000000,0.000000,0.000000}%
\pgfsetstrokecolor{currentstroke}%
\pgfsetdash{}{0pt}%
\pgfsys@defobject{currentmarker}{\pgfqpoint{-0.048611in}{0.000000in}}{\pgfqpoint{-0.000000in}{0.000000in}}{%
\pgfpathmoveto{\pgfqpoint{-0.000000in}{0.000000in}}%
\pgfpathlineto{\pgfqpoint{-0.048611in}{0.000000in}}%
\pgfusepath{stroke,fill}%
}%
\begin{pgfscope}%
\pgfsys@transformshift{0.800000in}{2.355361in}%
\pgfsys@useobject{currentmarker}{}%
\end{pgfscope}%
\end{pgfscope}%
\begin{pgfscope}%
\definecolor{textcolor}{rgb}{0.000000,0.000000,0.000000}%
\pgfsetstrokecolor{textcolor}%
\pgfsetfillcolor{textcolor}%
\pgftext[x=0.386419in, y=2.307147in, left, base]{\color{textcolor}\rmfamily\fontsize{10.000000}{12.000000}\selectfont \(\displaystyle {0.050}\)}%
\end{pgfscope}%
\begin{pgfscope}%
\pgfpathrectangle{\pgfqpoint{0.800000in}{0.528000in}}{\pgfqpoint{4.960000in}{3.696000in}}%
\pgfusepath{clip}%
\pgfsetrectcap%
\pgfsetroundjoin%
\pgfsetlinewidth{0.803000pt}%
\definecolor{currentstroke}{rgb}{0.690196,0.690196,0.690196}%
\pgfsetstrokecolor{currentstroke}%
\pgfsetdash{}{0pt}%
\pgfpathmoveto{\pgfqpoint{0.800000in}{2.776244in}}%
\pgfpathlineto{\pgfqpoint{5.760000in}{2.776244in}}%
\pgfusepath{stroke}%
\end{pgfscope}%
\begin{pgfscope}%
\pgfsetbuttcap%
\pgfsetroundjoin%
\definecolor{currentfill}{rgb}{0.000000,0.000000,0.000000}%
\pgfsetfillcolor{currentfill}%
\pgfsetlinewidth{0.803000pt}%
\definecolor{currentstroke}{rgb}{0.000000,0.000000,0.000000}%
\pgfsetstrokecolor{currentstroke}%
\pgfsetdash{}{0pt}%
\pgfsys@defobject{currentmarker}{\pgfqpoint{-0.048611in}{0.000000in}}{\pgfqpoint{-0.000000in}{0.000000in}}{%
\pgfpathmoveto{\pgfqpoint{-0.000000in}{0.000000in}}%
\pgfpathlineto{\pgfqpoint{-0.048611in}{0.000000in}}%
\pgfusepath{stroke,fill}%
}%
\begin{pgfscope}%
\pgfsys@transformshift{0.800000in}{2.776244in}%
\pgfsys@useobject{currentmarker}{}%
\end{pgfscope}%
\end{pgfscope}%
\begin{pgfscope}%
\definecolor{textcolor}{rgb}{0.000000,0.000000,0.000000}%
\pgfsetstrokecolor{textcolor}%
\pgfsetfillcolor{textcolor}%
\pgftext[x=0.386419in, y=2.728030in, left, base]{\color{textcolor}\rmfamily\fontsize{10.000000}{12.000000}\selectfont \(\displaystyle {0.075}\)}%
\end{pgfscope}%
\begin{pgfscope}%
\pgfpathrectangle{\pgfqpoint{0.800000in}{0.528000in}}{\pgfqpoint{4.960000in}{3.696000in}}%
\pgfusepath{clip}%
\pgfsetrectcap%
\pgfsetroundjoin%
\pgfsetlinewidth{0.803000pt}%
\definecolor{currentstroke}{rgb}{0.690196,0.690196,0.690196}%
\pgfsetstrokecolor{currentstroke}%
\pgfsetdash{}{0pt}%
\pgfpathmoveto{\pgfqpoint{0.800000in}{3.197127in}}%
\pgfpathlineto{\pgfqpoint{5.760000in}{3.197127in}}%
\pgfusepath{stroke}%
\end{pgfscope}%
\begin{pgfscope}%
\pgfsetbuttcap%
\pgfsetroundjoin%
\definecolor{currentfill}{rgb}{0.000000,0.000000,0.000000}%
\pgfsetfillcolor{currentfill}%
\pgfsetlinewidth{0.803000pt}%
\definecolor{currentstroke}{rgb}{0.000000,0.000000,0.000000}%
\pgfsetstrokecolor{currentstroke}%
\pgfsetdash{}{0pt}%
\pgfsys@defobject{currentmarker}{\pgfqpoint{-0.048611in}{0.000000in}}{\pgfqpoint{-0.000000in}{0.000000in}}{%
\pgfpathmoveto{\pgfqpoint{-0.000000in}{0.000000in}}%
\pgfpathlineto{\pgfqpoint{-0.048611in}{0.000000in}}%
\pgfusepath{stroke,fill}%
}%
\begin{pgfscope}%
\pgfsys@transformshift{0.800000in}{3.197127in}%
\pgfsys@useobject{currentmarker}{}%
\end{pgfscope}%
\end{pgfscope}%
\begin{pgfscope}%
\definecolor{textcolor}{rgb}{0.000000,0.000000,0.000000}%
\pgfsetstrokecolor{textcolor}%
\pgfsetfillcolor{textcolor}%
\pgftext[x=0.386419in, y=3.148914in, left, base]{\color{textcolor}\rmfamily\fontsize{10.000000}{12.000000}\selectfont \(\displaystyle {0.100}\)}%
\end{pgfscope}%
\begin{pgfscope}%
\pgfpathrectangle{\pgfqpoint{0.800000in}{0.528000in}}{\pgfqpoint{4.960000in}{3.696000in}}%
\pgfusepath{clip}%
\pgfsetrectcap%
\pgfsetroundjoin%
\pgfsetlinewidth{0.803000pt}%
\definecolor{currentstroke}{rgb}{0.690196,0.690196,0.690196}%
\pgfsetstrokecolor{currentstroke}%
\pgfsetdash{}{0pt}%
\pgfpathmoveto{\pgfqpoint{0.800000in}{3.618010in}}%
\pgfpathlineto{\pgfqpoint{5.760000in}{3.618010in}}%
\pgfusepath{stroke}%
\end{pgfscope}%
\begin{pgfscope}%
\pgfsetbuttcap%
\pgfsetroundjoin%
\definecolor{currentfill}{rgb}{0.000000,0.000000,0.000000}%
\pgfsetfillcolor{currentfill}%
\pgfsetlinewidth{0.803000pt}%
\definecolor{currentstroke}{rgb}{0.000000,0.000000,0.000000}%
\pgfsetstrokecolor{currentstroke}%
\pgfsetdash{}{0pt}%
\pgfsys@defobject{currentmarker}{\pgfqpoint{-0.048611in}{0.000000in}}{\pgfqpoint{-0.000000in}{0.000000in}}{%
\pgfpathmoveto{\pgfqpoint{-0.000000in}{0.000000in}}%
\pgfpathlineto{\pgfqpoint{-0.048611in}{0.000000in}}%
\pgfusepath{stroke,fill}%
}%
\begin{pgfscope}%
\pgfsys@transformshift{0.800000in}{3.618010in}%
\pgfsys@useobject{currentmarker}{}%
\end{pgfscope}%
\end{pgfscope}%
\begin{pgfscope}%
\definecolor{textcolor}{rgb}{0.000000,0.000000,0.000000}%
\pgfsetstrokecolor{textcolor}%
\pgfsetfillcolor{textcolor}%
\pgftext[x=0.386419in, y=3.569797in, left, base]{\color{textcolor}\rmfamily\fontsize{10.000000}{12.000000}\selectfont \(\displaystyle {0.125}\)}%
\end{pgfscope}%
\begin{pgfscope}%
\pgfpathrectangle{\pgfqpoint{0.800000in}{0.528000in}}{\pgfqpoint{4.960000in}{3.696000in}}%
\pgfusepath{clip}%
\pgfsetrectcap%
\pgfsetroundjoin%
\pgfsetlinewidth{0.803000pt}%
\definecolor{currentstroke}{rgb}{0.690196,0.690196,0.690196}%
\pgfsetstrokecolor{currentstroke}%
\pgfsetdash{}{0pt}%
\pgfpathmoveto{\pgfqpoint{0.800000in}{4.038893in}}%
\pgfpathlineto{\pgfqpoint{5.760000in}{4.038893in}}%
\pgfusepath{stroke}%
\end{pgfscope}%
\begin{pgfscope}%
\pgfsetbuttcap%
\pgfsetroundjoin%
\definecolor{currentfill}{rgb}{0.000000,0.000000,0.000000}%
\pgfsetfillcolor{currentfill}%
\pgfsetlinewidth{0.803000pt}%
\definecolor{currentstroke}{rgb}{0.000000,0.000000,0.000000}%
\pgfsetstrokecolor{currentstroke}%
\pgfsetdash{}{0pt}%
\pgfsys@defobject{currentmarker}{\pgfqpoint{-0.048611in}{0.000000in}}{\pgfqpoint{-0.000000in}{0.000000in}}{%
\pgfpathmoveto{\pgfqpoint{-0.000000in}{0.000000in}}%
\pgfpathlineto{\pgfqpoint{-0.048611in}{0.000000in}}%
\pgfusepath{stroke,fill}%
}%
\begin{pgfscope}%
\pgfsys@transformshift{0.800000in}{4.038893in}%
\pgfsys@useobject{currentmarker}{}%
\end{pgfscope}%
\end{pgfscope}%
\begin{pgfscope}%
\definecolor{textcolor}{rgb}{0.000000,0.000000,0.000000}%
\pgfsetstrokecolor{textcolor}%
\pgfsetfillcolor{textcolor}%
\pgftext[x=0.386419in, y=3.990680in, left, base]{\color{textcolor}\rmfamily\fontsize{10.000000}{12.000000}\selectfont \(\displaystyle {0.150}\)}%
\end{pgfscope}%
\begin{pgfscope}%
\definecolor{textcolor}{rgb}{0.000000,0.000000,0.000000}%
\pgfsetstrokecolor{textcolor}%
\pgfsetfillcolor{textcolor}%
\pgftext[x=0.222838in,y=2.376000in,,bottom,rotate=90.000000]{\color{textcolor}\rmfamily\fontsize{10.000000}{12.000000}\selectfont \(\displaystyle \vartheta,\ рад\)}%
\end{pgfscope}%
\begin{pgfscope}%
\pgfpathrectangle{\pgfqpoint{0.800000in}{0.528000in}}{\pgfqpoint{4.960000in}{3.696000in}}%
\pgfusepath{clip}%
\pgfsetrectcap%
\pgfsetroundjoin%
\pgfsetlinewidth{1.505625pt}%
\definecolor{currentstroke}{rgb}{0.121569,0.466667,0.705882}%
\pgfsetstrokecolor{currentstroke}%
\pgfsetdash{}{0pt}%
\pgfpathmoveto{\pgfqpoint{1.025455in}{1.513595in}}%
\pgfpathlineto{\pgfqpoint{1.184926in}{1.514681in}}%
\pgfpathlineto{\pgfqpoint{1.189736in}{1.517631in}}%
\pgfpathlineto{\pgfqpoint{1.194245in}{1.523125in}}%
\pgfpathlineto{\pgfqpoint{1.198904in}{1.532506in}}%
\pgfpathlineto{\pgfqpoint{1.204015in}{1.548218in}}%
\pgfpathlineto{\pgfqpoint{1.209576in}{1.573132in}}%
\pgfpathlineto{\pgfqpoint{1.215738in}{1.611748in}}%
\pgfpathlineto{\pgfqpoint{1.223855in}{1.678194in}}%
\pgfpathlineto{\pgfqpoint{1.235728in}{1.796122in}}%
\pgfpathlineto{\pgfqpoint{1.264286in}{2.085655in}}%
\pgfpathlineto{\pgfqpoint{1.278715in}{2.203962in}}%
\pgfpathlineto{\pgfqpoint{1.294347in}{2.313835in}}%
\pgfpathlineto{\pgfqpoint{1.310880in}{2.414591in}}%
\pgfpathlineto{\pgfqpoint{1.327714in}{2.503732in}}%
\pgfpathlineto{\pgfqpoint{1.344548in}{2.581200in}}%
\pgfpathlineto{\pgfqpoint{1.361682in}{2.649780in}}%
\pgfpathlineto{\pgfqpoint{1.378967in}{2.709992in}}%
\pgfpathlineto{\pgfqpoint{1.396553in}{2.763353in}}%
\pgfpathlineto{\pgfqpoint{1.414439in}{2.810643in}}%
\pgfpathlineto{\pgfqpoint{1.432625in}{2.852567in}}%
\pgfpathlineto{\pgfqpoint{1.451263in}{2.890042in}}%
\pgfpathlineto{\pgfqpoint{1.470502in}{2.923777in}}%
\pgfpathlineto{\pgfqpoint{1.490342in}{2.954108in}}%
\pgfpathlineto{\pgfqpoint{1.510933in}{2.981548in}}%
\pgfpathlineto{\pgfqpoint{1.532427in}{3.006504in}}%
\pgfpathlineto{\pgfqpoint{1.554972in}{3.029299in}}%
\pgfpathlineto{\pgfqpoint{1.578870in}{3.050317in}}%
\pgfpathlineto{\pgfqpoint{1.604422in}{3.069833in}}%
\pgfpathlineto{\pgfqpoint{1.632078in}{3.088139in}}%
\pgfpathlineto{\pgfqpoint{1.662138in}{3.105338in}}%
\pgfpathlineto{\pgfqpoint{1.695055in}{3.121580in}}%
\pgfpathlineto{\pgfqpoint{1.731428in}{3.137020in}}%
\pgfpathlineto{\pgfqpoint{1.772010in}{3.151794in}}%
\pgfpathlineto{\pgfqpoint{1.817552in}{3.165961in}}%
\pgfpathlineto{\pgfqpoint{1.868805in}{3.179527in}}%
\pgfpathlineto{\pgfqpoint{1.926521in}{3.192460in}}%
\pgfpathlineto{\pgfqpoint{1.991602in}{3.204721in}}%
\pgfpathlineto{\pgfqpoint{2.029479in}{3.210132in}}%
\pgfpathlineto{\pgfqpoint{2.041804in}{3.209481in}}%
\pgfpathlineto{\pgfqpoint{2.052926in}{3.206646in}}%
\pgfpathlineto{\pgfqpoint{2.064199in}{3.201420in}}%
\pgfpathlineto{\pgfqpoint{2.076223in}{3.193322in}}%
\pgfpathlineto{\pgfqpoint{2.089450in}{3.181627in}}%
\pgfpathlineto{\pgfqpoint{2.104029in}{3.165656in}}%
\pgfpathlineto{\pgfqpoint{2.120262in}{3.144452in}}%
\pgfpathlineto{\pgfqpoint{2.138448in}{3.116889in}}%
\pgfpathlineto{\pgfqpoint{2.158890in}{3.081709in}}%
\pgfpathlineto{\pgfqpoint{2.181886in}{3.037562in}}%
\pgfpathlineto{\pgfqpoint{2.208039in}{2.982391in}}%
\pgfpathlineto{\pgfqpoint{2.238099in}{2.913601in}}%
\pgfpathlineto{\pgfqpoint{2.273120in}{2.827693in}}%
\pgfpathlineto{\pgfqpoint{2.315055in}{2.718670in}}%
\pgfpathlineto{\pgfqpoint{2.368412in}{2.573388in}}%
\pgfpathlineto{\pgfqpoint{2.455888in}{2.327711in}}%
\pgfpathlineto{\pgfqpoint{2.557644in}{2.043769in}}%
\pgfpathlineto{\pgfqpoint{2.618366in}{1.880947in}}%
\pgfpathlineto{\pgfqpoint{2.670070in}{1.748513in}}%
\pgfpathlineto{\pgfqpoint{2.716514in}{1.635520in}}%
\pgfpathlineto{\pgfqpoint{2.759501in}{1.536669in}}%
\pgfpathlineto{\pgfqpoint{2.799782in}{1.449509in}}%
\pgfpathlineto{\pgfqpoint{2.837959in}{1.372116in}}%
\pgfpathlineto{\pgfqpoint{2.874332in}{1.303328in}}%
\pgfpathlineto{\pgfqpoint{2.909052in}{1.242331in}}%
\pgfpathlineto{\pgfqpoint{2.942570in}{1.187876in}}%
\pgfpathlineto{\pgfqpoint{2.974885in}{1.139562in}}%
\pgfpathlineto{\pgfqpoint{3.006148in}{1.096774in}}%
\pgfpathlineto{\pgfqpoint{3.036359in}{1.059136in}}%
\pgfpathlineto{\pgfqpoint{3.065818in}{1.025943in}}%
\pgfpathlineto{\pgfqpoint{3.094376in}{0.997060in}}%
\pgfpathlineto{\pgfqpoint{3.122332in}{0.971900in}}%
\pgfpathlineto{\pgfqpoint{3.149687in}{0.950230in}}%
\pgfpathlineto{\pgfqpoint{3.176592in}{0.931721in}}%
\pgfpathlineto{\pgfqpoint{3.203045in}{0.916187in}}%
\pgfpathlineto{\pgfqpoint{3.229198in}{0.903374in}}%
\pgfpathlineto{\pgfqpoint{3.255200in}{0.893088in}}%
\pgfpathlineto{\pgfqpoint{3.281202in}{0.885184in}}%
\pgfpathlineto{\pgfqpoint{3.307205in}{0.879597in}}%
\pgfpathlineto{\pgfqpoint{3.333358in}{0.876241in}}%
\pgfpathlineto{\pgfqpoint{3.359961in}{0.875076in}}%
\pgfpathlineto{\pgfqpoint{3.387016in}{0.876129in}}%
\pgfpathlineto{\pgfqpoint{3.414672in}{0.879440in}}%
\pgfpathlineto{\pgfqpoint{3.443229in}{0.885120in}}%
\pgfpathlineto{\pgfqpoint{3.472688in}{0.893260in}}%
\pgfpathlineto{\pgfqpoint{3.503501in}{0.904108in}}%
\pgfpathlineto{\pgfqpoint{3.535816in}{0.917878in}}%
\pgfpathlineto{\pgfqpoint{3.569935in}{0.934875in}}%
\pgfpathlineto{\pgfqpoint{3.606158in}{0.955440in}}%
\pgfpathlineto{\pgfqpoint{3.645086in}{0.980137in}}%
\pgfpathlineto{\pgfqpoint{3.687472in}{1.009706in}}%
\pgfpathlineto{\pgfqpoint{3.734516in}{1.045298in}}%
\pgfpathlineto{\pgfqpoint{3.788475in}{1.089000in}}%
\pgfpathlineto{\pgfqpoint{3.854308in}{1.145326in}}%
\pgfpathlineto{\pgfqpoint{3.955612in}{1.235337in}}%
\pgfpathlineto{\pgfqpoint{4.081265in}{1.346334in}}%
\pgfpathlineto{\pgfqpoint{4.152359in}{1.406106in}}%
\pgfpathlineto{\pgfqpoint{4.213081in}{1.454311in}}%
\pgfpathlineto{\pgfqpoint{4.268092in}{1.495195in}}%
\pgfpathlineto{\pgfqpoint{4.319496in}{1.530672in}}%
\pgfpathlineto{\pgfqpoint{4.368495in}{1.561815in}}%
\pgfpathlineto{\pgfqpoint{4.415539in}{1.589109in}}%
\pgfpathlineto{\pgfqpoint{4.461232in}{1.613070in}}%
\pgfpathlineto{\pgfqpoint{4.505872in}{1.633993in}}%
\pgfpathlineto{\pgfqpoint{4.549910in}{1.652193in}}%
\pgfpathlineto{\pgfqpoint{4.593498in}{1.667812in}}%
\pgfpathlineto{\pgfqpoint{4.636936in}{1.681021in}}%
\pgfpathlineto{\pgfqpoint{4.680373in}{1.691912in}}%
\pgfpathlineto{\pgfqpoint{4.724112in}{1.700588in}}%
\pgfpathlineto{\pgfqpoint{4.768451in}{1.707112in}}%
\pgfpathlineto{\pgfqpoint{4.813542in}{1.711491in}}%
\pgfpathlineto{\pgfqpoint{4.859685in}{1.713732in}}%
\pgfpathlineto{\pgfqpoint{4.907331in}{1.713805in}}%
\pgfpathlineto{\pgfqpoint{4.956781in}{1.711644in}}%
\pgfpathlineto{\pgfqpoint{5.008635in}{1.707133in}}%
\pgfpathlineto{\pgfqpoint{5.063646in}{1.700089in}}%
\pgfpathlineto{\pgfqpoint{5.122865in}{1.690230in}}%
\pgfpathlineto{\pgfqpoint{5.187947in}{1.677095in}}%
\pgfpathlineto{\pgfqpoint{5.262046in}{1.659807in}}%
\pgfpathlineto{\pgfqpoint{5.352528in}{1.636304in}}%
\pgfpathlineto{\pgfqpoint{5.499074in}{1.595561in}}%
\pgfpathlineto{\pgfqpoint{5.534545in}{1.585634in}}%
\pgfpathlineto{\pgfqpoint{5.534545in}{1.585634in}}%
\pgfusepath{stroke}%
\end{pgfscope}%
\begin{pgfscope}%
\pgfpathrectangle{\pgfqpoint{0.800000in}{0.528000in}}{\pgfqpoint{4.960000in}{3.696000in}}%
\pgfusepath{clip}%
\pgfsetbuttcap%
\pgfsetroundjoin%
\pgfsetlinewidth{1.505625pt}%
\definecolor{currentstroke}{rgb}{1.000000,0.498039,0.054902}%
\pgfsetstrokecolor{currentstroke}%
\pgfsetdash{{5.550000pt}{2.400000pt}}{0.000000pt}%
\pgfpathmoveto{\pgfqpoint{1.025455in}{1.513595in}}%
\pgfpathlineto{\pgfqpoint{1.180659in}{1.513945in}}%
\pgfpathlineto{\pgfqpoint{1.183768in}{1.515774in}}%
\pgfpathlineto{\pgfqpoint{1.187896in}{1.523148in}}%
\pgfpathlineto{\pgfqpoint{1.193136in}{1.545224in}}%
\pgfpathlineto{\pgfqpoint{1.199666in}{1.597458in}}%
\pgfpathlineto{\pgfqpoint{1.206713in}{1.683700in}}%
\pgfpathlineto{\pgfqpoint{1.222112in}{1.939405in}}%
\pgfpathlineto{\pgfqpoint{1.243194in}{2.310483in}}%
\pgfpathlineto{\pgfqpoint{1.255273in}{2.501964in}}%
\pgfpathlineto{\pgfqpoint{1.268468in}{2.691056in}}%
\pgfpathlineto{\pgfqpoint{1.286415in}{2.917907in}}%
\pgfpathlineto{\pgfqpoint{1.305177in}{3.121500in}}%
\pgfpathlineto{\pgfqpoint{1.327685in}{3.326311in}}%
\pgfpathlineto{\pgfqpoint{1.359731in}{3.556496in}}%
\pgfpathlineto{\pgfqpoint{1.394830in}{3.742938in}}%
\pgfpathlineto{\pgfqpoint{1.426754in}{3.865760in}}%
\pgfpathlineto{\pgfqpoint{1.456575in}{3.948191in}}%
\pgfpathlineto{\pgfqpoint{1.483903in}{4.000997in}}%
\pgfpathlineto{\pgfqpoint{1.510987in}{4.035107in}}%
\pgfpathlineto{\pgfqpoint{1.540171in}{4.054300in}}%
\pgfpathlineto{\pgfqpoint{1.577269in}{4.056000in}}%
\pgfpathlineto{\pgfqpoint{1.604289in}{4.043337in}}%
\pgfpathlineto{\pgfqpoint{1.631309in}{4.020275in}}%
\pgfpathlineto{\pgfqpoint{1.662098in}{3.982631in}}%
\pgfpathlineto{\pgfqpoint{1.700127in}{3.921292in}}%
\pgfpathlineto{\pgfqpoint{1.726008in}{3.871099in}}%
\pgfpathlineto{\pgfqpoint{1.751889in}{3.814829in}}%
\pgfpathlineto{\pgfqpoint{1.782812in}{3.740355in}}%
\pgfpathlineto{\pgfqpoint{1.823539in}{3.631581in}}%
\pgfpathlineto{\pgfqpoint{1.847922in}{3.561430in}}%
\pgfpathlineto{\pgfqpoint{1.872305in}{3.487872in}}%
\pgfpathlineto{\pgfqpoint{1.903602in}{3.389113in}}%
\pgfpathlineto{\pgfqpoint{1.968960in}{3.170130in}}%
\pgfpathlineto{\pgfqpoint{2.026843in}{2.965976in}}%
\pgfpathlineto{\pgfqpoint{2.156214in}{2.495813in}}%
\pgfpathlineto{\pgfqpoint{2.233803in}{2.217413in}}%
\pgfpathlineto{\pgfqpoint{2.297432in}{1.997619in}}%
\pgfpathlineto{\pgfqpoint{2.321926in}{1.915825in}}%
\pgfpathlineto{\pgfqpoint{2.388301in}{1.703924in}}%
\pgfpathlineto{\pgfqpoint{2.415149in}{1.622754in}}%
\pgfpathlineto{\pgfqpoint{2.449288in}{1.523705in}}%
\pgfpathlineto{\pgfqpoint{2.481377in}{1.435122in}}%
\pgfpathlineto{\pgfqpoint{2.513466in}{1.351147in}}%
\pgfpathlineto{\pgfqpoint{2.543072in}{1.277915in}}%
\pgfpathlineto{\pgfqpoint{2.570427in}{1.213957in}}%
\pgfpathlineto{\pgfqpoint{2.602301in}{1.144069in}}%
\pgfpathlineto{\pgfqpoint{2.636649in}{1.074430in}}%
\pgfpathlineto{\pgfqpoint{2.667514in}{1.016926in}}%
\pgfpathlineto{\pgfqpoint{2.692477in}{0.974008in}}%
\pgfpathlineto{\pgfqpoint{2.718723in}{0.932329in}}%
\pgfpathlineto{\pgfqpoint{2.752543in}{0.883805in}}%
\pgfpathlineto{\pgfqpoint{2.788918in}{0.838133in}}%
\pgfpathlineto{\pgfqpoint{2.818764in}{0.805623in}}%
\pgfpathlineto{\pgfqpoint{2.844215in}{0.781344in}}%
\pgfpathlineto{\pgfqpoint{2.873139in}{0.757619in}}%
\pgfpathlineto{\pgfqpoint{2.905007in}{0.736113in}}%
\pgfpathlineto{\pgfqpoint{2.936518in}{0.719504in}}%
\pgfpathlineto{\pgfqpoint{2.967332in}{0.707615in}}%
\pgfpathlineto{\pgfqpoint{3.000764in}{0.699404in}}%
\pgfpathlineto{\pgfqpoint{3.029912in}{0.696114in}}%
\pgfpathlineto{\pgfqpoint{3.054905in}{0.696000in}}%
\pgfpathlineto{\pgfqpoint{3.081170in}{0.698473in}}%
\pgfpathlineto{\pgfqpoint{3.118744in}{0.706452in}}%
\pgfpathlineto{\pgfqpoint{3.145961in}{0.715293in}}%
\pgfpathlineto{\pgfqpoint{3.173177in}{0.726585in}}%
\pgfpathlineto{\pgfqpoint{3.206661in}{0.743609in}}%
\pgfpathlineto{\pgfqpoint{3.238980in}{0.763100in}}%
\pgfpathlineto{\pgfqpoint{3.271299in}{0.785359in}}%
\pgfpathlineto{\pgfqpoint{3.300853in}{0.807922in}}%
\pgfpathlineto{\pgfqpoint{3.327639in}{0.830012in}}%
\pgfpathlineto{\pgfqpoint{3.358530in}{0.857268in}}%
\pgfpathlineto{\pgfqpoint{3.418907in}{0.915243in}}%
\pgfpathlineto{\pgfqpoint{3.480694in}{0.979714in}}%
\pgfpathlineto{\pgfqpoint{3.541233in}{1.046399in}}%
\pgfpathlineto{\pgfqpoint{3.667332in}{1.190172in}}%
\pgfpathlineto{\pgfqpoint{3.752485in}{1.286413in}}%
\pgfpathlineto{\pgfqpoint{3.810570in}{1.349787in}}%
\pgfpathlineto{\pgfqpoint{3.874337in}{1.416088in}}%
\pgfpathlineto{\pgfqpoint{3.934114in}{1.474327in}}%
\pgfpathlineto{\pgfqpoint{3.999075in}{1.532600in}}%
\pgfpathlineto{\pgfqpoint{4.058651in}{1.580979in}}%
\pgfpathlineto{\pgfqpoint{4.111614in}{1.619497in}}%
\pgfpathlineto{\pgfqpoint{4.174633in}{1.659646in}}%
\pgfpathlineto{\pgfqpoint{4.201127in}{1.674652in}}%
\pgfpathlineto{\pgfqpoint{4.239708in}{1.694469in}}%
\pgfpathlineto{\pgfqpoint{4.294227in}{1.718436in}}%
\pgfpathlineto{\pgfqpoint{4.326019in}{1.730218in}}%
\pgfpathlineto{\pgfqpoint{4.367480in}{1.743248in}}%
\pgfpathlineto{\pgfqpoint{4.412788in}{1.754352in}}%
\pgfpathlineto{\pgfqpoint{4.443848in}{1.760246in}}%
\pgfpathlineto{\pgfqpoint{4.507028in}{1.767995in}}%
\pgfpathlineto{\pgfqpoint{4.531704in}{1.769523in}}%
\pgfpathlineto{\pgfqpoint{4.570114in}{1.770346in}}%
\pgfpathlineto{\pgfqpoint{4.625053in}{1.768373in}}%
\pgfpathlineto{\pgfqpoint{4.690706in}{1.761599in}}%
\pgfpathlineto{\pgfqpoint{4.751675in}{1.751508in}}%
\pgfpathlineto{\pgfqpoint{4.813365in}{1.738047in}}%
\pgfpathlineto{\pgfqpoint{4.876035in}{1.721544in}}%
\pgfpathlineto{\pgfqpoint{4.927803in}{1.706235in}}%
\pgfpathlineto{\pgfqpoint{5.019799in}{1.676160in}}%
\pgfpathlineto{\pgfqpoint{5.143518in}{1.632413in}}%
\pgfpathlineto{\pgfqpoint{5.290238in}{1.580241in}}%
\pgfpathlineto{\pgfqpoint{5.378073in}{1.550777in}}%
\pgfpathlineto{\pgfqpoint{5.471254in}{1.522166in}}%
\pgfpathlineto{\pgfqpoint{5.534545in}{1.504652in}}%
\pgfpathlineto{\pgfqpoint{5.534545in}{1.504652in}}%
\pgfusepath{stroke}%
\end{pgfscope}%
\begin{pgfscope}%
\pgfsetrectcap%
\pgfsetmiterjoin%
\pgfsetlinewidth{0.803000pt}%
\definecolor{currentstroke}{rgb}{0.000000,0.000000,0.000000}%
\pgfsetstrokecolor{currentstroke}%
\pgfsetdash{}{0pt}%
\pgfpathmoveto{\pgfqpoint{0.800000in}{0.528000in}}%
\pgfpathlineto{\pgfqpoint{0.800000in}{4.224000in}}%
\pgfusepath{stroke}%
\end{pgfscope}%
\begin{pgfscope}%
\pgfsetrectcap%
\pgfsetmiterjoin%
\pgfsetlinewidth{0.803000pt}%
\definecolor{currentstroke}{rgb}{0.000000,0.000000,0.000000}%
\pgfsetstrokecolor{currentstroke}%
\pgfsetdash{}{0pt}%
\pgfpathmoveto{\pgfqpoint{5.760000in}{0.528000in}}%
\pgfpathlineto{\pgfqpoint{5.760000in}{4.224000in}}%
\pgfusepath{stroke}%
\end{pgfscope}%
\begin{pgfscope}%
\pgfsetrectcap%
\pgfsetmiterjoin%
\pgfsetlinewidth{0.803000pt}%
\definecolor{currentstroke}{rgb}{0.000000,0.000000,0.000000}%
\pgfsetstrokecolor{currentstroke}%
\pgfsetdash{}{0pt}%
\pgfpathmoveto{\pgfqpoint{0.800000in}{0.528000in}}%
\pgfpathlineto{\pgfqpoint{5.760000in}{0.528000in}}%
\pgfusepath{stroke}%
\end{pgfscope}%
\begin{pgfscope}%
\pgfsetrectcap%
\pgfsetmiterjoin%
\pgfsetlinewidth{0.803000pt}%
\definecolor{currentstroke}{rgb}{0.000000,0.000000,0.000000}%
\pgfsetstrokecolor{currentstroke}%
\pgfsetdash{}{0pt}%
\pgfpathmoveto{\pgfqpoint{0.800000in}{4.224000in}}%
\pgfpathlineto{\pgfqpoint{5.760000in}{4.224000in}}%
\pgfusepath{stroke}%
\end{pgfscope}%
\begin{pgfscope}%
\pgfsetbuttcap%
\pgfsetmiterjoin%
\definecolor{currentfill}{rgb}{1.000000,1.000000,1.000000}%
\pgfsetfillcolor{currentfill}%
\pgfsetfillopacity{0.800000}%
\pgfsetlinewidth{1.003750pt}%
\definecolor{currentstroke}{rgb}{0.800000,0.800000,0.800000}%
\pgfsetstrokecolor{currentstroke}%
\pgfsetstrokeopacity{0.800000}%
\pgfsetdash{}{0pt}%
\pgfpathmoveto{\pgfqpoint{3.299872in}{3.725556in}}%
\pgfpathlineto{\pgfqpoint{5.662778in}{3.725556in}}%
\pgfpathquadraticcurveto{\pgfqpoint{5.690556in}{3.725556in}}{\pgfqpoint{5.690556in}{3.753334in}}%
\pgfpathlineto{\pgfqpoint{5.690556in}{4.126778in}}%
\pgfpathquadraticcurveto{\pgfqpoint{5.690556in}{4.154556in}}{\pgfqpoint{5.662778in}{4.154556in}}%
\pgfpathlineto{\pgfqpoint{3.299872in}{4.154556in}}%
\pgfpathquadraticcurveto{\pgfqpoint{3.272094in}{4.154556in}}{\pgfqpoint{3.272094in}{4.126778in}}%
\pgfpathlineto{\pgfqpoint{3.272094in}{3.753334in}}%
\pgfpathquadraticcurveto{\pgfqpoint{3.272094in}{3.725556in}}{\pgfqpoint{3.299872in}{3.725556in}}%
\pgfpathclose%
\pgfusepath{stroke,fill}%
\end{pgfscope}%
\begin{pgfscope}%
\pgfsetrectcap%
\pgfsetroundjoin%
\pgfsetlinewidth{1.505625pt}%
\definecolor{currentstroke}{rgb}{0.121569,0.466667,0.705882}%
\pgfsetstrokecolor{currentstroke}%
\pgfsetdash{}{0pt}%
\pgfpathmoveto{\pgfqpoint{3.327650in}{4.050389in}}%
\pgfpathlineto{\pgfqpoint{3.605428in}{4.050389in}}%
\pgfusepath{stroke}%
\end{pgfscope}%
\begin{pgfscope}%
\definecolor{textcolor}{rgb}{0.000000,0.000000,0.000000}%
\pgfsetstrokecolor{textcolor}%
\pgfsetfillcolor{textcolor}%
\pgftext[x=3.716539in,y=4.001778in,left,base]{\color{textcolor}\rmfamily\fontsize{10.000000}{12.000000}\selectfont Нелинейная модель, \(\displaystyle M=0.61\)}%
\end{pgfscope}%
\begin{pgfscope}%
\pgfsetbuttcap%
\pgfsetroundjoin%
\pgfsetlinewidth{1.505625pt}%
\definecolor{currentstroke}{rgb}{1.000000,0.498039,0.054902}%
\pgfsetstrokecolor{currentstroke}%
\pgfsetdash{{5.550000pt}{2.400000pt}}{0.000000pt}%
\pgfpathmoveto{\pgfqpoint{3.327650in}{3.856723in}}%
\pgfpathlineto{\pgfqpoint{3.605428in}{3.856723in}}%
\pgfusepath{stroke}%
\end{pgfscope}%
\begin{pgfscope}%
\definecolor{textcolor}{rgb}{0.000000,0.000000,0.000000}%
\pgfsetstrokecolor{textcolor}%
\pgfsetfillcolor{textcolor}%
\pgftext[x=3.716539in,y=3.808112in,left,base]{\color{textcolor}\rmfamily\fontsize{10.000000}{12.000000}\selectfont Линейная модель, \(\displaystyle M=0.61\)}%
\end{pgfscope}%
\end{pgfpicture}%
\makeatother%
\endgroup%
}
    \caption{Изменение угла тангажа для линейной и нелинейной модели}
    \label{fig:model_theta}
\end{minipage}
\end{figure}

\begin{table}[htpb]
    \centering
    \caption{Сравнение параметров переходного процесса $\Delta H(t)$}
    \label{tab:stat_lin_nonlin}
    \begin{tabular}{|c|c|c|}
        \hline
        {} &  Линейная модель &  Нелинейная модель \\
        \hline
        $t_{рег},\ с$ &            26.67 &              26.69 \\
        \hline
        $\sigma,\ \%$ &            28.00 &              24.66 \\
        \hline
    \end{tabular}
\end{table}

\subsection{Вывод}
При моделировании различных скоростей отклонения руля высоты, переходный
процесс практически не изменился (см. рисунок \ref{fig:model_DD_Delta_H}), время
регулирования привода с наибольшей максимальной скоростью отклонения было
меньше на $\approx 0.01\ с$, что незначительно. Максимальное отклонение руля высоты в случае с
$\dot{\delta}_{{в}_{max}} = 15\, \frac{\text{град.}}{\text{сек.}}$ было меньше
на $40 \%$ (см. рисунок \ref{fig:model_DD_delta_elevator}). Максимальная
угловая скорость тангажа равна $\approx 0.115\ \text{рад/с}$ у модели с
$\dot{\delta}_{{в}_{max}} = 15\, \frac{\text{град.}}{\text{сек.}}$ (см. рисунок
\ref{fig:model_DD_omega_z}). Характер изменения угла тангажа имеет различие до
5 секунды моделирования (см. рисунок \ref{fig:model_DD_theta}).

Разница во времени регулирования между линейной и нелинейной моделью в $\approx
0.03\ с$ (см. таблицу \ref{tab:stat_lin_nonlin}), но разница во времени
срабатывания порядка $\approx 1.8\, с$. У линейной модели максимальное
отклонение руля высоты имеет недопустимое значение $max(\delta_{в}) >
-21^\circ$ (см. рисунок \ref{fig:delta_elevator}). В следствии этого
максимальная угловая скорость тангажа $\approx 0.16 \ \text{рад/с}$ (см.
рисунок \ref{fig:model_omega_z}). Изменение угла тангажа у нелинейной модели
ограничено 0.11 рад, что примерно равно $6.5^\circ$ (см. рисунок \ref{fig:model_theta}).

