\chapter{Расчет лётно – технических характеристик самолета}

Определим следующие характеристики самолета:
\begin{enumerate}
    \item Зависимости от числа M (скорости) и H (высоты) полета результаты
        сведем в таблицы 2.1-2.7:
    \begin{itemize}
    \item  располагаемой и потребной для горизонтального установившегося полета
    тяги силовой установки, 
    \item энергетической скороподъемности,
    \item часового расхода топлива,
    \item километрового расхода топлива.
    \end{itemize}
    \item Зависимости от высоты:
        \begin{itemize}
            \item максимальной энергетической скороподъемности,
            \item минимального часового расхода топлива,
            \item минимального километрового расхода топлива,
            \item минимального и максимального числа M (скорости) полета (с учетом
                ограничений по безопасности полета),
            \item числа $M$ (скорости) полета, соответствующего минимальной
                потребной тяги,
            \item числа $M$ (скорости) полета, соответствующего максимальной
                энергетической скороподъемности,
            \item скорости полета, соответствующей минимальному часовому расходу топлива,
            \item скорости полета, соответствующему минимальному километровому
                расходу топлива
        \end{itemize}
    \item Статический и практический потолки самолета.
\end{enumerate}

Соотношения для расчета:
Узловые точки по числу Маха:
\[
    M = [0.2 0.3 0.4 0.5 0.6 0.7 0.8 0.9 0.95]
\]
\begin{equation}
    V = M a_H,
    \label{eq:V_speed}
\end{equation}
где $a_H$ --- скорость звука на высоте $H$.
\begin{equation}
    q = \frac{\rho_H V^2}{2},
    \label{eq:q_value}
\end{equation}
где $\rho_H$ --- плотность воздуха на высоте $H$.
\begin{equation}
    C_{y_n} = \frac{\bar{m} p_s 10}{q},
    \label{eq:c_y_n}
\end{equation}
где $\bar{m} = 0.95$ --- относительная масса самолета, $p_s$ --- удельная 
нагрузка на крыло.
\begin{equation}
    C_{x_n}(C_y, M) = C_{x_m}(M) + A(M) \left[ C_{y_n} - C_{y_m}(M)\right]^2
    \label{eq:c_x_n}
\end{equation}
где $C_{y_m}$ --- коэффициент подъемной силы при $C_x = C_{x_m}$, $C_{x_m}$ ---
минимальный коэффициент лобового сопротивления, $A$ --- коэффициент отвала
поляры.
\begin{equation}
    K_n = \frac{C_{y_n}}{C_{x_n}}
    \label{eq:K_n}
\end{equation}
\begin{equation}
    P_n = \frac{\bar{m} m_0 g}{K_n}
    \label{eq:P_potr}
\end{equation}
\begin{equation}
    P_p(M,H) = \bar{P}_0 m_0 g \tilde{P}(H,M)
    \label{eq:P_rasp}
\end{equation}
\begin{equation}
    n_x = \Delta \bar{P} = \frac{(P_p - P_n)}{\bar{m} m_0 g}
    \label{eq:n_x}
\end{equation}
\begin{equation}
    V_y^* = \Delta \bar{P} V
    \label{eq:Vy}
\end{equation}
\begin{equation}
    \bar{R} = \frac{P_n}{P_p}
    \label{eq:R_dross}
\end{equation}
\begin{equation}
    q_{ч} = Ce(M,H,\bar{R})P_n = Ce_0 \tilde{Ce}(H,M) \hat{Ce}_{др}(R) P_n
    \label{eq:q_chas}
\end{equation}
\begin{equation}
    q_{км} = \frac{q_{ч}}{3.6V},
    \label{eq:q_km}
\end{equation}
где $q_{ч}$ --- часовой расход топлива, $q_{км}$ --- километровый расход топлива.


\begin{sidewaystable}
    \centering
    \caption{Результаты расчета для высоты $H=0$ км}
    \label{tab:result_0}
    \begin{tabular}{lllllllllllll}
$M$ & $V$ & $V$ & $q$ & $C_{y_n}$ & $K_n$ & $P_n*10^{-5}$ & $P_p*10^{-5}$ & $\Delta \bar{p}(n_x)$ & $V_y^*$ & $\bar{R}_{кр}$ & $q_{ч}$ & $q_{км}$ \\
$-$ & $\frac{м}{с}$ & $\frac{км}{ч}$ & $\frac{H}{м^2}$ & $-$ & $-$ & $H$ & $H$ & $-$ & $\frac{м}{с}$ & $-$ & $\frac{кг}{ч}$ & $\frac{кг}{км}$ \\
0.10 & 34.03 & 122.50 & 709. & 7.17 & 2.063 & 6.325 & 4.045 & -0.17 & -5.95 & 1.56 & 34701. & 283.26 \\
0.20 & 68.06 & 245.01 & 2837. & 1.79 & 8.400 & 1.553 & 3.798 & 0.17 & 11.71 & 0.41 & 11821. & 48.25 \\
0.30 & 102.09 & 367.51 & 6383. & 0.8 & 14.970 & 0.872 & 3.569 & 0.21 & 21.11 & 0.24 & 8315. & 22.62 \\
0.40 & 136.12 & 490.02 & 11348. & 0.45 & 14.968 & 0.872 & 3.396 & 0.19 & 26.34 & 0.26 & 8619. & 17.59 \\
0.5 & 170.15 & 612.52 & 17731. & 0.29 & 11.505 & 1.134 & 3.279 & 0.16 & 27.97 & 0.35 & 10763. & 17.57 \\
0.60 & 204.17 & 735.03 & 25533. & 0.2 & 8.379 & 1.557 & 3.201 & 0.13 & 25.73 & 0.49 & 13413. & 18.25 \\
0.7 & 238.20 & 857.53 & 34754. & 0.15 & 6.217 & 2.099 & 3.167 & 0.08 & 19.50 & 0.66 & 15761. & 18.38 \\
0.80 & 272.23 & 980.04 & 45393. & 0.11 & 4.606 & 2.833 & 3.158 & 0.02 & 6.79 & 0.9 & 20914. & 21.34 \\
0.9 & 306.26 & 1102.54 & 57450. & 0.09 & 3.018 & 4.323 & 3.193 & -0.09 & -26.53 & 1.35 & 34825. & 31.59 \\
0.95 & 323.28 & 1163.79 & 64011. & 0.08 & 1.97 & 6.624 & 3.219 & -0.26 & -84.37 & 2.06 & 53864. & 46.28 \\
\end{tabular}

\end{sidewaystable}

\begin{sidewaystable}
    \centering
    \caption{Результаты расчета для высоты $H=2$ км}
    \label{tab:result_2}
    \begin{tabular}{|c|c|c|c|c|c|c|c|c|c|c|c|c|}
\hline
$M$ & $V$ & $V$ & $q$ & $C_{y_n}$ & $K_n$ & $P_n*10^{-5}$ & $P_p*10^{-5}$ & $\Delta \bar{p}(n_x)$ & $V_y^*$ & $\bar{R}_{кр}$ & $q_{ч}$ & $q_{км}$ \\ 
\hline
$-$ & $\frac{м}{с}$ & $\frac{км}{ч}$ & $\frac{H}{м^2}$ & $-$ & $-$ & $H$ & $H$ & $-$ & $\frac{м}{с}$ & $-$ & $\frac{кг}{ч}$ & $\frac{кг}{км}$ \\ 
\hline
0.10 & 33. & 120. & 557. & 9.129 & 1.61 & 8.108 & 3.708 & -0.337 & -11.2 & 2.19 & 43913. & 366.82 \\ 
\hline
0.20 & 67. & 239. & 2227. & 2.282 & 6.65 & 1.962 & 3.483 & 0.117 & 7.7 & 0.56 & 12280. & 51.29 \\ 
\hline
0.30 & 100. & 359. & 5011. & 1.014 & 13.28 & 0.983 & 3.266 & 0.175 & 17.5 & 0.30 & 8608. & 23.97 \\ 
\hline
0.40 & 133. & 479. & 8908. & 0.571 & 15.77 & 0.827 & 3.085 & 0.173 & 23.0 & 0.27 & 7858. & 16.41 \\ 
\hline
0.5 & 166. & 599. & 13919. & 0.365 & 13.57 & 0.961 & 2.963 & 0.153 & 25.5 & 0.32 & 9006. & 15.05 \\ 
\hline
0.60 & 200. & 718. & 20043. & 0.254 & 10.36 & 1.259 & 2.877 & 0.124 & 24.7 & 0.44 & 11002. & 15.32 \\ 
\hline
0.7 & 233. & 838. & 27281. & 0.186 & 7.83 & 1.666 & 2.847 & 0.091 & 21.1 & 0.59 & 12882. & 15.37 \\ 
\hline
0.80 & 266. & 958. & 35632. & 0.143 & 5.83 & 2.239 & 2.838 & 0.046 & 12.2 & 0.79 & 15478. & 16.16 \\ 
\hline
0.9 & 299. & 1077. & 45097. & 0.113 & 3.81 & 3.428 & 2.86 & -0.044 & -13.0 & 1.2 & 26469. & 24.57 \\ 
\hline
0.95 & 316. & 1137. & 50247. & 0.101 & 2.48 & 5.256 & 2.879 & -0.182 & -57.6 & 1.83 & 41009. & 36.06 \\ 
\hline
\end{tabular}
\end{sidewaystable}

\begin{sidewaystable}
    \centering
    \caption{Результаты расчета для высоты $H=4$ км}
    \label{tab:result_4}
    \begin{tabular}{|c|c|c|c|c|c|c|c|c|c|c|c|c|}
\hline
$M$ & $V$ & $V$ & $q$ & $C_{y_n}$ & $K_n$ & $P_n*10^{-5}$ & $P_p*10^{-5}$ & $\Delta \bar{p}(n_x)$ & $V_y^*$ & $\bar{R}_{кр}$ & $q_{ч}$ & $q_{км}$ \\ 
\hline
$-$ & $\frac{м}{с}$ & $\frac{км}{ч}$ & $\frac{H}{м^2}$ & $-$ & $-$ & $H$ & $H$ & $-$ & $\frac{м}{с}$ & $-$ & $\frac{кг}{ч}$ & $\frac{кг}{км}$ \\ 
\hline
0.10 & 32 & 117 & 431 & 11.780 & 1.24 & 10.522 & 3.357 & -0.549 & -17.8 & 3.13 & 56927 & 487.17 \\ 
\hline
0.20 & 65 & 234 & 1726 & 2.945 & 5.15 & 2.536 & 3.093 & 0.043 & 2.8 & 0.82 & 13116 & 56.12 \\ 
\hline
0.30 & 97 & 351 & 3883 & 1.309 & 11.06 & 1.180 & 2.890 & 0.131 & 12.8 & 0.41 & 9060 & 25.84 \\ 
\hline
0.40 & 130 & 467 & 6903 & 0.736 & 15.35 & 0.850 & 2.726 & 0.144 & 18.7 & 0.31 & 7540 & 16.13 \\ 
\hline
0.50 & 162 & 584 & 10786 & 0.471 & 15.21 & 0.858 & 2.609 & 0.134 & 21.8 & 0.33 & 7797 & 13.34 \\ 
\hline
0.60 & 195 & 701 & 15532 & 0.327 & 12.51 & 1.043 & 2.522 & 0.113 & 22.1 & 0.41 & 9043 & 12.90 \\ 
\hline
0.70 & 227 & 818 & 21141 & 0.240 & 9.77 & 1.336 & 2.470 & 0.087 & 19.8 & 0.54 & 10455 & 12.78 \\ 
\hline
0.80 & 260 & 935 & 27612 & 0.184 & 7.35 & 1.776 & 2.453 & 0.052 & 13.5 & 0.72 & 12129 & 12.97 \\ 
\hline
0.90 & 292 & 1052 & 34947 & 0.145 & 4.80 & 2.720 & 2.509 & -0.016 & -4.7 & 1.08 & 19916 & 18.94 \\ 
\hline
0.95 & 308 & 1110 & 38938 & 0.131 & 3.13 & 4.162 & 2.542 & -0.124 & -38.3 & 1.64 & 30744 & 27.69 \\ 
\hline
\end{tabular}
\end{sidewaystable}

\begin{sidewaystable}
    \centering
    \caption{Результаты расчета для высоты $H=6$ км}
    \label{tab:result_6}
    \begin{tabular}{lllllllllllll}
$M$ & $V$ & $V$ & $q$ & $C_{y_n}$ & $K_n$ & $P_n*10^{-5}$ & $P_p*10^{-5}$ & $\Delta \bar{p}(n_x)$ & $V_y^*$ & $\bar{R}_{кр}$ & $q_{ч}$ & $q_{км}$ \\
$-$ & $\frac{м}{с}$ & $\frac{км}{ч}$ & $\frac{H}{м^2}$ & $-$ & $-$ & $H$ & $H$ & $-$ & $\frac{м}{с}$ & $-$ & $\frac{кг}{ч}$ & $\frac{кг}{км}$ \\
0.10 & 31.65 & 113.92 & 330. & 15.38 & 0.945 & 13.803 & 2.769 & -0.85 & -26.76 & 4.99 & 73934. & 648.99 \\
0.20 & 63.29 & 227.84 & 1322. & 3.84 & 3.916 & 3.332 & 2.574 & -0.06 & -3.67 & 1.29 & 18493. & 81.16 \\
0.30 & 94.94 & 341.77 & 2974. & 1.71 & 8.777 & 1.487 & 2.444 & 0.07 & 6.97 & 0.61 & 8970. & 26.25 \\
0.40 & 126.58 & 455.69 & 5287. & 0.96 & 13.705 & 0.952 & 2.336 & 0.11 & 13.43 & 0.41 & 7460. & 16.37 \\
0.5 & 158.22 & 569.61 & 8262. & 0.62 & 15.771 & 0.827 & 2.263 & 0.11 & 17.41 & 0.37 & 7090. & 12.45 \\
0.60 & 189.87 & 683.53 & 11897. & 0.43 & 14.338 & 0.91 & 2.224 & 0.10 & 19.12 & 0.41 & 7770. & 11.37 \\
0.7 & 221.51 & 797.45 & 16193. & 0.31 & 11.792 & 1.106 & 2.206 & 0.08 & 18.67 & 0.50 & 8845. & 11.09 \\
0.80 & 253.16 & 911.38 & 21150. & 0.24 & 9.077 & 1.437 & 2.215 & 0.06 & 15.09 & 0.65 & 10105. & 11.09 \\
0.9 & 284.80 & 1025.3 & 26768. & 0.19 & 5.964 & 2.188 & 2.263 & 0.01 & 1.64 & 0.97 & 15225. & 14.85 \\
0.95 & 300.63 & 1082.26 & 29824. & 0.17 & 3.926 & 3.323 & 2.299 & -0.08 & -23.58 & 1.45 & 23971. & 22.15 \\
\end{tabular}

\end{sidewaystable}

\begin{sidewaystable}
    \centering
    \caption{Результаты расчета для высоты $H=8$ км}
    \label{tab:result_8}
    \begin{tabular}{|c|c|c|c|c|c|c|c|c|c|c|c|c|}
\hline
$M$ & $V$ & $V$ & $q$ & $C_{y_n}$ & $K_n$ & $P_n*10^{-5}$ & $P_p*10^{-5}$ & $\Delta \bar{p}(n_x)$ & $V_y^*$ & $\bar{R}_{кр}$ & $q_{ч}$ & $q_{км}$ \\ 
\hline
$-$ & $\frac{м}{с}$ & $\frac{км}{ч}$ & $\frac{H}{м^2}$ & $-$ & $-$ & $H$ & $H$ & $-$ & $\frac{м}{с}$ & $-$ & $\frac{кг}{ч}$ & $\frac{кг}{км}$ \\ 
\hline
0.10 & 31. & 111. & 250. & 20.357 & 0.71 & 18.343 & 2.358 & -1.225 & -37.7 & 7.78 & 97560. & 879.56 \\ 
\hline
0.20 & 62. & 222. & 999. & 5.089 & 2.93 & 4.447 & 2.206 & -0.172 & -10.6 & 2.02 & 24373. & 109.87 \\ 
\hline
0.30 & 92. & 333. & 2247. & 2.262 & 6.71 & 1.945 & 2.077 & 0.010 & 0.9 & 0.94 & 10501. & 31.56 \\ 
\hline
0.40 & 123. & 444. & 3995. & 1.272 & 11.31 & 1.153 & 1.990 & 0.064 & 7.9 & 0.58 & 7288. & 16.43 \\ 
\hline
0.5 & 154. & 555. & 6242. & 0.814 & 14.83 & 0.88 & 1.96 & 0.083 & 12.8 & 0.45 & 6736. & 12.15 \\ 
\hline
0.60 & 185. & 666. & 8988. & 0.565 & 15.14 & 0.862 & 1.947 & 0.083 & 15.4 & 0.44 & 6934. & 10.42 \\ 
\hline
0.7 & 216. & 776. & 12234. & 0.415 & 13.39 & 0.975 & 1.973 & 0.076 & 16.5 & 0.49 & 7662. & 9.87 \\ 
\hline
0.80 & 246. & 887. & 15979. & 0.318 & 10.72 & 1.217 & 2.020 & 0.062 & 15.2 & 0.60 & 8729. & 9.84 \\ 
\hline
0.9 & 277. & 998. & 20223. & 0.251 & 7.18 & 1.816 & 2.072 & 0.02 & 5.4 & 0.88 & 11788. & 11.81 \\ 
\hline
0.95 & 293. & 1054. & 22533. & 0.226 & 4.81 & 2.712 & 2.103 & -0.047 & -13.7 & 1.29 & 19081. & 18.11 \\ 
\hline
\end{tabular}
\end{sidewaystable}

\begin{sidewaystable}
    \centering
    \caption{Результаты расчета для высоты $H=10$ км}
    \label{tab:result_10}
    \begin{tabular}{lllllllllllll}
$M$ & $V$ & $V$ & $q$ & $C_{y_n}$ & $K_n$ & $P_n*10^{-5}$ & $P_p*10^{-5}$ & $\Delta \bar{p}(n_x)$ & $V_y^*$ & $\bar{R}_{кр}$ & $q_{ч}$ & $q_{км}$ \\
$-$ & $\frac{м}{с}$ & $\frac{км}{ч}$ & $\frac{H}{м^2}$ & $-$ & $-$ & $H$ & $H$ & $-$ & $\frac{м}{с}$ & $-$ & $\frac{кг}{ч}$ & $\frac{кг}{км}$ \\
0.10 & 29.95 & 107.83 & 186. & 27.37 & 0.527 & 24.742 & 1.774 & -1.76 & -52.73 & 13.95 & 130922. & 1214.14 \\
0.20 & 59.91 & 215.66 & 743. & 6.84 & 2.163 & 6.031 & 1.674 & -0.33 & -20.01 & 3.60 & 32730. & 151.76 \\
0.30 & 89.86 & 323.49 & 1671. & 3.04 & 4.981 & 2.619 & 1.601 & -0.08 & -7.02 & 1.64 & 14594. & 45.11 \\
0.40 & 119.81 & 431.32 & 2971. & 1.71 & 8.77 & 1.488 & 1.544 & 0.00 & 0.52 & 0.96 & 8315. & 19.28 \\
0.5 & 149.76 & 539.15 & 4643. & 1.09 & 12.626 & 1.033 & 1.542 & 0.04 & 5.84 & 0.67 & 6131. & 11.37 \\
0.60 & 179.72 & 646.98 & 6686. & 0.76 & 14.344 & 0.91 & 1.549 & 0.05 & 8.80 & 0.59 & 6088. & 9.41 \\
0.7 & 209.67 & 754.82 & 9100. & 0.56 & 13.786 & 0.946 & 1.570 & 0.05 & 10.03 & 0.60 & 6454. & 8.55 \\
0.80 & 239.62 & 862.65 & 11886. & 0.43 & 11.697 & 1.115 & 1.627 & 0.04 & 9.39 & 0.69 & 7286. & 8.45 \\
0.9 & 269.58 & 970.48 & 15043. & 0.34 & 8.141 & 1.603 & 1.739 & 0.01 & 2.82 & 0.92 & 10413. & 10.73 \\
0.95 & 284.55 & 1024.39 & 16761. & 0.30 & 5.630 & 2.317 & 1.815 & -0.04 & -10.96 & 1.28 & 16091. & 15.71 \\
\end{tabular}

\end{sidewaystable}

\begin{sidewaystable}
    \centering
    \caption{Результаты расчета для высоты $H=12$ км}
    \label{tab:result_12}
    \begin{tabular}{|c|c|c|c|c|c|c|c|c|c|c|c|c|}
\hline
$M$ & $V$ & $V$ & $q$ & $C_{y_n}$ & $K_n$ & $P_n*10^{-5}$ & $P_p*10^{-5}$ & $\Delta \bar{p}(n_x)$ & $V_y^*$ & $\bar{R}_{кр}$ & $q_{ч}$ & $q_{км}$ \\ 
\hline
$-$ & $\frac{м}{с}$ & $\frac{км}{ч}$ & $\frac{H}{м^2}$ & $-$ & $-$ & $H$ & $H$ & $-$ & $\frac{м}{с}$ & $-$ & $\frac{кг}{ч}$ & $\frac{кг}{км}$ \\ 
\hline
0.10 & 30 & 106 & 136 & 37.420 & 0.38 & 33.920 & 1.294 & -2.501 & -73.8 & 26.21 & 176742 & 1663.84 \\ 
\hline
0.20 & 59 & 212 & 543 & 9.355 & 1.57 & 8.314 & 1.202 & -0.545 & -32.2 & 6.92 & 44443 & 209.19 \\ 
\hline
0.30 & 89 & 319 & 1222 & 4.158 & 3.61 & 3.611 & 1.146 & -0.189 & -16.7 & 3.15 & 19832 & 62.23 \\ 
\hline
0.40 & 118 & 425 & 2173 & 2.339 & 6.49 & 2.010 & 1.120 & -0.068 & -8.1 & 1.79 & 11380 & 26.78 \\ 
\hline
0.50 & 148 & 531 & 3396 & 1.497 & 9.88 & 1.321 & 1.113 & -0.016 & -2.4 & 1.19 & 7761 & 14.61 \\ 
\hline
0.60 & 177 & 637 & 4890 & 1.039 & 12.14 & 1.075 & 1.128 & 0.004 & 0.7 & 0.95 & 6355 & 9.97 \\ 
\hline
0.70 & 207 & 744 & 6655 & 0.764 & 12.57 & 1.038 & 1.165 & 0.010 & 2.0 & 0.89 & 6192 & 8.33 \\ 
\hline
0.80 & 236 & 850 & 8693 & 0.585 & 11.39 & 1.146 & 1.231 & 0.007 & 1.5 & 0.93 & 7189 & 8.46 \\ 
\hline
0.90 & 266 & 956 & 11002 & 0.462 & 8.39 & 1.556 & 1.331 & -0.017 & -4.6 & 1.17 & 10556 & 11.04 \\ 
\hline
0.95 & 280 & 1009 & 12258 & 0.415 & 6.09 & 2.142 & 1.386 & -0.058 & -16.2 & 1.54 & 14712 & 14.58 \\ 
\hline
\end{tabular}
\end{sidewaystable}

\begin{sidewaystable}
    \centering
    \caption{Результаты расчета для высоты $H=12.40$ км}
    \label{tab:result_12_40}
    \begin{tabular}{lllllllllllll}
$M$ & $V$ & $V$ & $q$ & $C_{y_n}$ & $K_n$ & $P_n*10^{-5}$ & $P_p*10^{-5}$ & $\Delta \bar{p}(n_x)$ & $V_y^*$ & $\bar{R}_{кр}$ & $q_{ч}$ & $q_{км}$ \\
$-$ & $\frac{м}{с}$ & $\frac{км}{ч}$ & $\frac{H}{м^2}$ & $-$ & $-$ & $H$ & $H$ & $-$ & $\frac{м}{с}$ & $-$ & $\frac{кг}{ч}$ & $\frac{кг}{км}$ \\
0.10 & 29.51 & 106.23 & 128. & 39.83 & 0.361 & 36.121 & 1.215 & -2.68 & -78.94 & 29.73 & 188212. & 1771.82 \\
0.20 & 59.01 & 212.45 & 510. & 9.96 & 1.472 & 8.862 & 1.128 & -0.59 & -34.98 & 7.85 & 47374. & 222.99 \\
0.30 & 88.52 & 318.68 & 1148. & 4.43 & 3.388 & 3.851 & 1.076 & -0.21 & -18.82 & 3.58 & 21150. & 66.37 \\
0.40 & 118.03 & 424.90 & 2042. & 2.49 & 6.098 & 2.14 & 1.052 & -0.08 & -9.84 & 2.03 & 12111. & 28.50 \\
0.5 & 147.53 & 531.13 & 3190. & 1.59 & 9.349 & 1.396 & 1.045 & -0.03 & -3.97 & 1.34 & 8201. & 15.44 \\
0.60 & 177.04 & 637.35 & 4594. & 1.11 & 11.619 & 1.123 & 1.059 & -0.00 & -0.87 & 1.06 & 6877. & 10.79 \\
0.7 & 206.55 & 743.58 & 6253. & 0.81 & 12.175 & 1.072 & 1.094 & 0.00 & 0.35 & 0.98 & 6714. & 9.03 \\
0.80 & 236.06 & 849.80 & 8167. & 0.62 & 11.164 & 1.169 & 1.156 & -0.00 & -0.23 & 1.01 & 7698. & 9.06 \\
0.9 & 265.56 & 956.03 & 10336. & 0.49 & 8.324 & 1.567 & 1.25 & -0.02 & -6.47 & 1.25 & 10637. & 11.13 \\
0.95 & 280.32 & 1009.14 & 11516. & 0.44 & 6.111 & 2.135 & 1.302 & -0.06 & -17.9 & 1.64 & 14663. & 14.53 \\
\end{tabular}

\end{sidewaystable}

Для построение таблицы (TODO: стр 40 в курсовой)

\begin{enumerate}
    \item Определим $M_{{\min}_P}$ и $M_{{\max}_P}$, как точка пересечения
        графиков $P_n(M, H_i)$ и $P_p(M, H_i)$ рисунки @@@ 
    \item Минимально допустимое число $M_{{\min}_{доп}}$, как точка пересечения
        графиков $C_{y_n}(M, H_i)$ и $C_{y_{доп}}(M)$ рисунки @@@
    \item Максимально допустимое число $M$ полета по условиям безопасности 
        определяется как: 
    \[
        M_{{\max}_{доп}} = \min \left\{ M_{пред}, M(V_{i_{\max}} \right\},
    \]
    где $M(V_{i_{\max}}) = \frac{V_{i_{\max}} \sqrt{\Delta^{-1}}}{3.6 a_H}$, 
    $\sqrt{\Delta^{-1}} = \sqrt{\frac{\rho_0}{\rho_H}}$
    \item Располагаемые значение минимального и максимального числа $M$
определяются как: 
\[
    M_{\min} = \max \left\{ M_{{\min}_{доп}}, M_{{\min}_P} \right\},
\]
\[
    M_{\max} = \min \left\{ M_{{\max}_{доп}}, M_{{\max}_P}, M_{пред} \right\},
\]
    \item Число $М_1$ полета, соответствующее минимальной потребной тяге
        определяется как:
        \[
            M_1 = M(P_{n_{\min}}) = \arg \min_{M} \Delta P_n (M)
        \]
    \item Число $М_2$ полета, соответствующее максимальной энергетической скороподъёмности
определяется как:
        \[
            M_2 = M(V_{y_{max}}^*) = \arg \max_{M} V_y^* (M, H_i)
        \]
    \item Минимальные значения часового $q_{ч_{min}}$ и километрового
        $q_{км_{min}}$ расхода топлива, и соответствующие им скорости полета
        определены на графике 2.4.1-7 и 2.5.1-7 или как:
        \[
            q_{ч_{min}} = \min_V q_ч(V, H_i), \, V_3 = V(q_{ч_{min}}) =
            \arg \min_V q_ч (V, H_i)
        \]
        \[
            q_{{км}_{min}} = \min_V q_{км}(V, H_i), \, V_4 = V(q_{{км}_{min}}) =
            \arg \min_V q_{км} (V, H_i)
        \]
\end{enumerate}

\begin{sidewaystable}
    \centering
    \caption{Результаты для построение графика высот и скоростей}
    \label{tab:H_M}
    \begin{tabular}{|c|c|c|c|c|c|c|c|c|c|c|c|c|}
\hline
$H$ & $V_{y_{max}}^*$ & $\underset{\min \, доп}{M [V]}$ & $\underset{\max \, доп}{M [V]}$ & $\underset{\min}{M [V]}$ & $\underset{\max}{M [V]}$ & $\underset{(P_п\, min)}{M_1 [V_1]}$ & $\underset{(V_{y_{max}}^*)}{M_2 [V_2]}$ & $\underset{(q_{ч_{\min}})}{V_3}$ & $\underset{(q_{{км}_{\min}})}{V_4}$ & $M_4$ & $q_{ч_{\min}}$ & $q_{{км}_{\min}}$ \\ 
\hline
$км$ & $\frac{м}{с}$ & $-\,[\frac{км}{ч}]$ & $-\,[\frac{км}{ч}]$ & $-\,[\frac{км}{ч}]$ & $-\,[\frac{км}{ч}]$ & $-\,[\frac{км}{ч}]$ & $-\,[\frac{км}{ч}]$ & $\frac{км}{ч}$ & $\frac{км}{ч}$ & $-$ & $\frac{кг}{ч}$ & $\frac{кг}{км}$ \\ 
\hline
0.0 & 17.79 & $0.240\, [293]$ & $0.612\, [750]$ & $0.240\, [293]$ & $0.612\, [750]$ & $0.300\, [368]$ & $0.380\, [466]$ & 99 & 126 & 0.370 & 6536.16 & 16.3 \\ 
\hline
2.0 & 16.56 & $0.270\, [324]$ & $0.675\, [808]$ & $0.270\, [324]$ & $0.671\, [803]$ & $0.340\, [407]$ & $0.420\, [503]$ & 110 & 133 & 0.400 & 6286.48 & 14.44 \\ 
\hline
4.0 & 13.58 & $0.307\, [359]$ & $0.748\, [874]$ & $0.307\, [359]$ & $0.699\, [817]$ & $0.380\, [444]$ & $0.460\, [538]$ & 120 & 149 & 0.460 & 6193.81 & 12.99 \\ 
\hline
6.0 & 11.23 & $0.352\, [401]$ & $0.800\, [911]$ & $0.352\, [401]$ & $0.726\, [827]$ & $0.440\, [501]$ & $0.500\, [570]$ & 130 & 161 & 0.510 & 6076.86 & 11.58 \\ 
\hline
8.0 & 8.1 & $0.406\, [451]$ & $0.800\, [887]$ & $0.406\, [451]$ & $0.744\, [825]$ & $0.500\, [555]$ & $0.540\, [599]$ & 145 & 176 & 0.570 & 5951.95 & 10.49 \\ 
\hline
10.0 & 4.33 & $0.475\, [513]$ & $0.800\, [863]$ & $0.475\, [513]$ & $0.739\, [796]$ & $0.540\, [582]$ & $0.590\, [636]$ & 156 & 183 & 0.610 & 5902.71 & 9.67 \\ 
\hline
11.56 & 0.5 & $0.544\, [578]$ & $0.800\, [850]$ & $0.544\, [578]$ & $0.664\, [705]$ & $0.590\, [627]$ & $0.600\, [637]$ & 168 & 195 & 0.660 & 6159.64 & 9.54 \\ 
\hline
\end{tabular}
\end{sidewaystable}
