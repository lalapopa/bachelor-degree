\chapter{Расчет лётно – технических характеристик самолета}

Определим следующие характеристики самолета:
\begin{enumerate}
    \item Зависимости от числа M (скорости) и H (высоты) полета результаты
        сведем в таблицы 2.1-2.7:
    \begin{itemize}
    \item  располагаемой и потребной для горизонтального установившегося полета
    тяги силовой установки, 
    \item энергетической скороподъемности,
    \item часового расхода топлива,
    \item километрового расхода топлива.
    \end{itemize}
    \item Зависимости от высоты:
        \begin{itemize}
            \item максимальной энергетической скороподъемности,
            \item минимального часового расхода топлива,
            \item минимального километрового расхода топлива,
            \item минимального и максимального числа M (скорости) полета (с учетом
                ограничений по безопасности полета),
            \item числа $M$ (скорости) полета, соответствующего минимальной
                потребной тяги,
            \item числа $M$ (скорости) полета, соответствующего максимальной
                энергетической скороподъемности,
            \item скорости полета, соответствующей минимальному часовому расходу топлива,
            \item скорости полета, соответствующему минимальному километровому
                расходу топлива
        \end{itemize}
    \item Статический и практический потолки самолета.
\end{enumerate}

Соотношения для расчета:
Узловые точки по числу Маха:
\[
    M = [0.2 0.3 0.4 0.5 0.6 0.7 0.8 0.9 0.95]
\]
\begin{equation}
    V = M a_H,
    \label{eq:V_speed}
\end{equation}
где $a_H$ --- скорость звука на высоте $H$.
\begin{equation}
    q = \frac{\rho_H V^2}{2},
    \label{eq:q_value}
\end{equation}
где $\rho_H$ --- плотность воздуха на высоте $H$.
\begin{equation}
    C_{y_n} = \frac{\bar{m} p_s 10}{q},
    \label{eq:c_y_n}
\end{equation}
где $\bar{m} = 0.95$ --- относительная масса самолета, $p_s$ --- удельная 
нагрузка на крыло.
\begin{equation}
    C_{x_n}(C_y, M) = C_{x_m}(M) + A(M) \left[ C_{y_n} - C_{y_m}(M)\right]^2
    \label{eq:c_x_n}
\end{equation}
где $C_{y_m}$ --- коэффициент подъемной силы при $C_x = C_{x_m}$, $C_{x_m}$ ---
минимальный коэффициент лобового сопротивления, $A$ --- коэффициент отвала
поляры.
\begin{equation}
    K_n = \frac{C_{y_n}}{C_{x_n}}
    \label{eq:K_n}
\end{equation}
\begin{equation}
    P_n = \frac{\bar{m} m_0 g}{K_n}
    \label{eq:P_potr}
\end{equation}
\begin{equation}
    P_p(M,H) = \bar{P}_0 m_0 g \tilde{P}(H,M)
    \label{eq:P_rasp}
\end{equation}
\begin{equation}
    n_x = \Delta \bar{P} = \frac{(P_p - P_n)}{\bar{m} m_0 g}
    \label{eq:n_x}
\end{equation}
\begin{equation}
    V_y^* = \Delta \bar{P} V
    \label{eq:Vy}
\end{equation}
\begin{equation}
    \bar{R} = \frac{P_n}{P_p}
    \label{eq:R_dross}
\end{equation}
\begin{equation}
    q_{ч} = Ce(M,H,\bar{R})P_n = Ce_0 \tilde{Ce}(H,M) \hat{Ce}_{др}(R) P_n
    \label{eq:q_chas}
\end{equation}
\begin{equation}
    q_{км} = \frac{q_{ч}}{3.6V},
    \label{eq:q_km}
\end{equation}
где $q_{ч}$ --- часовой расход топлива, $q_{км}$ --- километровый расход топлива.


\begin{sidewaystable}
    \centering
    \caption{Результаты расчета для высоты $H=0$ км}
    \label{tab:result_0}
    \begin{tabular}{|c|c|c|c|c|c|c|c|c|c|c|c|c|}
\hline
$M$ & $V$ & $V$ & $q$ & $C_{y_n}$ & $K_n$ & $P_n*10^{-5}$ & $P_p*10^{-5}$ & $\Delta \bar{p}(n_x)$ & $V_y^*$ & $\bar{R}_{кр}$ & $q_{ч}$ & $q_{км}$ \\ 
\hline
$-$ & $\frac{м}{с}$ & $\frac{км}{ч}$ & $\frac{H}{м^2}$ & $-$ & $-$ & $H$ & $H$ & $-$ & $\frac{м}{с}$ & $-$ & $\frac{кг}{ч}$ & $\frac{кг}{км}$ \\ 
\hline
0.10 & 34. & 123. & 709. & 7.166 & 2.06 & 6.325 & 4.045 & -0.175 & -5.9 & 1.56 & 34701. & 283.26 \\ 
\hline
0.20 & 68. & 245. & 2837. & 1.791 & 8.40 & 1.553 & 3.798 & 0.172 & 11.7 & 0.41 & 11821. & 48.25 \\ 
\hline
0.30 & 102. & 368. & 6383. & 0.796 & 14.97 & 0.872 & 3.569 & 0.207 & 21.1 & 0.24 & 8315. & 22.62 \\ 
\hline
0.40 & 136. & 490. & 11348. & 0.448 & 14.97 & 0.872 & 3.396 & 0.193 & 26.3 & 0.26 & 8619. & 17.59 \\ 
\hline
0.5 & 170. & 613. & 17731. & 0.287 & 11.50 & 1.134 & 3.279 & 0.164 & 28. & 0.35 & 10763. & 17.57 \\ 
\hline
0.60 & 204. & 735. & 25533. & 0.199 & 8.38 & 1.557 & 3.201 & 0.126 & 25.7 & 0.49 & 13413. & 18.25 \\ 
\hline
0.7 & 238. & 858. & 34754. & 0.146 & 6.22 & 2.099 & 3.167 & 0.082 & 19.5 & 0.66 & 15761. & 18.38 \\ 
\hline
0.80 & 272. & 980. & 45393. & 0.112 & 4.61 & 2.833 & 3.158 & 0.025 & 6.8 & 0.9 & 20914. & 21.34 \\ 
\hline
0.9 & 306. & 1103. & 57450. & 0.088 & 3.02 & 4.323 & 3.193 & -0.087 & -26.5 & 1.35 & 34825. & 31.59 \\ 
\hline
0.95 & 323. & 1164. & 64011. & 0.079 & 1.97 & 6.624 & 3.219 & -0.261 & -84.4 & 2.06 & 53864. & 46.28 \\ 
\hline
\end{tabular}
\end{sidewaystable}

\begin{sidewaystable}
    \centering
    \caption{Результаты расчета для высоты $H=2$ км}
    \label{tab:result_2}
    \begin{tabular}{|c|c|c|c|c|c|c|c|c|c|c|c|c|}
\hline
$M$ & $V$ & $V$ & $q$ & $C_{y_n}$ & $K_n$ & $P_n*10^{-5}$ & $P_p*10^{-5}$ & $\Delta \bar{p}(n_x)$ & $V_y^*$ & $\bar{R}_{кр}$ & $q_{ч}$ & $q_{км}$ \\ 
\hline
$-$ & $\frac{м}{с}$ & $\frac{км}{ч}$ & $\frac{H}{м^2}$ & $-$ & $-$ & $H$ & $H$ & $-$ & $\frac{м}{с}$ & $-$ & $\frac{кг}{ч}$ & $\frac{кг}{км}$ \\ 
\hline
0.10 & 33 & 120 & 557 & 9.129 & 1.61 & 8.108 & 3.708 & -0.337 & -11.2 & 2.19 & 43913 & 366.82 \\ 
\hline
0.20 & 67 & 239 & 2227 & 2.282 & 6.65 & 1.962 & 3.483 & 0.117 & 7.7 & 0.56 & 12280 & 51.29 \\ 
\hline
0.30 & 100 & 359 & 5011 & 1.014 & 13.28 & 0.983 & 3.266 & 0.175 & 17.5 & 0.30 & 8608 & 23.97 \\ 
\hline
0.40 & 133 & 479 & 8908 & 0.571 & 15.77 & 0.827 & 3.085 & 0.173 & 23.0 & 0.27 & 7858 & 16.41 \\ 
\hline
0.50 & 166 & 599 & 13919 & 0.365 & 13.57 & 0.961 & 2.963 & 0.153 & 25.5 & 0.32 & 9006 & 15.05 \\ 
\hline
0.60 & 200 & 718 & 20043 & 0.254 & 10.36 & 1.259 & 2.877 & 0.124 & 24.7 & 0.44 & 11002 & 15.32 \\ 
\hline
0.70 & 233 & 838 & 27281 & 0.186 & 7.83 & 1.666 & 2.847 & 0.091 & 21.1 & 0.59 & 12882 & 15.37 \\ 
\hline
0.80 & 266 & 958 & 35632 & 0.143 & 5.83 & 2.239 & 2.838 & 0.046 & 12.2 & 0.79 & 15478 & 16.16 \\ 
\hline
0.90 & 299 & 1077 & 45097 & 0.113 & 3.81 & 3.428 & 2.860 & -0.044 & -13.0 & 1.20 & 26469 & 24.57 \\ 
\hline
0.95 & 316 & 1137 & 50247 & 0.101 & 2.48 & 5.256 & 2.879 & -0.182 & -57.6 & 1.83 & 41009 & 36.06 \\ 
\hline
\end{tabular}
\end{sidewaystable}

\begin{sidewaystable}
    \centering
    \caption{Результаты расчета для высоты $H=4$ км}
    \label{tab:result_4}
    \begin{tabular}{|c|c|c|c|c|c|c|c|c|c|c|c|c|}
\hline
$M$ & $V$ & $V$ & $q$ & $C_{y_n}$ & $K_n$ & $P_n*10^{-5}$ & $P_p*10^{-5}$ & $\Delta \bar{p}(n_x)$ & $V_y^*$ & $\bar{R}_{кр}$ & $q_{ч}$ & $q_{км}$ \\ 
\hline
$-$ & $\frac{м}{с}$ & $\frac{км}{ч}$ & $\frac{H}{м^2}$ & $-$ & $-$ & $H$ & $H$ & $-$ & $\frac{м}{с}$ & $-$ & $\frac{кг}{ч}$ & $\frac{кг}{км}$ \\ 
\hline
0.10 & 32. & 117. & 431. & 11.780 & 1.24 & 10.522 & 3.357 & -0.549 & -17.8 & 3.13 & 56927. & 487.17 \\ 
\hline
0.20 & 65. & 234. & 1726. & 2.945 & 5.15 & 2.536 & 3.093 & 0.043 & 2.8 & 0.82 & 13116. & 56.12 \\ 
\hline
0.30 & 97. & 351. & 3883. & 1.309 & 11.06 & 1.18 & 2.89 & 0.131 & 12.8 & 0.41 & 9060. & 25.84 \\ 
\hline
0.40 & 130. & 467. & 6903. & 0.736 & 15.35 & 0.850 & 2.726 & 0.144 & 18.7 & 0.31 & 7540. & 16.13 \\ 
\hline
0.5 & 162. & 584. & 10786. & 0.471 & 15.21 & 0.858 & 2.609 & 0.134 & 21.8 & 0.33 & 7797. & 13.34 \\ 
\hline
0.60 & 195. & 701. & 15532. & 0.327 & 12.51 & 1.043 & 2.522 & 0.113 & 22.1 & 0.41 & 9043. & 12.9 \\ 
\hline
0.7 & 227. & 818. & 21141. & 0.240 & 9.77 & 1.336 & 2.470 & 0.087 & 19.8 & 0.54 & 10455. & 12.78 \\ 
\hline
0.80 & 260. & 935. & 27612. & 0.184 & 7.35 & 1.776 & 2.453 & 0.052 & 13.5 & 0.72 & 12129. & 12.97 \\ 
\hline
0.9 & 292. & 1052. & 34947. & 0.145 & 4.8 & 2.720 & 2.509 & -0.016 & -4.7 & 1.08 & 19916. & 18.94 \\ 
\hline
0.95 & 308. & 1110. & 38938. & 0.131 & 3.13 & 4.162 & 2.542 & -0.124 & -38.3 & 1.64 & 30744. & 27.69 \\ 
\hline
\end{tabular}
\end{sidewaystable}

\begin{sidewaystable}
    \centering
    \caption{Результаты расчета для высоты $H=6$ км}
    \label{tab:result_6}
    \begin{tabular}{|c|c|c|c|c|c|c|c|c|c|c|c|c|}
\hline
$M$ & $V$ & $V$ & $q$ & $C_{y_n}$ & $K_n$ & $P_n*10^{-5}$ & $P_p*10^{-5}$ & $\Delta \bar{p}(n_x)$ & $V_y^*$ & $\bar{R}_{кр}$ & $q_{ч}$ & $q_{км}$ \\ 
\hline
$-$ & $\frac{м}{с}$ & $\frac{км}{ч}$ & $\frac{H}{м^2}$ & $-$ & $-$ & $H$ & $H$ & $-$ & $\frac{м}{с}$ & $-$ & $\frac{кг}{ч}$ & $\frac{кг}{км}$ \\ 
\hline
0.10 & 32. & 114. & 330. & 15.38 & 0.95 & 13.803 & 2.769 & -0.846 & -26.8 & 4.99 & 73934. & 648.99 \\ 
\hline
0.20 & 63. & 228. & 1322. & 3.845 & 3.92 & 3.332 & 2.574 & -0.058 & -3.7 & 1.29 & 18493. & 81.16 \\ 
\hline
0.30 & 95. & 342. & 2974. & 1.709 & 8.78 & 1.487 & 2.444 & 0.073 & 7. & 0.61 & 8970. & 26.25 \\ 
\hline
0.40 & 127. & 456. & 5287. & 0.961 & 13.71 & 0.952 & 2.336 & 0.106 & 13.4 & 0.41 & 7460. & 16.37 \\ 
\hline
0.5 & 158. & 570. & 8262. & 0.615 & 15.77 & 0.827 & 2.263 & 0.110 & 17.4 & 0.37 & 7090. & 12.45 \\ 
\hline
0.60 & 190. & 684. & 11897. & 0.427 & 14.34 & 0.91 & 2.224 & 0.101 & 19.1 & 0.41 & 7770. & 11.37 \\ 
\hline
0.7 & 222. & 797. & 16193. & 0.314 & 11.79 & 1.106 & 2.206 & 0.084 & 18.7 & 0.50 & 8845. & 11.09 \\ 
\hline
0.80 & 253. & 911. & 21150. & 0.240 & 9.08 & 1.437 & 2.215 & 0.06 & 15.1 & 0.65 & 10105. & 11.09 \\ 
\hline
0.9 & 285. & 1025. & 26768. & 0.19 & 5.96 & 2.188 & 2.263 & 0.006 & 1.6 & 0.97 & 15225. & 14.85 \\ 
\hline
0.95 & 301. & 1082. & 29824. & 0.170 & 3.93 & 3.323 & 2.299 & -0.078 & -23.6 & 1.45 & 23971. & 22.15 \\ 
\hline
\end{tabular}
\end{sidewaystable}

\begin{sidewaystable}
    \centering
    \caption{Результаты расчета для высоты $H=8$ км}
    \label{tab:result_8}
    \begin{tabular}{|c|c|c|c|c|c|c|c|c|c|c|c|c|}
\hline
$M$ & $V$ & $V$ & $q$ & $C_{y_n}$ & $K_n$ & $P_n*10^{-5}$ & $P_p*10^{-5}$ & $\Delta \bar{p}(n_x)$ & $V_y^*$ & $\bar{R}_{кр}$ & $q_{ч}$ & $q_{км}$ \\ 
\hline
$-$ & $\frac{м}{с}$ & $\frac{км}{ч}$ & $\frac{H}{м^2}$ & $-$ & $-$ & $H$ & $H$ & $-$ & $\frac{м}{с}$ & $-$ & $\frac{кг}{ч}$ & $\frac{кг}{км}$ \\ 
\hline
0.10 & 31. & 111. & 250. & 20.357 & 0.71 & 18.343 & 2.358 & -1.225 & -37.7 & 7.78 & 97560. & 879.56 \\ 
\hline
0.20 & 62. & 222. & 999. & 5.089 & 2.93 & 4.447 & 2.206 & -0.172 & -10.6 & 2.02 & 24373. & 109.87 \\ 
\hline
0.30 & 92. & 333. & 2247. & 2.262 & 6.71 & 1.945 & 2.077 & 0.010 & 0.9 & 0.94 & 10501. & 31.56 \\ 
\hline
0.40 & 123. & 444. & 3995. & 1.272 & 11.31 & 1.153 & 1.990 & 0.064 & 7.9 & 0.58 & 7288. & 16.43 \\ 
\hline
0.5 & 154. & 555. & 6242. & 0.814 & 14.83 & 0.88 & 1.96 & 0.083 & 12.8 & 0.45 & 6736. & 12.15 \\ 
\hline
0.60 & 185. & 666. & 8988. & 0.565 & 15.14 & 0.862 & 1.947 & 0.083 & 15.4 & 0.44 & 6934. & 10.42 \\ 
\hline
0.7 & 216. & 776. & 12234. & 0.415 & 13.39 & 0.975 & 1.973 & 0.076 & 16.5 & 0.49 & 7662. & 9.87 \\ 
\hline
0.80 & 246. & 887. & 15979. & 0.318 & 10.72 & 1.217 & 2.020 & 0.062 & 15.2 & 0.60 & 8729. & 9.84 \\ 
\hline
0.9 & 277. & 998. & 20223. & 0.251 & 7.18 & 1.816 & 2.072 & 0.02 & 5.4 & 0.88 & 11788. & 11.81 \\ 
\hline
0.95 & 293. & 1054. & 22533. & 0.226 & 4.81 & 2.712 & 2.103 & -0.047 & -13.7 & 1.29 & 19081. & 18.11 \\ 
\hline
\end{tabular}
\end{sidewaystable}

\begin{sidewaystable}
    \centering
    \caption{Результаты расчета для высоты $H=10$ км}
    \label{tab:result_10}
    \begin{tabular}{lllllllllllll}
$M$ & $V$ & $V$ & $q$ & $C_{y_n}$ & $K_n$ & $P_n*10^{-5}$ & $P_p*10^{-5}$ & $\Delta \bar{p}(n_x)$ & $V_y^*$ & $\bar{R}_{кр}$ & $q_{ч}$ & $q_{км}$ \\
$-$ & $\frac{м}{с}$ & $\frac{км}{ч}$ & $\frac{H}{м^2}$ & $-$ & $-$ & $H$ & $H$ & $-$ & $\frac{м}{с}$ & $-$ & $\frac{кг}{ч}$ & $\frac{кг}{км}$ \\
0.10 & 29.95 & 107.83 & 186. & 27.37 & 0.527 & 24.742 & 1.774 & -1.76 & -52.73 & 13.95 & 130922. & 1214.14 \\
0.20 & 59.91 & 215.66 & 743. & 6.84 & 2.163 & 6.031 & 1.674 & -0.33 & -20.01 & 3.60 & 32730. & 151.76 \\
0.30 & 89.86 & 323.49 & 1671. & 3.04 & 4.981 & 2.619 & 1.601 & -0.08 & -7.02 & 1.64 & 14594. & 45.11 \\
0.40 & 119.81 & 431.32 & 2971. & 1.71 & 8.77 & 1.488 & 1.544 & 0.00 & 0.52 & 0.96 & 8315. & 19.28 \\
0.5 & 149.76 & 539.15 & 4643. & 1.09 & 12.626 & 1.033 & 1.542 & 0.04 & 5.84 & 0.67 & 6131. & 11.37 \\
0.60 & 179.72 & 646.98 & 6686. & 0.76 & 14.344 & 0.91 & 1.549 & 0.05 & 8.80 & 0.59 & 6088. & 9.41 \\
0.7 & 209.67 & 754.82 & 9100. & 0.56 & 13.786 & 0.946 & 1.570 & 0.05 & 10.03 & 0.60 & 6454. & 8.55 \\
0.80 & 239.62 & 862.65 & 11886. & 0.43 & 11.697 & 1.115 & 1.627 & 0.04 & 9.39 & 0.69 & 7286. & 8.45 \\
0.9 & 269.58 & 970.48 & 15043. & 0.34 & 8.141 & 1.603 & 1.739 & 0.01 & 2.82 & 0.92 & 10413. & 10.73 \\
0.95 & 284.55 & 1024.39 & 16761. & 0.30 & 5.630 & 2.317 & 1.815 & -0.04 & -10.96 & 1.28 & 16091. & 15.71 \\
\end{tabular}

\end{sidewaystable}

\begin{sidewaystable}
    \centering
    \caption{Результаты расчета для высоты $H=12$ км}
    \label{tab:result_12}
    \begin{tabular}{lllllllllllll}
$M$ & $V$ & $V$ & $q$ & $C_{y_n}$ & $K_n$ & $P_n*10^{-5}$ & $P_p*10^{-5}$ & $\Delta \bar{p}(n_x)$ & $V_y^*$ & $\bar{R}_{кр}$ & $q_{ч}$ & $q_{км}$ \\
$-$ & $\frac{м}{с}$ & $\frac{км}{ч}$ & $\frac{H}{м^2}$ & $-$ & $-$ & $H$ & $H$ & $-$ & $\frac{м}{с}$ & $-$ & $\frac{кг}{ч}$ & $\frac{кг}{км}$ \\
0.10 & 29.51 & 106.23 & 136. & 37.42 & 0.385 & 33.92 & 1.294 & -2.50 & -73.78 & 26.21 & 176742. & 1663.84 \\
0.20 & 59.01 & 212.45 & 543. & 9.35 & 1.569 & 8.314 & 1.202 & -0.55 & -32.17 & 6.92 & 44443. & 209.19 \\
0.30 & 88.52 & 318.68 & 1222. & 4.16 & 3.613 & 3.611 & 1.146 & -0.19 & -16.72 & 3.15 & 19832. & 62.23 \\
0.40 & 118.03 & 424.90 & 2173. & 2.34 & 6.49 & 2.010 & 1.120 & -0.07 & -8.05 & 1.79 & 11380. & 26.78 \\
0.5 & 147.53 & 531.13 & 3396. & 1.5 & 9.879 & 1.321 & 1.113 & -0.02 & -2.35 & 1.19 & 7761. & 14.61 \\
0.60 & 177.04 & 637.35 & 4890. & 1.04 & 12.141 & 1.075 & 1.128 & 0.00 & 0.72 & 0.95 & 6355. & 9.97 \\
0.7 & 206.55 & 743.58 & 6655. & 0.76 & 12.571 & 1.038 & 1.165 & 0.01 & 2.01 & 0.89 & 6192. & 8.33 \\
0.80 & 236.06 & 849.80 & 8693. & 0.58 & 11.388 & 1.146 & 1.231 & 0.01 & 1.55 & 0.93 & 7189. & 8.46 \\
0.9 & 265.56 & 956.03 & 11002. & 0.46 & 8.388 & 1.556 & 1.331 & -0.02 & -4.57 & 1.17 & 10556. & 11.04 \\
0.95 & 280.32 & 1009.14 & 12258. & 0.41 & 6.091 & 2.142 & 1.386 & -0.06 & -16.23 & 1.54 & 14712. & 14.58 \\
\end{tabular}

\end{sidewaystable}

\begin{sidewaystable}
    \centering
    \caption{Результаты расчета для высоты $H=12.40$ км}
    \label{tab:result_12_40}
    \begin{tabular}{|c|c|c|c|c|c|c|c|c|c|c|c|c|}
\hline
$M$ & $V$ & $V$ & $q$ & $C_{y_n}$ & $K_n$ & $P_n*10^{-5}$ & $P_p*10^{-5}$ & $\Delta \bar{p}(n_x)$ & $V_y^*$ & $\bar{R}_{кр}$ & $q_{ч}$ & $q_{км}$ \\ 
\hline
$-$ & $\frac{м}{с}$ & $\frac{км}{ч}$ & $\frac{H}{м^2}$ & $-$ & $-$ & $H$ & $H$ & $-$ & $\frac{м}{с}$ & $-$ & $\frac{кг}{ч}$ & $\frac{кг}{км}$ \\ 
\hline
0.10 & 30. & 106. & 128. & 39.831 & 0.36 & 36.121 & 1.215 & -2.675 & -78.9 & 29.73 & 188212. & 1771.82 \\ 
\hline
0.20 & 59. & 212. & 510. & 9.958 & 1.47 & 8.862 & 1.128 & -0.593 & -35. & 7.85 & 47374. & 222.99 \\ 
\hline
0.30 & 89. & 319. & 1148. & 4.426 & 3.39 & 3.851 & 1.076 & -0.213 & -18.8 & 3.58 & 21150. & 66.37 \\ 
\hline
0.40 & 118. & 425. & 2042. & 2.489 & 6.1 & 2.14 & 1.052 & -0.083 & -9.8 & 2.03 & 12111. & 28.50 \\ 
\hline
0.5 & 148. & 531. & 3190. & 1.593 & 9.35 & 1.396 & 1.045 & -0.027 & -4. & 1.34 & 8201. & 15.44 \\ 
\hline
0.60 & 177. & 637. & 4594. & 1.106 & 11.62 & 1.123 & 1.059 & -0.005 & -0.9 & 1.06 & 6877. & 10.79 \\ 
\hline
0.7 & 207. & 744. & 6253. & 0.813 & 12.18 & 1.072 & 1.094 & 0.002 & 0.3 & 0.98 & 6714. & 9.03 \\ 
\hline
0.80 & 236. & 850. & 8167. & 0.622 & 11.16 & 1.169 & 1.156 & -0.001 & -0.2 & 1.01 & 7698. & 9.06 \\ 
\hline
0.9 & 266. & 956. & 10336. & 0.492 & 8.32 & 1.567 & 1.25 & -0.024 & -6.5 & 1.25 & 10637. & 11.13 \\ 
\hline
0.95 & 280. & 1009. & 11516. & 0.441 & 6.11 & 2.135 & 1.302 & -0.064 & -17.9 & 1.64 & 14663. & 14.53 \\ 
\hline
\end{tabular}
\end{sidewaystable}

Для построение таблицы (TODO: стр 40 в курсовой)

\begin{enumerate}
    \item Определим $M_{{\min}_P}$ и $M_{{\max}_P}$, как точка пересечения
        графиков $P_n(M, H_i)$ и $P_p(M, H_i)$ рисунки @@@ 
    \item Минимально допустимое число $M_{{\min}_{доп}}$, как точка пересечения
        графиков $C_{y_n}(M, H_i)$ и $C_{y_{доп}}(M)$ рисунки @@@
    \item Максимально допустимое число $M$ полета по условиям безопасности 
        определяется как: 
    \[
        M_{{\max}_{доп}} = \min \left\{ M_{пред}, M(V_{i_{\max}} \right\},
    \]
    где $M(V_{i_{\max}}) = \frac{V_{i_{\max}} \sqrt{\Delta^{-1}}}{3.6 a_H}$, 
    $\sqrt{\Delta^{-1}} = \sqrt{\frac{\rho_0}{\rho_H}}$
    \item Располагаемые значение минимального и максимального числа $M$
определяются как: 
\[
    M_{\min} = \max \left\{ M_{{\min}_{доп}}, M_{{\min}_P} \right\},
\]
\[
    M_{\max} = \min \left\{ M_{{\max}_{доп}}, M_{{\max}_P}, M_{пред} \right\},
\]
    \item Число $М_1$ полета, соответствующее минимальной потребной тяге
        определяется как:
        \[
            M_1 = M(P_{n_{\min}}) = \arg \min_{M} \Delta P_n (M)
        \]
    \item Число $М_2$ полета, соответствующее максимальной энергетической скороподъёмности
определяется как:
        \[
            M_2 = M(V_{y_{max}}^*) = \arg \max_{M} V_y^* (M, H_i)
        \]
    \item Минимальные значения часового $q_{ч_{min}}$ и километрового
        $q_{км_{min}}$ расхода топлива, и соответствующие им скорости полета
        определены на графике 2.4.1-7 и 2.5.1-7 или как:
        \[
            q_{ч_{min}} = \min_V q_ч(V, H_i), \, V_3 = V(q_{ч_{min}}) =
            \arg \min_V q_ч (V, H_i)
        \]
        \[
            q_{{км}_{min}} = \min_V q_{км}(V, H_i), \, V_4 = V(q_{{км}_{min}}) =
            \arg \min_V q_{км} (V, H_i)
        \]
\end{enumerate}

\begin{sidewaystable}
    \centering
    \caption{Результаты для построение графика высот и скоростей}
    \label{tab:H_M}
    \begin{tabular}{|c|c|c|c|c|c|c|c|c|c|c|c|c|}
\hline
$H$ & $V_{y_{max}}^*$ & $\underset{\min \, доп}{M [V]}$ & $\underset{\max \, доп}{M [V]}$ & $\underset{\min}{M [V]}$ & $\underset{\max}{M [V]}$ & $\underset{(P_п\, min)}{M_1 [V_1]}$ & $\underset{(V_{y_{max}}^*)}{M_2 [V_2]}$ & $\underset{(q_{ч_{\min}})}{V_3}$ & $\underset{(q_{{км}_{\min}})}{V_4}$ & $M_4$ & $q_{ч_{\min}}$ & $q_{{км}_{\min}}$ \\ 
\hline
$км$ & $\frac{м}{с}$ & $-\,[\frac{км}{ч}]$ & $-\,[\frac{км}{ч}]$ & $-\,[\frac{км}{ч}]$ & $-\,[\frac{км}{ч}]$ & $-\,[\frac{км}{ч}]$ & $-\,[\frac{км}{ч}]$ & $\frac{км}{ч}$ & $\frac{км}{ч}$ & $-$ & $\frac{кг}{ч}$ & $\frac{кг}{км}$ \\ 
\hline
0.0 & 17.79 & $0.240\, [293]$ & $0.612\, [750]$ & $0.240\, [293]$ & $0.612\, [750]$ & $0.300\, [368]$ & $0.380\, [466]$ & 99 & 126 & 0.370 & 6536.16 & 16.3 \\ 
\hline
2.0 & 16.56 & $0.270\, [324]$ & $0.675\, [808]$ & $0.270\, [324]$ & $0.671\, [803]$ & $0.340\, [407]$ & $0.420\, [503]$ & 110 & 133 & 0.400 & 6286.48 & 14.44 \\ 
\hline
4.0 & 13.58 & $0.307\, [359]$ & $0.748\, [874]$ & $0.307\, [359]$ & $0.699\, [817]$ & $0.380\, [444]$ & $0.460\, [538]$ & 120 & 149 & 0.460 & 6193.81 & 12.99 \\ 
\hline
6.0 & 11.23 & $0.352\, [401]$ & $0.800\, [911]$ & $0.352\, [401]$ & $0.726\, [827]$ & $0.440\, [501]$ & $0.500\, [570]$ & 130 & 161 & 0.510 & 6076.86 & 11.58 \\ 
\hline
8.0 & 8.1 & $0.406\, [451]$ & $0.800\, [887]$ & $0.406\, [451]$ & $0.744\, [825]$ & $0.500\, [555]$ & $0.540\, [599]$ & 145 & 176 & 0.570 & 5951.95 & 10.49 \\ 
\hline
10.0 & 4.33 & $0.475\, [513]$ & $0.800\, [863]$ & $0.475\, [513]$ & $0.739\, [796]$ & $0.540\, [582]$ & $0.590\, [636]$ & 156 & 183 & 0.610 & 5902.71 & 9.67 \\ 
\hline
11.0 & 1.98 & $0.518\, [550]$ & $0.800\, [850]$ & $0.518\, [550]$ & $0.717\, [762]$ & $0.580\, [616]$ & $0.600\, [638]$ & 162 & 186 & 0.630 & 6040.57 & 9.55 \\ 
\hline
11.74 & 0.0 & $0.554\, [588]$ & $0.800\, [850]$ & $0.600\, [637]$ & $0.609\, [647]$ & $0.600\, [637]$ & $0.600\, [637]$ & 3 & 3 & 0.010 & -4038345663.93 & -380168988.72 \\ 
\hline
\end{tabular}
\end{sidewaystable}
