\chapter{Расчет лётно – технических характеристик самолета}

Определим следующие характеристики самолета:
\begin{enumerate}
    \item Зависимости от числа M (скорости) и H (высоты) полета результаты
        сведем в таблицы 2.1-2.7:
    \begin{itemize}
    \item  располагаемой и потребной для горизонтального установившегося полета
    тяги силовой установки, 
    \item энергетической скороподъемности,
    \item часового расхода топлива,
    \item километрового расхода топлива.
    \end{itemize}
    \item Зависимости от высоты:
        \begin{itemize}
            \item максимальной энергетической скороподъемности,
            \item минимального часового расхода топлива,
            \item минимального километрового расхода топлива,
            \item минимального и максимального числа M (скорости) полета (с учетом
                ограничений по безопасности полета),
            \item числа $M$ (скорости) полета, соответствующего минимальной
                потребной тяги,
            \item числа $M$ (скорости) полета, соответствующего максимальной
                энергетической скороподъемности,
            \item скорости полета, соответствующей минимальному часовому расходу топлива,
            \item скорости полета, соответствующему минимальному километровому
                расходу топлива
        \end{itemize}
    \item Статический и практический потолки самолета.
\end{enumerate}

Соотношения для расчета:
Узловые точки по числу Маха:
\[
    M = [0.2 0.3 0.4 0.5 0.6 0.7 0.8 0.9 0.95]
\]
\begin{equation}
    V = M a_H,
    \label{eq:V_speed}
\end{equation}
где $a_H$ --- скорость звука на высоте $H$.
\begin{equation}
    q = \frac{\rho_H V^2}{2},
    \label{eq:q_value}
\end{equation}
где $\rho_H$ --- плотность воздуха на высоте $H$.
\begin{equation}
    C_{y_n} = \frac{\bar{m} p_s 10}{q},
    \label{eq:c_y_n}
\end{equation}
где $\bar{m} = 0.95$ --- относительная масса самолета, $p_s$ --- удельная 
нагрузка на крыло.
\begin{equation}
    C_{x_n}(C_y, M) = C_{x_m}(M) + A(M) \left[ C_{y_n} - C_{y_m}(M)\right]^2
    \label{eq:c_x_n}
\end{equation}
где $C_{y_m}$ --- коэффициент подъемной силы при $C_x = C_{x_m}$, $C_{x_m}$ ---
минимальный коэффициент лобового сопротивления, $A$ --- коэффициент отвала
поляры.
\begin{equation}
    K_n = \frac{C_{y_n}}{C_{x_n}}
    \label{eq:K_n}
\end{equation}
\begin{equation}
    P_n = \frac{\bar{m} m_0 g}{K_n}
    \label{eq:P_potr}
\end{equation}
\begin{equation}
    P_p(M,H) = \bar{P}_0 m_0 g \tilde{P}(H,M)
    \label{eq:P_rasp}
\end{equation}
\begin{equation}
    n_x = \Delta \bar{P} = \frac{(P_p - P_n)}{\bar{m} m_0 g}
    \label{eq:n_x}
\end{equation}
\begin{equation}
    V_y^* = \Delta \bar{P} V
    \label{eq:Vy}
\end{equation}
\begin{equation}
    \bar{R} = \frac{P_n}{P_p}
    \label{eq:R_dross}
\end{equation}
\begin{equation}
    q_{ч} = Ce(M,H,\bar{R})P_n = Ce_0 \tilde{Ce}(H,M) \hat{Ce}_{др}(R) P_n
    \label{eq:q_chas}
\end{equation}
\begin{equation}
    q_{км} = \frac{q_{ч}}{3.6V},
    \label{eq:q_km}
\end{equation}
где $q_{ч}$ --- часовой расход топлива, $q_{км}$ --- километровый расход топлива.




Для построение таблицы (TODO: стр 40 в курсовой)

\begin{enumerate}
    \item Определим $M_{{\min}_P}$ и $M_{{\max}_P}$, как точка пересечения
        графиков $P_n(M, H_i)$ и $P_p(M, H_i)$ рисунки @@@ 
    \item Минимально допустимое число $M_{{\min}_{доп}}$, как точка пересечения
        графиков $C_{y_n}(M, H_i)$ и $C_{y_{доп}}(M)$ рисунки @@@
    \item Максимально допустимое число $M$ полета по условиям безопасности 
        определяется как: 
    \[
        M_{{\max}_{доп}} = \min \left\{ M_{пред}, M(V_{i_{\max}} \right\},
    \]
    где $M(V_{i_{\max}}) = \frac{V_{i_{\max}} \sqrt{\Delta^{-1}}}{3.6 a_H}$, 
    $\sqrt{\Delta^{-1}} = \sqrt{\frac{\rho_0}{\rho_H}}$
    \item Располагаемые значение минимального и максимального числа $M$
определяются как: 
\[
    M_{\min} = \max \left\{ M_{{\min}_{доп}}, M_{{\min}_P} \right\},
\]
\[
    M_{\max} = \min \left\{ M_{{\max}_{доп}}, M_{{\max}_P}, M_{пред} \right\},
\]
    \item Число $М_1$ полета, соответствующее минимальной потребной тяге
        определяется как:
        \[
            M_1 = M(P_{n_{\min}}) = \arg \min_{M} \Delta P_n (M)
        \]
    \item Число $М_2$ полета, соответствующее максимальной энергетической скороподъёмности
определяется как:
        \[
            M_2 = M(V_{y_{max}}^*) = \arg \max_{M} V_y^* (M, H_i)
        \]
    \item Минимальные значения часового $q_{ч_{min}}$ и километрового
        $q_{км_{min}}$ расхода топлива, и соответствующие им скорости полета
        определены на графике 2.4.1-7 и 2.5.1-7 или как:
        \[
            q_{ч_{min}} = \min_V q_ч(V, H_i), \, V_3 = V(q_{ч_{min}}) =
            \arg \min_V q_ч (V, H_i)
        \]
        \[
            q_{{км}_{min}} = \min_V q_{км}(V, H_i), \, V_4 = V(q_{{км}_{min}}) =
            \arg \min_V q_{км} (V, H_i)
        \]

\end{enumerate}


