\chapter{Расчет траектории полета}

\section{Расчет характеристик набора высоты}

Начальные условия:
\[
    H_0 = 0;\, M_0 = 1.2 M_{{min}_{доп}}(V_0 = 1.2 V_{{min}_{доп}}.
\]
Конечные условия: 
\[
    (H_к, M_к) = \arg \min_{H, M} q_{км} (M, H)
\]

Конечная высота принимается равная $H_к = 11, \, \text{км}$
Соотношения для расчета :

\begin{equation}
    \frac{dV}{dH} = \frac{V^{i + 1} - V^i}{H^{i + 1} - H^i}
\end{equation}
\begin{equation}
    \kappa = \frac{1}{1 + \frac{V}{g} \frac{d {V}}{d {H}}}
\end{equation}
\begin{equation}
    \theta_{наб} = n_x \kappa 57.3
\end{equation}
\begin{equation}
    V_{y_{наб}} = V_{y_{max}}^* \kappa
\end{equation}
\begin{equation}
    H_э^i = H^i + \frac{(V^i)^2}{2g}
\end{equation}
\begin{equation}
    \Delta H_э = H_э (V_{наб}^{i+1}, H^{i+1}) - H_э (V_{наб}^{i}, H^{i})
\end{equation}
\begin{equation}
    \left( \frac{1}{n_x} \right)_{ср}  = 0.5 \left[ \frac{1}{n_x(H_э^i)} + \frac{1}{n_x(H_э^{i+1})} \right] 
\end{equation}
\begin{equation}
    \left( \frac{1}{V_y^*} \right)_{ср}  = 0.5 \left[ \frac{1}{V_y^*(H_э^i)} + \frac{1}{V_y^*(H_э^{i+1})} \right] 
\end{equation}
\begin{equation}
    \left( \frac{CeP}{V_y^*} \right)_{ср}  = 0.5 \left[ \frac{CeP}{V_y^*(H_э^i)} + \frac{CeP}{V_y^*(H_э^{i+1})} \right] 
\end{equation}
\begin{equation}
    L_{наб} = \sum \left( \frac{1}{n_x} \right)_{ср} \frac{\Delta H_{э}}{1000}
\end{equation}
\begin{equation}
    t_{наб} = \sum \left( \frac{1}{V_y^*} \right)_{ср} \frac{\Delta H_{э}}{60}
\end{equation}
\begin{equation}
    m_{T_{наб}} = \sum \left( \frac{CeP}{V_y^*} \right)_{ср} \frac{\Delta H_{э}}{3600}
\end{equation}




\section{Расчет характеристик крейсерского полета}

Для расчета времени $T_{кр}$ и дальности $L_{кр}$ крейсерского полета:
\begin{equation}
T_{кр} = \frac{60 K_{ГП}}{gCe} \ln{\frac{1 - \bar{m}_{T_{наб}} - \bar{m}_{T_{пр}}}{1 - \bar{m}_{T_{кр}}-\bar{m}_{T_{наб}}-\bar{m}_{T_{пр}}}}
\end{equation}
\begin{equation}
L_{кр} = \frac{36 V K_{ГП}}{gCe} \ln{\frac{1 - \bar{m}_{T_{наб}} - \bar{m}_{T_{пр}}}{1 - \bar{m}_{T_{кр}}-\bar{m}_{T_{наб}}-\bar{m}_{T_{пр}}}}
\end{equation}
где $\bar{m}_{Т_{кр}} = 1 - \bar{m}_{сн} - \bar{m}_цн - \bar{m}_{Т_{наб}} -
\bar{m}_{Т_{снп}} - \bar{m}_{Т_{анз}} - \bar{m}_{Т_{пр}} = 0.1827$ 

Принимаем:
$m_{цн} = 0,26$ – относительная масса пустого снаряженного самолета;\\
$m_{сн} =0,46$ – относительная масса целевой нагрузки;\\
$m_{T_{cнп}} =0.015$ - относительная масса топлива, расходуемая при снижении и
посадке;\\
$\bar{m}_{T_{наб}} \frac{m_{T_{наб}}}{m_0} = $- относительная масса топлива, расходуемая при наборе;
высоты\\
$m_{Т_{анз}} = 0.05$ - аэронавигационный запас топлива;
$m_{Т_{пр}} = 0.01$ - запас топлива для маневрирования по аэродрому, опробования
двигателей, взлета;
$K_{ГП} = 13.51$
$V = 206\, \frac{м}{с^2}$
$Ce= 0.0617\, \frac{Кг}{Н∗ч} $ – удельный расход топлива на высоте крейсерского
полета

Высота в конце крейсерского полета $H_{к\, кр}$ определяется как:
\begin{equation}
    \rho_{H\, {кр}} = \frac{2 \bar{m}_{к\, кр} Ps 10 }{C_{y_{ГП}} V_к^2}
\end{equation}
где $\bar{m}_{к\, кр} = 1 - \bar{m}_{T_{наб}} - \bar{m}_{T_{пр}} - \bar{m}_{T_{кр}}$



\section{Расчет характеристик участка снижения}
Расчет аналогичен расчету участка набора высоты раздел 3.1.
Только в качестве программы снижения принимается зависимость $M_{сн}(H)$,
соответствующая минимуму потребной тяги.

Начальные условия:

Скорость соответствует минимуму потребной тяги. Определяется по графику
$M(P_{n\, \min})=f(H)$ (Рисунок 2.2).
\[
    M_0=0.6; H_0 = 10\, \text{км}
\]
Конечные условия:\\
Скорость в конце снижения соответствует наивыгоднейшей скорости при $Н=0$.
$M_к = 0.30$; $H_к = 0$
Результаты расчетов приведены на таблице №3.3.2, по этим данным построили


\section{Расчет диаграммы транспортных возможностей}
Определим зависимость целевой нагрузки от дальности полета самолета
$m_{цн}(L)$ (Рисунок 3.4.1)
Расчет ведется для трех режимов:
\begin{enumerate}
    \item Полет с максимальной коммерческой нагрузкой,
    \item Полет с максимальным запасом топлива,
    \item Полет без коммерческой нагрузки ( $m_{цн}=0$ ) с максимальным запасом топлива.
\end{enumerate}

Режим 1.

Для данного режима определили в разделах 3.1, 3.2,3.3

$m_{цн} = \frac{m_{ЦН}}{m_0}$

Режим 2.

$ L = L_{наб} + L_{кр} + L_{сн} $

Для упрощения для дальности полета и расход топлива при наб

