\chapter{Расчет траектории полета}

\section{Расчет характеристик набора высоты}

Начальные условия:
\[
    H_0 = 0;\, M_0 = 1.2 M_{{min}_{доп}}(V_0 = 1.2 V_{{min}_{доп}}.
\]
Конечные условия: 
\[
    (H_к, M_к) = \arg \min_{H, M} q_{км} (M, H)
\]

Конечная высота принимается равная $H_к = 11, \, \text{км}$
Соотношения для расчета :

\begin{equation}
    \frac{dV}{dH} = \frac{V^{i + 1} - V^i}{H^{i + 1} - H^i}
\end{equation}
\begin{equation}
    \kappa = \frac{1}{1 + \frac{V}{g} \frac{d {V}}{d {H}}}
\end{equation}
\begin{equation}
    \theta_{наб} = n_x \kappa 57.3
\end{equation}
\begin{equation}
    V_{y_{наб}} = V_{y_{max}}^* \kappa
\end{equation}
\begin{equation}
    H_э^i = H^i + \frac{(V^i)^2}{2g}
\end{equation}
\begin{equation}
    \Delta H_э = H_э (V_{наб}^{i+1}, H^{i+1}) - H_э (V_{наб}^{i}, H^{i})
\end{equation}
\begin{equation}
    \left( \frac{1}{n_x} \right)_{ср}  = 0.5 \left[ \frac{1}{n_x(H_э^i)} + \frac{1}{n_x(H_э^{i+1})} \right] 
\end{equation}
\begin{equation}
    \left( \frac{1}{V_y^*} \right)_{ср}  = 0.5 \left[ \frac{1}{V_y^*(H_э^i)} + \frac{1}{V_y^*(H_э^{i+1})} \right] 
\end{equation}
\begin{equation}
    \left( \frac{CeP}{V_y^*} \right)_{ср}  = 0.5 \left[ \frac{CeP}{V_y^*(H_э^i)} + \frac{CeP}{V_y^*(H_э^{i+1})} \right] 
\end{equation}
\begin{equation}
    L_{наб} = \sum \left( \frac{1}{n_x} \right)_{ср} \frac{\Delta H_{э}}{1000}
\end{equation}
\begin{equation}
    t_{наб} = \sum \left( \frac{1}{V_y^*} \right)_{ср} \frac{\Delta H_{э}}{60}
\end{equation}
\begin{equation}
    m_{T_{наб}} = \sum \left( \frac{CeP}{V_y^*} \right)_{ср} \frac{\Delta H_{э}}{3600}
\end{equation}




\section{Расчет характеристик крейсерского полета}

Для расчета времени $T_{кр}$ и дальности $L_{кр}$ крейсерского полета:
\begin{equation}
T_{кр} = \frac{60 K_{ГП}}{gCe} \ln{\frac{1 - \bar{m}_{T_{наб}} - \bar{m}_{T_{пр}}}{1 - \bar{m}_{T_{кр}}-\bar{m}_{T_{наб}}-\bar{m}_{T_{пр}}}}
\end{equation}
\begin{equation}
L_{кр} = \frac{36 V K_{ГП}}{gCe} \ln{\frac{1 - \bar{m}_{T_{наб}} - \bar{m}_{T_{пр}}}{1 - \bar{m}_{T_{кр}}-\bar{m}_{T_{наб}}-\bar{m}_{T_{пр}}}}
\end{equation}
где $\bar{m}_{Т_{кр}} = 1 - \bar{m}_{сн} - \bar{m}_цн - \bar{m}_{Т_{наб}} -
\bar{m}_{Т_{снп}} - \bar{m}_{Т_{анз}} - \bar{m}_{Т_{пр}} = 0.1827$ 

Принимаем:
$m_{цн} = 0,26$ – относительная масса пустого снаряженного самолета;\\
$m_{сн} =0,46$ – относительная масса целевой нагрузки;\\
$m_{T_{cнп}} =0.015$ - относительная масса топлива, расходуемая при снижении и
посадке;\\
$\bar{m}_{T_{наб}} \frac{m_{T_{наб}}}{m_0} = $- относительная масса топлива, расходуемая при наборе;
высоты\\
$m_{Т_{анз}} = 0.05$ - аэронавигационный запас топлива;
$m_{Т_{пр}} = 0.01$ - запас топлива для маневрирования по аэродрому, опробования
двигателей, взлета;
$K_{ГП} = 13.51$
$V = 206\, \frac{м}{с^2}$
$Ce= 0.0617\, \frac{Кг}{Н∗ч} $ – удельный расход топлива на высоте крейсерского
полета

Высота в конце крейсерского полета $H_{к\, кр}$ определяется как:
\begin{equation}
    \rho_{H\, {кр}} = \frac{2 \bar{m}_{к\, кр} Ps 10 }{C_{y_{ГП}} V_к^2}
\end{equation}
где $\bar{m}_{к\, кр} = 1 - \bar{m}_{T_{наб}} - \bar{m}_{T_{пр}} - \bar{m}_{T_{кр}}$



\section{Расчет характеристик участка снижения}
Расчет аналогичен расчету участка набора высоты раздел 3.1.
Только в качестве программы снижения принимается зависимость $M_{сн}(H)$,
соответствующая минимуму потребной тяги.

Начальные условия:

Скорость соответствует минимуму потребной тяги. Определяется по графику
$M(P_{n\, \min})=f(H)$ (Рисунок 2.2).
\[
    M_0=0.6; H_0 = 10\, \text{км}
\]
Конечные условия:\\
Скорость в конце снижения соответствует наивыгоднейшей скорости при $Н=0$.
$M_к = 0.30$; $H_к = 0$
Результаты расчетов приведены на таблице №3.3.2, по этим данным построили


\section{Расчет диаграммы транспортных возможностей}
Определим зависимость целевой нагрузки от дальности полета самолета
$m_{цн}(L)$ (Рисунок 3.4.1)
Расчет ведется для трех режимов:
\begin{enumerate}
    \item Полет с максимальной коммерческой нагрузкой,
    \item Полет с максимальным запасом топлива,
    \item Полет без коммерческой нагрузки ( $m_{цн}=0$ ) с максимальным запасом топлива.
\end{enumerate}

Режим 1.

Для данного режима определили в разделах 3.1, 3.2,3.3

$m_{цн} = \frac{m_{ЦН}}{m_0}$

Режим 2.

$ L = L_{наб} + L_{кр} + L_{сн} $

Для упрощения для дальности полета и расход топлива при наборе и снижении,
для всех режимов соответствует первому режиму.

\[
    \bar{m}_{взл} = 1
\]
\[
    \bar{m}_{T_{кр}} = \bar{m}_{T_{max}}- \bar{m}_{T_{наб}} -
\bar{m}_{T_{сн}} - \bar{m}_{T_{анз}} - \bar{m}_{T_{пр}} 
\]
\[
\bar{m}_{T_{max}} = 0.5258
\]
\[
    L_{кр} = \frac{36 V K}{gCe} \ln{\frac{\bar{m}_{взл} - \bar{m}_{T_{наб}} - \bar{m}_{T_{пр}}}{\bar{m}_{взл}-\bar{m}_{T_{кр}}-\bar{m}_{T_{наб}} - \bar{m}_{T_{пр}}}}
\]
\[
    \bar{m}_{цн} = 1 - \bar{m}_{пуст} - \bar{m}_{T_{max}}
\]
\[
    \bar{m}_{пуст} = \frac{88500}{m_0}
\]

Режим 3.
\[
    \bar{m}_{взл} = \bar{m}_{пуст} + \bar{m}_{T_{max}}
\]

\section{Расчет взлетно-посадочных характеристик самолета}


Для расчета: скорости отрыва при взлете $V_{отр}$, длины разбега
$L_{р}$, взлетной дистанции
$L_{вд}$, скорости касания ВПП при посадке
$V_{кас}$, длины пробега
$L_{пр}$, посадочной дистанции
$L_{пд}$.

Предполагается что:

1. Угол атаки при разбеге и пробеге $\alpha_р = \alpha_{п} = 2^\circ$ 

2. Угол атаки при отрыве и касании ВПП $\alpha_{отр} = \alpha_{кас} = 6^\circ$

3. Безопасная высота пролета препятствий $H_{взл} = 10.7\, м$ и $H_{пос} = 15\, м$

4. Тяга двигателей $P_{взл} = (1.2 ... 1.3)P$, $Ce_{взл} = (1.03 ... 1.05) Ce_0$ 

5. При пробеге по ВПП используется реверс тяги.

Соотношения для расчета:

\begin{equation}
    V_{отр} = \sqrt{\frac{20 P_s (1 - 0.9 \bar{P}_{взл} \sin{\alpha_{отр}})}
    {\rho_0 C_{y_{отр}}}}
\end{equation}
\begin{equation}
    C_p = 0.9 \bar{P}_{взл} - f_p
\end{equation}
\begin{equation}
    b_p = (C_{x_p} - f_p C_{y_p}) \frac{\rho_0}{2 P_s 10},
\end{equation}
где $f_p = 0.02$
\begin{equation}
    L_p = \frac{1}{2 g b_p} \ln{\frac{C_p}{C_p -  b_p V_{отр}^2}} 
\end{equation}
\begin{equation}
    V_2 = 1.1 V_{отр}
\end{equation}
\begin{equation}
    \hat{V}_{ср} = \sqrt{\frac{V_2^2 + V_{отр}^2}{2}}
\end{equation}
\begin{equation}
    \hat{n}_{x_{ср}} = \bar{P}_{взл} - \frac{C_{x_{отр}} \rho_0 \hat{V}_{ср}^2}
    {P_s 20}
\end{equation}
\begin{equation}
    L_{вув} = \frac{1}{\hat{n}_{x_{ср}}} \left( \frac{V_2^2 + V_{отр}^2}{2g} +
    H_{взл}\right) 
\end{equation}
\begin{equation}
    \bar{m}_{пос} = \bar{m}_{к\, кр} - \bar{m}_{T_{снп}}
\end{equation}
\begin{equation}
    V_{кас} = \sqrt{\frac{2 \bar{m}_{пос} P_s 10 }{ C_{y_{кас}} \rho_0 }}
\end{equation}
\begin{equation}
    \bar{P}_{рев} = \frac{P_{рев}}{m_{пос} g}
\end{equation}
\begin{equation}
    a_n = - \bar{P}_{рев} - f_n
\end{equation}
\begin{equation}
    b_n = \frac{\rho_0}{\bar{m}_{пос} P_s 20} (C_{x_{проб}} - f_n C_{y_{проб}})
\end{equation}
\begin{equation}
    L_{проб} = \frac{1}{2g b_n} \ln{\frac{a_n - b_n V_{кас}^2}{a_n}}
\end{equation}
\begin{equation}
    C_{y_{пос}} = 0.7 C_{y_{кас}}(\alpha_{кас})
\end{equation}
\begin{equation}
    V_{пл} = \sqrt{\frac{2 \bar{m}_{пос} P_s 10 }{C_{y_{пос}} \rho_0}}
\end{equation}
\begin{equation}
    K_{пос} = \frac{C_{y_{пос}}}{C_{x_{пос}}}
\end{equation}
\begin{equation}
    L_{вуп} = K_{пос} \left( H_{пос} + \frac{V_{пл}^2 - V_{кас}^2}{2g} \right) 
\end{equation}
\begin{equation}
    L_{пд} = L_{проб} + L_{вуп}
\end{equation}

Результаты расчетов на таблице № 3.5.1

\section{Расчет характеристик маневренности самолета}

В данном разделе определим характеристики правильного виража.

Расчеты ведутся для высоты $H=6\, \text{км}$.

Характеристики маневренности рассчитываются при 50\%-ом выгорании
топлива для массы самолета: $\bar{m}_с = 1 - 0.5 \bar{m}_T$

Для расчета таблицы №3.6.1:
\begin{enumerate}
    \item Максимальная допустимая нормальная перегрузка:
        \[
            n_{y_{доп}}=\min \left\{ n_{y_{э}},\,n_y(C_{y_{доп}}) \right\} 
        \]
        $n_{y_{э}} = 3$, $n_y(C_{y_{доп}}) = \frac{C_{y_{доп}}}{C_{y_{ГП}}}$,
        $C_{y_{ГП}} = \frac{\bar{m}_с P_s 10}{q}$
    \item Нормальная перегрузка предельного правильного виража
        \[
            n_{y_{вир}}=\min \left\{ n_{y_{доп}},\,n_{y_P} \right\} 
        \]
        $n_{y_{P}} = \frac{1}{C_{y_{a}ГП}} \left( C_{y_m} + \sqrt{
        \frac{\bar{P} C_{y_{a}ГП} - C_{x_{м}} }{A}} \right) $, $\bar{P} = \frac{P_p}{mg}$
    \item Кинематические параметры виража:
        \[
            \omega_{вир}  = \frac{g}{V} \sqrt{n_{y\, вир}^2 - 1}
        \]
        \[
            r_{вир} = \frac{V}{\omega_{вир}}
        \]
        \[
            t_{вир} = \frac{2 \pi r_{вир}}{V}
        \]
        \item Диапазон Маха берется:
            $M = [0.4,\, 0.5,\, 0.6,\,0.7,\,0.8]$
\end{enumerate}


\section{Расчет характеристик продольной статической устойчивости и
управляемости}

Для расчета продольной статической устойчивости и управляемости
необходимо определить безразмерную площадь горизонтального оперения
$\bar{S}_{ГО}$ из условия устойчивости и
балансировки.

Для определения $\bar{S}_{ГО}$ рассчитываются
предельно передняя $\bar{x}_{ТПП}$ для режима
посадки ($H=0,\, M=0.2$) и предельно задняя
$\bar{x}_{ТПЗ}$ центровки:
\[
    \bar{x}_{ТПЗ} = \frac{-m_{Z_0\, БГО} + \bar{x}_{F\, БГО}C_{y\, БГО}+ 
    C_{y\, ГО} \bar{S}_{ГО} K_{ГО} \bar{L}_{ГО}}{C_{y\, БГО}},
\]
Где $C_{y \, БГО} = C_{y_0\,{БГО}} + C_{y\, БГО}^\alpha \alpha$, 
$C_{y\, ГО} = C_{y\, ГО}^{\alpha_{ГО}} \left[ \alpha(1-\epsilon^\alpha) + 
\varphi_{эф}\right] < 0$, $\varphi_{эф} = \varphi_{уст} + n_в \delta_{max}$,
$\delta_{\max} = -25^\circ$, $\varphi_{уст} = -4^\circ$. 
\[
    \bar{x}_{ТПЗ} = \bar{x}_{H} + \sigma_{n\, \min}
\]
$\bar{x}_{H} = \bar{x}_F - \frac{m_z^{\bar{\omega}_z}}{\mu}$, $\mu = \frac{2 P_s 10 }{\rho g b_a}$,
$m_z^{\bar{\omega}_z} = m_{z\, БГО}^{\bar{\omega}_z}+m_{z\, ГО}^{\bar{\omega}_z}$,
$m_{z\, ГО}^{\bar{\omega}_z} = - C_{y\, {ГО}}^{\alpha_{ГО}} \bar{S}_{ГО} \bar{L}_{ГО}^2
\sqrt{K_{ГО}}$
\[
    \bar{x}_F = \bar{x}_{F\, БГО} + \Delta \bar{x}_F
\]
$ \Delta \bar{x}_F \approx \frac{C_{y\, ГО}^{\alpha_{ГО}} }{C_{y}^\alpha}
(1-\varepsilon^\alpha)\bar{S}_{ГО} \bar{L}_{ГО}^2
K_{ГО}$, $\sigma_{n\, \min} = -0.1$

По приведенным формулам для ряда значений
$\bar{S}_{ГО} = (0.01,\, 0.2)$ рассчитывается таблица
3.7.1

Затем графически определяется потребная площадь ГО из условия:

\[
    \bar{x}_{ТПЗ}(\bar{S}_{ГО}) - \bar{x}_{ТПП}(\bar{S}_{ГО}) = \Delta \bar{x}_{э} 1.2 
\]
$\Delta \bar{x}_{э} \approx 0.15$


Далее расчеты характеристик устойчивости и управляемости производятся
для средней центровки: 
\[
    \bar{x}_{T} = 0.5 \left[  \bar{x}_{ТПЗ}(\bar{S}_{ГО}^*) + \bar{x}_{ТПП}(\bar{S}_{ГО}^*) \right] 
\]

Значения величин $\bar{x}_F$, $\bar{x}_H$, $\bar{x}_{ТПЗ}$, $\sigma_n$ определяются в
узловых точках по $M$ на высоте $H=0$ для таблицы 3.7.
\[
    \sigma_n = \bar{x}_{T} - \bar{x}_{F} + \frac{m_z^{\bar{\omega}_z}}{\mu}
\]

Зависимости $\varphi_{бал}(M)$, $\varphi^n(M)$, $n_{y_р}(M)$ для трех значений
высот: $H=(0 \, км,\, 6 \, км,\, H_{кр})$.
\[
    m_z^{C_y} = \bar{x}_T - \bar{x}_F
\]
$\bar{x}_{F} = \bar{x}_{F\, БГО} + \Delta \bar{x}_{F\, ГО}$, $m_z^{\delta_в} = 
-C_{y\, {ГО}}^{\alpha_{ГО}} \bar{S}_{ГО} \bar{L}_{ГО} K_{ГО} n_в$, $C_{y\, {ГО}}
= \frac{10 P_s \bar{m}}{q}$, $\bar{m} = 1 - 0.5 \bar{m}_{T}$,
\[
    m_{Z_0} = m_{Z_0\, БГО} - 
    (1-\varepsilon^\alpha)\bar{S}_{ГО} \bar{L}_{ГО} K_{ГО} C_{y\,{ГО}}^{\alpha_{ГО}}
    \alpha_0
\]
\[
    \delta_{бал}  = - \frac{m_{z_0} m_z^{C_y} C_{y\, {ГП}}}{ m_z^{\delta_в}
    \left(1 + \frac{m_z^{C_y}}{\bar{L}_{го}}\right)} + \frac{\varphi_{уст}}{n_в}
\]
\[
    \delta^n = -57.3 \frac{C_{y\, ГП} \sigma_n}{ m_z^{\delta_в}}
\]
\[
    n_{y_р} = 1 + \frac{\delta_{\max} + \varphi_{уст} - \delta_{бал}}{\delta^n}
\]

